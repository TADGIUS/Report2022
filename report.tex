\documentclass[utf8,usehyperref,12pt]{G7-32}

%removes border around refs, cites
\hypersetup{
	urlbordercolor={1 1 1},
	citebordercolor={1 1 1}, 
	linkbordercolor={1 1 1}
}

\usepackage{amsthm,amsfonts,amsmath,amssymb,amscd}
\usepackage[T2A]{fontenc}
\usepackage[utf8]{inputenc}
\usepackage[english,russian]{babel}
\usepackage{float}
\usepackage{graphicx}
\usepackage{cmap}
\usepackage{color}

\usepackage[all,cmtip]{xy}
\graphicspath{{pictures/}}

\usepackage{subcaption}
\usepackage{cite}

\usepackage{dsfont}
\usepackage{mathrsfs}
%\newtheorem{theoremA}{Теорема}
%\renewcommand*{\thetheoremA}{\Alph{theoremA}}

\TableInChaper
\PicInChaper
\setlength\GostItemGap{2mm}


\NirOrgLongName{\MakeUppercase{ФЕДЕРАЛЬНОЕ АГЕНТСТВО НАУЧНЫХ ОРГАНИЗАЦИЙ\\
ФЕДЕРАЛЬНОЕ ГОСУДАРСТВЕННОЕ БЮДЖЕТНОЕ УЧРЕЖДЕНИЕ НАУКИ \\ ДАГЕСТАНСКИЙ ФЕДЕРАЛЬНЫЙ ИССЛЕДОВАТЕЛЬСКИЙ ЦЕНТР РОССИЙСКОЙ АКАДЕМИИ НАУК}}

\NirBoss{Директор ДФИЦ РАН}{Муртазаев А.К.} %% Заказчик, утверждающий НИР
\NirManager{Врио зав. Отделом математики и информатики ИФ ДФИЦ РАН, кандидат физ.-мат. наук}{Шарапудинов Т.И.}

\NirTown{Махачкала,}
\NirYear{2023}

\NirUdk{УДК \No }
\NirGosNo{Регистрационный \No }

\NirStage{
}{промежуточный, за 2022 г.}{
}

\bibliographystyle{unsrt}

\newtheorem{theorem}{Теорема}[chapter]
\newtheorem*{theorem*}{Теорема}
\newtheorem*{lemma*}{Лемма}
\newtheorem*{abstract}{Аннотация}
\newtheorem{property}{Свойство}
\newtheorem{lemma}{Лемма}[chapter]
\newtheorem{statement}{Утверждение}[chapter]
\newtheorem{definition}{Определение}[chapter]
\newtheorem{example}{Пример}[chapter]
\newtheorem{corollary}{Следствие}[chapter]
\newtheorem*{corollary*}{Следствие}
\newtheorem{remark}{Замечание}[chapter]
\newtheorem*{remark*}{Замечание}
\newtheorem{hypothesis}{Гипотеза}[chapter]

\newtheorem{cond}{Условие}
\newtheorem{theoremA}{Теорема}[chapter]
\newtheorem{lemmaA}{Лемма}[chapter]
\newtheorem{corollaryA}{Следствие}[chapter]

\newtheorem{state}{Предложение}
\newtheorem{proposition}{Предложение}[chapter]
\renewcommand{\thetheoremA}{\thechapter.\Alph{theoremA}}
\renewcommand{\thelemmaA}{\thechapter.\Alph{lemmaA}}
\renewcommand{\thecorollaryA}{\thechapter.\Alph{corollaryA}}
\newcommand{\No}{\textnumero}

\newcommand{\norm}[1]{\|#1\|_{p(\cdot),w}}
\newcommand{\ip}[2]{\langle #1, #2 \rangle}

\DeclareMathOperator*{\esssup}{ess\,sup}
\DeclareMathOperator*{\essinf}{ess\,inf}

\numberwithin{equation}{chapter} %
\renewcommand{\theequation}{\thechapter.\arabic{equation}}

\newenvironment{description}{}{}

\DeclareMathOperator*{\sign}{sign}
\newtheorem{theoremrus}{Теорема}
\newtheorem{lemmarus}{Лемма}
\newtheorem{statementrus}{Утверждение}
\newtheorem{remarkrus}{Замечание}
\newtheorem{corollaryrus}{Следствие}



\def\Ker{\operatorname{Ker}}

\newcommand{\row}[3]{
  \small{#1}&
  \centering{\rule[-2mm]{5cm}{0.2mm}} \newline \centering \footnotesize{{подпись, дата}} &
  #2 \newline \small{{(#3)}} \\ & & \\
}

\usepackage[shortlabels]{enumitem}

%%%%%%%<------------- НАЧАЛО ДОКУМЕНТА
\begin{document}

% \usefont{T2A}{ftm}{m}{} %%% Использование шрифтов Т2 для возможности скопировать текст из PDF-файлов.

\frontmatter %%% <-- это выключает нумерацию ВСЕГО; здесь начинаются ненумерованные главы типа Исполнители, Обозначения и прочее

\NirTitle{\begin{center}
{\large
[НАЗВАНИЕ ТЕМЫ]
}
\\[12pt]
\end{center}
}

\Executors %% Список исполнителей здесь
% %%% это рисует линию размера 3мм и толщиной 0.1 пункт
[ДОЛЖНОСТЬ И ФИО НАУЧНОГО РУКОВОДИТЕЛЯ], & \rule{1\linewidth}{0.1pt} & \\
\vspace{1cm}

[ДОЛЖНОСТЬ И ФИО РУКОВОДИТЕЛЯ ОТДЕЛА], & \rule{1\linewidth}{0.1pt} & \\
\vspace{1cm}

[ДОЛЖНОСТЬ И ФИО СОТРУДНИКА], & \rule{1\linewidth}{0.1pt} & \\

\vspace{0.5cm}


\Referat %Реферат отчёта, не более 1 страницы

Отчет X с., X рис., X таблиц, X источников.

\MakeUppercase{полиномы Якоби -- Соболева, полиномы Мейкснера -- Соболева, ряд Фурье, равномерная сходимость, пространства Соболева и Лебега, условия Макенхоупта, рациональные сплайны, уравнение Бельтрами, задача Римана -- Гильберта, W-метод, уравнение Ито, метод Ванга-Ландау.
}

Объектом исследования являются системы полиномов, ортогональные относительно дискретно-непрерывного скалярного произведения Соболева; интерполяционные рациональные сплайн-функции;
базисность полиномов Лежандра в пространствах Лебега с переменным показателем;
стохастические системы с запаздыванием;
усреднения периодической задачи для уравнения Бельтрами;
интегральные преобразования скалярных и векторных полей;
проверка интервальной раскрашиваемости всех биграфов заданного порядка;
чистые и разбавленные модели Поттса.

В ходе выполнения НИР изучена сходимость рядов Фурье по полиномам Якоби -- Соболева, Мейкснера -- Соболева и исследованы аппроксимативные свойства их частичных сумм в различных функциональных пространствах. Получены динамическое решение интегрального уравнения Вольтерры второго рода в виде коллокационных рациональных сплайн-функций и точные по порядку оценки скорости сходимости приближенных решений. Получено также приближенное решение интегрального уравнения Фредгольма второго рода в случае произвольных сеток узлов.
Получены утверждения, играющие важную роль при изучении базисности системы полиномов Лежандра в пространствах Лебега с переменным показателем.
Предложен и обоснован модифицированный метод регуляризации для анализа различных видов устойчивости стохастических систем, содержащий одновременно компоненты с непрерывным и дискретным временем, и получены достаточные условия моментной устойчивости решений для таких систем.
Изучены вопросы усреднения обобщенного уравнения Бельтрами с локально периодическими коэффициентами и получены оценки погрешности усреднения периодической задачи для обобщенного уравнения Бельтрами в пространствах Соболева и Лебега.
Получены новые формулы обращения интегральных преобразований скалярных и векторных полей, определенных на некоторых семействах ломаных на плоскости.
Разработано программное обеспечение для решения перечислительных проблем дискретной математики, задач компьютерной графики, а также для создания демонстрационного материала по дисциплинам компьютерных наук.
Методом Ванга-Ландау исследована трехвершинная модель Поттса на решетке Кагоме. 
Определены структуры основного состояния и построена фазовая диаграмма. 

Полученные результаты могут найти применение в задачах математической физики, цифровой обработки сигналов, теории управления и в задачах составления расписаний мультипроцессорной системы.


\setcounter{tocdepth}{2} %hide subsections

\tableofcontents

%\NormRefs % Нормативные ссылки
%\Defines % Необходимые определения

\Abbreviations %% Список обозначений и сокращений в тексте
В настоящем отчете о НИР применяют следующие сокращения и обозначения.
\begin{abbreviation}
\item[ДФИЦ] Дагестанский федеральный исследовательский центр
\item[ОМИ] Отдел математики и информатики
\item[РАН] Российская академия наук
\item[ФП] Фазовый переход
\item[ПО] Программное обеспечение
\end{abbreviation}

\Introduction

[В этом файле должно быть записано общее для отдела введение]

\mainmatter %% это включает нумерацию глав и секций в документе ниже

 \chapter{Cистемы функций, ортогональные относительно дискретно-непрерывных скалярных произведений типа Соболева}



\section*{Введение}
\textcolor{red}{Обозначим через $W^r_{L^p_\rho}=W^r_{L^p_\rho}[a,b]$ пространство Соболева, состоящее из $r-1$ непрерывно дифференцируемых на $[a,b]$ функций $f$, таких что $f^{(r-1)}$ абсолютно непрерывна и $f^{(r)} \in L^p_\rho[a,b]$, где $L^p_\rho=L^p_\rho[a,b]$ --- весовое пространство Лебега: $L^p_\rho[a,b] = \{ f: \int_a^b |f(x)|^p\rho(x)dx \}$. При $p=2$ в пространстве $W^r_{L^p_\rho}$ можно ввести скалярное произведение:
\begin{equation}\label{mmg-sob-prod}
	\langle f, g \rangle = \sum_{k=0}^{r-1}f^{(k)}(a)g^{(k)}(a)+\int_a^b f^{(r)}(x)g^{(r)}(x)\rho(x)dx.
\end{equation}
Указанное скалярное произведение принято называть дискретно-непрерывным скалярным произведением типа Соболева.}

\textcolor{red}{Норму в пространстве $W^r_{L^p_\rho}$ определим следующим образом:
\begin{equation}\label{mmg-sob-norm-def}
	\|f\|_{W^r_{L^p_\rho}} = \Bigl[
	\sum_{k=0}^{r-1}|f^{(k)}(a)|^p + \int_a^b |f^{(r)}(t)|^p \rho(t)dt
	\Bigr]^{1/p}.
\end{equation}}

\textcolor{red}{Скалярное произведение типа Соболева характеризуется тем, что оно включает в себя производные перемножаемых функций. Достаточно общий вид скалярного произведения типа Соболева может быть выражен формулой
\begin{equation}\label{mmg-inner-prod-sob-common}
	\langle f,g \rangle = \sum_{k=0}^{m}\int\limits_{\mathbb{R}}f^{(k)}(x)g^{(k)}(x)d\mu_k,
\end{equation}
где $d\mu_k$ --- борелевские меры. Подробный обзор результатов, полученных для систем полиномов, ортогональных относительно различных соболевских скалярных произведений вида \eqref{mmg-inner-prod-sob-common}, можно найти в обзорной работе \cite{mmg-MarcellanXu2015}.}

\textcolor{red}{Как правило, рассматриваются следующие виды соболевских скалярных произведений \eqref{mmg-inner-prod-sob-common}:
\begin{itemize}
	\item
	непрерывные (все меры $\mu_k$, $k \ge 0$, абсолютно непрерывны),
	\item
	дискретные ($\mu_0$ абсолютна непрерывна, а $\mu_k$, $k \ge 1$, --- дискретны),
	\item
	дискретно-непрерывные ($\mu_m$ абсолютна непрерывна, а $\mu_k$, $k < m$, --- дискретны).	
\end{itemize}}

\textcolor{red}{Исследованиям вопросов сходимости рядов Фурье по полиномам, ортогональным относительно непрерывного скалярного произведения, посвящены, например, работы \cite{mmg-MarcellanJacobiSobolev,mmg-CiaurriJacobiSobolev,mmg-CiaurriCoherentPairs,mmg-Fejzullahu2010,mmg-Fejzullahu2013}. В \cite{mmg-MarcellanJacobiSobolev,mmg-CiaurriJacobiSobolev} рассматривается случай, когда $d\mu_k(x)=w_{\alpha+k,\beta+k}(x)dx$, где $w_{\alpha+k,\beta+k}(x)=(1-x)^{\alpha+k}(1+x)^{\beta+k}$ --- веса Якоби. В работе \cite{mmg-CiaurriCoherentPairs} исследуется случай $m=1$, причем меры $\mu_0$ и $\mu_1$ образуют так называемую когерентную пару (см. \cite{mmg-IserlesKoch1991,mmg-MarcellanXu2015}). В \cite{mmg-Fejzullahu2010} получены необходимые условия сходимости по норме ряда Фурье для полиномов, ортогональных относительно \eqref{mmg-inner-prod-sob-common} при $m=1$ и $d\mu_0(x)=d\mu_1(x)=(1-x^2)^{\alpha-1/2}dx$.}

\textcolor{red}{В случае дискретных скалярных произведений сходимость соответствующих рядов Фурье исследовалась в работах \cite{mmg-Marcellan2002,mmg-Rocha2003,mmg-OsilenkerFourier2012,mmg-OsilenkerLinearMethods2015,mmg-Fejzullahu2009,mmg-CiaurriSigma2018}. В работах \cite{mmg-Marcellan2002,mmg-Rocha2003} рассмотрены вопросы поточечной и равномерной сходимости рядов Фурье в случае скалярного произведения вида:
\begin{equation*}
	\langle f,g \rangle = \int_{-1}^{1}f(x)g(x)d\mu(x)+
	\sum_{k=1}^K\sum_{i=0}^{N_k} M_{k,i}f^{(i)}(a_k)g^{(i)}(a_k),
\end{equation*}
где $d\mu(x)$ --- мера Якоби. В \cite{mmg-OsilenkerFourier2012,mmg-OsilenkerLinearMethods2015,mmg-Fejzullahu2009,mmg-CiaurriSigma2018} рассмотрен частный случай приведенного выше скалярного произведения ($K=2$, $N_1=N_2=1$, $a_1=-1$, $a_2=1$) и исследована сходимость рядов Фурье и их линейных средних по соответствующим ортогональным полиномам.}

\textcolor{red}{Ряды Фурье в дискретно-непрерывном случае довольно подробно исследовались в работах Шарапудинова И.И. (см. \cite{mmg-SharapudinovUMN,mmg-SharapudinovIzvRan2019} и приведенные там списки литературы). В них рассматривается скалярное произведение вида \eqref{mmg-sob-prod}.
В \cite{mmg-SharapudinovUMN,mmg-SharapudinovIzvRan2019,mmg-MMG2019,mmg-Gadzhimirzaev2019} исследованы вопросы поточечной и равномерной сходимости рядов Фурье по системам, ортогональным относительно \eqref{mmg-sob-prod} и ассоциированным с такими классическими системами, как система полиномов Якоби, Лагерра, система функций Хаара, Уолша, Лагерра и система косинусов. При некоторых значениях параметров, фигурирующих в определениях классических систем, изучены аппроксимативные свойства рядов Фурье по указанным системам.}

В настоящем пункте остановимся более подробно на некоторых результатах, связанных со скалярным произведением \eqref{mmg-sob-prod} в случае $\rho(x)=\rho(\alpha,\beta; x)=(1-x)^\alpha(1+x)^\beta$.

Пусть $\mathcal{P}^{\alpha,\beta}=\{ \hat{P}_n^{\alpha,\beta} \}_{n=0}^\infty$ --- система полиномов, ортонормированных на отрезке $[-1,1]$ с весом Якоби $\rho(\alpha,\beta; x)$ (система полиномов Якоби). Введем в рассмотрение новую систему функций $\mathcal{P}^{\alpha,\beta}_r$, $r \ge 1$, с помощью равенств:
\begin{gather}
	\label{mmg-sob-def1}
	P_{r,k}^{\alpha,\beta}(x) =\frac{(x+1)^k}{k!}, \quad k=0,1,\ldots, r-1,\\
	\label{mmg-sob-def2}
	P_{r,k}^{\alpha,\beta}(x) =\frac{1}{(r-1)!}\int\limits_{-1}^x(x-t)^{r-1}\hat{P}_{k-r}^{\alpha,\beta}(t)dt, \quad k=r,r+1,\ldots.
\end{gather}
Можно показать, что таким образом определенная система будет ортонормирована относительно \eqref{mmg-sob-prod} при $\rho(x)=\rho(\alpha,\beta; x)$ \cite[с.~231]{mmg-Shii-izvran2018}. Систему $\mathcal{P}_r^{\alpha,\beta}$ будем называть системой полиномов, ортогональной в смысле Соболева и ассоциированной с полиномами Якоби $P_n^{\alpha,\beta}$.
Ряд Фурье функции $f \in W^r_{L^2_{\rho(\alpha,\beta)}}[-1,1]$ по системе $\mathcal{P}_r^{\alpha,\beta}$ и частичная сумма этого ряда имеют следующий вид \cite[с.~227]{mmg-Shii-izvran2018}:
\begin{gather}
	\label{mmg-sob-fourier-series}
	f(x) \sim \sum_{k=0}^{r-1} f^{(k)}(-1)\frac{(x+1)^k}{k!}+ \sum_{k=r}^\infty c^{\alpha,\beta}_{r,k}(f) P_{r,k}^{\alpha,\beta}(x),\\
	\label{mmg-sob-part-sum}
	S^{\alpha,\beta}_{r,n}(f,x) = \sum_{k=0}^{r-1} f^{(k)}(-1)\frac{(x+1)^k}{k!}+ \sum_{k=r}^n c^{\alpha,\beta}_{r,k}(f) P_{r,k}^{\alpha,\beta}(x), \quad n \ge r,
\end{gather}
где $c^{\alpha,\beta}_{r,k}(f)=\int_{-1}^1 f^{(r)}(t)\hat{P}_{k-r}^{\alpha,\beta}(t)\rho(\alpha,\beta; t)dt$.

Одно из замечательных свойств соболевского ряда Фурье \eqref{mmg-sob-fourier-series} состоит в том, что его частичная сумма \eqref{mmg-sob-part-sum} при $n \ge r$ совпадает $r$-кратно с исходной функцией $f(x)$ в точке $x=-1$ \cite[с. 228]{mmg-Shii-izvran2018}:
\begin{equation}\label{mmg-sob-part-sum-coins}
	(S^{\alpha,\beta}_{r,n})^{(\nu)}(f,-1)=f^{(\nu)}(-1), \quad 0\le\nu\le r-1.
\end{equation}
Это очень важное свойство, которое в сочетании с хорошими аппроксимативными свойствами сумм Фурье \eqref{mmg-sob-part-sum}
делает их весьма эффективным инструментом приближённого решения краевых задач для обыкновенных дифференциальных уравнений спектральными методами. Заметим при этом, что суммы Фурье по классическим ортогональным полиномам таким свойством не обладают.

\section{Вспомогательные сведения}
%Рассмотрим некоторые свойства частичных сумм \eqref{mmg-sob-part-sum}.
Из \eqref{mmg-sob-def1}, \eqref{mmg-sob-def2} вытекает соотношение
\begin{equation}\label{mmg-Prk-deriv-prop}
	(P^{\alpha,\beta}_{r,k}(x))^{(\nu)} =
	\begin{cases}
		P^{\alpha,\beta}_{r-\nu,k-\nu}(x), &\nu \le \min\{k,r\},\\
		(P^{\alpha,\beta}_{k-r})^{(\nu-r)}(x), &r < \nu \le k,\\
		0, &\nu > k,
	\end{cases}
\end{equation}
где полагаем $P^{\alpha,\beta}_{0,k}(x)=P^{\alpha,\beta}_{k}(x)$, $r \ge 0$.

Как отмечалось во введении, коэффициенты ряда Фурье по системе $\mathcal{P}^{\alpha,\beta}_r$ имеют вид \cite[с. 10]{mmg-SharapudinovIzvRan2019}:
\begin{equation}\label{mmg-crk}
	c^{\alpha,\beta}_{r,k}(f)=
	\begin{cases}
		f^{(k)}(-1), &k < r,\\
		\int_{-1}^1 f^{(r)}(t)\hat{P}_{k-r}^{\alpha,\beta}(t)\rho(t;\alpha,\beta)dt, &k \ge r.
	\end{cases}
\end{equation}

Отсюда нетрудно показать, что коэффициенты $c^{\alpha,\beta}_{r,k}(f)$ при различных $r$ связаны равенством
\begin{equation}\label{mmg-crk-deriv}
	c^{\alpha,\beta}_{r,k}(f)=c^{\alpha,\beta}_{r-\nu,k-\nu}(f^{(\nu)}), \quad 0 \le \nu \le \min\{r,k\}.
\end{equation}

Из соотношений \eqref{mmg-Prk-deriv-prop} и \eqref{mmg-crk-deriv} получаем:
\begin{equation*}
	(S^{\alpha,\beta}_{r,n})^{(\nu)}(f,x)=
	\sum_{k=0}^{n}c^{\alpha,\beta}_{r,k}(f)(P^{\alpha,\beta}_{r,k})^{(\nu)}(x)=
	\sum_{k=\nu}^{n}c^{\alpha,\beta}_{r-\nu,k-\nu}(f^{(\nu)})P^{\alpha,\beta}_{r-\nu,k-\nu}(x).
\end{equation*}
Последнее выражение представляет собой частичную сумму порядка $n-\nu$ ряда Фурье функции $f^{(\nu)}$ по системе $\mathcal{P}^{\alpha,\beta}_{r-\nu}$. Таким образом, имеет место равенство
\begin{equation}\label{mmg-Srn-deriv}
	(S^{\alpha,\beta}_{r,n})^{(\nu)}(f,x)=
	S^{\alpha,\beta}_{r-\nu,n-\nu}(f^{(\nu)},x), \quad 0 \le \nu \le \min\{r,n\},
\end{equation}
при этом считаем, что $S^{\alpha,\beta}_{0,n}(f)=S^{\alpha,\beta}_n(f)$ --- частичная сумма ряда Фурье по полиномам Якоби $\hat{P}^{\alpha,\beta}_n$.

Поскольку $P^{\alpha,\beta}_{r,k}(x)$ --- полином степени $k$, то $S^{\alpha,\beta}_{r,n}(f,x)$ будет полиномом степени $n$. Кроме того, $S^{\alpha,\beta}_{r,m}(f,x)$, $m \ge n$, оставляет на месте полиномы степени $n$:
\begin{equation}\label{mmg-Srk-proj}
	S^{\alpha,\beta}_{r,m}(p_n,x)=p_n(x), \quad m \ge n,
\end{equation}
где $p_n(x)$ --- полином степени $n$, что нетрудно показать индукцией по $r$. В самом деле, для $r=0$ соотношение \eqref{mmg-Srk-proj}, очевидно, выполнено. Шаг индукции доказывается с помощью следующего соотношения, которое вытекает из \eqref{mmg-Srn-deriv} и \eqref{mmg-sob-part-sum-coins}:
\begin{equation}\label{mmg-Srn-int-1}
	S^{\alpha,\beta}_{r,m}(f,x)=f(-1)+\int_{-1}^x S^{\alpha,\beta}_{r-1,m-1}(f',t)dt, \quad r \ge 1.
\end{equation}

Далее, из \eqref{mmg-sob-part-sum} можно получить следующее интегральное представление для частичных сумм $S^{\alpha,\beta}_{r,n}(f,x)$ при $r=1$:
\begin{equation}\label{mmg-S1n-int-repr}
	S^{\alpha,\beta}_{1,1+n}(f,x) = f(-1)+\int_{-1}^{1}f'(t)\rho(\alpha, \beta; t)K^{\alpha,\beta}_{1,1+n}(x,t)dt,
\end{equation}
где
\begin{equation*}
	K^{\alpha,\beta}_{1,1+n}(x,t)=\sum_{k=1}^{n+1}\hat{P}^{\alpha,\beta}_{k-1}(t)P^{\alpha,\beta}_{1,k}(x).
\end{equation*}
С помощью формулы \eqref{mmg-sob-def2} для ядра $K^{\alpha,\beta}_{1,1+n}(x,t)$ можно получить соотношение:
\begin{equation}\label{mmg-K1n-Kn-repr}
	K^{\alpha,\beta}_{1,1+n}(x,t)=\sum_{k=1}^{n+1}\hat{P}^{\alpha,\beta}_{k-1}(t)\int_{-1}^{x}\hat{P}^{\alpha,\beta}_{k-1}(u)du=\int_{-1}^{x}K^{\alpha,\beta}_n(t,u)du,
\end{equation}
где $K^{\alpha,\beta}_n(t,u)=\sum_{k=0}^{n}\hat{P}^{\alpha,\beta}_{k}(t)\hat{P}^{\alpha,\beta}_{k}(u)$.

\section{Пространство Соболева}
В этом разделе будут рассмотрены некоторые свойства пространств Соболева $W^r_{L^1_{\rho(\alpha, \beta)}}$, $\rho(\alpha, \beta; x)=(1-x)^\alpha(1+x)^\beta$.

\begin{lemma}\label{mmg-complete-W1L1rho}
	Пространство Соболева $W^r_{L^1_{\rho(\alpha, \beta)}}$ является полным при $-1 < \alpha, \beta \le 0$.
\end{lemma}

Пространство $W^r_{L^1_\rho}=W^r_{L^1_{\rho(\alpha, \beta)}}$ перестает быть полным при $\max\{\alpha,\beta\} > 0$. Покажем это для случая $r=1$. Пусть для определенности $\alpha>0$. Рассмотрим последовательность функций
\begin{equation*}
	f_n(x)=
	\begin{cases}
		0, &-1 \le x \le 0,\\
		-\ln (1-x), &0<x<1-\frac{1}{n},\\
		\ln n + n(x-(1-\frac{1}{n})), &1-\frac{1}{n} \le x \le 1.
	\end{cases}
\end{equation*}
Легко видеть, что $f_n(x)$ --- абсолютные непрерывные на $[-1,1]$ функции, производные которых
\begin{equation*}
	f_n'(x)=
	\begin{cases}
		0, &-1 \le x < 0,\\
		\frac{1}{1-x}, &0<x<1-\frac{1}{n},\\
		n, &1-\frac{1}{n} \le x \le 1,
	\end{cases}
\end{equation*}
принадлежат $L^1_{\rho}$. Следовательно, $f_n(x) \in W^1_{L^1_{\rho}}$.
Нетрудно также убедиться в том, что $\|f_n'-g\|_{L^1_\rho} \to 0$, где
\begin{equation*}
	g(x)=
	\begin{cases}
		0, &-1 \le x < 0,\\
		\frac{1}{1-x}, &0<x<1.
	\end{cases}	
\end{equation*}
Отсюда в силу равенства $\|f_n-f_m\|_{W^1_{L^1_{\rho}}} = \|f_n'-f_m'\|_{L^1_\rho}$ вытекает фундаментальность последовательности $\{f_n\}$ в $W^1_{L^1_{\rho}}$. Покажем теперь, что указанная последовательность не имеет предела в $W^1_{L^1_{\rho}}$. Предположим противное, что существует функция $f \in W^1_{L^1_{\rho}}$, такая что $\|f_n-f\|_{W^1_{L^1_{\rho}}} \to 0$. Тогда, пользуясь определением нормы, получаем $\|f'_n-f'\|_{L^1_{\rho}} \to 0$, откуда в силу единственности предела следует равенство $f'(x)=g(x)$ почти всюду. Но такого быть не может, поскольку $f' \in L^1$, а $g \in L^1_\rho \setminus L^1$.

\begin{lemma}\label{mmg-pol-dense}
	Множество полиномов всюду плотно в пространстве $W^1_{L^1_{\rho(\alpha,\beta)}}$, $\alpha, \beta > -1$.
\end{lemma}




\section{Сходимость в равномерной метрике и метрике пространств Соболева}
Из результатов, полученных в общем случае для систем вида \eqref{mmg-sob-def1}, \eqref{mmg-sob-def2} \cite[с. 38, теорема 2]{mmg-SharapudinovUMN}, следует, что при $-1 < \alpha, \beta < 1$ ряд Фурье \eqref{mmg-sob-fourier-series} сходится равномерно на $[-1,1]$ к функциям $f \in W^r_{L^2_{\rho(\alpha,\beta)}}[-1,1]$ \cite[с. 68, следствие 5]{mmg-SharapudinovUMN}.
Возникает естественный вопрос о том, сохранится ли свойство равномерной сходимости рядов Фурье \eqref{mmg-sob-fourier-series} для функций $f(x)\in W^r_{L^p_{\rho(\alpha,\beta)}}[-1,1]$, когда $1 \le p < 2$.
В статье \cite{mmg-Shii-izvran2018}, опираясь на результаты Макенхоупта, получены условия равномерной сходимости рядов Фурье по соболевской системе $\mathcal{P}^{-\frac12,-\frac12}_r=\{\hat{P}^{-\frac12,-\frac12}_{r,k}\}_{k=0}^\infty$, порождённой системой полиномов Чебышева первого рода $\{\hat{P}^{-\frac12,-\frac12}_{k}\}$.
\begin{theoremA}\label{mmg-st-sob-cheb-uniconv}
	Пусть $A, B \in \mathbb{R}$, $p>1$ таковы, что
	\begin{equation}\label{mmg-}
		\left|\frac{A+1}{p}-\frac{1}{4}\right|<\frac{1}{4},\quad
		\left|\frac{B+1}{p}-\frac{1}{4}\right|<\frac{1}{4}.
	\end{equation}
	Тогда если $f\in W^r_{L_{\rho(A,B)}^p}[-1,1]$, $r \ge 1$, то ряд Фурье функции $f$ по системе $\mathcal{P}^{-\frac12,-\frac12}_r$ равномерно на $[-1,1]$ сходится к $f(x)$.
\end{theoremA}
Из этой теоремы непосредственно выводится следствие.
\begin{corollaryA}
	Если $f \in W^r_{L_{\rho(-\frac{1}{2},-\frac{1}{2})}^p}[-1,1]$, $p>1$, то ряд Фурье функции $f$ по системе $\mathcal{P}^{-\frac12,-\frac12}_r$ равномерно на $[-1,1]$ сходится к $f(x)$.
\end{corollaryA}
Заметим, что ряд Фурье по системе $\mathcal{P}^{-\frac12,-\frac12}_r$ может быть построен для любой функции $f \in W^r_{L_{\rho(-\frac{1}{2},-\frac{1}{2})}^1}[-1,1]$. Однако приведённое выше следствие справедливо только для функций $W^r_{L_{\rho(-\frac{1}{2},-\frac{1}{2})}^p}[-1,1]$ при $p>1$. Интерес представляет вопрос о том, будет ли справедливо утверждение следствия для более общего случая, когда $f \in W^r_{L_{\rho(-\frac{1}{2},-\frac{1}{2})}^1}[-1,1]$. Положительный ответ на этот вопрос получен в \cite[теорема 6]{mmg-Shii-izvran2018}.
\begin{theoremA}
	Если $f \in W^r_{L_{\rho(-\frac{1}{2},-\frac{1}{2})}^1}[-1,1]$, то ряд Фурье функции $f$ по системе $\mathcal{P}^{-\frac12,-\frac12}_r$ равномерно на $[-1,1]$ сходится к $f(x)$.
\end{theoremA}

Аналог теоремы \ref{mmg-st-sob-cheb-uniconv} доказан также для системы соболевских функций $\mathcal{P}_r^{\alpha,0}=\{P_{r,k}^{\alpha,0}\}_{k=0}^\infty$, порождённых полиномами Якоби $P_k^{\alpha,0}(x)$ при $-1<\alpha\le\frac{1}{2}$, $\alpha$ --- дробное \cite[с. 10, следствие 1]{mmg-Shii-matzam2017}.
\begin{theoremA}\label{mmg-st-sob-jac-0-uniconv}
	Пусть
	$-1<\alpha\le\frac{1}{2}$, $A,B\in\mathbb{R}$, $p>1$ таковы, что
	\begin{equation*}
		\left|\frac{A+1}{p}-\frac{\alpha+1}{2}\right|<
		\min\left\{\frac{1}{4},\frac{\alpha+1}{2}\right\}, \quad
		\left|\frac{B+1}{p}-\frac{1}{2}\right|<\frac{1}{4}.
	\end{equation*}
	Тогда если $f\in W^r_{L_{\rho(A,B)}^p}[-1,1]$, $r \ge 1$, то ряд Фурье функции $f$ по системе $\mathcal{P}_r^{\alpha,\beta}$ равномерно на $[-1,1]$ сходится к $f(x)$.
\end{theoremA}

Одним из результатов данного раздела является обобщение теорем \ref{mmg-st-sob-cheb-uniconv}, \ref{mmg-st-sob-jac-0-uniconv}.
\begin{theorem}\label{mmg-sob-uni-conv-muck}
	Пусть $-1 < \alpha,\beta$, $A,B \in \mathbb{R}$, $p>1$, таковы, что
	\begin{gather}
		\label{mmg-muck-A}
		\left|\frac{A+1}{p}-\frac{\alpha+1}{2}\right|<
		\min\left\{\frac{1}{4},\frac{\alpha+1}{2}\right\},\\
		\label{mmg-muck-B}
		\left|\frac{B+1}{p}-\frac{\beta+1}{2}\right|<\min\left\{\frac{1}{4},\frac{\beta+1}{2}\right\},\\
		\label{mmg-Lpw-in-L1-cond}
		\frac{A+1}{p} < 1, \quad \frac{B+1}{p} < 1.
	\end{gather}
	Тогда если $f\in W^r_{L_{\rho(A,B)}^p}[-1,1]$, $r \ge 1$, то ряд Фурье функции $f$ по системе $\mathcal{P}_r^{\alpha,\beta}$ равномерно на $[-1,1]$ сходится к $f(x)$.
\end{theorem}
На самом деле указанную теорему можно несколько усилить.
\begin{theorem}\label{mmg-sob-conv-muck}
	Пусть $\alpha,\beta > -1$, $A,B \in \mathbb{R}$, $p>1$. Для каждой функции $f\in W^r_{L_{\rho(A,B)}^p}[-1,1]$, $r \ge 1$, ряд Фурье по системе $\mathcal{P}_r^{\alpha,\beta}$ сходится к $f(x)$ по норме пространства $W^r_{L_{\rho(A,B)}^p}[-1,1]$ тогда и только тогда, когда
	\begin{gather}
		\label{mmg-muck-A1}
		\left|\frac{A+1}{p}-\frac{\alpha+1}{2}\right|<
		\min\left\{\frac{1}{4},\frac{\alpha+1}{2}\right\},\\
		\label{mmg-muck-B1}
		\left|\frac{B+1}{p}-\frac{\beta+1}{2}\right|<\min\left\{\frac{1}{4},\frac{\beta+1}{2}\right\}.
	\end{gather}
\end{theorem}

При условиях \eqref{mmg-Lpw-in-L1-cond} норма $W^r_{L_{\rho(A,B)}^p}[-1,1]$ сильнее равномерной нормы, поэтому теорема \ref{mmg-sob-uni-conv-muck} вытекает из теоремы \ref{mmg-sob-conv-muck}.

Для $A=\alpha$, $B=\beta$ аналог теоремы \ref{mmg-sob-conv-muck} получен в работе \cite{mmg-Diaz-Gonzalez2020} (см. теорему 5). В ней получены достаточные условия сходимости в пространстве $W^{\alpha,\beta}_{\overline{w},p}$, $w=(w_0,\ldots,w_{r-1}) \in \mathbb{R}^r$, рядов Фурье по системе полиномов, ортогональных относительно более общего скалярного произведения
\begin{equation*}
	\langle f, g \rangle = \sum_{k=0}^{r-1} f^{(k)}(\omega_k)g^{(k)}(\omega_k)+\int_{-1}^{1}f^{(r)}(t)g^{(r)}(t)\rho(\alpha,\beta;t)dt.
\end{equation*}



Как и в случае системы $\mathcal{P}^{-\frac12,-\frac12}_r$, для построения ряда Фурье по системе $\mathcal{P}_r^{\alpha,\beta}$ необходимо и достаточно, чтобы $f\in W^r_{L_{\rho(\alpha,\beta)}^1}[-1,1]$. Теорема \ref{mmg-sob-uni-conv-muck} справедлива только при $p>1$. Основным результатом настоящего раздела является следующая теорема.

\begin{theorem}\label{mmg-sob-uni-conv-L1}
	Если $f \in W^r_{L^1_{\rho(\alpha,\beta)}}$, $r \ge 1$, $-1<\alpha,\beta \le 0$, то ряд Фурье функции $f$ по системе полиномов $\mathcal{P}_r^{\alpha,\beta}$ равномерно на $[-1,1]$ сходится к $f$.
\end{theorem}

Данная теорема носит окончательный характер в том смысле, что при заданных $\alpha, \beta$ нельзя расширить множество рассматриваемых функций (ряд Фурье по $\mathcal{P}_r^{\alpha,\beta}$ не определяется для функций $f \notin  W^r_{L^1_{\rho(\alpha,\beta)}}$).

Доказательство теоремы \ref{mmg-sob-uni-conv-L1} основано на теореме \ref{mmg-st-S1n-norm-est}. Теорема \ref{mmg-st-S1n-norm-est} доказывается с помощью леммы \ref{mmg-st-K1n-bounded}. В доказательстве леммы \ref{mmg-st-K1n-bounded} используется следующее утверждение, которое вытекает из лемм, доказанных в работе А.\,В. Зорщикова \cite[леммы 2 и 3]{mmg-Zorschikov1967}.
\begin{lemma}\label{mmg-st-Zor}
	Пусть $-1 < \alpha,\beta \le 0$, $x \in [-1,1]$. Тогда существует такая постоянная $c(\alpha,\beta)$, зависящая только от $\alpha,\beta$, что для любых $n$  и любых $-1 \le a \le b \le 1$ выполняется неравенство
	\begin{equation*}
		\Bigl| \int_a^b \rho(\alpha,\beta;u)K_n^{\alpha,\beta}(t,u)du \Bigr| \le c(\alpha,\beta).
	\end{equation*}
\end{lemma}

\begin{lemma}\label{mmg-st-K1n-bounded}
	Справедлива оценка
	\begin{equation}\
		|K^{\alpha,\beta}_{1,1+n}(x,t)| \le c(\alpha, \beta), \quad -1 < \alpha, \beta \le 0, x,t \in [-1,1], n \ge 0.
	\end{equation}
\end{lemma}
\begin{theorem}\label{mmg-st-S1n-norm-est}
	Для частичных сумм рядов Фурье по системе полиномов $\mathcal{P}_1^{\alpha,\beta}$ при $-1< \alpha, \beta \le 0$ имеет место неравенство:
	\begin{equation}\label{mmg-S1n-bound}
		\|S^{\alpha,\beta}_{1,n}\|_{W^1_{L^1_{\rho(\alpha,\beta)}} \to C} \le c(\alpha,\beta),
	\end{equation}
	где $c(\alpha,\beta)$ --- константа, не зависящая от $n$.
\end{theorem}










\chapter{Ряды Фурье по системам функций, ортогональным относительно скалярных произведений типа Соболева}

\begin{center}
\textbf{ Аннотация}
\end{center}
Рассмотрена задача об отклонении от функции $f$ из пространства $W^r$ частичных сумм ряда Фурье по системе полиномов Якоби $\{P_n^{\alpha-r,-r}(x)\}$, ортогональной относительно скалярного произведения типа Соболева. Исследовано поведение функции типа Лебега частичных сумм ряда Фурье по системе $\{P_n^{\alpha-r,-r}(x)\}$. Получены оценки в терминах модуля непрерывности $r$ - ой производной функции $f$.

Исследована задача о сходимости ряда Фурье по системе полиномов $\{m_{n,N}^{\alpha,r}(x)\}$, ортонормированной по Соболеву и порожденной системой модифицированных полиномов Мейкснера. В частности, показано, что ряд Фурье по этой системе сходится к $f\in W^r_{l^p_{\rho_N}(\Omega_\delta)}$ поточечно на сетке $\Omega_\delta$ при $p\ge2$. Получены оценки для соответствующей функции Лебега частичных сумм ряда Фурье по системе $\{m_{n,N}^{0,r}(x)\}$.


\section*{Введение}

В последние годы теории ортогональных по Соболеву полиномов посвящено большое число работ. В частности, это связано с тем, что соболевские скалярные произведения и соответствующие им ортогональные системы (и их дифференциальные аналоги) играют важную роль во многих проблемах теории функций, квантовой механики, математической физики, вычислительной математики и т.д. В частности, ряды Фурье по ним обладают важными для приложений свойствами, которые отсутствуют у рядов Фурье по классическим ортогональным системам (см., например, \cite{Ram-Ba-Ra-Pe,Ram-Mar-Xu,Ram-Shar-UMN}).
Например, ряды Фурье по полиномам, ортогональным по Соболеву, представляется более естественным аппаратом, чем ряды Фурье по классическим ортогональным полиномам, для приближенного решения краевых задач, в которых требуется анализ поведения приближенного решения в одной или нескольких точках.
При решении таких задач важную роль играют сходимость и скорость сходимости рядов Фурье. В этом параграфе эти вопросы будут рассмотрены для систем полиномов, ортогональных по Соболеву и порожденных полиномами Якоби (см. п. \ref{Ram-Jac}) и Мейкснера (см. п. \ref{Ram-Mex}).

\section{Об аппроксимативных свойствах рядов Фурье по полиномам Якоби $P_n^{\alpha-r,-r}(x)$, ортогональным по Соболеву}\label{Ram-Jac}

\textbf{ Аннотация.} Рассмотрена задача об отклонении от функции $f$ из пространства $W^r$ частичных сумм ряда Фурье по системе полиномов Якоби $\{P_n^{\alpha-r,-r}(x)\}$, ортогональной относительно скалярного произведения типа Соболева. Исследовано поведение функции типа Лебега частичных сумм ряда Фурье по системе $\{P_n^{\alpha-r,-r}(x)\}$. Получены оценки в терминах модуля непрерывности $r$ - ой производной функции $f$.

\subsection{Введение}

Пусть $-1<\alpha$ -- нецелое, $\rho(x)=(1-x)^\alpha$, $L_\rho^2$ -- пространство Лебега, состоящее из измеримых на $[-1,1]$ функций $f$, для которых
$$
\int_{-1}^{1}f^2(x)\rho(x)dx<\infty.
$$
Для $r\in\mathbb{N}$ через $W^r_{L_\rho^2}$ обозначим пространство функций $f$, непрерывно дифференцируемых $r-1$ раз, причем $f^{(r-1)}$ абсолютно непрерывна на $[-1,1]$, а $f^{(r)}\in L_\rho^2$, $W^r$ -- класс $r$ раз непрерывно дифференцируемых функций $f$, заданных на $[-1,1]$ и для которых $|f^{(r)}|\le 1$.
Для $f,g\in W^r_{L_\rho^2}$ определим скалярное произведение Соболева следующего вида
\begin{equation}\label{Ram-Sob_inner_product}
\langle f,g\rangle_S=\sum_{\nu=0}^{r-1}f^{(\nu)}(-1)g^{(\nu)}(-1)+\int_{-1}^{1}f^{(r)}(x)g^{(r)}(x)\rho(x)dx.
\end{equation}

Рассмотрим систему полиномов
\begin{equation}\label{Ram-Ort_Sob_pol}
	\varphi_n(x)=
	\begin{cases}
		\frac{(x+1)^n}{n!}, & 0\le n\le r-1 \\
		\frac{2^r}{(n+\alpha-r)^{[r]}\sqrt{h_{n-r}^{\alpha,0}}}P_n^{\alpha-r,-r}(x), & r\le n,
	\end{cases}
\end{equation}
где $P_n^{\alpha-r,-r}(x)$ -- полином Якоби степени $n$. В работе~\cite{Ram-SharMN} было показано, что система~\eqref{Ram-Ort_Sob_pol} полна в $W^r_{L_\rho^2}$ и ортонормирована относительно скалярного произведения~\eqref{Ram-Sob_inner_product}. Ряд Фурье по этой системе имеет следующий вид
\begin{equation}\label{Ram-Fourier_series} f(x)=\sum_{k=0}^{r-1}\frac{f^{(k)}(-1)}{k!}(x+1)^k+\sum_{k=r}^{\infty}\frac{2^r\widehat{f_k}P_k^{\alpha-r,-r}(x)}{\sqrt{h_{k-r}^{\alpha,0}}(k+\alpha-r)^{[r]}},
\end{equation}
где
$$
\widehat{f_k}=\langle f,\varphi_k\rangle_S=\int_{-1}^{1}f^{(r)}(t)\frac{P_{k-r}^{\alpha,0}(t)}{\sqrt{h_{k-r}^{\alpha,0}}}(1-t)^\alpha dt, \quad k\ge r.
$$

Через $S^\alpha_{n+2r}(f)=S^\alpha_{n+2r}(f,x)$ обозначим частичную сумму ряда~\eqref{Ram-Fourier_series}:
$$
S^\alpha_{n+2r}(f)=\sum_{k=0}^{r-1}\frac{f^{(k)}(-1)}{k!}(x+1)^k+
\sum_{k=r}^{n+2r}\frac{2^r\widehat{f_k}P_k^{\alpha-r,-r}(x)}{\sqrt{h_{k-r}^{\alpha,0}}(k+\alpha-r)^{[r]}}.
$$
В той же работе были исследованы аппроксимативные свойства сумм $S^\alpha_{n+2r}(f)$ для функций из пространства $W^r$. В частности была доказана следующая (см.~\cite[теорема 4]{Ram-SharMN})

\textbf{Теорема А.}
\textit{Пусть $-1<\alpha$ -- нецелое, $r\in\mathbb{N}$, $f\in W^r$. Тогда
\begin{multline}\label{Ram-ineq_f-S}
|f(x)-S^\alpha_{n+2r}(f)|\le c(r)\left(\frac{\sqrt{1-x^2}}{n+2r}\right)^r\omega\left(f^{(r)},\frac{\sqrt{1-x^2}}{n+2r}\right)+ \\
c(r)\left[\omega\left(f^{(r)},\frac{1}{n+2r}\right)\frac{I_{r,n}^\alpha(x)}{(n+2r)^r}+\omega\left(f^{(r)},\frac{1}{(n+2r)^2}\right)J_{r,n}^\alpha(x)\right],
\end{multline}
где
\begin{equation}\label{Ram-value_I}
I_{r,n}^\alpha(x)=(1+x)^r\int_{-1}^{1-1/n^2}(1-t)^{\alpha-\frac{r}{2}}(1+t)^{\frac{r}{2}}|K_{n+r}^{\alpha-r,r}(x,t)|dtб,
\end{equation}
\begin{equation}\label{Ram-value_J}
J_{r,n}^\alpha(x)=(1+x)^r\int_{1-1/n^2}^{1}(1-t)^{\alpha}|K_{n+r}^{\alpha-r,r}(x,t)|dt,
\end{equation}
$$
\omega(g,\delta)=\sup_{x,t\in[-1,1], |x-t|\le\delta}|f(x)-f(t)|.
$$
}

В связи с неравенством~\eqref{Ram-ineq_f-S} возникает задача об оценке величин $I_{r,n}^\alpha(x)$ и $J_{r,n}^\alpha(x)$, определенных равенствами~\eqref{Ram-value_I} и~\eqref{Ram-value_J} соответственно. Основными результатами настоящего пункта являются теоремы~\ref{Ram-theo1} и~\ref{Ram-theo2}, в которых получены оценки сверху для $I_{r,n}^\alpha(x)$, $J_{r,n}^\alpha(x)$ при $x\in(-1,1)$.

\subsection{Некоторые сведения о полиномах Якоби}

Для произвольных действительных чисел $\alpha$ и $\beta$ полиномы Якоби $P_n^{\alpha,\beta}(x)$ можно определить~\cite{Ram-Sege} с помощью формулы Родрига
$$
P_n^{\alpha,\beta}(x)=\frac{(-1)^n}{2^nn!}\frac{1}{\kappa(x)}\frac{d^n}{dx^n}\{\kappa(x)\sigma^n(x)\},
$$
где $\kappa(x)=\kappa(x;\alpha,\beta)=(1-x)^\alpha(1+x)^\beta$, $\sigma(x)=1-x^2$. Отметим также следующие свойства полиномов $P_n^{\alpha,\beta}(x)$, которые можно найти в~\cite{Ram-Sege}:
\begin{itemize}
	\item
	соотношение ортогональности
	$$
	\int_{-1}^{1}P_n^{\alpha,\beta}(x)P_m^{\alpha,\beta}(x)\kappa(x)dx=h_n^{\alpha,\beta}\delta_{n,m},\quad \alpha, \beta>-1,
	$$
	где
	$$
	h_n^{\alpha,\beta}=\frac{\Gamma(n+\alpha+1)\Gamma(n+\beta+1)2^{\alpha+\beta+1}}{n!(2n+\alpha+\beta+1)\Gamma(n+\alpha+\beta+1)};
	$$
	
	\item
	равенства
	\begin{multline}\label{alpha_1}
		(1-x)P_n^{\alpha+1,\beta}(x)=\\
		\frac{2}{2n+\alpha+\beta+2}\left[(n+\alpha+1)P_n^{\alpha,\beta}(x)-(n+1)P_{n+1}^{\alpha,\beta}(x)\right],
	\end{multline}
	\begin{multline}\label{beta_1}
		(1+x)P_n^{\alpha,\beta+1}(x)=\\
		\frac{2}{2n+\alpha+\beta+2}\left[(n+\beta+1)P_n^{\alpha,\beta}(x)+(n+1)P_{n+1}^{\alpha,\beta}(x)\right];
	\end{multline}
	
	\item
	формула Кристоффеля -- Дарбу
	\begin{multline}\label{Kris_Dar}
		K_n^{\alpha,\beta}(x,t)=\sum_{k=0}^{n}\frac{P_k^{\alpha,\beta}(x)P_k^{\alpha,\beta}(t)}{h_k^{\alpha,\beta}}=\frac{2^{-\alpha-\beta}}{2n+\alpha+\beta+2}\times\\
		\frac{\Gamma(n+2)\Gamma(n+\alpha+\beta+2)}{\Gamma(n+\alpha+1)\Gamma(n+\beta+1)}
		\frac{P_{n+1}^{\alpha,\beta}(x)P_n^{\alpha,\beta}(t)-P_n^{\alpha,\beta}(x)P_{n+1}^{\alpha,\beta}(t)}{x-t};
	\end{multline}
	
	\item
	весовая оценка ($-1\le x\le1$)
	\begin{equation*}
		\sqrt{n}|P_n^{\alpha,\beta}(x)|\le c(\alpha,\beta)\left(\sqrt{1-x}+\frac{1}{n}\right)^{-\alpha-\frac12}
		\left(\sqrt{1+x}+\frac{1}{n}\right)^{-\beta-\frac12},
	\end{equation*}
\end{itemize}
где здесь и всюду в дальнейшем $c$, $c(\alpha,\beta)$, $c(\alpha,\beta,r)$ -- положительные числа, зависящие только от указанных параметров и различные в разных местах.

\subsection{Формулировка основных результатов}

В дальнейшем при оценке величин $I_{r,n}^\alpha(x)$, $J_{r,n}^\alpha(x)$ нам понадобятся некоторые преобразования формулы Кристоффеля -- Дарбу, определенной равенством~\eqref{Kris_Dar}. Для этого воспользуемся равенством~\eqref{alpha_1}, из которого находим
$$
P_{n+1}^{\alpha,\beta}(y)=\frac{n+\alpha+1}{n+1}P_{n}^{\alpha,\beta}(y)-\frac{2n+\alpha+\beta+2}{2(n+1)}(1-y)P_{n}^{\alpha+1,\beta}(y).
$$
С учетом этого равенства нетрудно получить оценку для ядра $K_n^{\alpha,\beta}(x,t)$:
$$
|K_n^{\alpha,\beta}(x,t)|\le
c(\alpha,\beta)\frac{n}{|x-t|}\left[(1-t)|P_{n}^{\alpha+1,\beta}(t)P_{n}^{\alpha,\beta}(x)|+
(1-x)|P_{n}^{\alpha+1,\beta}(x)P_{n}^{\alpha,\beta}(t)|\right].
$$
Если воспользоваться равенством~\eqref{beta_1}, то можно получить аналогичную оценку для $K_n^{\alpha,\beta}(x,t)$:
$$
|K_n^{\alpha,\beta}(x,t)|\le
c(\alpha,\beta)\frac{n}{|x-t|}\left[(1+t)|P_{n}^{\alpha,\beta+1}(t)P_{n}^{\alpha,\beta}(x)|+
(1+x)|P_{n}^{\alpha,\beta+1}(x)P_{n}^{\alpha,\beta}(t)|\right].
$$

Теперь перейдем к формулировке основных результатов. Имеет место следующая
\begin{theorem}\label{Ram-theo1}
	Пусть $r-1<\alpha$ -- нецелое, $x\in(-1,1)$. Тогда для величины $I_{r,n}^\alpha(x)$ справедливы следующие оценки:
	
	1) если $x\in\left[0,1-\frac{1}{2n^2}\right]$, то
	\begin{equation*}
		I_{r,n}^\alpha(x)\le c(\alpha,r)(1-x)^{\frac{r}{2}}\left[\ln(n\sqrt{1-x}+1)+(1-x)^{-\frac{\alpha}{2}-\frac14}+1\right];
	\end{equation*}
	
	2) если $x\in\left(1-\frac{1}{2n^2},1\right)$, то
	\begin{equation*}
		I_{r,n}^\alpha(x)\le c(\alpha,r)
		\begin{cases}
			1, & \alpha\le r-\frac12, \\
			(1-x)^{\frac{r-\alpha}{2}-\frac14}, & \alpha>r-\frac12;
		\end{cases}
	\end{equation*}
	
	3) если $x\in\left[-1+\frac{1}{2n^2},0\right)$, то
	\begin{equation*}
		I_{r,n}^\alpha(x)\le c(\alpha,r)(1+x)^{\frac{r}{2}}\left(\ln(n\sqrt{1+x}+1)+(1+x)^{-\frac14}+1\right);
	\end{equation*}
	
	4) если $x\in\left(-1,-1+\frac{1}{2n^2}\right)$, то
	\begin{equation*}
		I_{r,n}^\alpha(x)\le c(\alpha,r)(1+x)^{\frac{r}{2}-\frac14}.
	\end{equation*}
\end{theorem}

Перейдем теперь к оценке величины $J_{r,n}^\alpha(x)$, определенной равенством~\eqref{Ram-value_J}. Справедлива следующая

\begin{theorem}\label{Ram-theo2}
	Пусть $r-1<\alpha$ -- нецелое, $x\in(-1,1)$. Тогда для величины $J_{r,n}^\alpha(x)$ справедливы следующие оценки:
	
	1) если $x\in\left(1-\frac{2}{n^2},1\right)$, то
	\begin{equation*}\label{est_for_J1_seg}
		J_{r,n}^\alpha(x)\le \frac{c(\alpha,r)}{n^{2r}},
	\end{equation*}
	
	2) если $x\in\left[0,1-\frac{2}{n^2}\right]$, то
	\begin{equation*}
		J_{r,n}^\alpha(x)\le c(\alpha,r)\frac{(1-x)^{\frac{r-\alpha}{2}-\frac34}}{n^{r+\alpha+\frac32}},
	\end{equation*}
	
	3) если $x\in(-1,0)$, то
	\begin{equation*}
		J_{r,n}^\alpha(x)\le c(\alpha,r)\frac{(1+x)^{\frac{r}{2}-\frac14}}{n^{r+\alpha+\frac32}}.
	\end{equation*}
\end{theorem}

Таким образом, сравнивая оценки для величин $I_{r,n}^\alpha(x)$ и $J_{r,n}^\alpha(x)$, мы можем переформулировать теорему {\textbf{A}} в следующем виде.
\begin{theorem}\label{Ram-theo3}
Пусть $r\in\mathbb{N}$, $r-1<\alpha$ -- нецелое, $f\in W^r$, $x\in(-1,1)$. Тогда
\begin{equation*}
|f(x)-S^\alpha_{n+2r}(f)|\le \frac{c(\alpha,r)}{(n+2r)^r}\omega\left(f^{(r)},\frac{1}{n+2r}\right)\mathcal{I}_{r,n}^\alpha(x),
\end{equation*}
где
$$
\mathcal{I}_{r,n}^\alpha(x)=
\begin{cases}
     (1-x)^{\frac{r}{2}}\left[\ln(n\sqrt{1-x}+1)+(1-x)^{-\frac{\alpha}{2}-\frac14}+1\right], & 0\le x\le 1-\frac{1}{2n^2}; \\
     \begin{cases}
	       1, & \alpha\le r-\frac12, \\
	       (1-x)^{\frac{r-\alpha}{2}-\frac14}, & \alpha>r-\frac12,
     \end{cases} & 1-\frac{1}{2n^2}<x<1;\\
    (1+x)^{\frac{r}{2}}\left(\ln(n\sqrt{1+x}+1)+(1+x)^{-\frac14}+1\right), & -1+\frac{1}{2n^2}\le x<0; \\
    (1+x)^{\frac{r}{2}-\frac14}, & -1<x<-1+\frac{1}{2n^2}.
\end{cases}
$$
\end{theorem}

\subsection{Заключение}
Была рассмотрена задача об отклонении от функции $f$ из пространства $W^r$ частичных сумм ряда Фурье по ортогональной по Соболеву системе полиномов $\{\varphi_n(x)\}_{n=0}^\infty$, в которой $\varphi_n(x)=\frac{(x+1)^n}{n!}$ при $0\le n\le r-1$ и $\varphi_n(x)=\frac{2^r}{(n+\alpha-r)^{[r]}\sqrt{h_{n-r}^{\alpha,0}}}P_n^{\alpha-r,-r}(x)$ при $n\ge r$, где $P_n^{\alpha-r,-r}(x)$ -- полином Якоби степени $n$. Получены оценки сверху для функции типа Лебега частичных сумм ряда Фурье по системе $\{\varphi_n(x)\}_{n=0}^\infty$.


\section{Сходимость ряда Фурье по полиномам Мейкснера -- Соболева и аппроксимативные свойства его частичных сумм}\label{Ram-Mex}

\textbf{ Аннотация.} Исследована задача о сходимости ряда Фурье по системе полиномов $\{m_{n,N}^{\alpha,r}(x)\}$, ортонормированной по Соболеву и порожденной системой модифицированных полиномов Мейкснера. В частности, показано, что ряд Фурье по этой системе сходится к $f\in W^r_{l^p_{\rho_N}(\Omega_\delta)}$ поточечно на сетке $\Omega_\delta$ при $p\ge2$. Кроме того, исследованы аппроксимативные свойства частичных сумм ряда Фурье по системе $\{m_{n,N}^{0,r}(x)\}$. Получены оценки для соответствующей функции Лебега.

\subsection{Введение}
В настоящее время теория полиномов, ортогональных по Соболеву, продолжает интенсивно развиваться. В частности, это связано с тем, что системы полиномов, ортогональные относительно соболевских скалярных произведений, и ряды Фурье по ним обладают важными для приложений свойствами, которые отсутствуют у классических ортогональных систем \cite{Ram-Ba-Ra-Pe,Ram-Mar-Xu,Ram-Shar-UMN,Ram-Shar-VMJ}.
В литературе можно встретить различные подходы к построению систем полиномов, ортогональных по Соболеву, отличающиеся выбором тех или иных скалярных произведений.
Приведем некоторые виды скалярных произведений, связанные с полиномами Мейкснера.
Например, в \cite{Ram-Ar-Go-Mar,Ram-Kh-Old} рассмотрено скалярное произведение Соболева следующего вида
$$
\langle f,g\rangle_S=\sum_{x=0}^{\infty}f(x)g(x)w(x)+\lambda\sum_{x=0}^{\infty}\Delta f(x)\Delta g(x)w(x),
$$
где $\lambda\ge 0$, $\Delta f(x)=f(x+1)-f(x)$, $w(x)$ -- вес Мейкснера. А в \cite{Ram-Bav1,Ram-Bav2} были рассмотрены частные случаи этого скалярного произведения, а именно, в \cite{Ram-Bav1} вместо второй суммы было рассмотрено одно слагаемое $\lambda f(0)g(0)$, в \cite{Ram-Bav2} -- два слагаемых $Mf(0)g(0)+N\Delta f(0)\Delta g(0)$, $M,N\ge 0$. При этом было показано, что полиномы $\{Q_n(x)\}$, ортогональные относительно этих скалярных произведений, можно определить посредством равенства $Q_n(x)=\sum_{k=0}^{n}c_{k,n}M_k^\alpha(x)$, где $M_k^\alpha(x)$ -- полином Мейкснера степени $k$. Далее, в \cite{Ram-Shar-VMJ,Ram-Shar-Sar} было рассмотрено скалярное произведение следующего вида
\begin{equation}\label{Ram-Sob-inner-Intro}
\langle f,g\rangle_S=\sum_{k=0}^{r-1}\Delta^kf(0)\Delta_\delta^kg(0)+\sum_{x=0}^\infty\Delta^rf(x)\Delta^rg(x)w(x)
\end{equation}
и показано, что полиномы, ортонормированные относительно \eqref{Ram-Sob-inner-Intro}, можно определить посредством равенств
$$
m_{r,n}^{\alpha}(x)=\frac{x^{[n]}}{n!},\ n=\overline{0,r-1},
$$
$$
m_{r,n}^{\alpha}(x)=
\frac{1}{\sqrt{h_{n-r}^\alpha}(r-1)!}\sum_{t=0}^{x-r}(x-1-t)^{[r-1]}M_{n-r}^\alpha(t),\ x\ge r,\ n\ge r,
$$
где $x^{[n]}=x(x-1)\cdots(x-n+1)$.
В дальнейшем нам понадобятся некоторые обозначения. Пусть $1\le p<\infty$, $l_w^p(\Omega)$ -- пространство дискретных функций $f$, заданных на сетке $\Omega=\{0, 1, \ldots\}$ и для которых $\|f\|_{l_{w}^p(\Omega)}^p=\sum_{x\in\Omega}|f(x)|^pw(x)<\infty$, а $W^r_{l_{w}^p(\Omega)}$ -- подпространство в $l_{w}^p(\Omega)$.
В \cite{Ram-Shar-Sar} была доказана следующая

\textbf{Теорема B.}
\textit{
Система полиномов $\{m_{r,n}^\alpha(x)\}$ полна в $W^r_{l_w^2(\Omega)}$.
}

Другие виды скалярных произведений Соболева, связанные с полиномами Мейкснера, можно найти в \cite{Ram-Mor-Bal,Ram-Co-So-Vil}.
Результаты, полученные в вышеприведенных работах \cite{Ram-Shar-VMJ,Ram-Ar-Go-Mar,Ram-Kh-Old,Ram-Bav1,Ram-Bav2,Ram-Shar-Sar,Ram-Mor-Bal,Ram-Co-So-Vil}, в основном связаны с исследованием распределения нулей полиномов Мейкснера --  Соболева, изучением их алгебраических, асимптотических и дифференциальных свойств. В то же время остаются мало изученными вопросы сходимости ряда Фурье по полиномам Мейкснера -- Соболева и аппроксимативные свойства его частичных сумм. В связи с этим в отчетном году была рассмотрена система полиномов $\{m_{n,N}^{\alpha,r}(x)\}$, ортонормированная по Соболеву и порожденная системой модифицированных полиномов Мейкснера $\{m_{n,N}^{\alpha}(x)\}$.
Показано, что ряд Фурье по этой системе сходится к $f\in W^r_{l^p_{\rho_N}(\Omega_\delta)}$ поточечно на сетке $\Omega_\delta$ при $p\ge2$. А в случае, когда $1\le p<2$ показано, что существуют функция и сетка $\Omega_\delta$, ряд Фурье которой расходится в некоторой точке $x_0\in\Omega_\delta$. Кроме того, исследованы аппроксимативные свойства частичных сумм ряда Фурье по системе $\{m_{n,N}^{0,r}(x)\}$.

\subsection{Некоторые сведения о полиномах Мейкснера}

Пусть $N>0$, $\delta=1/N$, $\Omega_\delta=\{0,\delta,2\delta,\ldots \}$. Через $M_{n,N}^{\alpha}(x)$ обозначим модифицированные полиномы Мейкснера, которые при $\alpha>-1$ ортогональны на сетке $\Omega_\delta$ относительно веса $\rho_N(x)=e^{-x}\frac{\Gamma(Nx+\alpha+1)}{\Gamma(Nx+1)}(1-e^{-\delta})^{\alpha+1}$. Соответствующие ортонормированные полиномы мы обозначим через $m_{n,N}^{\alpha}(x)=\frac{1}{\sqrt{h_n^\alpha}}M_{n,N}^{\alpha}(x)$, где $h_n^\alpha={n+\alpha\choose n}e^{n\delta}\Gamma(\alpha+1)$. Приведем некоторые свойства полиномов $M_{n,N}^{\alpha}(x)$, которые можно найти в \cite{Ram-SharBook}:
\begin{itemize}
\item
формула Родрига
\begin{equation}\label{Ram-for-Rod}
M_{n,N}^{\alpha}(x)=\frac{\Gamma(Nx+1)e^{n\delta+x}}{n!\Gamma(Nx+\alpha+1)}
\Delta^n_\delta\left\{\frac{\Gamma(Nx+\alpha+1)}
{\Gamma(Nx-n+1)}e^{-x}\right\};
\end{equation}
\item
явный вид
\begin{equation}\label{Ram-explicit-rep}
M_{n,N}^\alpha(x)={n+\alpha\choose n}\sum_{k=0}^n{n^{[k]}(Nx)^{[k]}\over(\alpha+1)_kk!}\left(1-e^\delta\right)^k;
\end{equation}
\item
равенства
\begin{equation}\label{Ram-deriv}
\Delta_\delta^r M_{n,N}^{\alpha}(x)=(1-e^{\delta})^rM_{n-r,N}^{\alpha+r}(x),
\end{equation}
\begin{equation}\label{Ram-parametr-r}
M^{-l}_{n,N}(x)=\frac{(n-l)!}{n!}(e^\delta-1)^l(-Nx)_lM_{n-l,N}^l(x-l\delta),\ 1\le l\le n;
\end{equation}

\item
формула Кристоффеля--Дарбу
\begin{multline}\label{Ram-Kric-Dar}
K_{n,N}^{\alpha}(x,y)=\sum_{k=0}^n m_{k,N}^{\alpha}(x)m_{k,N}^{\alpha}(y)=\\
\frac{\delta}{(e^{\delta}-1)e^{n\delta}}\frac{(n+1)!}{\Gamma(n+\alpha+1)}\frac{M_{n,N}^\alpha(x)M_{n+1,N}^\alpha(y)-
M_{n+1,N}^\alpha(x)M_{n,N}^\alpha(y)}{x-y},
\end{multline}
\end{itemize}
которую посредством элементарных преобразований можно записать в следующем виде
$$
K_{n,N}^\alpha(x,y)={\alpha_n\over(\alpha_n+\alpha_{n-1})}m_{n,N}^{\alpha}(x)m_{n,N}^{\alpha}(y)+
{\alpha_n\alpha_{n-1}\over\alpha_n+\alpha_{n-1}}{\delta\over e^{\delta\over2}-e^{-{\delta\over2}}} {1\over y-x}\times
$$
$$
\left[m_{n,N}^\alpha(y)\left(m_{n+1,N}^\alpha(x)- m_{n-1,N}^\alpha(x)\right)
-m_{n,N}^\alpha(x)\left(m_{n+1,N}^\alpha(y)-m_{n-1,N}^\alpha(y)
\right)\right],
$$
где $\alpha_n=\sqrt{(n+1)(n+\alpha+1)}$.

Далее, при $\alpha>-1$, $0\le x<\infty$, $\theta_n=4n+2\alpha+2$, $\lambda>0$, $1\le n\le \lambda N$, $s\geq0$ справедливы следующие весовые оценки~\cite{Ram-SharBook}:
\begin{equation*}
e^{-x/2}\left|m_{n,N}^\alpha(x\pm s\delta)\right|\le c(\alpha,\lambda,s)\theta_n^{-\frac{\alpha}{2}}A_n^\alpha(x),
\end{equation*}
\begin{equation*}
A_n^\alpha(x)=\begin{cases}
\theta_n^{\alpha},&  0\le x\le \frac{1}{\theta_n},\\
\theta_n^{\alpha/2-1/4}x^{-\alpha/2-1/4},&     \frac{1}{\theta_n}<x\le {\theta_n\over 2},\\
\left[\theta_n(\theta_n^{1/3}+|x-\theta_n|)\right]^{-1/4},& {\theta_n\over2}<x\leq{3\theta_n\over2},\\
e^{-x/4}, & {3\theta_n\over2}<x<\infty,
\end{cases}
\end{equation*}
$$
e^{-x/2}\left|m_{n+1,N}^{\alpha}(x)-m_{n-1,N}^{\alpha}(x)\right|\leq
$$
\begin{equation*}
c(\alpha,\lambda)\begin{cases}
\theta_n^{\alpha/2-1},&  0\le x\le \frac{1}{\theta_n},\\
\theta_n^{-3/4}x^{-\alpha/2+1/4},&     \frac{1}{\theta_n}<x\le {\theta_n\over 2},\\
x^{-\alpha/2}\theta_n^{-3/4}\left[\theta_n^{1/3}+|x-\theta_n|\right]^{1/4},& {\theta_n\over2}<x\leq{3\theta_n\over2},\\
e^{-x/4}, & {3\theta_n\over2}<x<\infty,
\end{cases}
\end{equation*}
где здесь и далее $c(\alpha)$, $c(\alpha, \lambda)$, $c(\alpha, \lambda, s)$ -- положительные числа, зависящие только от указанных параметров, причем различные в разных местах.

В дальнейшем нам также понадобятся следующие утверждения.
\begin{lemma}
Пусть $0\le l$ -- целое, $r\in\mathbb{N}$, $l\le r$. Тогда имеет место равенство:
\begin{equation*}
\Delta^l_\delta\left((Nx)^{[r]}M^r_{n,N}(x-r\delta)\right)=(n-l+r+1)_l(Nx)^{[r-l]}M^{r-l}_{n,N}(x-(r-l)\delta).
\end{equation*}
\end{lemma}

\begin{lemma}[\cite{Ram-MN2019}]
Пусть $-1<\alpha\in\mathbb{R}$, $\theta_n=4n+2\alpha+2$, $\lambda>0$, $N=1/\delta$, $0<\delta\leq1$. Тогда для $1\leq n\leq \lambda N$ имеет место следующая оценка
\begin{equation*}
e^{-x}K_{n,N}^\alpha(x,x)\le c(\alpha,\lambda)
\begin{cases}
n^{1-\alpha}(A_n^\alpha(x))^2, & x\in[0,\frac{\theta_n}{2}]\cup[\frac{3\theta_n}{2},\infty), \\
n^{-\alpha}, & x\in[\frac{\theta_n}{2},\frac{3\theta_n}{2}].
\end{cases}
\end{equation*}
\end{lemma}

\subsection{Ряд Фурье по полиномам Мейкснера -- Соболева и аппроксимативные свойства его частичных сумм}

Пусть $\alpha>-1$. Рассмотрим систему полиномов $\{m_{n,N}^{\alpha,r}(x)\}$:
\begin{equation*}
m_{n,N}^{\alpha,r}(x)=\frac{(Nx)^{[n]}}{n!},\ n=\overline{0,r-1},
\end{equation*}
\begin{equation}\label{Ram-Pol-second}
m_{n,N}^{\alpha,r}(x)=
\frac{1}{(r-1)!}\sum\limits_{t\in \Omega_\delta^x}(Nx-1-Nt)^{[r-1]}m_{n-r,N}^\alpha(t),\ x\ge r\delta,\ n\ge r.
\end{equation}
где $x^{[n]}=x(x-1)\cdots(x-n+1)$, $\Omega_\delta^x=\{0, \delta, \ldots, x-r\delta\}$. Заметим, что $m_{n,N}^{\alpha,r}(x)=0$ при $n\ge r$, $x\in\{0, \delta, \ldots, (r-1)\delta\}$. Действительно, из~\eqref{Ram-Pol-second} и~\eqref{Ram-explicit-rep} имеем
$$
m_{n,N}^{\alpha,r}(x)=\frac{1}{\sqrt{h_{n-r}^\alpha}}
{n-r+\alpha\choose n-r}\sum_{k=0}^{n-r}{(n-r)^{[k]}\left(1-e^\delta\right)^k\over(\alpha+1)_kk!}P_{k+r}(x),
$$
где $P_{k+r}(x)=\frac{1}{(r-1)!}\sum\limits_{t\in \Omega_\delta^x}(Nx-1-Nt)^{[r-1]}(Nx)^{[k]}$. Запишем дискретный аналог формулы Тейлора для функции $d(x)=(Nx)^{[k+r]}$:
$$
d(x)=\sum_{k=0}^{r-1}\Delta_\delta^kd(0){(Nx)^{[k]}\over k!}+\frac{1}{(r-1)!}\sum\limits_{t\in \Omega_\delta^x}(Nx-1-Nt)^{[r-1]}\Delta^r_\delta d(x)=
$$
$$
\sum_{k=0}^{r-1}\Delta_\delta^kd(0){(Nx)^{[k]}\over k!}+(k+r)^{[r]}P_{k+r}(x)=(k+r)^{[r]}P_{k+r}(x).
$$
Отсюда $P_{k+r}(x)=\frac{d(x)}{(k+r)^{[r]}}=\frac{(Nx)^{[k+r]}}{(k+r)^{[r]}}$. А поскольку $(Nx)^{[k+r]}=0$ для $k\ge0$, $x\in\{0, \delta, \ldots, (r-1)\delta\}$, то $m_{n,N}^{\alpha,r}(x)=0$ при $n\ge r$, $x\in\{0, \delta, \ldots, (r-1)\delta\}$.

Если теперь запишем дискретный аналог формулы Тейлора для полинома $M_{n,N}^{\alpha-r}(x)$ и воспользуемся равенством~\eqref{Ram-deriv}, то получим
$$
M_{n,N}^{\alpha-r}(x)=\sum_{k=0}^{r-1}\Delta_\delta^kM_{n,N}^{\alpha-r}(0){(Nx)^{[k]}\over k!}+
\frac{(1-e^\delta)^r}{(r-1)!}\sum\limits_{t\in \Omega_\delta^x}(Nx-1-Nt)^{[r-1]}M_{n-r,N}^{\alpha}(x).
$$
Отсюда с учетом~\eqref{Ram-Pol-second} имеем
\begin{equation}\label{Ram-rep-alpha-r}
m_{n,N}^{\alpha,r}(x)= \frac{1}{(1-e^\delta)^r}\frac{1}{\sqrt{h_{n}^{\alpha}}}\left[M_{n,N}^{\alpha-r}(x)-\sum_{k=0}^{r-1}\Delta_\delta^kM_{n,N}^{\alpha-r}(0){(Nx)^{[k]}\over k!}\right],
\end{equation}
где $\Delta^k_\delta M_{n,N}^{\alpha-r}(0)=(1-e^\delta)^k\frac{\Gamma(n+\alpha-r+1)}{(n-k)!\Gamma(\alpha-r+k+1)}$. Заметим, что $\Delta^k_\delta M_{n,N}^{\alpha-r}(0)=0$ при $\alpha=0$.

Из определения системы $\{m_{n,N}^{\alpha,r}(x)\}$ также вытекает следующее свойство:
\begin{equation}\label{Ram-deriv-for-sys}
\Delta_\delta^\nu m_{n,N}^{\alpha,r}(x)=
\begin{cases}
m_{n-\nu,N}^{\alpha,r-\nu}(x),& 0\le\nu\le r-1, r\le n,\\
m_{n-r,N}^{\alpha}(x),& \nu=r\le n,\\
m_{n-\nu,N}^{\alpha,r-\nu}(x),& \nu\le n<r,\\
0,& n<\nu\le r.
\end{cases}
\end{equation}
Из~\eqref{Ram-deriv-for-sys} следует, что система $\{m_{n,N}^{\alpha,r}(x)\}$ ортонормирована относительно скалярного произведения следующего вида
\begin{equation*}
\langle f,g\rangle_S=\sum_{k=0}^{r-1}\Delta_\delta^kf(0)\Delta_\delta^kg(0)+\sum_{x\in\Omega_\delta}\Delta_\delta^rf(x)\Delta_\delta^rg(x)\rho_N(x).
\end{equation*}

Из теоремы \textbf{B} следует, что система полиномов $\{m_{n,N}^{\alpha,r}(x)\}$ полна в пространстве $W^r_{l^2_{\rho_N}(\Omega_\delta)}$. Нетрудно проверить, что ряд Фурье функции $f\in W^r_{l^2_{\rho_N}(\Omega_\delta)}$ по этой системе имеет следующий вид
\begin{equation}\label{Ram-Fourier-series}
f(x)\sim \sum_{k=0}^{r-1}\Delta_\delta^kf(0){(Nx)^{[k]}\over k!}+\sum_{k=r}^\infty c^\alpha_{r,k}(f)m^{\alpha,r}_{k,N}(x),
\end{equation}
где
\begin{equation*}
c^\alpha_{r,k}(f)=\sum_{t\in\Omega_\delta}\Delta_\delta^r f(t)m^\alpha_{k-r,N}(t)\rho_N(t),\ k\ge r.
\end{equation*}
Из неравенства Гельдера следует, что коэффициенты $c^\alpha_{r,k}(f)$ существуют для любой функции $f\in W^r_{l^p_{\rho_N}(\Omega_\delta)}$ при $p\ge 1$. В связи с этим возникает вопрос о сходимости ряда Фурье~\eqref{Ram-Fourier-series} к функции $f\in W^r_{l^p_{\rho_N}(\Omega_\delta)}$. Справедлива следующая
\begin{theorem}\label{Ram-theoMex1}
Пусть $\alpha>-1$, $1\le p<\infty$. Тогда, если $f\in W^r_{l^p_{\rho_N}(\Omega_\delta)}$, то при $p\ge2$ ряд~\eqref{Ram-Fourier-series} сходится поточечно к $f$ на $\Omega_\delta$. Если же $1\le p<2$, то существуют сетка $\Omega_\delta$ и функция $f\in W^r_{l^p_{\rho_N}(\Omega_\delta)}$, ряд Фурье которой расходится в некоторой точке $x_0\in \Omega_\delta$.
\end{theorem}

Далее, через $S_{n+r,N}^{\alpha,r}(f,x)$ обозначим частичную сумму ряда~\eqref{Ram-Fourier-series}:
\begin{equation}\label{Ram-Part-sum}
S_{n+r,N}^{\alpha,r}(f,x)=\sum_{k=0}^{r-1}\Delta_\delta^kf(0){(Nx)^{[k]}\over k!}+\sum_{k=r}^{n+r} c^\alpha_{r,k}(f)m^{\alpha,r}_{k,N}(x).
\end{equation}
Из~\eqref{Ram-Part-sum} следует, что для $S_{n+r,N}^{\alpha,r}(x)$ имеют место равенства
\begin{equation*}
S_{n+r,N}^{\alpha,r}(f,x)=f(x), \quad x\in\{0, \delta, \ldots, (r-1)\delta\}.
\end{equation*}
Кроме того, если $f(x)=p_{n+r}(x)$ -- алгебраический полином степени не выше $n+r$, то
\begin{equation*}
S_{n+r,N}^{\alpha,r}(p_{n+r},x)\equiv p_{n+r}(x).
\end{equation*}

Рассмотрим теперь вопрос об аппроксимативных свойствах частичных сумм $S_{n+r,N}^{\alpha,r}(f,x)$ при $\alpha=0$. В этом случае из~\eqref{Ram-rep-alpha-r} и~\eqref{Ram-parametr-r} следует, что для $m_{n+r,N}^{0,r}(x)$ имеет место равенство
$$
m_{n+r,N}^{0,r}(x)=\frac{(Nx)^{[r]}}{\sqrt{(n+r)^{[r]}}}m_{n,N}^r(x-r\delta).
$$
Тогда~\eqref{Ram-Part-sum} можно записать в виде
\begin{equation*}
S_{n+r,N}^{0,r}(f,x)=\sum_{k=0}^{r-1}\Delta_\delta^kf(0){(Nx)^{[k]}\over k!}+(Nx)^{[r]}\sum_{k=r}^{n+r} c^0_{r,k}(f)\frac{m^{r}_{k-r,N}(x-r\delta)}{\sqrt{k^{[r]}}}.
\end{equation*}

Далее, через $q_{n+r}(x)$ алгебраический полином степени $n+r,$ для которого
$
\Delta^i_\delta f(0)=\Delta^i_\delta q_{n+r}(0)\ (i=\overline{0, r-1}).
$
Тогда
$$
\left|f(x)-S_{n+r,N}^{0,r}(f,x)\right|\leq\left|f(x)-q_{n+r}(x)\right|+\left|S_{n+r,N}^{0,r}(q_{n+r}-f,x)\right|.
$$
Отсюда для $x\in\Omega_{r,\delta}=\{r\delta, (r+1)\delta, \ldots\}$
$$
e^{-{x\over 2}}x^{-{r\over 2}+{1\over 4}}\left|f(x)-S_{n+r,N}^{0,r}(f,x)\right|\leq e^{-{x\over 2}}x^{-{r\over 2}+{1\over 4}}\left|f(x)-q_{n+r}(x)\right|+
$$
\begin{equation}\label{Ram-gadz-eq15}
e^{-{x\over 2}}x^{-{r\over 2}+{1\over 4}}\left|S_{n+r,N}^{0,r}(q_{n+r}-f,x)\right|.
\end{equation}
Так как $\sum\limits_{k=0}^{r-1}\Delta_\delta^k(q_{n+r}(0)-f(0)){(Nx)^{[k]}\over k!}=0$, то
$$
S_{n+r,N}^{0,r}(q_{n+r}-f,x)=
(Nx)^{[r]}\sum_{k=r}^{n+r}c_{r,k}^0(q_{n+r}-f) \frac{m^{r}_{k-r,N}(x-r\delta)}{\sqrt{k^{[r]}}}=
$$
$$
(Nx)^{[r]}\sum_{k=r}^{n+r}\frac{m^{r}_{k-r,N}(x-r\delta)}{\sqrt{k^{[r]}}}
\sum_{t\in\Omega_{\delta}}\Delta_\delta^r(q_{n+r}(t)-f(t))e^{-t}(1-e^{-\delta})m_{k-r,N}^0(t).
$$
К внутренней сумме применим преобразование Абеля (попутно воспользуемся равенствами \eqref{Ram-for-Rod}, \eqref{Ram-parametr-r}) и получим
$$
\sum_{t\in\Omega_{\delta}}\Delta_\delta^r(q_{n+r}(t)-f(t))e^{-t}(1-e^{-\delta})m_{k-r,N}^0(t)=
$$
$$
(-1)^r(1-e^{-\delta})\sum_{t\in\Omega_{\delta}}(q_{n+r}(t+r\delta)-f(t+r\delta))\frac{\sqrt{e^{(k-r)\delta}}}{(k-r)!}
\Delta_\delta^k\left\{\frac{\Gamma(Nt+1)e^{-t}}{\Gamma(Nt-k+r+1)}\right\}=
$$
$$
(-1)^r\frac{1-e^{-\delta}}{\sqrt{e^{(k-r)\delta}}}\sum_{t\in\Omega_{r,\delta}}(q_{n+r}(t)-f(t))\frac{\Gamma(Nt-r+1)}{\Gamma(Nt+1)}e^{-t}M_{k,N}^{-r}(t)=
$$
$$
\frac{(e^{\delta}-1)^{r+1}}{e^\delta}\sum_{t\in\Omega_{r,\delta}}(q_{n+r}(t)-f(t))e^{-t}\sqrt{k^{[r]}}m_{k-r,N}^{r}(t-r\delta).
$$
Тогда
$$
S_{n+r,N}^{0,r}(q_{n+r}-f,x)=\frac{(e^{\delta}-1)^{r+1}}{e^\delta}(Nx)^{[r]}\sum_{t\in\Omega_{r,\delta}}(q_{n+r}(t)-f(t))e^{-t}K_{n,N}^r(t-r\delta,x-r\delta).
$$
Отсюда и из \eqref{Ram-gadz-eq15} выводим
\begin{multline}\label{Ram-gadz-eq16}
e^{-{x\over 2}}x^{-{r\over 2}+{1\over 4}}\left|f(x)-S_{n+r,N}^{0,r}(f,x)\right|\leq e^{-{x\over 2}}x^{-{r\over 2}+{1\over 4}}\left|f(x)-q_{n+r}(x)\right|+\\
\frac{(e^{\delta}-1)^{r+1}}{e^\delta}e^{-{x\over 2}}x^{-{r\over 2}+{1\over 4}}(Nx)^{[r]}\sum_{t\in\Omega_{r,\delta}}(q_{n+r}(t)-f(t))e^{-t}K_{n,N}^r(t-r\delta,x-r\delta).
\end{multline}

Положим
\begin{equation}\label{Ram-gadz-eq17}
E_{k}^r(f,\delta)=\inf_{q_{k}}\sup_{x\in\Omega_{r,\delta}} e^{-{x\over 2}}x^{-{r\over 2}+{1\over 4}}\left|f(x)-q_{k}(x)\right|,
\end{equation}
где нижняя грань берется по всем алгебраическим полиномам $q_{k}(x)$ степени $k,$ для которых $\Delta_\delta^i f(0)=\Delta_\delta^i q_{k}(0)\ (i=\overline{0, r-1}).$
Тогда из \eqref{Ram-gadz-eq16} и \eqref{Ram-gadz-eq17} получаем
\begin{equation}\label{Ram-LebIne}
e^{-{x\over 2}}x^{-{r\over 2}+{1\over 4}}\left|f(x)-S_{n+r,N}^{0,r}(f,x)\right|\leq E_{n+r}^r(f,\delta)(1+\lambda_{n,N}^{r}(x)),
\end{equation}
где
$$
\lambda_{n,N}^{r}(x)=\frac{(e^{\delta}-1)^{r+1}}{e^\delta}e^{-{x\over 2}}x^{-{r\over 2}+{1\over 4}}(Nx)^{[r]}
\sum_{t\in\Omega_{r,\delta}}e^{-{t\over 2}}t^{{r\over 2}-{1\over 4}}
\left|K_{n,N}^r(t-r\delta,x-r\delta)\right|.
$$
В связи с неравенством \eqref{Ram-LebIne} возникает задача об оценке величины $\lambda_{n,N}^{r}(x)$ на $[r\delta,\infty)$. Пусть $\nu=4n+2r+2$. Введем обозначения:
$X_1=\left[r\delta, \frac{3}{\nu}\right]$,
$X_2=\left(\frac{3}{\nu}, \frac{\nu}{2}\right]$,
$X_3=\left(\frac{\nu}{2}, \frac{3\nu}{2}\right]$,
$X_4=\left(\frac{3\nu}{2}, \infty\right)$.
Справедлива следующая
\begin{theorem}\label{Ram-theoMex2}
Пусть $r\in\mathbb{N}$, $\lambda> 0$, $n\leq\lambda N.$ Тогда имеют место следующие оценки:\\
1) если $x\in X_1\cup X_2$, то
$$
\lambda_{n,N}^{r}(x)\leq c(\lambda, r)\ln (n+1);
$$
2) если $x\in X_3$, то
$$
\lambda_{n,N}^{r}(x)\leq c(\lambda,r)\left[\ln(n+1)+\left(\frac{\nu}{\nu^{1\over3}+|x-\nu|}\right)^{\frac{1}{4}}\right];
$$
3) если $x\in X_4$, то
$$
\lambda_{n,N}^{r}(x)\leq c(\lambda,r)n^{-{r\over 2}+{7\over 4}}x^{{r\over 2}+{1\over 4}}e^{-\frac{x}{4}}.
$$
\end{theorem}

\subsection{Заключение}
Была исследована задача о сходимости ряда Фурье по полиномам, ортогональным по Соболеву и порожденным полиномами Мейкснера.
Кроме того, были исследованы аппроксимативные свойства частичных сумм Фурье по указанным полиномам. В частности, получены оценки для функции Лебега, зависящие от расположения переменной $x$ на полуоси $[r\delta, \infty)$.




	





\chapter{Рациональные сплайн-функции и их приложения}\label{ARK}

\section{Динамическое решение интегрального уравнения Вольтерры в виде
рациональных сплайн-функций}
%\begin{abstract}
%
%Приближенное решение интегрального уравнения Вольтерры второго рода
%представлено в виде коллокационных рациональных сплайн-функций на последовательных отрезках,
%исчерпывающих всю область решения. Получены также оценки скорости сходимости
%приближенных решений к точному в равномерной метрике через модуль непрерывности решения и его
%производных первого и второго порядков.
%\end{abstract}

\subsection{Введение}

Возросший в последние годы интерес к дальнейшему исследованию интегральных уравнений
Вольтерры, в частности, связан с поиском более эффективных методов решения задач
математической физики, с востребованностью решения задач, описывающих модели
биологии, экологии, экономики, а также с потребностью приложений интегральных уравнений
к решению задач по моделированию развивающихся динамических систем (см., напр., \cite{ark-3}
и цитированные там источники). Эти задачи показывают также специфику
уравнений Вольтерры, которая не позволяет получить полное их решение методами
исследований более общих интегральных уравнений Фредгольма, и востребованность новых методов
приближенного решения уравнений Вольтерры.

В данном подразделе рассматривается вопрос приближенного решения интегрального уравнения
 Вольтерры
\begin{equation}\label{ark-eq-1}
y(x)-\lambda \int_a^x K(x,t)y(t) dt=\varphi(x),\quad x\in[a,b],
\end{equation}
с непрерывным на треугольнике $a\leqslant t\leqslant x\leqslant b$ ядром $K(x,t)$
и непрерывной на отрезке $[a,b]$ правой частью $\varphi(x)$. Именно, для натурального
$N\geqslant 2$ берется сетка произвольных узлов $\Delta:a=x_0<x_1<\dots <x_N=b$,  и
в качестве приближенного решения уравнения \eqref{ark-eq-1} на отрезках вида $[a,x_n]$,
$n=1,2,\dots,N$, строятся коллокационные рациональные сплайн-функции с параметром.

Отметим, что вопросы приближенного решения интегральных уравнений с помощью полиномиальных
сплайнов рассматривались в \cite{ark-4,ark-5,ark-6,ark-7} и других работах. Но известно
\cite{ark-7, ark-8}, что классические полиномиальные сплайны для непрерывных функций
по произвольным сеткам узлов с бесконечно малыми диаметрами могут не сходиться.
В отличие от них сплайн-функции по рациональным интерполянтам \cite{ark-9} для любой непрерывной
на данном отрезке функции по любой последовательности сеток узлов с бесконечно малыми диаметрами
равномерно сходятся. В подобной <<безусловной>> сходимости сплайн-функций существенную роль
играет выбор полюсов рациональных интерполянтов, на основе которых строятся сплайн-функции.
При этом интерполянты содержат также параметр, соответствующий выбор которого
влияет на скорость сходимости. Представлен также достаточно эффективный способ нахождения
соответствующего дискретного решения.


\subsection{Основные результаты}
Для данной сетки произвольных узлов $\Delta: a=x_0<x_1<\dots <x_N=b$ $(N\geqslant 2)$
рассмотрим последовательно отрезки вида $[a,x_n]$, $n=1,2,\dots,N$, и промежуточные
сетки узлов $\Delta_n: a=x_0<x_1<\dots<x_n$.

Обозначим $h_i=x_i-x_{i-1}$, $i=1,2,\dots,N$, и относительно произвольного параметра
$\mu>0$ определим набор чисел $g=\{g_1,g_2,\dots,g_{N-1}\}$ таких, что при $i=1,2,\dots,N-1$
имеем
\begin{equation}\label{ark-eq-1.1}
g_i=\begin{cases}
x_{i+1}+\mu h_{i+1} \quad\text{ при }\quad h_{i+1}\leqslant h_i,\\
x_{i-1}-\mu h_i \quad \text{ при }\quad h_{i+1}> h_i.
\end{cases}
\end{equation}

Положим также $G_n=\{g_1,g_2,\dots,g_{n-1}\}$, $n=2,3,\dots,N$.
Всюду ниже $y(x)$ считается точным непрерывным решением интегрального уравнения \eqref{ark-eq-1}.

При $n=2,3,\dots,N$ для троек узлов $x_{i-1}<x_i<x_{i+1}$, $i=1,2,\dots,n-1$,
построим рациональные интерполянты вида
\begin{equation}\label{ark-eq-1.2}
R_i(x,y)=R_i(x,y,[a,x_n])=\alpha_i+\beta_i(x-x_i)+\gamma_i \frac 1{x-g_i},
\end{equation}
которые однозначно определяются условиями $R_i(x_j,y)=y(x_j)$ при $j=i-1,i,i+1$.

Функцию $R_i(x,y, [a,x_n])$ при $x\in[x_{i-1},x_{i+1}]$,
$i=1,2,\dots,n-1$, используя интерполяционные условия,  можно представить также в виде
\begin{equation}\label{ark-eq-1.3}
R_i(x,y,[a,x_n])=a_i(x) y(x_{i-1})+b_i(x) y(x_i)+c_i(x) y(x_{i+1}),
\end{equation}
$$
a_i(x)=\frac{(x-x_i)(x-x_{i+1})(x_{i-1}-g_i)}{(x_{i-1}-x_i)(x_{i-1}-x_{i+1})(x-g_i)},\quad
b_i(x)=\frac{(x-x_{i-1})(x-x_{i+1})(x_i-g_i)}{(x_i-x_{i-1})(x_i-x_{i+1})(x-g_i)},
$$
$$
c_i(x)=\frac{(x-x_{i-1})(x-x_i)(x_{i+1}-g_i)}{(x_{i+1}-x_{i-1})(x_{i+1}-x_i)(x-g_i)}.
$$

При этом $a_i(x)+b_i(x)+c_i(x)=1$ для $x\in[x_{i-1}, x_{i+1}]$, а с учетом расположения узлов
и полюса $g_i$ при $x\in [x_{i-1}, x_i]$ получим неравенства $a_i(x)\geqslant 0$,
$b_i(x)\geqslant 0$, $c_i(x)\leqslant 0$, при $x\in[x_i, x_{i+1}]$ -- неравенства
$a_i(x)\leqslant 0$, $b_i(x)\geqslant 0$, $c_i(x)\geqslant 0$.

Учитывая эти неравенства, в случае $h_{i+1}\leqslant h_i$ последовательно имеем:
$$
|a_i(x)|=\frac{|x-x_i|}{x_i-x_{i-1}}\cdot \frac{x_{i+1}-x}{g_i-x}\cdot
 \frac{g_i-x_{i-1}}{x_{i+1}-x_{i-1}}\leqslant 1,\quad \text{если}\quad x\in [x_{i-1},x_{i+1}];
$$
$$
|c_i(x)|=\mu \frac{x-x_{i-1}}{x_{i+1}-x_{i-1}}\cdot \frac{x_i-x}{g_i-x}
\leqslant \mu,\quad \text{если}\quad x\in [x_{i-1},x_i];
$$
$$
|c_i(x)|=\mu \frac{x-x_{i-1}}{g_i-x}\cdot \frac{x-x_i}{x_{i+1}-x_{i-1}}
\leqslant 1,\quad \text{если}\quad x\in [x_i,x_{i+1}];
$$
$$
|b_i(x)|=b_i(x)=1-a_i(x)-c_i(x)\leqslant 1-c_i(x)\leqslant 1+\mu,
\quad \text{если}\quad x\in [x_{i-1},x_i];
$$
$$
|b_i(x)|=b_i(x)=1-a_i(x)-c_i(x)\leqslant 1-a_i(x)\leqslant 2,
\quad \text{если}\quad x\in [x_i,x_{i+1}].
$$

Значит, если $x\in[x_{i-1}, x_i]$, то $0\leqslant a_i(x)\leqslant 1$,
$0\leqslant b_i(x)\leqslant 1+\mu$,  $-\mu\leqslant c_i(x)\leqslant 0$.

Если же $x\in[x_i, x_{i+1}]$, то $-1\leqslant a_i(x)\leqslant 0$,
$ 0\leqslant b_i(x)\leqslant 2$, $0\leqslant c_i(x)\leqslant 1$.

В случае $h_{i+1}>h_i$ аналогично получим неравенства $0\leqslant a_i(x)\leqslant 1$,
$0\leqslant b_i(x)\leqslant 2$,   $-1\leqslant c_i(x)\leqslant 0$, если $x\in[x_{i-1}, x_i]$,
и неравенства $-\mu\leqslant a_i(x)\leqslant 0$, $0\leqslant b_i(x)\leqslant 1+\mu$,
$0\leqslant c_i(x)\leqslant 1$, если $x\in[x_i,x_{i+1}]$.

Из полученных неравенств, в частности, при $x\in[x_{i-1}, x_{i+1}]$ для $i=1,2,\dots,n-1$ и
$\mu>0$ вытекают единые оценки
\begin{equation}\label{ark-eq-1.5}
|a_i(x)|\leqslant 1+\mu,\quad 0\leqslant b_i(x)\leqslant \max\{2,1+\mu\},
\quad |c_i(x)|\leqslant 1+\mu.
\end{equation}

Положим также
\begin{equation}\label{ark-eq-1.6}
\begin{array}{l}
 R_0(x,y,[a,x_n])=R_1(x,y,[a,x_n]),\quad n=1,2,\dots, N-1,\\
 R_n(x,y,[a,x_n])=R_{n-1}(x,y,[a,x_n]),\quad n=2,3,\dots, N.
\end{array}
\end{equation}

Равенства \eqref{ark-eq-1.6} играют роль краевых условий, с учетом которых
 для $r=1,2$ построим рациональную сплайн-функцию
$R_{n,r}(x,y,[a,x_n])=R_{n,r}(x,y,\Delta_n, G_n, \mu)$
на отрезке $[a,x_n]$, $n=1,2,\dots,N$, такую, что
при $x\in[x_{i-1},x_i]$, $i=1,2,\dots,n$, выполняется равенство
\begin{equation}\label{ark-eq-1.7}
R_{n,r}(x,y,[a,x_n])=R_i(x,y,[a,x_n])A_{i,r}(x)+R_{i-1}(x,y,[a,x_n])B_{i,r}(x),
\end{equation}
где
$$
A_{i,r}(x)=\frac{(x-x_{i-1})^r}{(x-x_{i-1})^r+(x_i-x)^r},\quad B_{i,r}(x)=1-A_{i,r}(x).
$$

Как следует из \cite{ark-9}, $R_{n,r}(x,y,[a,x_n])$ является гладкой функцией класса $C^r_{[a,b]}$.
В \cite{ark-11} показано, что если $f \in C^2_{[a,b]}$, то для $k=0,1,2$ соответственно
$R_{n,2}^{(k)}(x,f,[a,b])$ сходятся равномерно к $f^{(k)}(x)$ на $[a,b]$ при
$\|\Delta\|=\max\{h_i:i=1,2,\dots,N\}\to 0$.

Отметим также, что из \eqref{ark-eq-1.7}, \eqref{ark-eq-1.6} и \eqref{ark-eq-1.2} имеем
\begin{equation}\label{ark-eq-1.8}
R_{n,r}(x_n,y,[a,x_n])=R_{n-1}(x_n,y,[a,x_n])=y(x_n).
\end{equation}

Важным является вопрос об оценке скорости сходимости приближенных решений интегрального
уравнения \eqref{ark-eq-1} к точному его решению из данного класса гладкости.

Для решения этой задачи применяются приводимые далее аппроксимативные свойства
трехточечных рациональных интерполянтов $R_i(x,f)=R_i(x, f,[a,b])$ вида \eqref{ark-eq-1.2}
для функций $f(x)$, определенных на отрезке $[a,b]$, и данной сетки узлов
$\Delta: a=x_0<x_1<\dots<x_N=b$ $(N\geqslant 2)$. Будем пользоваться также
обозначениями
 $\rho_\Delta=\max\{h_i /h_j:\,|i-j|=1, 1\leqslant i, j\leqslant N\}$,
\newline $\omega(\delta, f)=\omega(\delta, f,[a,b])=\sup\{|f(x+h)-f(x)|:
0\leqslant h\leqslant \delta;\, x, x+h\in[a,b]\}$,
\newline $\|f\|_{[a,b]}=\sup\{|f(x)|: x\in[a,b]\}$.

Как показано в \cite{ark-10}, если $f\in C_{[a,b]}$, то при $x\in[x_{i-1}, x_{i+1}]$,
$i=1,2,\dots,N-1$, выполняется неравенство
\begin{equation}\label{ark-eq-1.9}
|f(x)-R_i(x,f)|\leqslant (2+\max\{1,\mu\})\omega(\|\Delta\|, f).
\end{equation}

Следует отметить, что
справедливость неравенства \eqref{ark-eq-1.9} для любой функции $f\in C_{[a,b]}$ и каждой сетки
произвольных узлов на отрезке $[a,b]$ обеспечивает рациональным сплайн-функциям вида
\eqref{ark-eq-1.7}  безусловную сходимость на всем классе функций $C_{[a,b]}$, а именно
без дополнительных ограничений на сетки узлов, кроме
стремления к нулю максимального расстояния между соседними узлами (в отличие от
классических полиномиальных сплайнов).

Если же $f\in C^1_{[a,b]}$, то при $x\in[x_{i-1}, x_{i+1}]$, $i=1,2,\dots,N-1$,
имеем

\begin{equation}\label{ark-eq-1.10}
|f(x)-R_i(x,f)|\leqslant
\left(4+\frac 2\mu\right)\|\Delta\|\omega(\|\Delta\|, f^\prime).
\end{equation}

В \cite{ark-11} установлено, что если $f\in C^2_{[a,b]}$, то при $x\in[x_{i-1},x_{i+1}]$,
$i=1,2,\dots,N-1$, выполняется неравенство
$$
|f(x)-R_i(x,f)|\leqslant  \left(\omega(x_{i+1}-x_{i-1}, f^{\prime\prime})+
\frac 1{4\mu} \rho_\Delta \|f^{\prime\prime}\|_{[x_{i-1},x_{i+1}]}\right)
\max\{h_i^2, h_{i+1}^2\}\leqslant
$$
\begin{equation}\label{ark-eq-1.11}
\leqslant \left(2\omega(\|\Delta\|, f^{\prime\prime})+\frac 1{4\mu}\rho_\Delta \|f^{\prime\prime}\|_{[a,b]}\right)
\|\Delta\|^2.
\end{equation}


Для данного интегрального уравнения \eqref{ark-eq-1}  с непрерывной на отрезке $[a,b]$ правой частью
и непрерывного на треугольнике $a\leqslant t \leqslant x \leqslant b$ ядра $K(x,t)$
будем строить параллельно дискретное решение и гладкие решения в виде рациональных сплайн-функций.

Введем дискретную функцию $Y(x)$ с искомыми значениями $y_0, y_1, \dots, y_N$
в соответствующих узлах данной сетки $\Delta_N: a=x_0<x_1<\dots<x_N=b$ $(N\geqslant 2)$
и всюду далее будем считать, что функция
 $R_{n,r}(x, Y, [a, x_n])= R_{n,r}(x, Y, \Delta_n, G_n, \mu)$ для $n=1,2,\dots,N$ и $r=1,2$
получается из выражения рациональной сплайн-функции $R_{n,r}(x,y,[a,x_n])$,
определенной равенством \eqref{ark-eq-1.7}, если там вместо решения $y(x)$ подставить дискретную функцию
$Y(x)$.

Значит, при $x\in[x_{i-1},x_i]$, $i=1,2,\dots,n$, будет выполняться равенство
\begin{equation}\label{ark-eq-2.1}
R_{n,r}(x,Y,[a,x_n])=R_i(x,Y,[a,x_n])A_{i,r}(x)+R_{i-1}(x,Y,[a,x_n]) B_{i,r}(x),
\end{equation}
а также в соответствии с \eqref{ark-eq-1.6} и \eqref{ark-eq-1.8} получим равенства
\begin{equation}\label{ark-eq-2.2}
\begin{array}{l}
 R_{1,r}(x,Y,[a,x_1])=R_1(x,Y,[a,x_1]),\quad R_1(x_1,Y,[a,x_1])=y_1,\\
 R_{n,r}(x,Y,[a,x_n])=R_{n-1}(x,Y,[a,x_n]),\\
R_{n,r}(x_n,Y,[a,x_n])=R_{n-1}(x_n,Y,[a,x_n])=y_n,\quad n=2,3,\dots,N.
\end{array}
\end{equation}

Положим $y_0=\varphi(x_0)=\varphi(a)$ и составим систему алгебраических уравнений относительно
$y_1, y_2, \dots, y_N$ вида
\begin{equation}\label{ark-eq-2.3}
y_n-\lambda \int_a^{x_n} K(x_n,t) R_{n,r}(t,Y,[a,x_n])dt=\varphi(x_n),\quad n=1,2,\dots,N.
\end{equation}

Следующая теорема дает условия однозначной разрешимости системы \eqref{ark-eq-2.3}.
\begin{theorem} \label{ark-theo1}
Если для данных значений параметров $\mu>0$ и $\lambda$ и ядра $K(x,t)$ выполняется
неравенство
\begin{equation}\label{ark-eq-2.4}
|\lambda| M(K)< \frac 1{3(1+\mu)},
\end{equation}
где
$$
M(K)=\sup\left\{\int_a^x |K(x,t)|dt: x\in [a,b]\right\},
$$
то система уравнений \eqref{ark-eq-2.3} имеет единственное решение $(y_1,y_2,\dots,y_N)$.
\end{theorem}


Как следует из теоремы \ref{ark-theo1}, если $y_0=\varphi(x_0)$ и для данных значений параметров
$\mu>0$, $\lambda$ и ядра $K(x,t)$ выполняется условие \eqref{ark-eq-2.4}, то дискретная функция
$Y(x)$ со значениями $y_0, y_1, \dots, y_N$ в соответствующих узлах сетки
$\Delta: a=x_0<x_1<\dots <x_N=b$ $(N\geqslant 2)$ однозначно определяется.

Более того, в качестве динамического решения интегрального уравнения \eqref{ark-eq-1} на
расширяющихся отрезках вида $[a,x_n]$, $n=1,2,\dots,N$, можно взять рациональные
сплайн-функции
$$
R_{n,r}(x,Y,[a,x_n])=R_{n,r}(x,Y,\Delta_n, G_n,\mu)
$$
из класса $C^r_{[a,x_n]}$ $(r=1,2)$, для которых при $x\in[x_{i-1},x_i]$, $i=1,2,\dots,n$,
выполняется равенство
\begin{equation}\label{ark-eq-2.8}
R_{n,r}(x,Y,[a,x_n])=R_i(x,Y,[a,x_n])A_{i,r}(x)+R_{i-1}(x,Y,[a,x_n])B_{i,r}(x).
\end{equation}


Как и выше, будем предполагать, что правая часть $\varphi(x)$ уравнения \eqref{ark-eq-1} и ядро
$K(x,t)$ являются непрерывными функциями соответственно на отрезке $[a,b]$ и на треугольнике
$a\leqslant t\leqslant x\leqslant b$.

Для данной сетки из произвольных узлов $\Delta_N: a=x_0<x_1<\dots <x_N=b$ $(N\geqslant 2)$
будем также придерживаться принятых выше обозначений, в соответствии с которыми рациональные
сплайн-функции $R_{n,r}(x,Y,[a,x_n])=R_{n,r}(x,Y,\Delta_n, G_n,\mu)$, $n=1,2,\dots,N$,
из класса $C^r_{[a,x_n]}$ $(r=1,2)$ определяются равенствами \eqref{ark-eq-2.8}.

Следующее утверждение дает оценку скорости сходимости приближенных решений
$R_{n,r}(x,Y,[a,x_n])$ интегрального уравнения \eqref{ark-eq-1} к его точному решению $y(x)$ на
отрезках вида $[a,x_n]$, $n=1,2,\dots,N$, которые расширяясь исчерпывают
всю область определения $[a,b]$ этого решения $y(x)$.

Оценка получена в терминах величины
\begin{equation}\label{ark-eq-3.1}
E_N(y)=\max\{\|y-R_j(\cdot, y, [a,b])\|_{[x_{j-1},x_{j+1}]}:j=1,2,\dots,N-1\},
\end{equation}
что позволяет оценить скорость сходимости приближенных решений к точному решению
$y(x)$ с учетом его гладкостных свойств. Для этого, как показано далее, можно воспользоваться
аппроксимативными свойствами трехточечных рациональных интерполянтов $R_j(x,y,[a,b])$
вида \eqref{ark-eq-1.2} с соответствующим выбором параметра $\mu>0$.

\begin{theorem} \label{ark-theo2}
Если для данных значений параметров $\mu>0$ и $\lambda$ и ядра $K(x,t)$ выполняется
условие
\begin{equation}\label{ark-eq-3.2}
|\lambda| M(K)< \frac 1{4(1+\mu)},
\end{equation}
то для непрерывного решения $y(x)$ уравнения \eqref{ark-eq-1} и при $r=1,2$ для
рациональных сплайн-функций $R_{n,r}(x,Y,[a,x_n])=R_{n,r}(x,Y,\Delta_n, G_n,\mu)$,
$n=1,2,\dots,N$, из \eqref{ark-eq-2.8} при $x\in[a,x_n]$ имеем
\begin{equation}\label{ark-eq-3.3}
|y(x)-R_{n,r}(x, Y,[a,x_n])|\leqslant  \frac 1{1-4(1+\mu)|\lambda|M(K)}E_N(y).
\end{equation}
\end{theorem}


\subsection{Заключение}
В случае равномерных сеток узлов $\Delta: a=x_0<x_1<\dots<x_N=b$ $(N\geqslant 2)$ для любого
значения $\mu>0$ вполне аналогично неравенствам \eqref{ark-eq-1.5} при $x\in[x_{i-1}, x_{i+1}]$
для $i=1,2,\dots,n-1$ получаются оценки
$$
|a_i(x)|\leqslant 1,\quad |b_i(x)|\leqslant 2,\quad |c_i(x)|\leqslant 1.
$$

Поэтому для равномерных сеток узлов $\Delta$ заключение
теоремы \ref{ark-theo1} остается справедливым, если в ней условие
на $|\lambda|M(K)$  заменить на неравенство $|\lambda|M(K)<1/3$.

Что касается теоремы \ref{ark-theo2}, в ней в случае равномерных сеток
узлов $\Delta$ условие на $|\lambda|M(K)$  можно заменить
на неравенство $|\lambda|M(K)<1/4$, а правую часть неравенства из
ее заключения -- на выражение $1/(1-4|\lambda|M(K))E_N(y)$.

Отсюда для равномерных сеток узлов и из теоремы \ref{ark-theo2} в общем случае с учетом
 неравенств \eqref{ark-eq-1.9}--\eqref{ark-eq-1.11}
непосредственно вытекают оценки скорости сходимости приближенных гладких решений
$R_{n,r}(x,Y,[a,x_n])$ к точному решению $y(x)$ интегрального уравнения \eqref{ark-eq-1}.
Эти оценки выражаются через модуль непрерывности решения $\omega(\|\Delta\|,y)$ в случае
непрерывности $y(x)$, а в случае существования гладких решений $y(x)$ --
через модули непрерывности производных
  $\omega(\|\Delta\|,y^\prime)$ и $\omega(\|\Delta\|,y^{\prime\prime})$
с соответствующим выбором параметра $\mu>0$.

К примеру, если уравнение \eqref{ark-eq-1} допускает решение $y(x)$ с непрерывной
второй производной  $y^{\prime\prime}(x)$
на отрезке $[a,b]$, то, как следует из \eqref{ark-eq-1.11}, в случае равномерных сеток узлов
$\Delta: a=x_0<x_1<\dots<x_N=b$ $(N\geqslant 2)$ с $x_i-x_{i-1}=h$ $(i=1,2,\dots,N)$ при
выполнении условия $|\lambda|M(K)<1/4$ и выборе параметра $\mu=1/(4h)$
справедливо неравенство
$$
\|y-R_{n,r}(\cdot,Y,[a,x_n])\|_{[a,x_n]}\leqslant \frac 1{1-4|\lambda|M(K)}
\left(2\omega(h,y^{\prime\prime},[a,x_n])+h\|y^{\prime\prime}\|_{[a,x_n]}\right) h^2
$$
для $r=1,2$ и каждого $n=2,3,\dots,N$.



\section{Решение интегральных уравнений Фредгольма методом
коллокационных рациональных сплайн-функций}

%\begin{abstract}
%Для произвольных сеток узлов получено приближенное решение интегрального
% уравнения Фредгольма второго рода в виде коллокационной рациональной
%сплайн-функции.
%
%Представлены оценки скорости равномерной сходимости приближенных решений
%к точному решению из класса гладкости $C^r$ для $r=0,1,2$.
%\end{abstract}

\subsection{Введение}
Многие непрерывно текущие процессы физического характера, химических реакций,
экологии и др., как хорошо известно, моделируются с привлечением
интегральных уравнений.
Учитывая, что к таким уравнениям точные методы решения не всегда применимы,
актуальным
остается вопрос об эффективных приближенных методах их решения. При этом
в случае наличия
элементов
неопределенности в изучаемых процессах востребованы уравнения с параметрами,
а если
предлагается приближенное решение, то желательно, чтобы это решение также
содержало некоторые управляемые параметры.

В данном подразделе изучается вопрос приближенного решения интегрального уравнения
Фредгольма второго рода
\begin{equation}\label{ark2-eq-1}
y(x)-\lambda \int_a^b K(x,t)y(t)dt=\varphi(x),\quad x\in[a,b],
\end{equation}
с помощью сплайн-функций относительно рациональных интерполянтов с параметрами.

Правую часть $\varphi(x)$ и ядро $K(x,t)$ полагаем непрерывными функциями соответственно
на отрезке $[a,b]$ и прямоугольнике $[a,b]\times[a,b]$, а величину $\lambda$ -- действительным
параметром, и при этом считаем, что уравнение \eqref{ark2-eq-1} имеет единственное решение $y(x)$, непрерывное
на отрезке $[a,b]$.

Схема приближенного решения интегральных уравнений вида \eqref{ark2-eq-1} с помощью кубических сплайнов
в общих чертах описана в \cite{ark-4} (гл. II, п.2.8).

В \cite{ark-7} (гл.VI, п.3) с помощью периодических кубических и параболических сплайнов
в случае равномерных сеток узлов дано приближенное решение уравнений вида \eqref{ark2-eq-1}
и изучена скорость сходимости приближенных решений к точному.

Оценка погрешности приближенного решения уравнений вида \eqref{ark2-eq-1} периодическими полиномиальными
сплайнами по равномерным сеткам узлов в интегральных метриках исследована в
\cite{ark-12} (гл. 5, п. 5.1).

Как известно \cite{ark-4, ark-7}, классические полиномиальные сплайны непрерывных функций в случае последовательностей произвольных сеток узлов с диаметрами, стремящимися к нулю, могут не сходиться.

Известно также \cite{ark-9}, что последовательность сплайн-функций относительно рациональных
 трехточечных интерполянтов для любой непрерывной на данном отрезке функции в случае
любой последовательности сеток узлов с диаметрами, стремящимися к нулю, сходится равномерно
 на этом отрезке.

В данном подразделе рассматриваются произвольные узлы, точнее, предлагается приближенное
решение интегральных уравнений вида \eqref{ark2-eq-1}, имеющих единственное решение, с помощью
 рациональных сплайн-функций в случае произвольных сеток узлов и получена оценка
 скорости сходимости приближенного решения к точному в зависимости от гладкостных
 свойств точного решения.

Отметим, что интегральные уравнения имеют многочисленные приложения самого разного
характера. В связи, в частности, с этим разработаны различные методы их решения, некоторые
из которых приведены также в \cite{ark-1,ark-2,ark-3}.


\subsection{Обозначения и вспомогательные утверждения}

Для сетки произвольных узлов $\Delta: a=x_0<x_1<\dots<x_N=b$ $(N\geqslant 2)$ положим
$h_i=x_i-x_{i-1},$ $i=1,2,\dots,N,$ и с параметром  $\mu>0$ построим набор чисел
$g=\{g_1,g_2, \dots,g_{N-1}\}$ таких, что
\begin{equation}\label{ark2-eq-2}
g_i=\begin{cases}
x_{i+1}+\mu h_{i+1}, \,\text{ если }\, h_{i+1}\leqslant h_i,\\
x_{i-1}-\mu h_i, \,\text{ при }\, h_{i+1}> h_i, \quad i=1,2,\dots,N-1.
\end{cases}
\end{equation}

Для функции $f(x)$, определенной на сетке узлов $\Delta$,
при $i=1,2,\dots,N-1$
рассмотрим рациональные интерполянты
\begin{equation}\label{ark2-eq-3}
R_i(x)=R_i(x,f)=\alpha_i+\beta_i(x-x_i)+\gamma_i\frac 1{x-g_i}
\end{equation}
такие, что $R_i(x_j)=f(x_j)$ для $j=i-1,i,i+1$. Из этих условий с использованием
 разделенных разностей имеем
$$
\begin{array}{lcl}
\alpha_i=f(x_i)-f(x_{i-1}, x_i, x_{i+1})(x_{i-1}-g_i)(x_{i+1}-g_i),\\
\beta_i=f(x_{i-1}, x_{i+1})+f(x_{i-1}, x_i, x_{i+1})(x_i-g_i),\\
\gamma_i=f(x_{i-1}, x_i, x_{i+1})(x_{i-1}-g_i)(x_i-g_i)(x_{i+1}-g_i).
\end{array}
$$

Будем считать также, что $R_0(x,f)\equiv R_1(x,f)$, $R_N(x,f)\equiv R_{N-1}(x,f)$.

Всюду ниже для натурального $r$ при $i=1,2,\dots,N$ обозначим
$$
A_{i,r}(x)=\frac{(x-x_{i-1})^r}{(x-x_{i-1})^r+(x_i-x)^r},\quad B_{i,r}(x)=1-A_{i,r}(x)
$$
 и рассмотрим рациональные  сплайн-функции
$R_{N,r}(x,f)=R_{N,r}(x,f,\Delta,g,\mu)$ такие, что при $x\in [x_{i-1}, x_i]$,
 $i=1,2,\dots,N$, выполняется равенство
\begin{equation}\label{ark2-eq-4}
R_{N,r}(x,f)=R_i(x,f)A_{i,r}(x)+R_{i-1}(x,f)B_{i,r}(x).
\end{equation}

Как следует из \cite{ark-9}, $R_{N,r}(x,f)$ представляет собой гладкую функцию
 класса $C^r_{[a,b]}$. При этом из \eqref{ark2-eq-4} имеем
\begin{equation}\label{ark2-eq-5}
R_{N,r}(x_i,f)=R_i(x_i,f)=f(x_i),\quad i=0,1,\dots,N.
\end{equation}

В \cite{ark-10} показано, что для любой непрерывной на данном отрезке $[a,b]$ функции $f(x)$,
произвольной сетки узлов $\Delta: a=x_0<x_1<\dots<x_N=b$ $(N\geqslant 2)$, любого
$\mu>0$ и соответствующей интерполяционной сплайн-функции
$R_{N,r}(x,f)=R_{N,r}(x,f,\Delta,g,\mu)$ для всех $x\in[a,b]$ выполняется неравенство
\begin{equation}\label{ark2-eq-6}
|f(x)-R_{N,r}(x,f)|\leqslant (3+\mu)\omega(\|\Delta\|,f),
\end{equation}
где $\|\Delta\|=\max\{h_i|i=1,2,\dots,N\}$ и, как обычно,
$$
\omega(\delta, f)=\sup\{|f(x+h)-f(x)|: |h|\leqslant \delta; x,x+h\in [a,b]\}
$$
означает равномерный модуль непрерывности функции $f(x)$ на данном отрезке $[a,b]$.

Заметим, что рациональные интерполянты $R_i(x,f)$ из \eqref{ark2-eq-3} для всех $x\in[x_{i-1},x_{i+1}]$
$(i=1,2,\dots,N-1)$ допускают в силу интерполяционности следующее представление:
\begin{equation}\label{ark2-eq-7}
R_i(x,f)=a_i(x)f(x_{i-1})+b_i(x)f(x_i)+c_i(x)f(x_{i+1}),
\end{equation}
в котором
$$
a_i(x)=\frac{(x-x_i)(x-x_{i+1})(x_{i-1}-g_i)}
{(x_{i-1}-x_i)(x_{i-1}-x_{i+1})(x-g_i)},
\quad
b_i(x)=\frac{(x-x_{i-1})(x-x_{i+1})(x_i-g_i)}
{(x_i-x_{i-1})(x_i-x_{i+1})(x-g_i)},
$$
$$
c_i(x)=\frac{(x-x_{i-1})(x-x_i)(x_{i+1}-g_i)}
{(x_{i+1}-x_{i-1})(x_{i+1}-x_i)(x-g_i)},
$$
причем $a_i(x)+b_i(x)+c_i(x)=1$.
Далее будут использоваться также неравенства
\begin{equation}\label{ark2-eq-8}
|a_i(x)|<1+\mu, \quad |b_i(x)|<2(1+\mu),\quad |c_i(x)|<1+\mu,
\end{equation}
которые справедливы при $\mu>0$ для $x\in[x_{i-1},x_{i+1}]$, $i=1,2,\dots,N-1$.

Эти неравенства проще получаются, если воспользоваться другими представлениями
для коэффициентов $a_i(x)$ и $b_i(x)$  , которые получаются \cite{ark-10} из \eqref{ark2-eq-3}.
Так, в случае $h_{i+1}\leqslant h_i$ последовательно имеем:
$$
|a_i(x)|=\left\vert \frac{x-x_i}{x_{i-1}-x_i}+\frac{(x-x_{i-1})(x-x_i)(x_{i+1}-g_i)}
{(x_{i-1}-x_i)(x_{i-1}-x_{i+1})(x-g_i)}\right\vert=
$$
$$
=\frac{|x-x_i|}{x_i-x_{i-1}}
\left[1-\frac{(x-x_{i-1})(g_i-x_{i+1})}{(x_{i+1}-x_{i-1})(g_i-x)}\right]\leqslant 1<1+\mu;
$$
$$
|b_i(x)|=\left\vert \frac{x-x_{i-1}}{x_i-x_{i-1}}+\frac{(x-x_{i-1})(x-x_i)(x_{i+1}-g_i)}
{(x_i-x_{i-1})(x_i-x_{i+1})(x-g_i)}\right\vert=
\frac{x-x_{i-1}}{x_i-x_{i-1}}\cdot\frac{|g_i-x-\mu(x-x_i)|}{g_i-x}=
$$
$$
=(1+\mu)\frac{(x-x_{i-1})(x_{i+1}-x)}{(x_i-x_{i-1})(g_i-x)}<2(1+\mu);
$$
$$
|c_i(x)|=\mu
\frac{(x-x_{i-1})|x-x_i|}{(x_{i+1}-x_{i-1})(g_i-x)},
$$
отсюда
$$
|c_i(x)|=\mu\frac{x-x_{i-1}}{x_{i+1}-x_{i-1}}\cdot \frac{x-x_i}{g_i-x}\leqslant 1<1+\mu,
$$
если $x\in[x_i, x_{i+1}]$, и
$$
|c_i(x)|=\mu\frac{x-x_{i-1}}{x_{i+1}-x_{i-1}}\cdot \frac{x_i-x}{g_i-x}\leqslant \mu<1+\mu,
$$
если $x\in[x_{i-1}, x_i]$.

Случай, когда $h_{i+1}>h_i$, рассматривается вполне аналогично. Неравенства
\eqref{ark2-eq-8} доказаны.


\subsection{Основные результаты}

Пусть интегральное уравнение \eqref{ark2-eq-1} для данных $\lambda$, непрерывной
на $[a,b]$ правой части $\varphi(x)$ и непрерывном на прямоугольнике $[a,b]\times [a,b]$
ядре $K(x,t)$ имеет единственное решение $y(x)$, непрерывное на $[a,b]$.

Рассмотрим дискретную функцию $Y(x)$, определенную на данной сетке с произвольными узлами
$\Delta: a=x_0<x_1<\dots<x_N=b$ $(N\geqslant 2)$, со значениями
$Y(x_i)=y_i$ для $i=0,1,\dots,N$.

Для этой функции $Y(x)$, параметра $\mu>0$ и набора полюсов $g=\{g_1,g_2,\dots,g_{N-1}\}$
в соответствии со значениями \eqref{ark2-eq-2} построим рациональные интерполянты $R_i(x,Y)$ вида \eqref{ark2-eq-3}
и соответствующую им сплайн-функцию
$R_{N,r}(x,Y)=R_{N,r}(x,Y,\Delta,g,\mu)$ типа \eqref{ark2-eq-4} для значений $r=1,2$.

Как следует из \eqref{ark2-eq-5} и конструкции рациональной сплайн-функции $R_{N,r}(x,Y)$,
будут выполняться равенства
\begin{equation}\label{ark2-eq-9}
R_{N,r}(x_i,Y)=R_i(x_i,Y)=y_i,\quad i=0,1,\dots,N.
\end{equation}

Составим систему линейных алгебраических уравнений относительно неизвестных
$y_0,y_1,\dots,y_N$ с помощью следующих условий коллокации узлов сетки:
\begin{equation}\label{ark2-eq-10}
R_{N,r}(x_i,Y)-\lambda \int_a^b K(x_i,t)R_{N,r}(t,Y)dt=\varphi(x_i),
\end{equation}
$i=0,1,\dots,N$.

Тогда имеет место

\begin{theorem}\label{teor1}
Если для данных значений $\lambda$ и $\mu>0$ и ядра $K(x,t)$ выполняется неравенство
\begin{equation}\label{ark2-eq-11}
|\lambda| \sup_{a\leqslant x\leqslant b} \,\int_a^b |K(x,t)|dt<\frac 1{8(1+\mu)},
\end{equation}
то системой \eqref{ark2-eq-10} однозначно определяется коллокационная рациональная сплайн- функция
$R_{N,r}(x,Y)=R_{N,r}(x,Y,\Delta,g,\mu)$ $(N\geqslant 2; r=1,2)$ вида \eqref{ark2-eq-4}.
 \end{theorem}

В условиях теоремы \ref{teor1} имеет место также

\begin{theorem}\label{teor2}
Если для данных значений $\lambda$ и $\mu>0$ и ядра $K(x,t)$ выполняется неравенство \eqref{ark2-eq-11},
то для непрерывного решения $y(x)$ интегрального уравнения \eqref{ark2-eq-1} и коллокационной
рациональной сплайн-функции $R_{N,r}(x,Y)$ $(N\geqslant 2; r=1,2)$, определяемой системой \eqref{ark2-eq-10},
при любом $x\in[a,b]$ выполняется неравенство
$$
|y(x)-R_{N,r}(x,Y)|\leqslant (10+8\mu)(3+\mu)\omega(\|\Delta\|,y).
$$
\end{theorem}

\subsection{Заключение}

 Отметим, что из неравенства (18) можно получить также оценки скорости сходимости
приближенных решений $R_{N,r}(x,Y)$ интегрального уравнения \eqref{ark2-eq-1} к его точному решению $y(x)$
класса $C^r_{[a,b]}$ в случаях $r=1$ и $r=2$.

Действительно, по теореме~1.1 из \cite{ark-10} для решения $y(x)$ из класса $C^1_{[a,b]}$ и
его интерполяционной рациональной сплайн-функции $R_{N,1}(x,y)$ вида \eqref{ark2-eq-4} получим
$$
|y(x)-R_{N,1}(x,y)|\leqslant
\left(4+\frac 2\mu\right)\|\Delta\|\omega(\|\Delta\|,y^\prime),\quad x\in[a,b].
$$

Если же решение $y(x)$ принадлежит классу $C^2_{[a,b]}$, то для
его интерполяционной рациональной сплайн-функции $R_{N,2}(x,y)$ вида \eqref{ark2-eq-4}
по теореме~1 из \cite{ark-11} при
$\rho_\Delta=\max\{h_ih_j^{-1}|\,|i-j|=1, 1\leqslant i,j\leqslant N\}$ имеем
$$
|y(x)-R_{N,2}(x,y)|\leqslant \|\Delta\|^2\left(2\omega(\|\Delta\|,y^{\prime\prime})+
\frac1{4\mu} \rho_\Delta \|y^{\prime\prime}\|\right),\quad x\in[a,b].
$$

Подставляя правые части последних двух неравенств в правую часть неравенства (18), получим
соответствующие оценки равномерной сходимости на отрезке $[a,b]$ коллокационных сплайн-функций
$R_{N,r} (x,Y)$ к решению $y(x)$ в случаях $r=1$ и $r=2$.




\section{Гладкая интерполяция локальными полиномиальными сплайнами}
%\begin{abstract}
%
%По дискретной функции, заданной в узлах произвольной сетки из данного отрезка
%$[a,b]$ числовой оси, построены локальные полиномиальные  интерполяционные
%сплайны пятой степени.
%
%Для произвольных сеток узлов доказано, что построенные интерполяционные
%сплайны имеют на отрезке $[a,b]$ непрерывные производные до второго порядка
%включительно.
%В случае непрерывных на отрезке $[a,b]$ функций и равномерных сеток узлов дана
%оценка равномерной сходимости построенных локальных сплайнов к функции
%на этом отрезке через равномерный модуль непрерывности функции.
%
%Изучены также аппроксимативные свойства самих сплайнов и их производных
%в случае функций, имеющих непрерывные производные второго порядка на данном
%отрезке $[a,b]$.
%
%Представлены оценки скорости одновременной равномерной сходимости самих
%сплайнов к функции, производных первого и второго порядков от сплайнов
%к соответствующим производным от исходной функции.
%
%При этом оценка скорости сходимости для самих локальных интерполяционных
%сплайнов к функции получена  в терминах модуля непрерывности этой функции,
%а оценки скорости сходимости производных первого  и второго порядков
%от сплайнов соответственно к первой и второй производным функции получены
%в терминах модуля непрерывности соответствующей производной функции.
%
%\end{abstract}

\subsection{Введение}

Вопросы о сплайн-аппроксимациях и сплайн-интерполяциях особенно актуальны
в численных методах современного анализа и математической физики.
Как математический аппарат для описания кривых и поверхностей сплайн-функции
применяются в задачах построения и оптимизации сложных поверхностей с помощью
компьютеров, а также для исследования различных физических, биологических и
других явлений,  для программного обеспечения медицинского диагностического
оборудования, для решения многих других прикладных задач
(см., например,\cite{ark-4,ark-5,ark-6,ark-7} и цитированные в них источники).

Следует отметить, что интерполяционные сплайны первой степени обладают
хорошими аппроксимативными свойствами, но не являются гладкими. Наибольший
интерес представляют интерполяционные сплайн-функции достаточно высокой
степени гладкости. Поэтому широкую известность получили глобальные
кубические сплайны Шенберга \cite{ark-4}, которые являются интерполяционными и
имеют максимальную для кубических сплайнов гладкость второго порядка.
Для их построения одновременно используются интерполяционные условия
во всех узлах исходной сетки, что приводит к решению систем уравнений
с большим числом неизвестных, а сама задача их построения имеет решение
при определенных краевых условиях.

Поэтому исследуются также локальные сплайны, для построения каждого фрагмента
которых используется лишь несколько интерполяционных условий.

В данном подразделе построены локальные полиномиальные сплайны пятой степени,
имеющие гладкость второго порядка, и изучены аппроксимативные свойства самих
сплайнов и их производных до второго порядка включительно.


\subsection{Основные результаты}

Пусть на некотором отрезке $[a,b]$ задана произвольная сетка узлов
$a=x_0<x_1<\dots <x_N=b$,  $N\geqslant 3$. Присоединим к ним точки
$ x_{-2}<x_{-1}<a, b<x_{N+1}<x_{N+2}$ и возьмем любую конечную функцию $f(x)$,
определенную на множестве $\Delta=\{x_k: -2\leqslant k \leqslant N+2\}$.
Тогда для $k=1,2,\dots,N$ однозначно определяются интерполяционные третьей степени
 полиномы Ньютона $p_k (x)=p_k (x,f)$
такие, что  $p_k (x_j )=f(x_j)$ при $j=k-2,k-1,k,k+1$.

Исходя из этих полиномов, для узлов сетки $\Delta$ построим новые полиномы
$$
P_j (x)=P_j (x,f)=p_j (x) \frac{x-x_{j-2}}{x_j-x_{j-2}}+
p_{j-1}(x)\frac{x_j-x}{x_j-x_{j-2}},
$$
$$
Q_k (x)=Q_k (x,f)=P_{k+1}(x)\frac{x-x_{k-1}}{x_k-x_{k-1}}+
P_k (x)\frac{x_k-x}{x_k-x_{k-1}}.
$$
На отрезке $[a,b]$  рассмотрим кусочно-полиномиальную функцию
 $S_N (x)=S_N (x,f,\Delta)$ такую, что при каждом $k=1,2,\dots,N$
для всех $x\in [x_{k-1},x_k]$ выполняется равенство $S_N (x)=Q_k (x,f)$.

Для сокращения записи далее рассмотрим случай равностоящих узлов.
Всюду ниже будем считать функцию $f(x)$ непрерывной $(b-a)-$периодической,
$h=(b-a)/N$, и рассмотрим узлы $\Delta:x_k=a+kh$, $k=0,\pm 1,\pm 2,\dots$

Доказаны следующие утверждения:

1) Для любой конечной функции $f(x)$, определенной на системе узлов
$\Delta=\{x_k:-2\leqslant k \leqslant N+2\}$, функция
$S_N (x)=S_N (x,f,\Delta)$ на отрезке $[a,b]$
является дважды непрерывно дифференцируемым полиномиальным сплайном.

2) Для любой $(b-a)-$периодической непрерывной функции $f(x)$
при всех $x\in [a,b]$ выполняется неравенство
$$
|S_N (x,f,\Delta)-f(x)|\leqslant 10 \omega(h,f).
$$

3) Для любой $(b-a)-$периодической непрерывной функции $f(x)$,имеющей
непрерывные производные второго порядка, при всех $x\in [a,b]$ выполняются
неравенства
$$
|S_N (x,f,\Delta)-f(x)|\leqslant 9h^2 \omega(h,f^{\prime\prime}),
$$
$$
|S_N^\prime(x,f,\Delta)-f^\prime(x)|\leqslant 18h\omega(h,f^{\prime\prime}),
$$
$$
|S_N^{\prime\prime}(x,f,\Delta)-f^{\prime\prime}(x)|\leqslant 99\omega(h,f^{\prime\prime}).
$$


\subsection{Заключение}
Отметим, что применения находят также локальные эрмитовы сплайны,
для построения которых
дополнительно требуются интерполяционные условия на производные,
и базисные сплайны, которые не являются интерполяционными.
Поэтому определенный интерес представляет задача построения локальных
интерполяционных сплайн-функций наперед заданной гладкости.





































\chapter{Модель Поттса с числом состояний спина \texorpdfstring{$q=3$}{q=3} на решетке Кагоме}

\section{Ведение}

Нами приводятся результаты исследования трехвершинной модели Поттса на решетке Кагоме методом Ванга-Ландау \cite{mma-bib-1, mma-bib-2}. Модель Поттса была предложена в 1952 году Поттсом по предложению С. Домба. Модель задается числом состояний $q$, в которых может находиться спин на произвольной решетке.

Гамильтониан модели Поттса с числом состояний $q=3$ на решетке Кагоме может быть представлен в следующем виде:
\begin{equation}
    \label{mma-eq-1}
    H = - J_1 \sum_{i, j} \cos \theta_{i, j} - J_2 \sum_{i, k} \cos \theta_{i, k},
\end{equation}
где $J_1$ и $J_2$ -- параметры обменного взаимодействия для ближайших и вторых ближайших соседей. $\theta_{i,j}$, $\theta_{i,k}$ -- углы между взаимодействующими спинами $S_i - S_j$ и $S_i - S_k$ соответственно. Отметим, что в данной работе рассматривается случай ферромагнитного обменного взаимодействия между ближайшими соседями ($J_1 = 1$) и конкурирующего антиферромагнитного взаимодействия между следующими за ближайшими соседями ($J_2 \leq 0$).

Схематическое и цветовое представление модели представлено на рисунке \ref{mma-fig-1}. На вставке приведены направления спинов для каждого из 3 значений спина и соответствующее цветовое представление. Также представлены взаимодействия между первыми и вторыми ближайшими соседями (каждый спин имеет 4 ближайших и 4 следующих за ближайшими соседа).
\begin{figure}[h]
    \begin{center}
        \includegraphics[width=0.4\textwidth]{mma/image2.png}
    \end{center}
    \caption{Модель Поттса с числом состояний спина $q = 3$ на решетке Кагоме.}
    \label{mma-fig-1}
\end{figure}

В данной работе значения обменных интегралов нами были приняты равными: $J_1 = 1$, а $J_2$ было принято антиферромагнитным и менялось по величине от 0 до 2. Таким образом, в исследованной нами в данной работе модели Поттса на решетке Кагоме учитывается ферромагнитное взаимодействие между ближайшими соседями спина и антиферромагнитное взаимодействие разной величины между следующими за ближайшими соседями. Для учета влияния размеров системы на термодинамические свойства исследовались системы с линейными размерами $L=12$ и $L=36$.


\section{Метод исследований}

Исследования проводились на основе алгоритма Ванга-Ландау метода Монте-Карло (МК) \cite{mma-bib-10, mma-bib-11, mma-bib-12, mma-bib-13, mma-bib-14, mma-bib-15}. Данный алгоритм является реализацией метода энтропийного моделирования и его основной особенностью является возможность расчета функции плотности состояний системы, зная которую можно легко вычислить любые интересующие нас термодинамические параметры системы.

Алгоритм Ванга-Ландау является разновидностью энтропийного моделирования и, как показывает опыт его применения в последние годы, является весьма эффективным для исследования различных дискретных спиновых систем \cite{mma-bib-14}.

Алгоритм Ванга-Ландау основан на том, что совершая случайное блуждание в пространстве энергий с вероятностями, обратно пропорциональными плотности состояний $g(E)$, мы получаем равномерное распределение по энергиям. Подобрав вероятности перехода такими, что посещение всех энергетических состояний стало бы равномерным, можно получить изначально неизвестную плотность состояний $g(E)$, зная которую можно вычислить значения необходимых термодинамических параметров при любой температуре. Так как плотность состояний $g(E)$ очень быстро растет с увеличением размеров исследуемых систем, для удобства хранения и обработки больших чисел пользуются величиной $\ln g(E)$.

Важным обстоятельством является то, что плотность состояний $g(E)$ не зависит от температуры, следовательно, рассчитав ее однократно, мы можем вычислить значения любых термодинамических параметров системы при любой ненулевой температуре.

В данной работе нами алгоритм Ванга-Ландау был использован в следующем виде \cite{mma-bib-10, mma-bib-11, mma-bib-12}:
\begin{itemize}
    \item Задается произвольная начальная конфигурация спинов. Стартовые значения плотности состояний $g(E) = 1$, гистограммы распределений по энергиям $H(E) = 0$ и начальное значение модификационного фактора $f = f_0 = e^1 \approx 2.71828$.
    \item Многократно совершаем шаги в фазовом пространстве, пока не получим относительно плоскую гистограмму $H(E)$ (т.е. пока не будут посещены примерно одинаковое количество раз все возможные энергетические состояния системы). В качестве критерия <<плоскости>> гистограммы нами принималось условие отклонения числа посещений всех возможных (с ненулевой плотностью $g(E) \neq 1$) энергетических состояний на величину не более чем на 10\% от среднего значения по системе.
    \item При этом вероятность перехода из состояния с энергией $E_1$ в состояние с энергией $E_2$ определяется по формуле $p = g(E_1)/g(E_2)$. Если переход в состояние с энергией $E_2$ состоялся, то для энергии $E_2$ проводится модификация плотности состояния $g(E_2) \to f \times g(E_2)$, и гистограммы $H(E_2) \to H(E_2) + 1$ иначе меняем параметры для энергии $E_1$ $g(E_1) \to f \times g(E_1)$, $H(E_1) \to H(E_1) + 1$.
    \item Если гистограмма стала <<плоской>> то: обнуляем гистограмму $H(E) \to 0$,  уменьшаем модификационный фактор $f \to \sqrt{f}$, и продолжаем снова и снова, пока модификационный фактор $f \geq f_{\min}$. В качестве минимального значения модификационного фактора нами принималось $f_{\min} = 1.0000000001$.
    \item Каждый раз при достижении энергетического минимума нами проводился анализ магнитной структуры основного состояния и его запись в графический файл. При этом проводилось сравнение полученной конфигурации с ранее полученными и только при обнаружении новой уникальной конфигурации производится ее сохранение в графический файл. Далее данная структура заносится в специальную базу данных для данной модели для дальнейшего сравнения. Данная процедура позволяет избежать дублирования в графических файлах многократно встречающихся состояний с одинаковой магнитной структурой.
    \item После расчета плотности состояний системы для любой интересующей нас температуры рассчитываются различные термодинамические параметры, такие как, энтропия, внутренняя энергия, свободная энергия, теплоемкость, намагниченность, восприимчивость и т.д. Некоторые формулы для расчета термодинамических параметров приведены ниже.
    Более подробно алгоритм Ванга-Ландау изложен в работах \cite{mma-bib-10, mma-bib-11, mma-bib-12, mma-bib-13, mma-bib-14, mma-bib-15}.
\end{itemize}

\section{Результаты исследований}

На рисунке \ref{mma-fig-2} приведены значения энергии основного состояния системы при различных значениях взаимодействий $J_1$ и $J_2$. Как видно из графика, в данной модели при низких температурах возможно ферромагнитное упорядочение (FM) (при $J_2 > -0.5$), триплетное антиферромагнитное упорядочение (TAFM) (при $J_2 < -0.5$) или возникновение неупорядоченного фрустрированного состояния ($J_2 = -0.5$).
\begin{figure}[h]
    \begin{center}
        \includegraphics[width=0.35\textwidth]{mma/image19.png}
    \end{center}
    \caption{Энергия основного состояния системы.}
    \label{mma-fig-2}
\end{figure}

Плотность состояний системы $g(E)$ при различных значениях обменных взаимодействий $J_1$ и $J_2$ для систем с линейными размерами $L = 12$ и $L = 36$ приведены на рисунке \ref{mma-fig-3}. На графике приведены плотности состояний для всех трех областей, приведенных на рисунке \ref{mma-fig-2}. Как видно из рисунка, основное состояние системы при $J_2 = -0.5$ сильно вырождено, что обусловлено наличием в данном случае фрустрации в системе, а в остальных случаях вырождение не наблюдается.
\begin{figure}[h]
    \begin{center}
        \begin{subfigure}{0.48\textwidth}
            \begin{center}
                \includegraphics[width=1.0\textwidth]{mma/image20.jpeg}
            \end{center}
        \end{subfigure}
        \begin{subfigure}{0.48\textwidth}
            \begin{center}
                \includegraphics[width=1.0\textwidth]{mma/image21.jpeg}
            \end{center}
        \end{subfigure}
    \end{center}
    \caption{Плотность состояний $g(E)$ для трехвершинной модели Поттса на решетке Кагоме при различных значениях обменных взаимодействий $J_1$ и $J_2$.}
    \label{mma-fig-3}
\end{figure}

На рисунках \ref{mma-fig-4}, \ref{mma-fig-5}, \ref{mma-fig-6} приведены структуры основного состояния для ферромагнитной области ($J_2 > -0.5$), области фрустраций ($J_2 = -0.5$) и триплетной антиферромагнитной области ($J_2 < -0.5$).
\begin{figure}[h]
    \begin{center}
        \begin{subfigure}{0.3\textwidth}
            \begin{center}
                \includegraphics[width=1.0\textwidth]{mma/image24.png}
            \end{center}
            \caption{$J_2 > -0.5$.}
            \label{mma-fig-4}
        \end{subfigure}
        \begin{subfigure}{0.3\textwidth}
            \begin{center}
                \includegraphics[width=1.0\textwidth]{mma/image25.png}
            \end{center}
            \caption{$J_2 = -0.5$.}
            \label{mma-fig-5}
        \end{subfigure}
        \begin{subfigure}{0.3\textwidth}
            \begin{center}
                \includegraphics[width=1.0\textwidth]{mma/image26.jpeg}
            \end{center}
            \caption{$J_2 < -0.5$.}
            \label{mma-fig-6}
        \end{subfigure}
    \end{center}
    \caption{Структура основного состояния при разных значениях $J_2$.}
\end{figure}

Энергия основного состояния для ферромагнитной области задается как:
\begin{equation}
    \label{mma-eq-2}
    E_{\min} = -2J_1 - 2J_2.
\end{equation}

Энергия основного состояния для триплетной антиферромагнитной области задается как:
\begin{equation}
    \label{mma-eq-3}
    E_{\min} = - \frac{1}{2} J_1 + J_2.
\end{equation}

Вычислив единожды плотность состояний системы $g(E)$ можно легко рассчитать температурную зависимость любой интересующей нас величины. Например, внутренняя энергия $E$, свободная энергия $F$, и энтропия $S$ системы могут быть рассчитаны следующим образом \cite{mma-bib-14}:
\begin{gather}
    \label{mma-eq-4}
    E(T) = \frac{\sum_{E} Eg(E) e^{-E/k_B T}}{\sum_{E} g(E) e^{-E/k_B T}},
    \\
    \label{mma-eq-5}
    F(T) = -k_B T \ln \left( \sum_E g(E) e^{-E/k_B T} \right),
    \\
    \label{mma-eq-6}
    S(T) = \frac{E(T) - F(T)}{T},
    \\
    \label{mma-eq-7}
    C(T) = \frac{\langle E^2 \rangle - \langle E \rangle^2}{k_B T^2}.
\end{gather}

Рассчитанные из плотности состояний $g(E)$ по формулам \eqref{mma-eq-4}, \eqref{mma-eq-5}, \eqref{mma-eq-6}, \eqref{mma-eq-7} температурные зависимости внутренней энергии E и теплоемкости C при различных значениях обменных взаимодействий $J_1$ и $J_2$, для систем с линейными размерами $L=12$ и $L=36$ приведены на рисунках \ref{mma-fig-7} и \ref{mma-fig-8} соответственно. Отметим важную особенность алгоритма Ванга-Ландау: значения любых термодинамических параметров можно определить для любой температуры, с любым шагом, при этом объем необходимых вычислений, в отличие от других классических алгоритмов метода Монте-Карло, вырастает незначительно.
\begin{figure}[h]
    \begin{center}
        \includegraphics[width=0.5\textwidth]{mma/image31.jpeg}
    \end{center}
    \caption{Температурные зависимости внутренней энергии $E$, рассчитанные из плотности состояний $g(E)$ при различных значениях $J_1$ и $J_2$.}
    \label{mma-fig-7}
\end{figure}
\begin{figure}[h]
    \begin{center}
        \includegraphics[width=0.5\textwidth]{mma/image32.png}
    \end{center}
    \caption{Температурные зависимости теплоемкости $C$, рассчитанные из плотности состояний $g(E)$ при различных значениях $J_1$ и $J_2$.}
    \label{mma-fig-8}
\end{figure}

Из рисунка \ref{mma-fig-7} видно, что в случае в случае $J_2=-1$ на графике наблюдается энергетический скачок, что говорит о фазовом переходе первого рода. Температурная зависимость теплоемкости, приведенная на рисунке \ref{mma-fig-8}, подтверждает эти предположения. В случае $J_2=-0.5$ скачка теплоемкости не наблюдается, в системе в данном случае не происходит фазового перехода. При $J_2=0$ происходит фазовый переход второго рода.

Для более подробного анализа фазовых переходов и определения типа перехода мы использовали гистограммный метод анализа данных.

Если рассчитать гистограмму по энергии по формуле:
\begin{equation}
    \label{mma-eq-8}
    P(E) = g(E) e^{-E/k_B T}
\end{equation}
то в области фазового перехода мы будем наблюдать два пика для фазового перехода первого рода и один максимум для фазового перехода второго рода.

На рисунке \ref{mma-fig-9} приведены гистограммы энергий в области фазового перехода при $J_2 = -1$ и $J_2 = 0$. Как видно из рисунка при $J_2 = -1$ в системе происходит фазовый переход первого рода, а при $J_2 = 0$ фазовый переход второго рода.
\begin{figure}[h]
    \begin{center}
        \begin{subfigure}{0.45\textwidth}
            \begin{center}
                \includegraphics[width=1.0\textwidth]{mma/image34.jpeg}
            \end{center}
        \end{subfigure}
        \begin{subfigure}{0.45\textwidth}
            \begin{center}
                \includegraphics[width=1.0\textwidth]{mma/image35.jpeg}
            \end{center}
        \end{subfigure}
    \end{center}
    \caption{Температурные зависимости энтропии $S$, рассчитанные из плотности состояний $g(E)$ при различных значениях $J_1$ и $J_2$.}
    \label{mma-fig-9}
\end{figure}

Температурные зависимости энтропии $S$ при различных значениях обменных взаимодействий $J_1$ и $J_2$ для систем с линейными размерами $L=12$ и $L=36$ приведены на рисунке \ref{mma-fig-10}. При $J_2 = -0.5$ с понижением температуры энтропия стремится к значению $S_0/N = 0.435$, что говорит о сильном вырождении основного состояния. В остальных случаях с понижением температуры энтропия стремится к нулю. С повышением температуры для всех систем энтропия стремится к значению $\ln3 = 1.09861$.
\begin{figure}[h]
    \begin{center}
        \includegraphics[width=0.5\textwidth]{mma/image36.jpeg}
    \end{center}
    \caption{Температурные зависимости энтропии $S$, рассчитанные из плотности состояний $g(E)$ при различных значениях $J_1$ и $J_2$.}
    \label{mma-fig-10}
\end{figure}

Анализ приведенных выше результатов позволило построить фазовую диаграмму, которая приведена на рисунке \ref{mma-fig-11}.
\begin{figure}[h]
    \begin{center}
        \includegraphics[width=0.5\textwidth]{mma/image36.jpeg}
    \end{center}
    \caption{Фазовая диаграмма.}
    \label{mma-fig-11}
\end{figure}


\section*{Заключение}

Исследование магнитных структур основного состояния, фазовых переходов и термодинамических свойств двумерной модели Поттса с числом состояний спина $q=3$ на решетке Кагоме с учетом взаимодействий первых и вторых ближайших соседей выполнено с использованием алгоритма Ванга-Ландау метода Монте-Карло. Получены магнитные структуры основного состояния в широком интервале значений величины взаимодействия вторых ближайших соседей. Построена фазовая диаграмма зависимости критической температуры от величины взаимодействия вторых ближайших соседей. Для значения $\left| J_2/J_1 = 0.5 \right|$ наблюдается вырождение основного состояния, и система становится фрустрированной.

Таким образом, по проделанной работе можно сделать следующие выводы:
\begin{itemize}
    \item Предложена модель Поттса с числом состояний $q=3$ на решетке Кагоме, учитывающая обменное взаимодействие между первыми и вторыми ближайшими соседями;
    \item Разработана программа для ЭВМ, основанная на новейшем алгоритме Ванга-Ландау, позволяющая исследовать модель Поттса с числом состояний $q=3$ на решетке Кагоме;
    \item Методом Ванга-Ландау вычислены плотности состояний $g(E)$ для модели Поттса с числом состояний $q=3$ на решетке Кагоме.
    \item Определены магнитные структуры основного состояния при различных значениях обменных взаимодействий и показано, что основное состояние может быть ферромагнитным (при $J_2 > -0.5$), триплетным антиферромагнитным (при $J_2 < -0.5$) или сильно вырожденным неупорядоченным фрустрированным (при $J_2 = -0.5$);
    \item Рассчитаны температурные зависимости различных термодинамических параметров, таких как свободная энергия $F$, внутренняя энергия $E$, энтропия $S$, теплоемкость $C$;
    \item Показано, что энтропия для данной модели при температурах близких к абсолютному нулю, при различных соотношениях обменных взаимодействий близка к нулю, кроме случая $J_2 = -0.5$, которое соответствует сильно вырожденному фрустрированному состоянию. С повышением температуры энтропия во всех случаях стремится к теоретически предсказанному значению $\ln 3$;
    \item Вычислены температуры фазовых переходов и определены типы фазовых переходов, происходящих в системе при различных значениях обменных взаимодействий. Построена фазовая диаграмма модели.
\end{itemize}

Результаты, полученные в ходе исследований, могут быть полезными для описания различных низкоразмерных магнитных материалов, имеющих структуру типа решетки Кагоме, таких как Гербертсметиты, Делафосситы, Капелласиты, Фольбортиты и т.д.

\chapter{Влияние магнитного поля на фазовые переходы двумерной модели Поттса с
\texorpdfstring{$q=4$}{q=4} на гексагональной решетке}\label{RMK}


%\section*{Аннотация}
%
%Методом Монте-Карло получены магнитные структуры основного состояния двумерной модели Поттса с числом состояний спина $q = 4$ на гексагональной решетке с учетом взаимодействий первых и вторых ближайших соседей во внешнем магнитном поле $h$. Установлено, что в интервалах значений магнитного поля $0 < h < 1.0$ и $2.0 \leq h \leq 3.5$ наблюдается фазовый переход первого рода, а при значении поля $h = 1.5$ --- фазовый переход второго рода. Показано, что в интервале $4.0 \leq h \leq 7.0$ магнитное поле снимает вырождение основного состояния, и фазовый переход размывается.


\section{Введение}

В течении последних десятилетий наблюдается повышенный интерес к изучению эффектов фрустрации в спиновых решеточных моделях. Конкуренция обменных взаимодействий может привести в магнитных спиновых системах к возникновению фрустрации, которые не позволяют системе одновременно минимизировать все ее локальные взаимодействия, что приводит к бесконечно вырожденному основному состоянию \cite{rmk-bib-1}. Спиновые системы с фрустрациями обладают богатой природой фазовых переходов (ФП) и имеют особенности магнитного, термодинамического и критического поведения. Особый интерес имеет изучение влияния возмущений различной природы, таких как внешнее магнитное поле, взаимодействие вторых ближайших соседей, немагнитные примеси, тепловые и квантовые флуктуации и др. на физические свойства магнитных спиновых систем с фрустрациями. Включение этих возмущающих факторов может привести к совершенно новому физическому поведению таких систем \cite{rmk-bib-2}.

В связи с этим, нами изучается влияние внешнего магнитного поля на характер ФП, магнитные и термодинамические свойства двумерной модели Поттса с фрустрациями. Для фрустрированной модели Поттса существует совсем немного надежно установленных фактов. Большинство имеющихся результатов получены для двумерной модели Поттса с числом состояний спина $q = 2$ и $q = 3$ \cite{rmk-bib-3}. Эта модель изучена достаточно хорошо и получены интересные результаты. Модель Поттса демонстрирует температурный ФП первого или второго порядка, в зависимости от числа состояний спина q, пространственной размерности и геометрии решетки. Двумерная модель Поттса с числом состояний спина $q = 4$ довольно уникальна и до сих пор малоизучена. Результаты исследований двумерной ферромагнитной модели Поттса с числом состояний спина $q = 4$ на треугольной, гексагональной решетках и на решетке Кагоме, полученные методом Монте-Карло (МК) показывают, что в данной модели наблюдается ФП первого рода.

Интерес к модели Поттса обусловлен еще и тем, что эта модель служит основой теоретического описания широкого круга физических свойств и явлений в физике конденсированных сред. К их числу относятся некоторые классы адсорбированных газов на графите, сложные анизотропные ферромагнетики кубической структуры, спиновые стекла, многокомпонентные сплавы и жидкие смеси. На основе модели Поттса с различным числом состояний спина могут быть описаны структурные ФП во многих материалах.

Работ, посвященных изучению влияния внешнего магнитного поля, как возмущающего фактора, на ФП, магнитные и термодинамические свойства модели Поттса с числом состояний спина $q = 4$ практически нет, и этот вопрос все еще остается открытым и малоизученным. В связи с этим, в данной работе нами основе метода МК изучено влияние внешнего магнитного поля на ФП, магнитные и термодинамические свойства двумерной модели Поттса с числом состояний спина $q = 4$ на гексагональной решетке с учетом обменных взаимодействий первых и вторых ближайших соседей. Исследования проводятся на основе современных методов и идей, что позволит получить ответ на ряд вопросов, связанных с характером и природой ФП фрустрированных спиновых систем.


\section{Результат исследования}

Получена фазовая диаграмма зависимости параметра порядка $m$ от величины магнитного поля $h$ в низкотемпературной области (рис. \ref{rmk-fig-1}). На рисунке мы наблюдаем ступенчатую зависимость параметра порядка. Наблюдаются четыре ступеньки: I, II, III и IV. Ступенька I соответствует магнитному упорядочению, при котором только одно состояние спина совпадает с направлением внешнего поля. При увеличении внешнего магнитного поля ($h = 1.5$) еще одно состояние спина выстраивается вдоль внешнего поля. В системе возникает частичный порядок. Это приводит к возникновению ступеньки II на графике. При дальнейшем увеличении поля ($h = 3.0)$, вдоль внешнего поля выстраивается еще одно состояние спина (третье). Этим обусловлено возникновение ступеньки III на графике. При значении поля $h = 4.5$, вдоль внешнего поля выстраивается следующее состояние спина (четвертое). С этим связано возникновение ступеньки IV на графике \cite{rmk-bib-4}.
\begin{figure}[h]
    \begin{center}
        \includegraphics[width=0.75\textwidth]{rmk/image1.jpeg}
    \end{center}
    \caption{Фазовая диаграмма зависимости параметра порядка $m$ от магнитного поля.}
    \label{rmk-fig-1}
\end{figure}

Анализируя рис. \ref{rmk-fig-1} можно предположить, что поля $h = 1.5$; $2.5$ и $4.0$ являются для данной модели фрустрирующими полями. Это также подтверждается поведением температурной зависимости теплоемкости. Теплоемкость в этих полях пологая и значительно ниже, чем в остальных (нефрустрирующих) полях \cite{rmk-bib-4, rmk-bib-5}.


\section*{Заключение}

Исследование влияния магнитного поля на фазовые переходы, магнитные структуры основного состояния и термодинамические свойства двумерной модели Поттса с числом состояний спина $q = 4$ на гексагональной решетке с взаимодействиями вторых ближайших соседей выполнено с использованием репличного обменного алгоритма метода Монте-Карло. На основе гистограммного метода проведен анализ характера фазовых переходов. Получены магнитные структуры основного состояния в широком интервале значений поля. Построена фазовая диаграмма зависимости параметра порядка от величины магнитного поля. Показано, что в интервале значений магнитного поля $0.0 \leq h \leq 3.5$, кроме значения $h = 1.5$ наблюдается фазовый переход первого рода. Для поля $h = 1.5$ наблюдается фазовый переход второго рода. Обнаружено, что при сильных полях $h \geq 4.0$ магнитное поле снимается вырождение основного состояния и фазовый переход в системе размывается.



\backmatter %% Здесь заканчивается нумерованная часть документа и начинаются заключение и ссылки

\Conclusion

%MMG

Для функций из $W^r_{L^1_\rho(\alpha,\beta)}$, $-1 <\alpha, \beta  \le 0$, исследована равномерная сходимость рядов Фурье по системам полиномов, ортогональных в смысле Соболева и порожденных системами полиномов Якоби с показателями $\alpha, \beta$.

Получены необходимые и достаточные условия сходимости в пространстве $W^r_{L^p_\rho(A,B)}$, $p > 1$, $A, B \in \mathbb{R}$ рядов Фурье по соболевской системе полиномов, порожденной полиномами Якоби с показателями $\alpha, \beta  > -1$. Показано также, что при дополнительном условии на $A, B$ и $p$ указанные ряды сходятся равномерно на отрезке $[-1,1]$.

%GRM

Продолжено исследование задачи об отклонении от функции $f\in W^r$ на $(-1,1)$ ряда Фурье по системе полиномов Якоби--Соболева $\{P_n^{\alpha-r,-r}(x)\}$, начатое в \cite{Ram-SharMN}. В частности были получены оценки для функций типа Лебега частичных сумм ряда Фурье по системе полиномов $\{P_n^{\alpha-r,-r}(x)\}$ (см. теоремы \ref{Ram-theo1}-\ref{Ram-theo3}).

Была рассмотрена система полиномов $\{m_{n,N}^{\alpha,r}(x)\}$, ортонормированная по Соболеву на сетке $\Omega_\delta=\{0, \delta, 2\delta, \ldots\}$ и порожденная системой модифицированных полиномов Мейкснера $\{m_{n,N}^{\alpha}(x)\}$. Показано, что ряд Фурье по этой системе сходится к $f\in W^r_{l^p_{\rho_N}(\Omega_\delta)}$ поточечно на сетке $\Omega_\delta$ при $p\ge2$. А в случае, когда $1\le p<2$ показано, что существуют функция и сетка $\Omega_\delta$, ряд Фурье которой расходится в некоторой точке $x_0\in\Omega_\delta$. Кроме того, исследованы аппроксимативные свойства частичных сумм ряда Фурье по системе $\{m_{n,N}^{0,r}(x)\}$.

% заключение к отчёту
\begin{thebibliography}{111}

  % Ниже указаны примеры форматирования литературы

  \bibitem{mmg-MarcellanXu2015}
  Marcellán F., Xu Y. On Sobolev orthogonal polynomials // Expositiones Math. --- 2015. --- Vol 33. P. 308---352.

  \bibitem{mmg-mmg-walsh-Shii-UMN}
  Шарапудинов И.И. Ортогональные по Соболеву системы функций и некоторые их приложения // УМН. --- 2019. --- Т. 74, \No 4(448). --- С. 87---164.

  \bibitem{ark-bib-2}
  Стечкин С.Б., Субботин Ю.Н.
  Сплайны в вычислительной математике.
  --- М.: Наука,
  1976. --- 248~с.

  \bibitem{ark-bib-3}
  Завьялов Ю.С., Квасов Б.И., Мирошниченко В.Л.
  Методы сплайн-функций.
  --- М.: Наука,
  1980.
  --- 352 c.
  
  %MMG
  	\bibitem{mmg-MarcellanXu2015}
	Marcellan F., Xu Y. On Sobolev orthogonal polynomials // Expo Math. --- 2015. --- Vol 33. P. 308---352.
	
	
	
	
	\bibitem{mmg-MarcellanJacobiSobolev}
	Marcellan, F., Quintana, Y., Urieles, A.: On the Pollard decomposition method applied to some Jacobi–
	Sobolev expansions. Turk. J. Math. 37(6), 934–948 (2013).
	
	
	
	
	\bibitem{mmg-CiaurriJacobiSobolev}
	Ciaurri, O., Minguez, J.: Fourier series of Jacobi–Sobolev polynomials. Integral Transf. Spec. Funct.
	30, 334–346 (2019).
	
	
	
	
	\bibitem{mmg-CiaurriCoherentPairs}
	Ciaurri, O., Minguez, J.: Fourier series for coherent pairs of Jacobi measures. Preprint.
	
	
	
	
	\bibitem{mmg-Fejzullahu2010}
	B.\,Xh. Fejzullahu. Asymptotic properties and Fourier expansions of orthogonal polynomials with a non-discrete Gegenbauer–Sobolev inner product. Journal of Approximation Theory, Volume 162, Issue 2, 2010, Pages 397-406, ISSN 0021-9045, https://doi.org/10.1016/j.jat.2009.07.002.
	
	
	
	
	\bibitem{mmg-Fejzullahu2013}
	B.\,Xh. Fejzullahu, F. Marcellan, J.J. Moreno-Balcazar, Jacobi–Sobolev orthogonal polynomials: Asymptotics and a Cohen type inequality, Journal of Approximation Theory,
	Volume 170, 2013, Pages 78-93, ISSN 0021-9045, https://doi.org/10.1016/j.jat.2012.05.015.
	
	
	
	
	
	\bibitem{mmg-IserlesKoch1991}
	Iserles, A., Koch, P.E., Norsett, S.P., Sanz-Serna, J.M.: On polynomials orthogonal with respect to
	certain Sobolev inner product. J. Approx. Theory 65, 151–175 (1991).
	
	
	
	
	\bibitem{mmg-Marcellan2002}
	F. Marcellan, B.P. Osilenker, I.A. Rocha, On Fourier Series of a Discrete Jacobi–Sobolev Inner Product, Journal of Approximation Theory, Volume 117, Issue 1, 2002,Pages 1-22, ISSN 0021-9045, https://doi.org/10.1006/jath.2002.3681.
	
	
	
	
	\bibitem{mmg-Rocha2003}
	Rocha I. A., Marcellan F., Salto L. Relative asymptotics and Fourier series of orthogonal
	polynomials with a discrete Sobolev inner product // J. Approx. Theory. 2003. V. 121.
	P. 336–356.
	
	
	
	
	\bibitem{mmg-OsilenkerFourier2012}
	Осиленкер Борис Петрович Сходимость и суммируемость рядов Фурье - Соболева // Вестник МГСУ. 2012. №5. URL: https://cyberleninka.ru/article/n/shodimost-i-summiruemost-ryadov-furie-soboleva-1 (дата обращения: 23.10.2020).
	
	
	
	
	\bibitem{mmg-OsilenkerLinearMethods2015}
	Б. П. Осиленкер, О линейных методах суммирования рядов Фурье по многочленам, ортогональным в дискретных пространствах Соболева, Сиб. матем. журн., 2015, том 56, номер 2, 420–435.
	
	
	
	
	\bibitem{mmg-Fejzullahu2009}
	Fejzullahu, Marcellan. On convergence and divergence of Fourier expansions with respect to some Gegenbauer-Sobolev type inner product.
	Communications in the Analytic Theory of Continued Fractions, 2009, n. 16, p. 1-11.
	
	
	
	
	\bibitem{mmg-CiaurriSigma2018}
	Ciaurri, O., Minguez, J.: Fourier series of Gegenbauer–Sobolev polynomials. SIGMASymm. Integrabi.
	Geom. Methods Appl. 14, 1–11 (2018).
	
	
	
	
	
	\bibitem{mmg-SharapudinovUMN}
	И. И. Шарапудинов, “Ортогональные по Соболеву системы функций и некоторые их приложения”, УМН, 74:4(448) (2019), 87–164.
	
	
	
	
	\bibitem{mmg-SharapudinovIzvRan2019}
	И. И. Шарапудинов, “Системы функций, ортогональные по Соболеву, ассоциированные с ортогональной системой”, Изв. РАН. Сер. матем., 82:1 (2018), 225–258.
	
	
	
	
	\bibitem{mmg-MMG2019}
	М. Г. Магомед-Касумов, “Система функций, ортогональная в смысле Соболева и порожденная системой Уолша”, Матем. заметки, 105:4 (2019), 545–552; Math. Notes, 105:4 (2019), 543–549.
	
	
	
	
	\bibitem{mmg-Gadzhimirzaev2019}
	R. M. Gadzhimirzaev, “Sobolev-orthonormal system of functions generated by the system of Laguerre functions”, Пробл. анал. Issues Anal., 8(26):1 (2019), 32–46.
	
	
	
	
	\bibitem{mmg-Diaz-Gonzalez2020}
	Diaz-Gonzalez, A., Marcellan, F., Pijeira-Cabrera, H. et al. Discrete–Continuous Jacobi–Sobolev Spaces and Fourier Series. Bull. Malays. Math. Sci. Soc. (2020). https://doi.org/10.1007/s40840-020-00950-7.
	
	
	
	
	
	\bibitem{mmg-Shii-izvran2018}
	И. И. Шарапудинов. Ортогональные по Соболеву системы функций, ассоциированные с ортогональной системой функций // Изв. РАН. Сер. матем., 2018, том 82, выпуск 1, с. 225--258. (\url{http://mi.mathnet.ru/izv8536}).  	
	
	
	
	
	\bibitem{mmg-Shii-matzam2017}
	И. И. Шарапудинов, Аппроксимативные свойства рядов Фурье по многочленам, ортогональным по Соболеву с весом Якоби и дискретными массами, Матем. заметки, 101:4 (2017), 611–629; Math. Notes, 101:4 (2017), 718–734.
	
	\bibitem{mmg-Muck1969}
	Muckenhoupt, B.: Mean convergence of Jacobi series. Proc. Am. Math. Soc. 23, 306–310 (1969).
	
	
	\bibitem{mmg-Zorschikov1967}
	А. В. Зорщиков, О равномерности сходимости рядов Фурье по многочленам Якоби, Докл. АН СССР, 1967, том 176, номер 1, 35–38.
	
	
	\bibitem{mmg-Fiht2}
	Г.М. Фихтенгольц. Курс дифференциального и интегрального исчисления. В 3 т. Т. II / Пред. и прим. А.А. Флоринского. --- 8-е изд. --- М.: ФИЗМАТЛИТ, 2003. --- 864 с. --- ISBN 5-9221-0157-9.

%GRM

\bibitem{Ram-Ba-Ra-Pe}
{Barry P. Rajkovi\'c P.M., Petkovi\'c M.D.} An application of Sobolev orthogonal polynomials to the computation of a special Hankel determinant // In book: Approximation and Computation (Chapter 4). 2011. Vol. 42. Pp. 53--60.

\bibitem{Ram-Mar-Xu}
{Marcell\'an F., Xu Y.} On Sobolev orthogonal polynomials // Expo Math. 2015. Vol. 33. Pp. 308--352.

\bibitem{Ram-Shar-UMN}
{Шарапудинов И.И.} Ортогональные по Соболеву системы функций и некоторые их приложения // УМН. 2019. Т. 74. Вып. 4. С. 87--164.

\bibitem{Ram-SharMN}
{Шарапудинов И.И.} Аппроксимативные свойства рядов Фурье по многочленам, ортогональным по Соболеву с весом Якоби и дискретными массами // Матем. заметки. 2017. Т. 101. Вып. 4. С. 611--629.	
	
\bibitem{Ram-Sege}
{Сеге Г.} Ортогональные многочлены. Москва. Физматгиз. 1962.	

\bibitem{Ram-Shar-VMJ}
{Шарапудинов И.И., Гаджиева З.Д., Гаджимирзаев Р.М.} Разностные уравнения и полиномы, ортогональные по Соболеву, порожденные многочленами Мейкснера //  Владикавк. матем. журн. 2017. Т. 19. Вып. 2. С. 58--72.

\bibitem{Ram-Ar-Go-Mar}
{Area I., Goboy E., Marcell\'an F.} Inner products involving differences: The Meixner–Sobolev polynomials // J. Differ. Equations Appl. 2000. Vol. 6. Pp. 1--31.

\bibitem{Ram-Kh-Old}
{Khwaja S.F., Olde-Daalhuis A.B.} Uniform asymptotic approximations for the Meixner–Sobolev polynomials // Analysis and Applications. 2012. Vol. 10. № 3. Pp. 345--361.

\bibitem{Ram-Bav1}
{Bavinck H., Haeringen H.V.} Difference equations for generalized Meixner polynomials // J. Math. Anal. Appl. 1994. Vol. 1994. Pp. 453--463.

\bibitem{Ram-Bav2}
{Bavinck H., Koekoek R.} Difference operators with sobolev type Meixner polynomials as eigenfunctions // Comput. Math. Appl. 1998. Vol. 36. Pp. 163--177.

\bibitem{Ram-Shar-Sar}
{Шарапудинов И.И., Гаджиева З.Д.} Полиномы, ортогональные по Соболеву, порожденные многочленами Мейкснера // Изв. Сарат. ун-та. Нов. сер. Сер. Математика. Механика. Информатика. 2016. Т. 16. Вып. 3. С. 310--321.

\bibitem{Ram-Mor-Bal}
{Moreno-Balc\'azar J.} $\delta$-Meixner-Sobolev orthogonal polynomials: Mehler–heine type formula and zeros // J. Comput. Appl. Math. 2015. Vol. 284. Pp. 228--234.

\bibitem{Ram-Co-So-Vil}
{Costas-Santos R.S., Soria-Lorente A., Vilaire J.-M.} On polynomials orthogonal with respect to an inner product involving higher-order differences: the Meixner case // Mathematics. 2022. Vol. 10. Pp. 1952.

\bibitem{Ram-SharBook}
{Шарапудинов И.И.} Многочлены, ортогональные на сетках. Махачкала. Изд-во Даг. гос. пед. ун-та. 1997.

\bibitem{Ram-MN2019}
{Гаджимирзаев Р.М.} Оценка функции Лебега сумм Фурье по модифицированным полиномам Мейкснера // Матем. заметки. 2019. Vol. 106. № 4. Pp. 519--530.

%ARK

\bibitem{ark-1} Цалюк~З.Б. Интегральные уравнения Вольтерра //
 Итоги науки и техн. Сер. Мат. анал. 1977. Т.~15.
  С.~131--198.

\bibitem{ark-2} Полянин~А.Д., Манжиров~А.В.
Интегральные уравнения. Часть 1:~справочник для вузов.  2-е изд.
 М.: Юрайт, 2017. 365~c.

\bibitem{ark-3} Сидоров~Д.Н. Методы анализа интегральных динамических
моделей: теория и приложения.  Иркутск: Изд-во ИГУ, 2013. 293~c.

\bibitem{ark-4} Алберг~Дж., Нильсон~Э., Уолш~Дж. Теория сплайнов и ее приложения. М.: Мир, 1972. 319~c.

\bibitem{ark-5} El Tom~M.E.A. Numerical solution of Volterra integral equations by spline
 functions //BIT.  1972. V.~13. P.~1--7.

\bibitem{ark-6} Netravali~A.N. Spline approximation to the solution of the Volterra integral equation of the
 second kind // Math. Comput. 1973. V.~27.
 Iss.~121. P.~99--106.

\bibitem{ark-7} Стечкин~С.Б., Субботин~Ю.Н. Сплайны в вычислительной математике.
 М.: Наука, 1976. 248~с.

\bibitem{ark-8} Nord~S. Approximation properties of the spline fit//~ BIT.
1967. V.~7. P.~132—144.

\bibitem{ark-9} Рамазанов~А.-Р.К., Магомедова~В.Г. Безусловно сходящиеся
интерполяционные рациональные сплайны // Мат. заметки. 2018. Т.~103.
 Вып.~4. С.~592--603.

\bibitem{ark-10} Рамазанов~А.-Р.К., Магомедова~В.Г. Сплайны по трехточечным рациональным интерполянтам
с автономными полюсами //
Дагестанские электронные математические известия. 2017. Вып.~7. C.~16--28.

\bibitem{ark-11} Рамазанов~А.-Р.К., Магомедова~В.Г. О приближенном решении дифференциальных
уравнений с помощью рациональных сплайн-функций // Журнал вычислительной математики и
математической физики. 2019. Т.~59. №~4. С.~579–586.

\bibitem{ark-12} Сендов~Б., Попов~В.А. Усредненные модули гладкости. М.: Мир,
1988. 328~c.

\bibitem{ark-13} Рамазанов~А.-Р.К., Рамазанов А.К., Магомедова~В.Г.
 О динамическом решении интегрального уравнения Вольтерры в виде
 рациональных сплайн-функций // Мат. заметки. 2022. Т.~111.
 Вып.~4. С.~581--591.

\bibitem{ark-14} Ramazanov~A.-R.K., Ramazanov A.K., Magomedova V.G.
On the Dynamic Solution of the Volterra Integral Equation in the Form
Rational Spline Functions// Mathematical Notes. 2022. Vol. 111, No. 4.
P.~596 – 603.

\bibitem{ark-15} Рамазанов~А.-Р.К., Магомедова~В.Г.
 О приближенном решении интегральных уравнений Фредгольма методом
коллокационных рациональных сплайн-функций //
Дагестанские электронные математические известия. 2022. Вып.~17. C.~20--31.

\bibitem{ark-16} Рамазанов~А.-Р.К., Алиева Р.Ш. О гладкой интерполяции
 локальными полиномиальными сплайнами // Вестник Дагестанского
государственного университета. Серия 1. Естественные науки. 2022. Том~37.
 Вып.~1. С.~32--39.


  % MMA
  
  \bibitem{mma-bib-1}
  Prewitt C.T., Shannon R.D., Rogers D.B.
  Chemistry of noble metal oxides. II. Crystal structures of PtCoO2, PdCoO2, CuFeO2 and AgFeO2.
  //
  Inorg. Chem.
  --- 1971.
  --- V. 10, №4.
  --- P. 719---723.
  
  \bibitem{mma-bib-2}
  Hirakawa K., Kadowaki H., Ubukoch K.
  Experimental studies of triangular lattice antiferromagnets with S = ½: NaTiO2 and LiNiO2.
  //
  J. Phys. Soc. Japan.
  --- 1985.
  --- V. 54, №9.
  --- P. 3526---3536.
  
  \bibitem{mma-bib-3}
  Townsend M.G., Longworth G. and Roudaut E.
  Triangular-spin, kagome plane in jarosites
  //
  Physical Review В.
  --- 1986.
  --- V. 33.
  --- P. 4919---4926.
  
  \bibitem{mma-bib-4}
  Li J., Sleight A.W.
  Structure of $\beta$-AgAlO2and structural systematics of tetrahedral MM'X2compounds.
  //
  J. Solid State Chem.
  --- 2004.
  --- V. 177, №3.
  --- P. 889---894.
  
  \bibitem{mma-bib-5}
  Sachdev, S.
  Kagome- and triangular-lattice Heisenberg antiferromagnets: ordering from quantum fluctuations and quantum-disordered ground states with un-confined bosonic spinons. Phys. Rev. B 45, 12377---12396 (1992).
  
  \bibitem{mma-bib-6}
  Xu G., Lian B., Zhang S.-C.
  Intrinsic quantum anomalous Hall effect in the Kagome lattice Cs2LiMn3F12
  //
  Phys. Rev. Lett. 115, 186802 (2015).
  
  \bibitem{mma-bib-7}
  Chen H., Niu Q., Macdonald A.H.
  Anomalous hall effect arising from noncollinear antiferromagnetism. Phys. Rev. Lett. 112, 17205 (2014)
  
  \bibitem{mma-bib-8}
  Balents L.
  Spin liquids in frustrated magnets. Nature 464, 199---208 (2010)
  
  \bibitem{mma-bib-9}
  Kang M., Ye L., Fang S., et al.
  Dirac fermions and flat bands in the ideal kagome metal FeSn
  //
  Nature Materials.
  --- 2020.
  --- V. 19.
  --- P. 163---170.
  
  \bibitem{mma-bib-10}
  Ramazanov M.K., Murtazaev A.K., Magomedov M.A., Badiev M.K.
  Phase transitions and thermodynamic properties of antiferromagnetic Ising model with next-nearest-neighbor interactions on the Kagomé lattice
  //
  Phase Transitions. ---2018. ---V. 91. ---P. 610-618.
  
  \bibitem{mma-bib-11}
  Магомедов М.А., Муртазаев А.К.
  Плотность состояний и структура основного состояния модели Изинга на решетке Кагоме с учетом взаимодействия ближайших и следующих соседей
  //
  ФТТ.
  --- 2018.
  --- Т. 60.
  --- С. 1173---1177.
  
  \bibitem{mma-bib-12}
  Ramazanov M.K., Murtazaev A.K., Magomedov M.A. Rizvanova T.R., Murtazaeva A.A.
  Phase diagram of the Potts model with the number of spin states q=4 on a Kagome lattice
  //
  Low Temperature Physics.
  --- 2021.
  --- V. 47, \No 5.
  --- P. 396---400.
  
  \bibitem{mma-bib-13}
  Landau D.P., Wang F., Tsai S.-H.
  Critical endpoint behavior: A Wang-Landau study
  //
  Comp. Phys. Comm.
  --- 2008.
  --- V. 179.
  --- P. 8.
  
  \bibitem{mma-bib-14}
  Körner M., Troyer M., in Computer Simulation Studies in Condensed-Matter Physics XVI, edited by D. Landau, S. Lewis and H.-B. Schütler (Springer Berlin Heidelberg, 2006), Vol. 103, p. 142.
  
  \bibitem{mma-bib-15}
  Chiaki Y., Yutaka O.
  Three-dimensional antiferromagnetic q -state Potts models: application of the Wang-Landau algorithm
  //
  Journal of Physics A: Mathematical and General.
  --- 2001.
  --- V. 34.
  --- 8781.
  
  \bibitem{mma-bib-16}
  Zhou C.Bhatt R.N.
  Understanding and improving the Wang-Landau algorithm
  //
  Physical Review E.
  --- 2005.
  --- V. 72(2).
  --- P. 025701.


  % RMK

  \bibitem{rmk-bib-1}
  Diep H.T.
  Frustrated Spin Systems.
  --- World Scientific Publishing Co. Pte. Ltd., Singapore, 2004. 
  --- P. 624.
  
  \bibitem{rmk-bib-2}
  Baxter R.J.
  Exactly Solved Models in Statistical Mechanics.
  --- Academic, New York, 1982;
  --- Mir, Moscow, 1985.
  
  \bibitem{rmk-bib-3}
  Wu F.Y.
  Exactly Solved Models: A Journey in Statistical Mechanics 
  --- World Scientific, New Jersey, 2008.
  
  \bibitem{rmk-bib-4}
  Ramazanov M.K., Murtazaev A.K., Magomedov M.A.
  Phase transitions in the frustrated Potts model in the magnetic field
  //
  Physica E: Low-dimensional Systems and Nanostructures.
  --- 2022.
  --- V. 140.
  --- P. 115226-1-115226-6.
  
  \bibitem{rmk-bib-5}
  Рамазанов М.К., Муртазаев А.К., Магомедов М.А.
  Фрустрированная модель Поттса с числом состояний спина $q = 4$ в магнитном поле
  //
  ЖЭТФ.
  --- 2022.
  --- Т. 161, вып. 6.
  --- С. 816---824.

\end{thebibliography} 
\chapter{ПРИЛОЖЕНИЕ А. Cписок работ, опубликованных \texorpdfstring{\\ }{} по теме НИР в [ГОД] г.}

% Ниже представлены примеры опубликованных работ

\section*{Список опубликованных научных статей}

\begin{enumerate}[1]
    % \item
    % Ramazanov A.-R.K., Magomedova V.G. Approximate Solution of Nonlinear Differential Equations with the Help of Rational Spline Functions // Computational Mathematics and Mathematical Physics. --- 2021. --- Vol. 61. No. 8. P.~1252---1259.

    % \item
    % Sultanakhmedov, M.S. Approximation of Functions by Discrete Fourier Sums in Polynomials Orthogonal on a Nonuniform Grid with Jacobi Weight. // Math Notes. --- 2021. --- Vol. 110. P.~418---431.

    % \item
    % Gadzhimirzaev, R.M., Shakh-Emirov, T.N. Approximation Properties of the Vallée-Poussin Means of Partial Sums of a Special Series in Laguerre Polynomials. // Math Notes. --- 2021. --- Vol. 110. P.~475---488.


    % MMA
    
    \item
    Ramazanov M.K., Murtazaev A.K., Magomedov M.A.
    Phase transitions in the frustrated Potts model in the magnetic field
    //
    Physica E: Low-dimensional Systems and Nanostructures.
    --- 2022.
    --- V. 140.
    --- P. 115226-1-115226-6. DOI: 10.1016/j.physe.2022.115226.
    
    \item
    Рамазанов М.К., Муртазаев А.К., Магомедов М.А., Курбанова Д.Р., Рамазанов К.М., Хизриев М.С.
    Энергетический анализ магнитных структур основного состояния модели Поттса
    //
    Дагестанские электронные математические известия.
    --- 2022.
    --- Вып. 17.
    --- С. 44---52.
    
    \item
    Магомедов М.А., Муртазаев А.К., Исаева М.М.
    Фазовая диаграмма и структура основного состояния трехвершинной модели Поттса на решетке Кагоме
    //
    Дагестанские электронные математические известия.
    --- 2022.
    --- Вып.17.
    --- С. 53---66.
    
    \item
    Рамазанов М.К., Муртазаев А.К., Магомедов М.А.
    Фрустрированная модель Поттса с числом состояний спина $q = 4$ в магнитном поле
    //
    ЖЭТФ.
    --- 2022.
    --- Т. 161, вып. 6.
    --- С. 816---824. DOI: 10.31857/S0044451022060049.
    
    \item
    Рамазанов М.К., Муртазаев А.К., Магомедов М.А., Мазагаева М.К., Джамалудинов М.Р.
    Исследование влияния слабых магнитных полей на термодинамические свойства модели Поттса с числом состояний спина $q = 4$ на гексагональной решетке
    //
    Физика твердого тела.
    --- 2022.
    --- Т. 64, вып. 2.
    --- С. 237---240. DOI: 10.21883/FTT.2022.02.51935.226


    % KRI
    
    \item
    Кадиев Р.И., Поносов А.В.
    Глобальная устойчивость систем нелинейных дифференциальных уравнений Ито с последействием и W-метод Н.В. Азбелева
    //
    Изв. Вузов. Матем.
    --- 2022.
    --- №1.
    --- С. 38---56.
    
    \item
    Кадиев Р.И., Поносов А.В.
    Исследование устойчивости решений непрерывно-дискретных стохастических систем с последействием методом регуляризации
    //
    Дифф. Урав.
    --- 2022.
    --- Т. 58, № 4.
    --- С. 435---455
    
    \item
    Kadiev R., Ponosov A.
    Positive invertibility of matrices and exponential stability of linear stochastic systems with delay
    //
    International Journal of Differential Equations.
    --- 2022.
    --- V. 2022.
    Article ID 5549693, 13 pages.
    
    \item
    Кадиев Р.И., Шахбанова З.И.
    Экспоненциальная устойчивость решений одной непрерывно-дискретной линейной системы Ито с ограниченными запаздываниями
    //
    Вестник ДГУ, Серия 1. Естественные науки.
    --- 2022.
    --- Т. 37, вып. 3.
    --- С. 7---17.


    % GRM

    \item
    Гаджимирзаев Р.М.
    Аппроксимативные свойства средних типа Валле-Пуссена частичных сумм ряда Фурье по полиномам Лагерра – Соболева
    //
    Сиб. Матем. Журн.
    --- 2022.
    --- Т. 63, № 3.
    --- С. 545---561. 
    
    \item
    Гаджимирзаев Р.М.
    Об аппроксимативных свойствах рядов Фурье по полиномам Якоби $Pn \alpha - r$, $-r(x)$, ортогональным по Соболеву
    //
    Матем. Заметки.
    --- 2022.
    --- Т. 111, № 6.
    --- С. 803---818.


    % MMG

    \item
    Магомед-Касумов М.Г., Шах-Эмиров Т.Н.
    О представлении соболевских систем, ортогональных относительно скалярного произведения с одной дискретной точкой
    //
    Матем. Заметки.
    --- 2022.
    --- Т. 111, № 4.
    --- С. 561---570. 
    Англоязычная версия:
    Mathematical Notes. 
    --- 2022.
    --- V. 111, \No 4.
    --- P. 561---570.
    
    \item
    Magomed-Kasumov M.G.
    Existence and uniqueness theorems for a differential equation with a discontinuous right-hand side.
    //
    Vladikavkaz Mathematical Journal.
    --- 2022.
    --- V. 24, iss. 1.
    --- P. 54---64.


    % ARK
    
    \item
    Рамазанов А.-Р.К., Рамазанов А.К., Магомедова В.Г.
    О динамическом решении интегрального уравнения Вольтерры в виде рациональных сплайн-функций
    //
    Матем. Заметки.
    --- 2022.
    --- Т. 111, № 4.
    --- С. 581---591. 
    
    \item
    Рамазанов А.-Р.К., Алиева Р.Ш.
    О гладкой интерполяции локальными полиномиальными сплайнами
    //
    Вестник ДГУ. Серия1. Естественные науки.
    --- 2022.
    --- Т. 37, вып. 1.
    --- С. 32---39.
    
    \item
    Рамазанов А.-Р.К., Магомедова В.Г.
    О приближенном решении интегральных уравнений Фредгольма методом коллокационных рациональных сплайн-функций
    //
    Дагестанские Электронные Математические Известия.
    --- 2022.
    --- № 17.
    --- С. 20---31.


    % SMM
    
    \item
    Сиражудинов М.М., Джамалудинова. С.П.
    Оценки локально-периодического усреднения задачи Римана – Гильберта для обобщённого уравнения Бельтрами
    //
    Дифф. Урав.
    --- 2022.
    --- Т. 58, № 6.
    --- С. 777---794. 
    
    \item
    Сиражудинов М.М., Ибрагимов М.Г., Магомедова М.Г.
    Оценки погрешности локально-периодического усреднения периодической задачи для уравнения Бельтрами
    //
    Вестник ДГУ. Серия1: Естественные науки
    --- 2022.
    --- № 1.
    --- С. 24---31.


    % MZG
    
    \item
    Меджидов З.Г.
    Обращение V-преобразования Радона со степенным весом на плоскости
    //
    Даг. Элект. Матем. Известия.
    --- 2022.
    --- № 17.
    --- С. 32---43.


    % AKM
    
    \item
    Магомедов А.М., Раджабова Н.Ш.
    Замощение клетчатой полосы шириной 4.
    //
    Информатика в школе.
    --- 2022.
    --- \No 1.
    --- P. 81---84. 
    
    \item
    Магомедов А.М., Лавренченко С.А.
    Некоторые свойства прямых рекуррентных соотношений для последовательностей димерных чисел
    //
    Вестник ДГУ. Серия 1. Естественные науки.
    --- 2022.
    --- Т. 37, вып. №1.
    --- С. 51---62.
    
    \item
    Магомедов А.М.
    "Компьютерное" решение и обобщение классической арифметической задачи
    //
    Вестник ДГУ. Серия 1. Естественные науки.
    --- 2022.
    --- Т. 37, вып. №3.
    --- С. 25---29.
    
    \item
    Магомедов А.М., Якубов Р.А.
    Некоторые подходы к определению четности числа разбиений прямоугольной полосы
    //
    Вестник ДГУ. Серия 1. Естественные науки.
    --- 2022.
    --- Т. 37, вып. №3.
    --- С. 30---33.


    % BAB
    
    \item
    Муртазаев А.К., Бабаев А.Б.
    Фазовые переходы в двумерных моделях Поттса на гексагональной решетке
    //
    Журнал экспериментальной и теоретической физики.
    --- 2022.
    --- Т. 161, вып. 6.
    --- С. 847---852.
    
    \item
    Муртазаев А.К., Бабаев А.Б.
    Компьютерное моделирование фазовых переходов в трехмерных слабо разбавленных спиновых системах
    //
    Поверхность. Рентгеновские, синхротронные и нейтронные исследования.
    --- 2022.
    --- №5.
    --- С. 37---41. 

\end{enumerate}

\section*{Список зарегистрированных программ для ЭВМ}

\begin{enumerate}[1]
    \item
    Магомедов А.М., Шарапудинов Т.И. Свидетельство №2021666365 о государственной регистрации программы для ЭВМ «Программа перечисления биграфов с переменной численностью вершин в каждой доле». Заявка №2021665799, дата поступления 13 октября 2021 г. Дата государственной регистрации в Реестре программ для ЭВМ 13 октября 2021. Правообладатель: ДФИЦ РАН.
    
    \item
    Султанахмедов М.С.~Свидетельство №2021682136 о государственной регистрации программы для ЭВМ «Программа для рекуррентного вычисления значений полиномов, ортогональных по Соболеву». Заявка № 2021682107, дата поступления 30 декабря 2021 г. Дата государственной регистрации в Реестре программ для ЭВМ 30 декабря 2021. Правообладатель: ДФИЦ РАН.
\end{enumerate}


\end{document}
