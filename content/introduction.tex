\Introduction

В последние годы теории ортогональных по Соболеву полиномов посвящено большое число работ. В частности, это связано с тем, что соболевские скалярные произведения и соответствующие им ортогональные системы (и их дифференциальные аналоги) играют важную роль во многих проблемах теории функций, квантовой механики, математической физики, вычислительной математики и т.д. В частности, ряды Фурье по ним обладают важными для приложений свойствами, которые отсутствуют у рядов Фурье по классическим ортогональным системам (см., например, \cite{Ram-Ba-Ra-Pe,Ram-Mar-Xu,Ram-Shar-UMN}).
Например, ряды Фурье по соболевским полиномам оказываются более естественным аппаратом, чем ряды Фурье по классическим ортогональным полиномам, для приближенного решения краевых задач, в которых требуется контроль поведения приближенного решения в одной или нескольких точках.
При решении таких задач важную роль играют сходимость и скорость сходимости рядов Фурье в различных функциональных пространствах.

Достаточно общий вид скалярного произведения типа Соболева может быть выражен формулой
\begin{equation}\label{mmg-inner-prod-sob-common}
	\langle f,g \rangle = \sum_{k=0}^{m}\int\limits_{\mathbb{R}}f^{(k)}(x)g^{(k)}(x)d\mu_k,
\end{equation}
где $d\mu_k$ -- борелевские меры.
Как правило, рассматриваются следующие виды соболевских скалярных произведений \eqref{mmg-inner-prod-sob-common}:
\begin{itemize}
	\item
	непрерывные (все меры $\mu_k$, $k \ge 0$, абсолютно непрерывны),
	\item
	дискретные ($\mu_0$ абсолютна непрерывна, а $\mu_k$, $k \ge 1$, -- дискретны),
	\item
	дискретно-непрерывные ($\mu_m$ абсолютна непрерывна, а $\mu_k$, $k < m$, -- дискретны).	
\end{itemize}

Исследованиям вопросов сходимости рядов Фурье по полиномам, ортогональным относительно непрерывного скалярного произведения, посвящены, например, работы \cite{mmg-MarcellanJacobiSobolev,mmg-CiaurriJacobiSobolev,mmg-CiaurriCoherentPairs,mmg-Fejzullahu2010,mmg-Fejzullahu2013}. В \cite{mmg-MarcellanJacobiSobolev,mmg-CiaurriJacobiSobolev} рассматривается случай, когда $d\mu_k(x)=w_{\alpha+k,\beta+k}(x)dx$, где $w_{\alpha+k,\beta+k}(x)=(1-x)^{\alpha+k}(1+x)^{\beta+k}$ -- веса Якоби. В работе \cite{mmg-CiaurriCoherentPairs} исследуется случай $m=1$, причем меры $\mu_0$ и $\mu_1$ образуют так называемую когерентную пару (см. \cite{mmg-IserlesKoch1991,mmg-MarcellanXu2015}). В \cite{mmg-Fejzullahu2010} получены необходимые условия сходимости по норме ряда Фурье для полиномов, ортогональных относительно \eqref{mmg-inner-prod-sob-common} при $m=1$ и $d\mu_0(x)=d\mu_1(x)=(1-x^2)^{\alpha-1/2}dx$.

В случае дискретных скалярных произведений сходимость соответствующих рядов Фурье и их линейных средних исследовалась в работах \cite{mmg-Marcellan2002,mmg-Rocha2003,mmg-OsilenkerFourier2012,mmg-OsilenkerLinearMethods2015,mmg-OsilenkerFourier2022,mmg-Fejzullahu2009,mmg-CiaurriSigma2018}. В работах \cite{mmg-Marcellan2002,mmg-Rocha2003} рассмотрены вопросы поточечной и равномерной сходимости рядов Фурье в случае скалярного произведения вида:
\begin{equation*}
	\langle f,g \rangle = \int_{-1}^{1}f(x)g(x)d\mu(x)+
	\sum_{k=1}^K\sum_{i=0}^{N_k} M_{k,i}f^{(i)}(a_k)g^{(i)}(a_k),
\end{equation*}
где $d\mu(x)$ -- мера Якоби. В \cite{mmg-OsilenkerFourier2012,mmg-OsilenkerLinearMethods2015,mmg-Fejzullahu2009,mmg-CiaurriSigma2018} рассмотрен частный случай приведенного выше скалярного произведения ($K=2$, $N_1=N_2=1$, $a_1=-1$, $a_2=1$) и исследована сходимость рядов Фурье и их линейных средних по соответствующим ортогональным полиномам.

Ряды Фурье в дискретно-непрерывном случае довольно подробно исследовались в работах Шарапудинова И.И. (см. \cite{mmg-SharapudinovUMN,mmg-SharapudinovIzvRan2019} и приведенные там списки литературы). В них рассматривается скалярное произведение вида
\begin{equation}\label{mmg-sob-prod}
	\langle f, g \rangle = \sum_{k=0}^{r-1}f^{(k)}(a)g^{(k)}(a)+\int_a^b f^{(r)}(x)g^{(r)}(x)\rho(x)dx.
\end{equation}
В \cite{mmg-SharapudinovUMN,mmg-SharapudinovIzvRan2019,mmg-MMG2019,mmg-Gadzhimirzaev2019} исследованы вопросы поточечной и равномерной сходимости рядов Фурье по системам, ортогональным относительно \eqref{mmg-sob-prod} и ассоциированным с такими классическими системами, как система полиномов Якоби, Лагерра, система функций Хаара, Уолша, Лагерра и система косинусов. При некоторых значениях параметров, фигурирующих в определениях классических систем, изучены аппроксимативные свойства рядов Фурье по указанным системам.

Приведем теперь краткое описание вопросов теории систем, ортогональных по Соболеву, исследованных в отчетном году. Для этого введем некоторые обозначения.
Через $W^r_{L^p_\rho}=W^r_{L^p_\rho}[a,b]$ обозначим пространство Соболева, состоящее из $r-1$ непрерывно дифференцируемых на $[a,b]$ функций $f$, таких что $f^{(r-1)}$ абсолютно непрерывна и $f^{(r)} \in L^p_\rho[a,b]$, где $L^p_\rho=L^p_\rho[a,b]$ -- весовое пространство Лебега: $L^p_\rho[a,b] = \{ f: \int_a^b |f(x)|^p\rho(x)dx \}$. При $p=2$ в пространстве $W^r_{L^p_\rho}$ можно ввести скалярное произведение \ref{mmg-sob-prod}.

Рассмотрим случай, когда $\rho(x)=\rho(\alpha,\beta; x)=(1-x)^\alpha(1+x)^\beta$. Пусть $\mathcal{P}^{\alpha,\beta}=\{ \hat{P}_n^{\alpha,\beta} \}_{n=0}^\infty$ --- система полиномов, ортонормированных на отрезке $[-1,1]$ с весом Якоби $\rho(\alpha,\beta; x)$ (система полиномов Якоби). Введем в рассмотрение новую систему полиномов $\mathcal{P}^{\alpha,\beta}_r$, $r \ge 1$, с помощью равенств:
\begin{gather}
	\label{mmg-sob-def1-intro}
	P_{r,k}^{\alpha,\beta}(x) =\frac{(x+1)^k}{k!}, \quad k=0,1,\ldots, r-1,\\
	\label{mmg-sob-def2-intro}
	P_{r,k}^{\alpha,\beta}(x) =\frac{1}{(r-1)!}\int\limits_{-1}^x(x-t)^{r-1}\hat{P}_{k-r}^{\alpha,\beta}(t)dt, \quad k=r,r+1,\ldots.
\end{gather}
Можно показать, что таким образом определенная система будет ортонормирована относительно \eqref{mmg-sob-prod} при $\rho(x)=\rho(\alpha,\beta; x)$ \cite[с.~231]{mmg-Shii-izvran2018}. Систему $\mathcal{P}_r^{\alpha,\beta}$ будем называть системой полиномов, ортогональной в смысле Соболева и ассоциированной с полиномами Якоби $P_n^{\alpha,\beta}$.
Ряд Фурье функции $f \in W^r_{L^2_{\rho(\alpha,\beta)}}[-1,1]$ по системе $\mathcal{P}_r^{\alpha,\beta}$ и частичная сумма этого ряда имеют следующий вид \cite[с.~227]{mmg-Shii-izvran2018}:
\begin{gather}
	\label{mmg-sob-fourier-series-intro}
	f(x) \sim \sum_{k=0}^{r-1} f^{(k)}(-1)\frac{(x+1)^k}{k!}+ \sum_{k=r}^\infty c^{\alpha,\beta}_{r,k}(f) P_{r,k}^{\alpha,\beta}(x),\\
	\label{mmg-sob-part-sum-intro}
	S^{\alpha,\beta}_{r,n}(f,x) = \sum_{k=0}^{r-1} f^{(k)}(-1)\frac{(x+1)^k}{k!}+ \sum_{k=r}^n c^{\alpha,\beta}_{r,k}(f) P_{r,k}^{\alpha,\beta}(x), \quad n \ge r,
\end{gather}
где $c^{\alpha,\beta}_{r,k}(f)=\int_{-1}^1 f^{(r)}(t)\hat{P}_{k-r}^{\alpha,\beta}(t)\rho(\alpha,\beta; t)dt$.

Одно из замечательных свойств соболевского ряда Фурье \eqref{mmg-sob-fourier-series-intro} состоит в том, что его частичная сумма \eqref{mmg-sob-part-sum-intro} при $n \ge r$ совпадает $r$-кратно с исходной функцией $f(x)$ в точке $x=-1$ \cite[с. 228]{mmg-Shii-izvran2018}:
\begin{equation*}
	(S^{\alpha,\beta}_{r,n})^{(\nu)}(f,-1)=f^{(\nu)}(-1), \quad 0\le\nu\le r-1.
\end{equation*}
Это очень важное свойство, которое в сочетании с хорошими аппроксимативными свойствами сумм Фурье \eqref{mmg-sob-part-sum-intro}
делает их весьма эффективным инструментом приближённого решения краевых задач для обыкновенных дифференциальных уравнений спектральными методами. При решении таких задач важную роль играют вопросы сходимости и скорости сходимости частичных сумм к искомому решению.
В разделе \ref{MMG} исследована равномерная сходимость рядов Фурье по системам полиномов, ортогональных в смысле Соболева и порожденных системами полиномов Якоби с показателями $\alpha, \beta$ для функций из $W^r_{L^1_\rho(\alpha,\beta)}$, $-1 <\alpha, \beta  \le 0$. Также для рядов Фурье по соболевской системе полиномов, порожденной полиномами Якоби с показателями $\alpha, \beta  > -1$, получены необходимые и достаточные условия сходимости в пространстве $W^r_{L^p_\rho(A,B)}$, $p > 1$, $A, B \in \mathbb{R}$. Кроме того, при дополнительном условии на $A, B$ и $p$ показано, что указанные ряды сходятся равномерно на отрезке $[-1,1]$.
В пункте \ref{Ram-Jac} рассмотрены ряды Фурье по системе $\{P_n^{\alpha-r,-r}(x)\}$, которая является частным случаем соболевской системы $\{P_{r,n}^{\alpha,\beta}(x)\}$. Исследована задача об отклонении от функции $f$ из пространства $W^r=\{f\in W^r_{L^2_\rho(\alpha,0)}: |f^{(r)}|\le1\}$ частичных сумм ряда Фурье по системе полиномов $\{P_n^{\alpha-r,-r}(x)\}$. В частности, исследовано поведение соответствующей функции типа Лебега и получены оценки в терминах модуля непрерывности $r$-ой производной функции $f$.

Перейдем к полиномам, ортогональным по Соболеву и порожденным дискретными ортогональными полиномами. В литературе можно встретить различные подходы к построению систем полиномов, ортогональных по Соболеву, отличающиеся выбором тех или иных скалярных произведений. Приведем некоторые виды скалярных произведений, связанные с полиномами Мейкснера. Например, в \cite{Ram-Ar-Go-Mar,Ram-Kh-Old} рассмотрено скалярное произведение Соболева следующего вида
$$
\langle f,g\rangle_S=\sum_{x=0}^{\infty}f(x)g(x)w(x)+\lambda\sum_{x=0}^{\infty}\Delta f(x)\Delta g(x)w(x),
$$
где $\lambda\ge 0$, $\Delta f(x)=f(x+1)-f(x)$, $w(x)$ -- вес Мейкснера. А в \cite{Ram-Bav1,Ram-Bav2} были рассмотрены частные случаи этого скалярного произведения, а именно, в \cite{Ram-Bav1} вместо второй суммы было рассмотрено одно слагаемое $\lambda f(0)g(0)$, в \cite{Ram-Bav2} -- два слагаемых $Mf(0)g(0)+N\Delta f(0)\Delta g(0)$, $M,N\ge 0$. При этом было показано, что полиномы $\{Q_n(x)\}$, ортогональные относительно этих скалярных произведений, можно определить посредством равенства $Q_n(x)=\sum_{k=0}^{n}c_{k,n}M_k^\alpha(x)$, где $M_k^\alpha(x)$ -- полином Мейкснера степени $k$. Далее, в \cite{Ram-Shar-VMJ,Ram-Shar-Sar} было рассмотрено скалярное произведение следующего вида
\begin{equation}\label{Intro-Ram-Sob-inner}
\langle f,g\rangle_S=\sum_{k=0}^{r-1}\Delta^kf(0)\Delta^kg(0)+\sum_{x=0}^\infty\Delta^rf(x)\Delta^rg(x)w(x)
\end{equation}
и показано, что полиномы, ортонормированные относительно \eqref{Intro-Ram-Sob-inner}, можно определить посредством равенств
$$
m_{r,n}^{\alpha}(x)=\frac{x^{[n]}}{n!},\ n=\overline{0,r-1},
$$
$$
m_{r,n}^{\alpha}(x)=
\frac{1}{\sqrt{h_{n-r}^\alpha}(r-1)!}\sum_{t=0}^{x-r}(x-1-t)^{[r-1]}M_{n-r}^\alpha(t),\ x\ge r,\ n\ge r,
$$
где $x^{[n]}=x(x-1)\cdots(x-n+1)$.
В дальнейшем нам понадобятся некоторые обозначения. Пусть $1\le p<\infty$, $l_w^p(\Omega)$ -- пространство дискретных функций $f$, заданных на сетке $\Omega=\{0, 1, \ldots\}$ и для которых $\|f\|_{l_{w}^p(\Omega)}^p=\sum_{x\in\Omega}|f(x)|^pw(x)<\infty$, а $W^r_{l_{w}^p(\Omega)}$ -- подпространство в $l_{w}^p(\Omega)$. При $p=2$ в $W^r_{l_{w}^p(\Omega)}$ можно определить скалярное произведение \eqref{Intro-Ram-Sob-inner}.
В \cite{Ram-Shar-Sar} были исследованы различные алгебраические свойства системы полиномов $\{m_{r,n}^{\alpha}(x)\}$. В частности, было показано, что она полна и ортонормирована в $W^r_{l_w^2(\Omega)}$.
Другие виды скалярных произведений Соболева, связанные с полиномами Мейкснера, можно найти в \cite{Ram-Mor-Bal,Ram-Co-So-Vil}.
Результаты, полученные в вышеприведенных работах \cite{Ram-Shar-VMJ,Ram-Ar-Go-Mar,Ram-Kh-Old,Ram-Bav1,Ram-Bav2,Ram-Shar-Sar,Ram-Mor-Bal,Ram-Co-So-Vil}, в основном связаны с исследованием распределения нулей полиномов Мейкснера --  Соболева, изучением их алгебраических, асимптотических и дифференциальных свойств. В то же время остаются мало изученными вопросы сходимости ряда Фурье по полиномам Мейкснера -- Соболева и аппроксимативные свойства его частичных сумм. В связи с этим в отчетном году была рассмотрена система полиномов $\{m_{n,N}^{\alpha,r}(x)\}$, ортонормированная по Соболеву и порожденная системой модифицированных полиномов Мейкснера $\{m_{n,N}^{\alpha}(x)\}$. В разделе \ref{Ram-Mex} исследован вопрос о сходимости к $f\in W^r_{l^p_{\rho_N}(\Omega_\delta)}$ ряда Фурье по системе $\{m_{n,N}^{\alpha,r}(x)\}$. Кроме того, исследованы аппроксимативные свойства частичных сумм ряда Фурье по системе $\{m_{n,N}^{0,r}(x)\}$.

В отчетном году были исследованы вопросы, связанные с приложением рациональных сплайнов к решению интегральных уравнений Вольтерры и Фредгольма. О важной роли теории этих уравнений в различных разделах математики и ее приложениях свидетельствует широта охватываемых этой теорией проблем и большое число посвященных ей научных работ. Для решения интегральных уравнений используются, в частности, численные методы с применением полиномиальных сплайнов. Однако специфика уравнений Вольтерры не позволяет получить полное их решение методами исследований общих интегральных уравнений и востребованы новые методы приближенного решения уравнений Вольтерры.
Нами построены гладкие сплайн-функции на базе трехточечных рациональных
интерполянтов, которые (в отличие от классических полиномиальных сплайнов)
сами и их производные первого и второго
порядков обладают свойством безусловной сходимости
на соответствующем классе $C^r[a,b]$ непрерывно дифференцируемых $r$ раз
($r=0,1,2$) на отрезке $[a,b]$ функций.
Это позволяет применить такие рациональные сплайн-функции для приближенного
решения интегральных уравнений Вольтерры и Фредгольма.
При этом решение представляет собой дважды гладкую на рассматриваемом отрезке
функцию.
В отчетном году разработан новый метод построения приближенного решения в виде коллокационной рациональной сплайн-функции
интегральных уравнений Вольтерры (см. раздел \ref{ARK}). Следует также отметить, что структура применяемых рациональных сплайн-функций
позволяет получить сравнительно более простые алгоритмы поиска решения. Получены также оценки скорости сходимости приближенных решений к точному решению.

Продолжены исследования вопроса базисности системы полиномов Лежандра в пространствах Лебега с переменным показателем $L^{p(\cdot)}$. Интерес к таким пространствам возрос с 1990-х годов ввиду их использования в различных приложениях (математическое моделирование электрореологических жидкостей, обработка зашумленных изображений и др.). Эти задачи, в свою очередь, приводят к поиску систем, образующих базисы в указанных пространствах. В работах И.И. Шарапудинова и его учеников \cite{tad-SHII-Haar,tad-SHII-AnalisysMath,tad-SHII-Leg,tad-MMG-Haar,tad-SHII-Jacob,tad-SHII-Ult,tad-RAM-Jacob} была показана базисность в $L^{p(\cdot)}$  тригонометрической системы, системы полиномов Якоби и системы функций Хаара при определенных условиях на показатель $p(x)$.
Нами была предпринята попытка ослабить эти условия на переменный показатель $p(x)$ (постоянство показателя на концах отрезка $[-1,1]$) в случае системы полиномов Лежандра. При исследовании этой задачи возникла необходимость изучения свойств ядра $K(x,y)$, связанного с ядром Кристоффеля -- Дарбу для системы полиномов Лежандра, и некоторых операторов с этим ядром. В отчетном году для $K(x,y)$ получены представления и оценки, в зависимости от расположения $x,y$ на квадрате $[-1,1]\times[-1,1]$, и доказаны некоторые утверждения, которые позволят показать ограниченность упомянутых операторов в пространстве Лебега $L^{p(\cdot)}$ (см. раздел \ref{ShTN}).

Другим направлением исследований, проводившихся в отчетном году, были исследования вопросов усреднения уравнения Бельтрами и обобщенного уравнения Бельтами с локально периодическими коэффициентами. Теория усреднения имеет многочисленные приложения в различных областях физики и механики сплошных сред (см. монографию \cite{smm-1} и литературу в  ней). Операторные оценки погрешности классических задач усреднения для дивергентных уравнений хорошо изучены Жиковым В. В., Бирманом   М. Ш., Суслиной  Т. А. и их учениками (см. \cite{smm-2,smm-3}).
Операторным оценкам погрешности усреднения  эллиптических операторов второго порядка дивергентного вида  с локально-периодическими коэффициентами посвящены работы \cite{smm-4,smm-5,smm-6,smm-7}.
В работе \cite{smm-8} получены оценки погрешности классического усреднения порядка  $O(\sqrt\varepsilon)$ задачи Римана -- Гильберта  для уравнения Бельтрами с периодическим коэффициентом. Оценки погрешности усреднения порядка $O(\varepsilon)$ периодической задачи для таких уравнений даются в работе \cite{smm-9}.  Оценки усреднения  порядка $O(\sqrt\varepsilon)$  задачи Римана -- Гильберта для уравнения Бельтрами с локально-периодическим коэффициентом получены  в \cite{smm-10}. В отчетном году нами изучено уравнение Бельтрами с локально периодическим коэффициентом $\mu(x,\varepsilon^{-1} x)$. Получены оценки
погрешности усреднения периодической задачи в пространствах Соболева и Лебега (см. раздел \ref{SMM}).


На основе $W$-метода Н.В. Азбелева были исследованы вопросы устойчивости линейных непрерывно-дискретных (гибридных)
стохастических уравнений с запаздываниями. Непрерывно-дискретные системы уравнений, образующие важный подкласс так называемых <<гибридных систем>>, характеризуются наличием двух составляющих в пространстве состояний, а именно компонент с непрерывным и дискретным временем. Интуитивно это означает, что динамика одной из компонент является чисто непрерывной, тогда как другая подвергается дополнительным воздействиям в определённые моменты времени. Такие системы естественным образом возникают в приложениях, например, в теории управления, экологических и биологических моделях. Поведение многих реальных процессов в природе и технике определяется состоянием не только в текущий, но и в предшествующие моменты времени. Примерами могут служить динамические системы, управляемые
на значительном расстоянии, а также системы с транспортным запаздыванием. Исследования показывают, что поведение решений систем
без учета запаздывания, даже при малой его величине, может существенно отличаться от поведения решений тех же систем с
запаздывающим аргументом. Это обстоятельство подчеркивает необходимость и принципиальную важность изучения таких систем. С
другой стороны, одним из основных условий физической реализуемости процесса является его устойчивость. Поэтому изучение непрерывно-дискретных
систем и их приложений естественным образом приводит к необходимости создания соответствующего направления в теории устойчивости.

Наконец, учет стохастических эффектов --- важная часть любого реалистичного подхода к моделированию. Например, в популяционной
динамике демографическая и экологическая стохастичность возникают из-за изменения во времени факторов, внешних по отношению к системе,
но влияющих на выживание популяции, а в теории управления случайные коэффициенты могут моделировать, например, неточности при
измерениях. Поэтому изучение гибридных стохастических систем привлекает последнее время внимание многих специалистов.

Исследования устойчивости систем со случайными параметрами часто проводятся методом функционалов Ляпунова-Красовского-Разумихина.
Однако применение прямого метода Ляпунова и его стохастических аналогов для функционально-дифференциальных уравнений во
многих случаях встречает серьезные трудности. В частности, эффективные признаки устойчивости обычно удаётся доказывать этими
методами лишь для сравнительно простых классов уравнений.
Использование функционалов Ляпунова-Красовского-Разумихина для изучения задачи устойчивости линейных непрерывно-дискретных (гибридных)
стохастических уравнений с запаздываниями не эффективно для получения конкретных признаков. В связи с этим для получения признаков устойчивости применен метод регуляризации, также известный как метод модельных (вспомогательных) уравнений или <<$W$-метод Н.В. Азбелева>>. Этот подход ранее доказал
свою эффективность как в теории стохастических дифференциальных уравнений, так и при исследовании стохастических разностных
уравнений. Полученные в отчетном году результаты отражены в разделе \ref{KRI}.

В 2022 г. продолжена работа над проблемами интервальной раскрашиваемости и вычисления димерных чисел. Задача об интервальной рёберной раскраске графа (и.р.), NP-полнота которой доказана Асратян А. С., Камалян Р. Р. в 1987 г. -- это задача о присвоении рёбрам графа целочисленных меток таким образом, чтобы в каждой вершине $v$ метки всех инцидентных вершине v рёбер были попарно различны и заполняли некоторый целочисленный интервал. С привлечением компьютерных вычислений К.Giaro в 1999 г. показал существование и.р. для всех биграфов порядка не более 14, в 2015 г. H. Khachatrian и T. Mamikonyan объявили, что  утверждение об и.р. справедливо для биграфов порядка 15. Доказательство аналогичного утверждения опубликовано Магомедовым А.М. (сотрудником ОМИ) в 2017 г. (в соавторстве) для биграфов $(X,Y)$ порядка 16 за исключением, быть может, двух случаев: $|X|=7, |Y|=9$  и $|X|=|Y|=8$. В текущем году усилен алгоритм проверки интервальной раскрашиваемости (см. раздел \ref{AKM}).

Далее, все известные формулы решения задачи вычисления димерных чисел используют операции с плавающей запятой, что требует значительных компьютерных ресурсов, к тому же чревато проблемами округления; исключение составляет лишь полученная в известной монографии Д. Кнута и др. для случая $m=3$ система из двух взаимно-рекуррентных формул (в.р.ф.), использующая лишь операции сложения целых чисел.
Магомедову А.М. принадлежит алгоритм генерации системы в.р.ф. для равносильной задачи -- подсчета разбиений на плитки $1\times2$ прямоугольника произвольных целочисленных размеров $m\times n$, из которой методом исключения определяется рекуррентное соотношение для $a_n$. На основе\textit{ метода первой светлой клетки}, автором которого является Магомедов А.М., удалось найти простое доказательство принципиального факта о существовании такого соотношения (в литературе удается найти лишь ссылки на то, что это удается сделать <<на основе символьной модификации метода матрицы переноса>>).
В текущем году при $m=3$ и $4$ удалось получить искомую формулу для $a_n$ без использования системы (в.р.ф.) и показать, что формула не единственна (см. раздел \ref{AKM}). 

В ОМИ продолжены исследования в области математического моделирования сложных систем. Исследована задача обращения $V$-преобразования, или обобщенного преобразования Радона на плоскости. $V$-образные ломаные, по которым берутся интегралы, имеют вершину внутри области восстановления. Полученная формула обобщает известные формулы на случай весовой степенной функции. Также исследована задача обращения интегрального преобразования на семействе ломаных, входящих в круг нормально к окружности и преломляющихся внутри круга (см. раздел \ref{MZG}). Результаты могут найти применение в областях науки и техники, связанных с неразрушающей реконструкцией: биомедицинская визуализация, национальная безопасность, гамма-астрономия и др. На основе точных формул обращения могут быть составлены алгоритмы численного восстановления.

Проведены исследования модели Поттса с числом состояний $q=3$ на решетке Кагоме с учетом ферромагнитного обменного взаимодействия между ближайшими
соседями и антиферромагнитного обменного взаимодействия между следующими ближайшими соседями. Исследованы фазовые переходы двумерных структур, описываемых моделями Поттса с $q=4$ и $q=5$ на гексагональной решетке. 
Для трехмерных структур, описываемых слабо разбавленной моделью Поттса с $q=5$ показано, что внесение слабого немагнитного беспорядка в
чистую систему не приводит к смене фазового перехода первого рода на фазовый переход второго рода (см. разделы \ref{MMA}, \ref{RMK}).





