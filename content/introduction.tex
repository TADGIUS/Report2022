\Introduction

В последние годы теории ортогональных по Соболеву полиномов посвящено большое число работ. В частности, это связано с тем, что соболевские скалярные произведения и соответствующие им ортогональные системы (и их дифференциальные аналоги) играют важную роль во многих проблемах теории функций, квантовой механики, математической физики, вычислительной математики и т.д. В частности, ряды Фурье по ним обладают важными для приложений свойствами, которые отсутствуют у рядов Фурье по классическим ортогональным системам (см., например, \cite{Ram-Ba-Ra-Pe,Ram-Mar-Xu,Ram-Shar-UMN}).
Например, ряды Фурье по полиномам, ортогональным по Соболеву, представляется более естественным аппаратом, чем ряды Фурье по классическим ортогональным полиномам, для приближенного решения краевых задач, в которых требуется анализ поведения приближенного решения в одной или нескольких точках.
При решении таких задач важную роль играют сходимость и скорость сходимости рядов Фурье. В этом параграфе эти вопросы будут рассмотрены для систем полиномов, ортогональных по Соболeву и порожденных полиномами Якоби (см. п. \ref{Ram-Jac}) и Мейкснера (см. п. \ref{Ram-Mex}).

