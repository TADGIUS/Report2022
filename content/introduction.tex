\Introduction

В последние годы теории ортогональных по Соболеву полиномов посвящено большое число работ. В частности, это связано с тем, что соболевские скалярные произведения и соответствующие им ортогональные системы (и их дифференциальные аналоги) играют важную роль во многих проблемах теории функций, квантовой механики, математической физики, вычислительной математики и т.д. В частности, ряды Фурье по ним обладают важными для приложений свойствами, которые отсутствуют у рядов Фурье по классическим ортогональным системам (см., например, \cite{Ram-Ba-Ra-Pe,Ram-Mar-Xu,Ram-Shar-UMN}).
Например, ряды Фурье по полиномам, ортогональным по Соболеву, представляется более естественным аппаратом, чем ряды Фурье по классическим ортогональным полиномам, для приближенного решения краевых задач, в которых требуется анализ поведения приближенного решения в одной или нескольких точках.
При решении таких задач важную роль играют сходимость и скорость сходимости рядов Фурье. В отчетном году эти вопросы были рассмотрены для систем полиномов, ортогональных по Соболeву и порожденных полиномами Якоби (см. п. \ref{Ram-Jac}) и Мейкснера (см. п. \ref{Ram-Mex}).

Обозначим через $W^r_{L^p_\rho}=W^r_{L^p_\rho}[a,b]$ пространство Соболева, состоящее из $r-1$ непрерывно дифференцируемых на $[a,b]$ функций $f$, таких что $f^{(r-1)}$ абсолютно непрерывна и $f^{(r)} \in L^p_\rho[a,b]$, где $L^p_\rho=L^p_\rho[a,b]$ --- весовое пространство Лебега: $L^p_\rho[a,b] = \{ f: \int_a^b |f(x)|^p\rho(x)dx \}$. При $p=2$ в пространстве $W^r_{L^p_\rho}$ можно ввести скалярное произведение:
\begin{equation}\label{mmg-sob-prod}
	\langle f, g \rangle = \sum_{k=0}^{r-1}f^{(k)}(a)g^{(k)}(a)+\int_a^b f^{(r)}(x)g^{(r)}(x)\rho(x)dx.
\end{equation}
Указанное скалярное произведение принято называть дискретно-непрерывным скалярным произведением типа Соболева.

Норму в пространстве $W^r_{L^p_\rho}$ определим следующим образом:
\begin{equation}\label{mmg-sob-norm-def}
	\|f\|_{W^r_{L^p_\rho}} = \Bigl[
	\sum_{k=0}^{r-1}|f^{(k)}(a)|^p + \int_a^b |f^{(r)}(t)|^p \rho(t)dt
	\Bigr]^{1/p}.
\end{equation}

Скалярное произведение типа Соболева характеризуется тем, что оно включает в себя производные перемножаемых функций. Достаточно общий вид скалярного произведения типа Соболева может быть выражен формулой
\begin{equation}\label{mmg-inner-prod-sob-common}
	\langle f,g \rangle = \sum_{k=0}^{m}\int\limits_{\mathbb{R}}f^{(k)}(x)g^{(k)}(x)d\mu_k,
\end{equation}
где $d\mu_k$ --- борелевские меры. Подробный обзор результатов, полученных для систем полиномов, ортогональных относительно различных соболевских скалярных произведений вида \eqref{mmg-inner-prod-sob-common}, можно найти в обзорной работе \cite{mmg-MarcellanXu2015}.

Как правило, рассматриваются следующие виды соболевских скалярных произведений \eqref{mmg-inner-prod-sob-common}:
\begin{itemize}
	\item
	непрерывные (все меры $\mu_k$, $k \ge 0$, абсолютно непрерывны),
	\item
	дискретные ($\mu_0$ абсолютна непрерывна, а $\mu_k$, $k \ge 1$, --- дискретны),
	\item
	дискретно-непрерывные ($\mu_m$ абсолютна непрерывна, а $\mu_k$, $k < m$, --- дискретны).	
\end{itemize}

Исследованиям вопросов сходимости рядов Фурье по полиномам, ортогональным относительно непрерывного скалярного произведения, посвящены, например, работы \cite{mmg-MarcellanJacobiSobolev,mmg-CiaurriJacobiSobolev,mmg-CiaurriCoherentPairs,mmg-Fejzullahu2010,mmg-Fejzullahu2013}. В \cite{mmg-MarcellanJacobiSobolev,mmg-CiaurriJacobiSobolev} рассматривается случай, когда $d\mu_k(x)=w_{\alpha+k,\beta+k}(x)dx$, где $w_{\alpha+k,\beta+k}(x)=(1-x)^{\alpha+k}(1+x)^{\beta+k}$ --- веса Якоби. В работе \cite{mmg-CiaurriCoherentPairs} исследуется случай $m=1$, причем меры $\mu_0$ и $\mu_1$ образуют так называемую когерентную пару (см. \cite{mmg-IserlesKoch1991,mmg-MarcellanXu2015}). В \cite{mmg-Fejzullahu2010} получены необходимые условия сходимости по норме ряда Фурье для полиномов, ортогональных относительно \eqref{mmg-inner-prod-sob-common} при $m=1$ и $d\mu_0(x)=d\mu_1(x)=(1-x^2)^{\alpha-1/2}dx$.

В случае дискретных скалярных произведений сходимость соответствующих рядов Фурье исследовалась в работах \cite{mmg-Marcellan2002,mmg-Rocha2003,mmg-OsilenkerFourier2012,mmg-OsilenkerLinearMethods2015,mmg-Fejzullahu2009,mmg-CiaurriSigma2018}. В работах \cite{mmg-Marcellan2002,mmg-Rocha2003} рассмотрены вопросы поточечной и равномерной сходимости рядов Фурье в случае скалярного произведения вида:
\begin{equation*}
	\langle f,g \rangle = \int_{-1}^{1}f(x)g(x)d\mu(x)+
	\sum_{k=1}^K\sum_{i=0}^{N_k} M_{k,i}f^{(i)}(a_k)g^{(i)}(a_k),
\end{equation*}
где $d\mu(x)$ --- мера Якоби. В \cite{mmg-OsilenkerFourier2012,mmg-OsilenkerLinearMethods2015,mmg-Fejzullahu2009,mmg-CiaurriSigma2018} рассмотрен частный случай приведенного выше скалярного произведения ($K=2$, $N_1=N_2=1$, $a_1=-1$, $a_2=1$) и исследована сходимость рядов Фурье и их линейных средних по соответствующим ортогональным полиномам.

Ряды Фурье в дискретно-непрерывном случае довольно подробно исследовались в работах Шарапудинова И.И. (см. \cite{mmg-SharapudinovUMN,mmg-SharapudinovIzvRan2019} и приведенные там списки литературы). В них рассматривается скалярное произведение вида \eqref{mmg-sob-prod}.
В \cite{mmg-SharapudinovUMN,mmg-SharapudinovIzvRan2019,mmg-MMG2019,mmg-Gadzhimirzaev2019} исследованы вопросы поточечной и равномерной сходимости рядов Фурье по системам, ортогональным относительно \eqref{mmg-sob-prod} и ассоциированным с такими классическими системами, как система полиномов Якоби, Лагерра, система функций Хаара, Уолша, Лагерра и система косинусов. При некоторых значениях параметров, фигурирующих в определениях классических систем, изучены аппроксимативные свойства рядов Фурье по указанным системам.

Приведем ниже краткое описание вопросов теории систем, ортогональных по Соболеву, исследованных в отчетном году.

Рассмотрена задача об отклонении от функции $f$ из пространства $W^r$ частичных сумм ряда Фурье по системе полиномов Якоби $\{P_n^{\alpha-r,-r}(x)\}$, ортогональной относительно скалярного произведения типа Соболева. Исследовано поведение функции типа Лебега частичных сумм ряда Фурье по системе $\{P_n^{\alpha-r,-r}(x)\}$. Получены оценки в терминах модуля непрерывности $r$ - ой производной функции $f$.

Исследована задача о сходимости ряда Фурье по системе полиномов $\{m_{n,N}^{\alpha,r}(x)\}$, ортонормированной по Соболеву и порожденной системой модифицированных полиномов Мейкснера. В частности, показано, что ряд Фурье по этой системе сходится к $f\in W^r_{l^p_{\rho_N}(\Omega_\delta)}$ поточечно на сетке $\Omega_\delta$ при $p\ge2$. Кроме того, исследованы аппроксимативные свойства частичных сумм ряда Фурье по системе $\{m_{n,N}^{0,r}(x)\}$. Получены оценки для соответствующей функции Лебега.

%Для функций из $W^r_{L^1_\rho(\alpha,\beta)}$, $-1 <\alpha, \beta  \le 0$, исследована равномерная сходимость рядов Фурье по системам полиномов, ортогональных в смысле Соболева и порожденных системами полиномов Якоби с показателями $\alpha, \beta$.

Исследована равномерная сходимость рядов Фурье по системам полиномов, ортогональных в смысле Соболева и порожденных системами полиномов Якоби с показателями $\alpha, \beta$ для функций из $W^r_{L^1_\rho(\alpha,\beta)}$, $-1 <\alpha, \beta  \le 0$.

%Получены необходимые и достаточные условия сходимости в пространстве $W^r_{L^p_\rho(A,B)}$, $p > 1$, $A, B \in \mathbb{R}$ рядов Фурье по соболевской системе полиномов, порожденной полиномами Якоби с показателями $\alpha, \beta  > -1$. Показано также, что при дополнительном условии на $A, B$ и $p$ указанные ряды сходятся равномерно на отрезке $[-1,1]$.

Для рядов Фурье по соболевской системе полиномов, порожденной полиномами Якоби с показателями $\alpha, \beta  > -1$, были получены необходимые и достаточные условия сходимости в пространстве $W^r_{L^p_\rho(A,B)}$, $p > 1$, $A, B \in \mathbb{R}$. Кроме того, при дополнительном условии на $A, B$ и $p$ было показано, что  указанные ряды сходятся равномерно на отрезке $[-1,1]$.


Другим направлением иследований в отчетном году является приложение рациональных сплайнов в решении интегральных уравнений Вольтерры и Фредгольма. О важной роли теории этих уравнений в различных разделах математики и ее приложениях свидетельствует широта охватываемых этой теорией проблем и большое число посвященных ей научных работ. Разработаны также численные методы решения интегральных уравнений, в частности, с использованием, в основном, полиномиальных сплайнов. Однако специфика уравнений Вольтерры не позволяет получить полное их решение методами исследований общих интегральных уравнений и востребованы новые методы приближенного решения уравнений Вольтерры.

Нами построены гладкие сплайн-функции на базе трехточечных рациональных
интерполянтов, которые (в отличие от классических полиномиальных сплайнов)
сами и их производные первого и второго
порядков обладают свойством безусловной сходимости
на соответствующем классе $C^r[a,b]$ непрерывно дифференцируемых $r$ раз
($r=0,1,2$) на отрезке $[a,b]$ функций.

Это позволяет применить такие рациональные сплайн-функции для приближенного
решения интегральных уравнений Вольтерры и Фредгольма.
При этом решение представляет собой дважды гладкую на рассматриваемом отрезке
функцию.

Разработан новый метод построения
приближенного решения в виде коллокационной рациональной сплайн-функции
интегральных уравнений Вольтерры.
Следует также отметить, что структура применяемых рациональных сплайн-функций
позволяет получить сравнительно более простые алгоритмы поиска решения.
Даны также
оценки скорости сходимости приближенных решений к точному решению.