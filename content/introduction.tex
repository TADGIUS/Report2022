\Introduction

%Соболя

В последние годы теории ортогональных по Соболеву полиномов посвящено большое число работ. В частности, это связано с тем, что соболевские скалярные произведения и соответствующие им ортогональные системы (и их дифференциальные аналоги) играют важную роль во многих проблемах теории функций, квантовой механики, математической физики, вычислительной математики и т.д. В частности, ряды Фурье по ним обладают важными для приложений свойствами, которые отсутствуют у рядов Фурье по классическим ортогональным системам (см., например, \cite{Ram-Ba-Ra-Pe,Ram-Mar-Xu,Ram-Shar-UMN}).
Например, ряды Фурье по полиномам, ортогональным по Соболеву, представляется более естественным аппаратом, чем ряды Фурье по классическим ортогональным полиномам, для приближенного решения краевых задач, в которых требуется анализ поведения приближенного решения в одной или нескольких точках.
При решении таких задач важную роль играют сходимость и скорость сходимости рядов Фурье. В отчетном году эти вопросы были рассмотрены для систем полиномов, ортогональных по Соболeву и порожденных полиномами Якоби (см. п. \ref{Ram-Jac}) и Мейкснера (см. п. \ref{Ram-Mex}).

Обозначим через $W^r_{L^p_\rho}=W^r_{L^p_\rho}[a,b]$ пространство Соболева, состоящее из $r-1$ непрерывно дифференцируемых на $[a,b]$ функций $f$, таких что $f^{(r-1)}$ абсолютно непрерывна и $f^{(r)} \in L^p_\rho[a,b]$, где $L^p_\rho=L^p_\rho[a,b]$ -- весовое пространство Лебега: $L^p_\rho[a,b] = \{ f: \int_a^b |f(x)|^p\rho(x)dx \}$. При $p=2$ в пространстве $W^r_{L^p_\rho}$ можно ввести скалярное произведение:
\begin{equation}\label{mmg-sob-prod}
	\langle f, g \rangle = \sum_{k=0}^{r-1}f^{(k)}(a)g^{(k)}(a)+\int_a^b f^{(r)}(x)g^{(r)}(x)\rho(x)dx.
\end{equation}
Указанное скалярное произведение принято называть дискретно-непрерывным скалярным произведением типа Соболева.

Норму в пространстве $W^r_{L^p_\rho}$ определим следующим образом:
\begin{equation}\label{mmg-sob-norm-def}
	\|f\|_{W^r_{L^p_\rho}} = \Bigl[
	\sum_{k=0}^{r-1}|f^{(k)}(a)|^p + \int_a^b |f^{(r)}(t)|^p \rho(t)dt
	\Bigr]^{1/p}.
\end{equation}

Скалярное произведение типа Соболева характеризуется тем, что оно включает в себя производные перемножаемых функций. Достаточно общий вид скалярного произведения типа Соболева может быть выражен формулой
\begin{equation}\label{mmg-inner-prod-sob-common}
	\langle f,g \rangle = \sum_{k=0}^{m}\int\limits_{\mathbb{R}}f^{(k)}(x)g^{(k)}(x)d\mu_k,
\end{equation}
где $d\mu_k$ -- борелевские меры. Подробный обзор результатов, полученных для систем полиномов, ортогональных относительно различных соболевских скалярных произведений вида \eqref{mmg-inner-prod-sob-common}, можно найти в обзорной работе \cite{mmg-MarcellanXu2015}.

Как правило, рассматриваются следующие виды соболевских скалярных произведений \eqref{mmg-inner-prod-sob-common}:
\begin{itemize}
	\item
	непрерывные (все меры $\mu_k$, $k \ge 0$, абсолютно непрерывны),
	\item
	дискретные ($\mu_0$ абсолютна непрерывна, а $\mu_k$, $k \ge 1$, -- дискретны),
	\item
	дискретно-непрерывные ($\mu_m$ абсолютна непрерывна, а $\mu_k$, $k < m$, -- дискретны).	
\end{itemize}

Исследованиям вопросов сходимости рядов Фурье по полиномам, ортогональным относительно непрерывного скалярного произведения, посвящены, например, работы \cite{mmg-MarcellanJacobiSobolev,mmg-CiaurriJacobiSobolev,mmg-CiaurriCoherentPairs,mmg-Fejzullahu2010,mmg-Fejzullahu2013}. В \cite{mmg-MarcellanJacobiSobolev,mmg-CiaurriJacobiSobolev} рассматривается случай, когда $d\mu_k(x)=w_{\alpha+k,\beta+k}(x)dx$, где $w_{\alpha+k,\beta+k}(x)=(1-x)^{\alpha+k}(1+x)^{\beta+k}$ -- веса Якоби. В работе \cite{mmg-CiaurriCoherentPairs} исследуется случай $m=1$, причем меры $\mu_0$ и $\mu_1$ образуют так называемую когерентную пару (см. \cite{mmg-IserlesKoch1991,mmg-MarcellanXu2015}). В \cite{mmg-Fejzullahu2010} получены необходимые условия сходимости по норме ряда Фурье для полиномов, ортогональных относительно \eqref{mmg-inner-prod-sob-common} при $m=1$ и $d\mu_0(x)=d\mu_1(x)=(1-x^2)^{\alpha-1/2}dx$.

В случае дискретных скалярных произведений сходимость соответствующих рядов Фурье исследовалась в работах \cite{mmg-Marcellan2002,mmg-Rocha2003,mmg-OsilenkerFourier2012,mmg-OsilenkerLinearMethods2015,mmg-Fejzullahu2009,mmg-CiaurriSigma2018}. В работах \cite{mmg-Marcellan2002,mmg-Rocha2003} рассмотрены вопросы поточечной и равномерной сходимости рядов Фурье в случае скалярного произведения вида:
\begin{equation*}
	\langle f,g \rangle = \int_{-1}^{1}f(x)g(x)d\mu(x)+
	\sum_{k=1}^K\sum_{i=0}^{N_k} M_{k,i}f^{(i)}(a_k)g^{(i)}(a_k),
\end{equation*}
где $d\mu(x)$ -- мера Якоби. В \cite{mmg-OsilenkerFourier2012,mmg-OsilenkerLinearMethods2015,mmg-Fejzullahu2009,mmg-CiaurriSigma2018} рассмотрен частный случай приведенного выше скалярного произведения ($K=2$, $N_1=N_2=1$, $a_1=-1$, $a_2=1$) и исследована сходимость рядов Фурье и их линейных средних по соответствующим ортогональным полиномам.

Ряды Фурье в дискретно-непрерывном случае довольно подробно исследовались в работах Шарапудинова И.И. (см. \cite{mmg-SharapudinovUMN,mmg-SharapudinovIzvRan2019} и приведенные там списки литературы). В них рассматривается скалярное произведение вида \eqref{mmg-sob-prod}.
В \cite{mmg-SharapudinovUMN,mmg-SharapudinovIzvRan2019,mmg-MMG2019,mmg-Gadzhimirzaev2019} исследованы вопросы поточечной и равномерной сходимости рядов Фурье по системам, ортогональным относительно \eqref{mmg-sob-prod} и ассоциированным с такими классическими системами, как система полиномов Якоби, Лагерра, система функций Хаара, Уолша, Лагерра и система косинусов. При некоторых значениях параметров, фигурирующих в определениях классических систем, изучены аппроксимативные свойства рядов Фурье по указанным системам.

Приведем ниже краткое описание вопросов теории систем, ортогональных по Соболеву, исследованных в отчетном году.

Рассмотрена задача об отклонении от функции $f$ из пространства $W^r$ частичных сумм ряда Фурье по системе полиномов Якоби $\{P_n^{\alpha-r,-r}(x)\}$, ортогональной относительно скалярного произведения типа Соболева. Исследовано поведение функции типа Лебега частичных сумм ряда Фурье по системе $\{P_n^{\alpha-r,-r}(x)\}$. Получены оценки в терминах модуля непрерывности $r$ - ой производной функции $f$.

Исследована задача о сходимости ряда Фурье по системе полиномов $\{m_{n,N}^{\alpha,r}(x)\}$, ортонормированной по Соболеву и порожденной системой модифицированных полиномов Мейкснера. В частности, показано, что ряд Фурье по этой системе сходится к $f\in W^r_{l^p_{\rho_N}(\Omega_\delta)}$ поточечно на сетке $\Omega_\delta$ при $p\ge2$. Кроме того, исследованы аппроксимативные свойства частичных сумм ряда Фурье по системе $\{m_{n,N}^{0,r}(x)\}$. Получены оценки для соответствующей функции Лебега.

%Для функций из $W^r_{L^1_\rho(\alpha,\beta)}$, $-1 <\alpha, \beta  \le 0$, исследована равномерная сходимость рядов Фурье по системам полиномов, ортогональных в смысле Соболева и порожденных системами полиномов Якоби с показателями $\alpha, \beta$.

Исследована равномерная сходимость рядов Фурье по системам полиномов, ортогональных в смысле Соболева и порожденных системами полиномов Якоби с показателями $\alpha, \beta$ для функций из $W^r_{L^1_\rho(\alpha,\beta)}$, $-1 <\alpha, \beta  \le 0$.

%Получены необходимые и достаточные условия сходимости в пространстве $W^r_{L^p_\rho(A,B)}$, $p > 1$, $A, B \in \mathbb{R}$ рядов Фурье по соболевской системе полиномов, порожденной полиномами Якоби с показателями $\alpha, \beta  > -1$. Показано также, что при дополнительном условии на $A, B$ и $p$ указанные ряды сходятся равномерно на отрезке $[-1,1]$.

Для рядов Фурье по соболевской системе полиномов, порожденной полиномами Якоби с показателями $\alpha, \beta  > -1$, были получены необходимые и достаточные условия сходимости в пространстве $W^r_{L^p_\rho(A,B)}$, $p > 1$, $A, B \in \mathbb{R}$. Кроме того, при дополнительном условии на $A, B$ и $p$ было показано, что  указанные ряды сходятся равномерно на отрезке $[-1,1]$.

%АРК

Другим направлением иследований в отчетном году является приложение рациональных сплайнов в решении интегральных уравнений Вольтерры и Фредгольма. О важной роли теории этих уравнений в различных разделах математики и ее приложениях свидетельствует широта охватываемых этой теорией проблем и большое число посвященных ей научных работ. Разработаны также численные методы решения интегральных уравнений, в частности, с использованием, в основном, полиномиальных сплайнов. Однако специфика уравнений Вольтерры не позволяет получить полное их решение методами исследований общих интегральных уравнений и востребованы новые методы приближенного решения уравнений Вольтерры.

Нами построены гладкие сплайн-функции на базе трехточечных рациональных
интерполянтов, которые (в отличие от классических полиномиальных сплайнов)
сами и их производные первого и второго
порядков обладают свойством безусловной сходимости
на соответствующем классе $C^r[a,b]$ непрерывно дифференцируемых $r$ раз
($r=0,1,2$) на отрезке $[a,b]$ функций.

Это позволяет применить такие рациональные сплайн-функции для приближенного
решения интегральных уравнений Вольтерры и Фредгольма.
При этом решение представляет собой дважды гладкую на рассматриваемом отрезке
функцию.

Разработан новый метод построения
приближенного решения в виде коллокационной рациональной сплайн-функции
интегральных уравнений Вольтерры.
Следует также отметить, что структура применяемых рациональных сплайн-функций
позволяет получить сравнительно более простые алгоритмы поиска решения.
Даны также оценки скорости сходимости приближенных решений к точному решению.

%ШЭТН

Интерес к пространствам Лебега с переменным показателем $L^{p(\cdot)}$ возрос с 1990-х годов ввиду
их использования в различных приложениях. Прежде всего, это математические
моделирование электрореологических жидкостей. Эти пространства также использовались для моделирования поведения других физических явлений, а также для изучения процессов обработки изображений и т. д. Эти задачи, в свою очередь, приводят к поиску систем, образующих базисы в этих пространствах. В работах И.И. Шарапудинова и его учеников \cite{tad-SHII-Haar,tad-SHII-AnalisysMath,tad-SHII-Leg,tad-MMG-Haar,tad-SHII-Jacob,tad-SHII-Ult,tad-RAM-Jacob} была показана базисность в $L^{p(\cdot)}$  тригонометрической системы, системы полиномов Якоби и системы функций Хаара при определенных условиях на показатель $p(x)$.

В отчетном году была предпринята попытка ослабить эти условия на переменный показатель $p(x)$ (постоянство показателя на концах отрезка $[-1,1]$) в случае системы полиномов Лежандра. При исследовании этой задачи возникла необходимость изучения свойств ядра $K(x,y)$, связанного с ядром Кристоффеля -- Дарбу для системы полиномов Лежандра. А именно для $K(x,y)$ получены  представления и оценки, в зависимости от расположения $x,y$ на квадрате $[-1,1]\times[-1,1]$.




%КРИ
Непрерывно-дискретные системы уравнений, образующие важный подкласс так называемых <<гибридных систем>>, характеризуются наличием двух составляющих в пространстве состояний, а именно компонент с непрерывным и дискретным временем. Интуитивно
это означает, что динамика одной из компонент является чисто
непрерывной, тогда как другая подвергается дополнительным
воздействиям в определённые моменты времени. Такие системы
естественным образом возникают в приложениях, например, в теории
управления, экологических   и биологических моделях.

Поведение многих реальных процессов в природе и технике определяется
состоянием не только в текущий, но и в предшествующие моменты
времени. Примерами могут служить динамические системы, управляемые
на значительном расстоянии, а также системы с транспортным
запаздыванием. Исследования показывают, что поведение решений систем
без учета запаздывания, даже при малой его величине, может
существенно отличаться от поведения решений тех же систем с
запаздывающим аргументом. Это обстоятельство подчеркивает
необходимость и принципиальную важность изучения таких систем. С
другой стороны, одним из основных условий физической реализуемости
процесса является его устойчивость. Поэтому изучение непрерывно-дискретных
систем и их приложений естественным образом приводит к необходимости
создания соответствующего направления в теории устойчивости.

Наконец, учет стохастических эффектов - важная часть любого
реалистичного подхода к моделированию. Например, в популяционной
динамике демографическая и экологическая стохастичность возникают
из-за изменения во времени факторов, внешних по отношению к системе,
но влияющих на выживание популяции, а в теории управления случайные
коэффициенты могут моделировать, например, неточности при
измерениях. Поэтому изучение гибридных стохастических систем
привлекает последнее время внимание многих специалистов.

Исследования устойчивости систем со случайными параметрами часто
проводятся методом функционалов Ляпунова-Красовского-Разумихина.
Однако применение прямого метода Ляпунова и его стохастических
аналогов для функцио\-нально-диф\-фе\-рен\-ци\-аль\-ных уравнений во
многих случаях встречает серьёзные трудности. В частности,
эффективные признаки устойчивости обычно удаётся доказывать этими
методами лишь для сравнительно простых классов уравнений.

Изучена задача, которая ранее, по-видимому, не исследовалась --
задача устойчивости линейных непрерывно-дискретных (гибридных)
стохастических уравнений с запаздываниями. Использование
функционалов Ляпунова-Красовского-Разумихина для изучение этой
задачи не эффективень  для получения конкретных признаков. Учитывая
это для получения признаков устойчивости применен метод
регуляризации, также известном как метод модельных (вспомогательных)
уравнений или <<$W$-метод Н.В. Азбелева>>. Этот подход ранее доказал
свою эффективность как в теории стохастических дифференциальных
уравнений, так и при исследовании стохастических разностных
уравнений. Суть метода заключается в том, что вместо функционала на
пространстве траекторий решений используется <<модельное>>
уравнение, которое обладает заданным свойством устойчивости и
которое используется для регуляризации исходного уравнения. Проверка
устойчивости последнего состоит в оценке нормы определённого
интегрального оператора или проверки положительной обратимости
некоторой матрицы. Последняя версия $W$-метода была разработана в
публикациях.

Непрерывно-дискретные системы уравнений, образующие важный подкласс так называемых <<гибридных систем>>, характеризуются наличием двух составляющих в пространстве состояний, а именно компонент с непрерывным и дискретным временем. Интуитивно
это означает, что динамика одной из компонент является чисто
непрерывной, тогда как другая подвергается дополнительным
воздействиям в определённые моменты времени. Такие системы
естественным образом возникают в приложениях, например, в теории
управления, экологических   и биологических моделях.

Поведение многих реальных процессов в природе и технике определяется
состоянием не только в текущий, но и в предшествующие моменты
времени. Примерами могут служить динамические системы, управляемые
на значительном расстоянии, а также системы с транспортным
запаздыванием. Исследования показывают, что поведение решений систем
без учета запаздывания, даже при малой его величине, может
существенно отличаться от поведения решений тех же систем с
запаздывающим аргументом. Это обстоятельство подчеркивает
необходимость и принципиальную важность изучения таких систем. С
другой стороны, одним из основных условий физической реализуемости
процесса является его устойчивость. Поэтому изучение непрерывно-дискретных
систем и их приложений естественным образом приводит к необходимости
создания соответствующего направления в теории устойчивости.

Наконец, учет стохастических эффектов - важная часть любого
реалистичного подхода к моделированию. Например, в популяционной
динамике демографическая и экологическая стохастичность возникают
из-за изменения во времени факторов, внешних по отношению к системе,
но влияющих на выживание популяции, а в теории управления случайные
коэффициенты могут моделировать, например, неточности при
измерениях. Поэтому изучение гибридных стохастических систем
привлекает последнее время внимание многих специалистов.

Исследования устойчивости систем со случайными параметрами часто
проводятся методом функционалов Ляпунова-Красовского-Разумихина.
Однако применение прямого метода Ляпунова и его стохастических
аналогов для функцио\-нально-диф\-фе\-рен\-ци\-аль\-ных уравнений во
многих случаях встречает серьёзные трудности. В частности,
эффективные признаки устойчивости обычно удаётся доказывать этими
методами лишь для сравнительно простых классов уравнений.

Изучена задача, которая ранее, по-видимому, не исследовалась --
задача устойчивости линейных непрерывно-дискретных (гибридных)
стохастических уравнений с запаздываниями. Использование
функционалов Ляпунова-Красовского-Разумихина для изучение этой
задачи не эффективен  для получения конкретных признаков. Учитывая
это для получения признаков устойчивости применен метод
регуляризации, также известном как метод модельных (вспомогательных)
уравнений или <<$W$-метод Н.В. Азбелева>>. Этот подход ранее доказал
свою эффективность как в теории стохастических дифференциальных
уравнений, так и при исследовании стохастических разностных
уравнений. Суть метода заключается в том, что вместо функционала на
пространстве траекторий решений используется <<модельное>>
уравнение, которое обладает заданным свойством устойчивости и
которое используется для регуляризации исходного уравнения. Проверка
устойчивости последнего состоит в оценке нормы определённого
интегрального оператора или проверки положительной обратимости
некоторой матрицы. Последняя версия $W$-метода была разработана в
публикациях.

%Сиражудинов

Теория усреднения имеет многочисленные приложения в различных областях физики и механики сплошных сред (см. монографию \cite{smm-1} и литературу в  ней). Операторные оценки погрешности классических задач усреднения для дивергентных уравнений хорошо изучены Жиковым В. В., Бирманом   М. Ш., Суслиной  Т. А. и их учениками (см. \cite{smm-2,smm-3}).

Операторным оценкам погрешности усреднения  эллиптических операторов второго порядка дивергентного вида  с локально-периодическими коэффициентами посвящены работы \cite{smm-4,smm-5,smm-6,smm-7}.
В этой работе получены оценки усреднения периодической задачи  для уравнения Бельтрами с локально-периодическим коэффициентом:
$$
	A_\varepsilon w_\varepsilon\equiv\partial_{\bar{z}}u_\varepsilon+\mu(x,\varepsilon^{-1}x)\,\partial_z w_\varepsilon
=f\in L_2(\square),\quad w_\varepsilon\in W_2^1(\square),
$$
где $\varepsilon>0$ -- малый параметр, $\square$ -- ячейка периодов со стороной 1.
комплекснозначная измеримая периодическая (периода 1 по всем переменным   функция, удовлетворяющая условию эллиптичности
 $\mathop{vrai\, sup}\limits_{(x,y)\in\square\times\square}|\mu(x,y)|\leqslant k_0,\quad k_0<1$  -- постоянная.
 Более того, $\mu(x,y)$  как функция $x$ равномерно непрерывна по Липшицу
$$
|\mu(x^\prime,y)-\mu(x,y)|\leqslant L|x^\prime-x|,\quad x^\prime,\ x\in\square,  \text{ \ п.в.\ }   y\in\square,
$$
где $L>0$ -- постоянная.

Это уравнение -- недивергентное, поэтому полученные оценки (см. Теорему \ref{smm-th1.8}) отличаются от оценок для дивергентных операторов.
В работе \cite{smm-8} получены оценки погрешности классического усреднения (порядка  $O(\sqrt\varepsilon)$ задачи Римана -- Гильберта  для уравнения Бельтрами с периодическим коэффициентом. Оценки погрешности усреднения (порядка $O(\varepsilon)$) периодической задачи для таких уравнений даются в работе \cite{smm-9}.  Оценки усреднения  порядка $O(\sqrt\varepsilon)$  задачи Римана -- Гильберта для уравнения Бельтрами с локально-периодическим коэффициентом получены  в \cite{smm-10}.

%Меджидов

В 1997 году Р. Баско в работе \cite{mzg-Basko} представил преобразование Радона для пары лучей, образующих букву V, в попытке смоделировать формирование изображения в так называемой одномерной камере Комптона. Ось этой $V$-образной линии поворачивается вокруг точки плоскости таким образом, что ее вершина лежит на прямой (представляющей детектор <<рассеяния>>), а угол раскрытия двух лучей (представляющий угол рассеяния Комптона) является переменной данных изображения.
Трунг Т. и Нгуен М. К. (\cite{mzg-Truong}) выдвинули идею $V$-образных преобразований Радона с фиксированным направлением оси симметрии. Такие преобразования могут представлять теоретический интерес в интегральной геометрии. Они возникают в результате связанного томографического процесса пропускания-отражения. Интегральные преобразования по преломленным лучам изучаются и во многих других работах. Из недавних работ можно отметить статьи \cite{mzg-Ambartsoumian2,mzg-Kuchment} и др.

В различных статьях упоминается, что в зависимости от конструкции детектора в криволинейном интеграле могут появляться различные веса. Новизна результатов нашей работы состоит в том, что в интегралах участвуют веса, равные некоторым степеням расстояния до вершины уголка ломаной.

Пусть в круге $S_R$ радиуса $R$ с центром в начале координат дано семейство ломаных вида
$$\Gamma(\beta,d)=L_d\cup L, \, 0\leq\beta\leq 2\pi, \, 0\leq d\leq R,$$
где
$$L_d=\{(x_1, x_2)=(R-s)\omega,  0\leq s\leq d\}, \omega=(\cos\beta, \sin \beta),$$
$$L=\{(x_1, x_2)=d\omega-s(\cos (\beta+\theta), \sin (\beta+\theta)), s\geq 0.$$
Здесь $\theta$ - острый угол.

Задача состоит в том, чтобы по заданным интегралам

\begin{equation}
\label{mzg-eq-one3}
Vf(\beta,d)=\int_{\Gamma(\beta,d)}f(x)ds
\end{equation}
функции $f(x)$ вдоль ломаных $\Gamma(\beta,d)$ определить эту функцию.

Легко заметить, в общем случае семейство ломаных $\Gamma(\beta,d)$ имеет размерность четыре. Можно, однако, ограничить данные 2D-набором, предположив, что лучи входят в круг нормально к его границе и ломаются под фиксированным углом $\theta$.

Г. Амбарцумян и С. Мун \cite{mzg-Ambartsoumian1} решили поставленную задачу методом разложений в ряды Фурье (методом Кормака).

В статье \cite{mzg-Medzhidov2} обнозначность восстановления неизвестной функции доказана, когда угол $\beta$ меняется в ограниченном угловом диапазоне: $0<\beta<\alpha_0$ и $|\pi-\beta|<\alpha_0$, где $\alpha_0$ - произвольный острый угол. При этих условиях доказана формула обращения.

В упомянутой выше работе \cite{mzg-Truong} рассматривается следующее двухпараметрическое семейство ломаных на плоскости

\begin{equation}
\label{mzg-eq-one4}
\Gamma(\xi,\varphi)=\{(x,y)=(\xi\pm r \sin \varphi, r \cos \varphi),\quad 0\leq\varphi\leq\frac{\pi}{2}, \quad \xi\in\mathbb R\},
\end{equation}

Интегральное преобразование на этом семействе ($V$-преобразование Радона) имеет вид:

$$g(\xi,\varphi)=\int_0^\infty f(\xi\pm r \sin \varphi, r \cos \varphi)dr.$$

В этой работе получена формула обращения

$$f(x,y)=\frac{1}{2\pi^2}\int_0^\infty \left( (p.v.)\int_\mathbb R d\xi\left(\frac{g'(\xi,\lambda)}{\xi-x-y\lambda}+\frac{g'(\xi,\lambda)}{\xi-x+y\lambda}\right)\right), \lambda=tg \varphi$$
(внутренний интеграл понимается в смысле главного значения).

Мы рассмотриваем аналогичное семейство ломаных в круге $S_R$:

$$L(\beta,\psi)=L_1(\beta,\psi)\cup L_2(\beta,\psi), 0\leq \beta\leq2\pi,\quad 0<\psi<\pi/2,$$
где
$$L_i(\beta,\psi)=\{x\in S_R:(x,\tau^{(i)})=R \sin \psi\},$$
$$\tau^{(1)}=(\sin(\psi-\beta), \cos(\psi-\beta)),\quad \tau^{(2)}=(\sin(\psi+\beta), -\cos(\psi+\beta)).$$

Пусть
$$g(\beta,\psi)=\int_{L(\beta,\psi)}f(x)ds.$$

Положим
$$h(\beta,r)=g\left(\beta, \arcsin \frac{r}{R}\right).$$

Пусть $f_n(\rho), \quad h_n(r)$ - коэффициенты разложений в ряды Фурье функций $f(\varphi,\rho)$ и $h(\beta,r)$, записанных в полярных координатах.
Эти коэффициенты связаны интегральным уравнением (\cite{mzg-Medzhidov1})
$$h_n(r)=4T_n\left(\frac{r}{R}\right)\int_r^{+\infty}f_n(\rho)\frac{T_n\left(\frac{r}{\rho}\right)}{\sqrt{\rho^2-r^2}}\rho d\rho,$$
где $T_n(t)=\cos (n \arccos t)$ -- полиномы Чебышева первого рода.

Решение этого интегрального уравнения имеет вид

$$f_n(\rho)=-\frac{1}{2\pi}\frac{d}{d\rho}\int_\rho^{+\infty}\frac{\rho}{r}\cdot\frac{T_n\left(\frac{r}{\rho}\right)}{T_n\left(\frac{r}{R}\right)}\frac{h_n(r)}{\sqrt{r^2-\rho^2}}dr.$$

Рассмотрим на семействе $\Gamma(\xi,\varphi)$ ломаных \eqref{mzg-eq-one4} и интегральное преобразование вида

\begin{equation}
\label{mzg-eq-one5}
Vf(\xi,\varphi)=\int_0^\infty\left( f(\xi+ r \sin \varphi, r \cos \varphi)+f(\xi-r \sin \varphi, r \cos \varphi)\right)r^k dr,
\end{equation}
заданное на ломаных этого семейства. Здесь $k$ -- неотрицательное вещественное число.

Пусть неизвестная функция $f$ имеет компактный носитель, принадлежащий верхней полуплоскости.

Мы решаем задачу восстановления функции $f$ по заданной функции $g(\xi,\varphi)=Vf(\xi,\varphi)$ для значений переменных $\xi$ и $\varphi$, $\xi\in\mathbb R,\, 0<\varphi<\frac{\pi}{2}$.

Похожая задача с другой весовой функцией решена в \cite{mzg-Kuchment}.

Формула обращения, полученная в работе \cite{mzg-Medzhidov2}, имеет вид

\begin{equation}
\label{mzg-eq-one8}
f(x,y)=\frac{1}{2\pi}\frac{1}{|x|^k y^k}\int_0^\infty \frac{\cos (\tau xy) }{\left(\sqrt{1+\tau^2}\right)^{k+1}}(-L)^{\frac{k+1}{2}}g(x,\tau)d\tau;
\end{equation}
здесь $Lh(\xi,\tau)=\frac{\partial^2}{\partial\xi^2 }h(\xi,\tau)$, а дробная степень оператора Лапласа $L$ определяется формулой

$$(-L)^lh(\xi,\tau)=I^{-2l}h(\xi,\tau),$$
где $I^\alpha$ -- потенциал Рисса:

$$\widetilde{\left(I^\alpha h\right)}(p,\tau)=|p|^{2l}\tilde h(p,\tau).$$




%АКМ
1. Задача об интервальной рёберной раскраске графа (и.р.), NP-полнота которой доказана Асратян А. С., Камалян Р. Р. в 1987 г. -- это задача о присвоении рёбрам графа целочисленных меток таким образом, чтобы в каждой вершине $v$ метки всех инцидентных вершине v рёбер были попарно различны и заполняли некоторый целочисленный интервал. С привлечением компьютерных вычислений К.Giaro в 1999 г. показал существование и.р. для всех биграфов порядка не более 14, в 2015 г. H. Khachatrian и T. Mamikonyan объявили, что  утверждение об и.р. справедливо для биграфов порядка 15. Доказательство аналогичного утверждения опубликовано автором в 2017 г. (в соавторстве) для биграфов $(X,Y)$ порядка 16 за исключением, быть может, двух случаев: $|X|=7, |Y|=9$  и $|X|=|Y|=8$.

В текущем году усилен алгоритм проверки интервальной раскрашиваемости. Получены два свидетельства о регистрации Программы ЭВМ в госреестре.

2. Известны классы химических соединений, которые синтезируются только тогда, когда графы соединений в топологической модели молекулы имеют совершенное паросочетание; более того, стабильность компонентов этих семейств зависит от количества совершенных паросочетаний в их графах.

Еще более актуальной задача вычисления димерных чисел оказалась в физике; вычисление числа способов объединения атомов в двухатомные молекулы (димеры) с соблюдением некоторых условий и привела известных ученых-физиков Kateleyn P. W., Temperley H. N. V. и Fisher M. E. к знаменитой <<формуле двойного произведения>>. Значительный вклад внесли в разработку проблемы Stanley R. P., Klarner D., Pollack J., Read Ronald C. Но все известные формулы решения задачи использовали операции с плавающей запятой, что требует значительных компьютерных ресурсов, к тому же чревато проблемами округления; исключение составляет лишь полученная в известной монографии Д. Кнута и др. для случая $m=3$ система из двух взаимно-рекуррентных формул (в.р.ф.), использующая лишь операции сложения целых чисел.

Магомедову А.М., в.н.с. ОМИ, принадлежит алгоритм генерации системы в.р.ф. для равносильной задачи -- подсчета разбиений на плитки $1\times2$ прямоугольника произвольных целочисленных размеров $m\times n$, из которой методом исключения определяется рекуррентное соотношение для $a_n$. На основе авторского \textit{ метода первой светлой клетки} удалось найти простое доказательство принципиального факта о существовании такого соотношения (в литературе удается найти лишь ссылки на то, что это удается сделать <<на основе символьной модификации метода матрицы переноса>>).

В текущем году при $m=3$ и $4$ удалось получить искомую формулу для $a_n$ без использования системы (в.р.ф.) и показать, что формула не единственна. Примечательно, что прежде компьютерным путем были сгенерированы разные формулы для $a_n$, после чего их строгое доказательство не представило труда.

В текущем году получено свидетельство о регистрации в гос.реестре Программы для ЭВМ, предназначенной для вычисления коэффициентов прямой рекурсии для $a_n$.

%Физики

В последние годы значительное внимание уделяется исследованию низкоразмерных
спиновых систем. Одним из таких интересных объектов является магнитная система на решетке
Кагоме. После обнаружения в 2011 году в реальном природном материале Гербертсметите,
имеющего структуру решетки Кагоме, перехода в квантовую спиновую жидкость, интерес к
данным системам возрос многократно \cite{mma-bib-2}. Кроме того, для моделей Поттса не имеется ни одного
точного решения до сегодняшнего дня. Изучение магнитных и тепловых свойств этих моделей на
различных двумерных решетках имеет важное фундаментальное и прикладное значение.
Первоначально при изучении фазовых переходов второго рода предполагалось, что
рассматриваемые системы являются идеально однородными. В реальных образцах, однако, всегда
присутствуют какие-либо дефекты, примеси или другие несовершенства, нарушающие
однородность системы. Кроме того, современная микроэлектроника достигла такого уровня
миниатюризации, что влиянием этих дефектов на поведение приборов и элементов
микроэлектроники невозможно пренебречь. Поэтому, проблема влияния вмороженных дефектов
структуры на критическое поведение является важной как с теоретической, так и с практической
точек зрения. Изучение различных физических свойства этих объектов открывает широкие
перспективы для экспериментальных приложений.
Особый интерес представляют исследования, связанные с влиянием внешнего магнитного
поля на спиновые системы с конкурирующими обменными взаимодействиями. При исследовании
ФП, термодинамических свойств и магнитных структур основного состояния таких систем нами
получены результаты, советующие мировому уровню. Эти результаты не уступают данным других
зарубежных исследователей по точности.
При выполнении проекта на данном этапе предполагалось:

--- разработка, тестирование и отладка программного обеспечения, основанное на
использовании репличного обменного алгоритма и алгоритма Ванга-Ландау метода МК для
исследования моделей спиновых систем с конкурирующими обменными взаимодействиями
и фрустрациями;

--- исследование влияния внешнего магнитного поля на ФП, магнитные и термодинамические
свойства моделей сложных спиновых систем с конкурирующими обменными
взаимодействиями и фрустрациями;

--- построение фазовых диаграмм в зависимости от величины обменных взаимодействий вторых
соседей, а также от величины внешнего магнитного поля.

В последние годы синтезировано большое количество материалов с кристаллической
структурой решетки Кагоме. Среди которых отметим Делафосситы, Капелласиты и Фольбортиты.
Данные материалы хорошо описываются моделью Поттса с различным числом состояний ($q=3$, $q=4$ и $q=5$). В отчетном году нами проведены исследования модели Поттса с числом состояний $q=3$
на решетке Кагоме с учетом ферромагнитного обменного взаимодействия между ближайшими
соседями и антиферромагнитного обменного взаимодействия между следующими ближайшими
соседями. Исследованы фазовые переходы двумерных структур, описываемых моделями Поттса с
$q=4$ и $q=5$ на гексагональной решетке. Для трехмерных структур, описываемых слабо
разбавленной моделью Поттса с $q=5$ показано, что внесение слабого немагнитного беспорядка в
чистую систему не приводит к смене фазового перехода первого рода на фазовый переход второго
рода.





