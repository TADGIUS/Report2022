\chapter{ПРИЛОЖЕНИЕ А. Cписок работ, опубликованных \texorpdfstring{\\ }{} по теме НИР в [ГОД] г.}

% Ниже представлены примеры опубликованных работ

\section*{Список опубликованных научных статей}

\begin{enumerate}[1]
    % \item
    % Ramazanov A.-R.K., Magomedova V.G. Approximate Solution of Nonlinear Differential Equations with the Help of Rational Spline Functions // Computational Mathematics and Mathematical Physics. --- 2021. --- Vol. 61. No. 8. P.~1252---1259.

    % \item
    % Sultanakhmedov, M.S. Approximation of Functions by Discrete Fourier Sums in Polynomials Orthogonal on a Nonuniform Grid with Jacobi Weight. // Math Notes. --- 2021. --- Vol. 110. P.~418---431.

    % \item
    % Gadzhimirzaev, R.M., Shakh-Emirov, T.N. Approximation Properties of the Vallée-Poussin Means of Partial Sums of a Special Series in Laguerre Polynomials. // Math Notes. --- 2021. --- Vol. 110. P.~475---488.


    % MMA
    
    \item
    Ramazanov M.K., Murtazaev A.K., Magomedov M.A.
    Phase transitions in the frustrated Potts model in the magnetic field
    //
    Physica E: Low-dimensional Systems and Nanostructures.
    --- 2022.
    --- V. 140.
    --- P. 115226-1-115226-6. DOI: 10.1016/j.physe.2022.115226.
    
    \item
    Рамазанов М.К., Муртазаев А.К., Магомедов М.А., Курбанова Д.Р., Рамазанов К.М., Хизриев М.С.
    Энергетический анализ магнитных структур основного состояния модели Поттса
    //
    Дагестанские электронные математические известия.
    --- 2022.
    --- Вып. 17.
    --- С. 44---52.
    
    \item
    Магомедов М.А., Муртазаев А.К., Исаева М.М.
    Фазовая диаграмма и структура основного состояния трехвершинной модели Поттса на решетке Кагоме
    //
    Дагестанские электронные математические известия.
    --- 2022.
    --- Вып.17.
    --- С. 53---66.
    
    \item
    Рамазанов М.К., Муртазаев А.К., Магомедов М.А.
    Фрустрированная модель Поттса с числом состояний спина $q = 4$ в магнитном поле
    //
    ЖЭТФ.
    --- 2022.
    --- Т. 161, вып. 6.
    --- С. 816---824. DOI: 10.31857/S0044451022060049.
    
    \item
    Рамазанов М.К., Муртазаев А.К., Магомедов М.А., Мазагаева М.К., Джамалудинов М.Р.
    Исследование влияния слабых магнитных полей на термодинамические свойства модели Поттса с числом состояний спина $q = 4$ на гексагональной решетке
    //
    Физика твердого тела.
    --- 2022.
    --- Т. 64, вып. 2.
    --- С. 237---240. DOI: 10.21883/FTT.2022.02.51935.226


    % KRI
    
    \item
    Кадиев Р.И., Поносов А.В.
    Глобальная устойчивость систем нелинейных дифференциальных уравнений Ито с последействием и W-метод Н.В. Азбелева
    //
    Изв. Вузов. Матем.
    --- 2022.
    --- №1.
    --- С. 38---56.
    
    \item
    Кадиев Р.И., Поносов А.В.
    Исследование устойчивости решений непрерывно-дискретных стохастических систем с последействием методом регуляризации
    //
    Дифф. Урав.
    --- 2022.
    --- Т. 58, № 4.
    --- С. 435---455
    
    \item
    Kadiev R., Ponosov A.
    Positive invertibility of matrices and exponential stability of linear stochastic systems with delay
    //
    International Journal of Differential Equations.
    --- 2022.
    --- V. 2022.
    Article ID 5549693, 13 pages.
    
    \item
    Кадиев Р.И., Шахбанова З.И.
    Экспоненциальная устойчивость решений одной непрерывно-дискретной линейной системы Ито с ограниченными запаздываниями
    //
    Вестник ДГУ, Серия 1. Естественные науки.
    --- 2022.
    --- Т. 37, вып. 3.
    --- С. 7---17.


    % GRM

    \item
    Гаджимирзаев Р.М.
    Аппроксимативные свойства средних типа Валле-Пуссена частичных сумм ряда Фурье по полиномам Лагерра – Соболева
    //
    Сиб. Матем. Журн.
    --- 2022.
    --- Т. 63, № 3.
    --- С. 545---561. 
    
    \item
    Гаджимирзаев Р.М.
    Об аппроксимативных свойствах рядов Фурье по полиномам Якоби $Pn \alpha - r$, $-r(x)$, ортогональным по Соболеву
    //
    Матем. Заметки.
    --- 2022.
    --- Т. 111, № 6.
    --- С. 803---818.


    % MMG

    \item
    Магомед-Касумов М.Г., Шах-Эмиров Т.Н.
    О представлении соболевских систем, ортогональных относительно скалярного произведения с одной дискретной точкой
    //
    Матем. Заметки.
    --- 2022.
    --- Т. 111, № 4.
    --- С. 561---570. 
    Англоязычная версия:
    Mathematical Notes. 
    --- 2022.
    --- V. 111, \No 4.
    --- P. 561---570.
    
    \item
    Magomed-Kasumov M.G.
    Existence and uniqueness theorems for a differential equation with a discontinuous right-hand side.
    //
    Vladikavkaz Mathematical Journal.
    --- 2022.
    --- V. 24, iss. 1.
    --- P. 54---64.


    % ARK
    
    \item
    Рамазанов А.-Р.К., Рамазанов А.К., Магомедова В.Г.
    О динамическом решении интегрального уравнения Вольтерры в виде рациональных сплайн-функций
    //
    Матем. Заметки.
    --- 2022.
    --- Т. 111, № 4.
    --- С. 581---591. 
    
    \item
    Рамазанов А.-Р.К., Алиева Р.Ш.
    О гладкой интерполяции локальными полиномиальными сплайнами
    //
    Вестник ДГУ. Серия1. Естественные науки.
    --- 2022.
    --- Т. 37, вып. 1.
    --- С. 32---39.
    
    \item
    Рамазанов А.-Р.К., Магомедова В.Г.
    О приближенном решении интегральных уравнений Фредгольма методом коллокационных рациональных сплайн-функций
    //
    Дагестанские Электронные Математические Известия.
    --- 2022.
    --- № 17.
    --- С. 20---31.


    % SMM
    
    \item
    Сиражудинов М.М., Джамалудинова. С.П.
    Оценки локально-периодического усреднения задачи Римана – Гильберта для обобщённого уравнения Бельтрами
    //
    Дифф. Урав.
    --- 2022.
    --- Т. 58, № 6.
    --- С. 777---794. 
    
    \item
    Сиражудинов М.М., Ибрагимов М.Г., Магомедова М.Г.
    Оценки погрешности локально-периодического усреднения периодической задачи для уравнения Бельтрами
    //
    Вестник ДГУ. Серия1: Естественные науки
    --- 2022.
    --- № 1.
    --- С. 24---31.


    % MZG
    
    \item
    Меджидов З.Г.
    Обращение V-преобразования Радона со степенным весом на плоскости
    //
    Даг. Элект. Матем. Известия.
    --- 2022.
    --- № 17.
    --- С. 32---43.


    % AKM
    
    \item
    Магомедов А.М., Раджабова Н.Ш.
    Замощение клетчатой полосы шириной 4.
    //
    Информатика в школе.
    --- 2022.
    --- \No 1.
    --- P. 81---84. 
    
    \item
    Магомедов А.М., Лавренченко С.А.
    Некоторые свойства прямых рекуррентных соотношений для последовательностей димерных чисел
    //
    Вестник ДГУ. Серия 1. Естественные науки.
    --- 2022.
    --- Т. 37, вып. №1.
    --- С. 51---62.
    
    \item
    Магомедов А.М.
    "Компьютерное" решение и обобщение классической арифметической задачи
    //
    Вестник ДГУ. Серия 1. Естественные науки.
    --- 2022.
    --- Т. 37, вып. №3.
    --- С. 25---29.
    
    \item
    Магомедов А.М., Якубов Р.А.
    Некоторые подходы к определению четности числа разбиений прямоугольной полосы
    //
    Вестник ДГУ. Серия 1. Естественные науки.
    --- 2022.
    --- Т. 37, вып. №3.
    --- С. 30---33.


    % BAB
    
    \item
    Муртазаев А.К., Бабаев А.Б.
    Фазовые переходы в двумерных моделях Поттса на гексагональной решетке
    //
    Журнал экспериментальной и теоретической физики.
    --- 2022.
    --- Т. 161, вып. 6.
    --- С. 847---852.
    
    \item
    Муртазаев А.К., Бабаев А.Б.
    Компьютерное моделирование фазовых переходов в трехмерных слабо разбавленных спиновых системах
    //
    Поверхность. Рентгеновские, синхротронные и нейтронные исследования.
    --- 2022.
    --- №5.
    --- С. 37---41. 

\end{enumerate}

\section*{Список зарегистрированных программ для ЭВМ}

\begin{enumerate}[1]
    \item
    Магомедов А.М., Шарапудинов Т.И. Свидетельство №2021666365 о государственной регистрации программы для ЭВМ «Программа перечисления биграфов с переменной численностью вершин в каждой доле». Заявка №2021665799, дата поступления 13 октября 2021 г. Дата государственной регистрации в Реестре программ для ЭВМ 13 октября 2021. Правообладатель: ДФИЦ РАН.
    
    \item
    Султанахмедов М.С.~Свидетельство №2021682136 о государственной регистрации программы для ЭВМ «Программа для рекуррентного вычисления значений полиномов, ортогональных по Соболеву». Заявка № 2021682107, дата поступления 30 декабря 2021 г. Дата государственной регистрации в Реестре программ для ЭВМ 30 декабря 2021. Правообладатель: ДФИЦ РАН.
\end{enumerate}
