
  \chapter{Отчет НИР за 2022 год Сиражудинова М.М.}

\section{Оценки погрешности усреднения периодической \\ задачи  для уравнения Бельтрами\\
	с локально-периодическим коэффициентом}
\begin{abstract}
Локальные характеристики математических моделей сильно неоднородных сред, как правило, описываются функциями вида $a(\varepsilon^{-1} x)$, или $b(x,\varepsilon^{-1} x)$, или $c(\varepsilon^{-1} x,\delta^{-1} x)$, \linebreak или $d(\varepsilon^{-1} x,\delta^{-1} x,\gamma^{-1} x)$ и т.\,д., где $\varepsilon$, $\delta$,$\gamma,\ldots>0$ – малые параметры, при этом функции $a$, $b$, $c$, $d$, $\ldots$ имеют упорядоченную структуру (они, например, периодические по переменным $y=\varepsilon^{-1} x$, $z=\delta^{-1} x$  и т.\,д.). Следовательно, соответствующие математические модели – дифференциальные уравнения с быстро осциллирующими коэффициентами.
Нами изучена уравнение Бельтрами с локально периодическим коэффициентом $\mu(x,\varepsilon^{-1} x)$. Получены оценки
погрешности усреднения периодической задачи в пространствах Соболева и Лебега.
\end{abstract}



\section*{Введение}
        \bigskip

Теория усреднения имеет многочисленные приложения в различных областях физики и механики сплошных сред (см. монографию \cite{smm-1} и литературу в  ней). Операторные оценки погрешности классических задач усреднения для дивергентных уравнений хорошо изучены Жиковым В. В., Бирманом   М. Ш., Суслиной  Т. А. и их учениками (см. \cite{smm-2,smm-3}).

Операторным оценкам погрешности усреднения  эллиптических операторов второго порядка дивергентного вида  с локально-периодическими коэффициентами посвящены работы \cite{smm-4,smm-5,smm-6,smm-7}.
В этой работе получены оценки усреднения периодической задачи  для уравнения Бельтрами с локально-периодическим коэффициентом:
$$
	A_\varepsilon w_\varepsilon\equiv\partial_{\bar{z}}u_\varepsilon+\mu(x,\varepsilon^{-1}x)\,\partial_z w_\varepsilon
=f\in L_2(\square),\quad w_\varepsilon\in W_2^1(\square),
$$
где $\varepsilon>0$ --- малый параметр, $\square$ --- ячейка периодов со стороной 1.
комплекснозначная измеримая периодическая (периода 1 по всем переменным   функция, удовлетворяющая условию эллиптичности
 $\mathop{vrai\, sup}\limits_{(x,y)\in\square\times\square}|\mu(x,y)|\leqslant k_0,\quad k_0<1$  --- постоянная.
 Более того, $\mu(x,y)$  как функция $x$ равномерно непрерывна по Липшицу
$$
|\mu(x^\prime,y)-\mu(x,y)|\leqslant L|x^\prime-x|,\quad x^\prime,\ x\in\square,  \text{ \ п.в.\ }   y\in\square,
$$
где $L>0$ --- постоянная.

Это уравнение --- недивергентное, поэтому полученные оценки (см. Теорему \ref{smm-th1.8}) отличаются от оценок для дивергентных операторов.
В работе \cite{smm-8} получены оценки погрешности классического усреднения (порядка  $O(\sqrt\varepsilon)$ задачи Римана -- Гильберта  для уравнения Бельтрами с периодическим коэффициентом. Оценки погрешности усреднения (порядка $O(\varepsilon)$) периодической задачи для таких уравнений даются в работе \cite{smm-9}.  Оценки усреднения  порядка $O(\sqrt\varepsilon)$  задачи Римана -- Гильберта для уравнения Бельтрами с локально-периодическим коэффициентом получены  в \cite{smm-10}.



 \subsection{Задача Римана -- Гильберта}
 В ограниченной односвязной области $Q$ с кусочно-гладкой границей рассмотрим задачу Римана -- Гильберта (Р -- Г) для уравнения Бельтрами:
\begin{equation}\label{smm-f:1.001}
	\left\{\begin{array}{l}
		A w\equiv\partial_{\bar{z}}w+\mu\,\partial_z w=f\in L_2(Q;\mathbb{C}), \\[3mm]
		w\in W_0(Q)=\{u\in W_2^1(Q) \mid { Re\ u}_{|\partial Q}=0,\quad \int_Q\text{ Im} u=0\},
	\end{array}\right.
\end{equation}
где коэффициент $\mu=\mu(x)$ --- измеримая ограниченная комплекснозначная функция, удовлетворяющая условию эллиптичности
\begin{equation}\label{smm-f:1.2}
	\mathop{vrai\,sup}_{x\in Q}\left(|\mu(x)|\right)\leqslant k_0 <1,
\end{equation}
$k_0>0$ --- постоянная (\textit{константа эллиптичности}).

Имеет место


\begin{theorem}\label{smm-th1.1}
Задача Римана -- Гильберта { \eqref{smm-f:1.001}} однозначно разрешима для любой правой части $f$ из $L_2(Q;\mathbb{C})$,
причем имеют место априорные оценки:
\begin{gather*}
(1-k_0)\left\|\partial_{\bar{z}}w\right\|^2_{L_2(Q;\mathbb{C})}\leqslant \text{ Re\,} \int_{Q} Aw\cdot \overline{\partial_{\bar{z}}w}\,dx,\quad w\in W_0(Q),\\
(1-k_0)\left\|\partial_{\bar{z}}w\right\|_{L_2(Q;\mathbb{C})}\leqslant \left\| Aw\right\|_{L_2(Q;\mathbb{C})} \leqslant
(1+k_0)\left\|\partial_{\bar{z}}w\right\|_{L_2(Q;\mathbb{C})}, \quad w\in W_0(Q),\\
	c_1\|w\|_{W^1_2(Q;\mathbb{C})}\leqslant\|Aw\|_{L_2(Q;\mathbb{C})}\leqslant c_2 \|w\|_{W^1_2(Q;\mathbb{C})},
\end{gather*}
где $c_1$, $c_2>0$ --- постоянные, зависящие только от постоянной эллиптичности $k_0$.


\hspace{5.mm} Пусть $w\in W_2^1(Q;\mathbb{C})$ --- решение уравнения  { \eqref{smm-f:1.001}}, коэффициент $\mu=\mu(x)$ которого равномерно непрерывен по Липшицу в $\overline Q$, т.е.
$$
|\mu(x^\prime)-\mu(x)|\leqslant L\,|x^\prime-x|,\quad x,\,x^\prime\in \overline{Q},
$$
где $L>0$ --- постоянная, И пусть  $f\in W_2^1(Q;\mathbb{C})$, тогда $w\in W_{2,\text{loc}}^2(Q)$ и в любой компактной подобласти $Q_1\Subset Q$ имеет место оценка
\begin{equation*}
	\|w\|_{W^2_2(Q_1; \mathbb{C})}\leqslant c\left(\|f\|_{W^1_2(Q; \mathbb{C})}+\|w\|_{W^1_2(Q; \mathbb{C})}\right),
\end{equation*}
где $c>0$ --- постоянная, зависящая только от $k_0$, $L$ и $\text{ dist\,}(Q_1,\partial Q)$.

\hspace{5.mm} Пусть дополнительно граница $\partial Q$ принадлежит классу $C^2$ и $w$ принадлежит $W_0(Q)\cap W_2^1(Q; \mathbb{C})$, тогда $w$ принадлежит $W_2^2(Q; \mathbb{C})$ и имеет место оценка
\begin{equation*}
\|w\|_{W^2_2(Q; \mathbb{C})}\leqslant c\|f\|_{W^1_2(Q; \mathbb{C})},
\end{equation*}
где $c>0$ --- постоянная, $c=\text{const\,}(k_0, L, Q)$.
\end{theorem}





\subsection{$G$-сходимость.}
Обозначим через $A(k_0)$ --- множество операторов Бельтрами.

\begin{definition}
Скажем, что последовательность операторов $\left\{ \textit{A}_k\right\}$ из класса $A(k_0)$
$G$-сходится в области $Q$ к оператору $A\in A(k_0)$
(и будем писать $A_k\overset{G}{\longrightarrow} A$),
если для любого $f\in L_2(Q;\mathbb{C})$ последовательность $w_k$ решений задачи Р -- Г: $A_kw_k=f$, $w_k\in W_0(Q)$,
сходится в $L_2(Q;\mathbb{C})$. {Сходимость в
	$L_2(Q;\mathbb{C})$ можно заменить на слабую сходимость в
	$W_2^1(Q;\mathbb{C})$. Эквивалентность полученных при этом определений следует из оценок теоремы \ref{smm-th1.1} и компактности вложения
	$W_2^1(Q;\mathbb{C})\subset L_2(Q;\mathbb{C})$.} к решению задачи Р -- Г: $Aw=f$, $w\in W_0(Q)$.
\end{definition}




Как известно \cite{smm-11}, \textit{ $G$-предел определен единственным образом.
Класс $A(k_0)$
		%обобщенных операторов Бельтрами, операторов Бельтрами
		$G$-компактен}.
	

	
	\subsection{Понятие усреднения}
	Рассмотрим задачу Р -- Г с малым параметром $\varepsilon$, $\varepsilon>0$:
	\begin{equation}\label{smm-f:1.8}
		\left\{\begin{array}{l}
			A_\varepsilon w_\varepsilon\equiv\partial_{\bar{z}}w_\varepsilon+\mu^{\varepsilon}\,\partial_z w_\varepsilon
			=f\in L_2(Q;\mathbb{C}), \\[3mm]
			w_\varepsilon\in W_0(Q).
		\end{array}\right.
	\end{equation}
	Всюду в пункте  { коэффициент
		$\mu^\varepsilon=\mu^\varepsilon(x)$ --- измеримая ограниченная функции, имеющая локально-периодическую структуру:
		$\mu^\varepsilon(x)=\mu(x,\varepsilon^{-1}x)$,
		т.\,е. функция $\mu(x,y)$ периодическая
		по второй переменной $y$ {(}периодичность по первой переменной $x$ не требуется{)}. Предполагаем, что функция $\mu(x,y)$ равномерно непрерывна по Липшицу  по первой переменной $x$
		\begin{equation*}
			|\mu(x^\prime,y)-\mu(x,y)|\leqslant L\,|x^\prime-x|,\quad x,\,x^\prime\in \overline{Q},\quad \text{п.\,в.\quad} y\in\square_T,
		\end{equation*}
		где $L>0$ --- постоянная. Кроме того, $\mu(x,y)$ удовлетворяет условию эллиптичности
		\begin{equation}\label{smm-f:1.2''}
			\mathop{vrai\,sup}\limits_{(x,y)\in \overline{Q}\times\square_T}\left(|\mu(x,y)|\right)\leqslant
			k_0 <1.
	\end{equation}}
	\hspace{5.mm}	Очевидно, что оператор $A_\varepsilon$ принадлежит классу $A(k_0)$.
	
\begin{definition}
	Скажем, что семейство $\left\{ A_\varepsilon \right\}$ операторов
краевой задачи \eqref{smm-f:1.8}  допускает усреднение, если $A_\varepsilon \overset{G}{\longrightarrow}A\in A(k_0) $ при
$\varepsilon\to 0$.
\end{definition}

	В вопросах усреднения важную роль играет ядро оператора $\mathcal{A}^*$, сопряженного оператору периодической по $y=(y_1,y_2)$ задачи:
	\begin{equation}\label{smm-f:1.7}
		\mathcal{A}w\equiv \partial_{\overline{\xi}}w +\mu(x,y) \partial_{\xi}w =f\in L_2(\square_T;\mathbb{C}),\quad
		w(x,\cdot)\in W_2^1(\square_T;\mathbb{C}),
	\end{equation}
	где $x\in \overline Q$, играет роль параметра,
	$$
	\partial_{\bar \xi}=2^{-1}\Big(\frac{\partial}{\partial y_1}+i\frac{\partial}{\partial y_2}\Big),\qquad  \partial_{\xi}=2^{-1}\Big(\frac{\partial}{\partial y_1}-i\frac{\partial}{\partial y_2}\Big).
	$$
	
	Сформулируем в виде теоремы результаты по периодической задаче, необходимые нам в дальнейшем.
	
	
\begin{theorem}\label{smm-th1.2}
Для каждого $x\in\overline Q$ справедливы следующие утверждения:
\begin{itemize}
\item  Для периодической задачи имеют место оценки
\begin{gather*}
	(1-k_0)\left<|\partial_{\bar{\xi}}w(x,\cdot)|^2\right>_y\leqslant \text{ Re\,}\left<\mathcal{A}w(x,\cdot)\cdot \overline{\partial_{\bar{\xi}}w(x,\cdot)}\,\right>_y,\quad  w(x,\cdot)\in W_2^1(\square_T;\mathbb{C}),\\
	c_1\left<|\partial_{\bar{\xi}}w(x,\cdot)|^2\right>_y^{1/2}\leqslant
	\left<|\mathcal{A}w(x,\cdot)|^2\right>_y^{1/2}\leqslant c_2\left<|\partial_{\bar{\xi}}w(x,\cdot)|^2\right>_y^{1/2}, \quad  w(x,\cdot)\in W_2^1(\square_T;\mathbb{C}),
\end{gather*}
где $c_1=1-k_0$, $c_2=1+k_0$. Первое из этих неравенств называют <<неравенством острого угла>>.
\item  Периодическая задача \eqref{smm-f:1.7} фредгольмова.
\item Ядра  $\text{Ker\,}\mathcal{A}^\ast$, $\text{ Ker\,}\mathcal{A}\,(=\mathbb{C})$
	--- одномерные подпространства соответствующих пространств,
	причем один из базисов $p$ ядра
	$\text{Ker\,}\mathcal{A}^\ast$ обладает свойством
	\begin{equation}\label{smm-df:1.10}
 \left< p(x,\cdot)\right>_y=1.
	\end{equation}
\end{itemize}
\end{theorem}

	\subsection{Задача на ячейке периодов}
	Для применения асимптотических методов при получении оценок погрешности усреднения
	уравнения \eqref{smm-f:1.8} нам потребуются периодическое
	решение следующей задачи на ячейке периодов
	\begin{equation}\label{smm-f:1.11}
		\begin{aligned}
			&\mathcal{A}N\equiv \partial_{\bar\xi}N_j(x,y)+\mu(x,y)\,\partial_\xi N(x,y) =\mu^0(x)-\mu(x,y), \quad \langle N(x,\cdot)\rangle_y=0,
				\end{aligned}
	\end{equation}
	где $x\in \overline Q$ выступает в роли параметра.

	
	{Здесь и далее  $\mu^0(x)$ --- функция, определённая
		формулой}
		\begin{equation*}
		\mu^0(x)=\left<\,\mu(x,\cdot)\,\overline{p(x,\cdot)}\right>
	\end{equation*}
	\textit{где} 	\textit{$p$ ---  базисный вектор из теоремы {\ref{smm-th1.2}}}.
	
	\begin{theorem}\label{smm-th1.3}
	Для каждого $x\in\overline{Q}$ задача \eqref{smm-f:1.11}
		однозначно разрешима.
	\end{theorem}

	\subsection{Периодические задачи}
	В дальнейшем нам потребуются периодические задачи трех видов. Сформулируем в виде   лемм  результаты по этим задачам.
	\begin{lemma}
		Рассмотрим периодическую задачу
		\begin{equation}\label{smm-Sirazhudinov:4}
			\mathscr{A}w\equiv\partial_{\bar{\xi}}w+\mu(x,y)\partial_\xi w=f\in L_2 (\square;\mathbb{C}),\quad w(x,\cdot)
			\in W_2^1 (\square),
		\end{equation}
		где $x\in \overline{\square}$ выступает в роли параметра, $\xi=y_1+iy_2$,
		$\partial_{\bar{\xi}}=2^{-1}(\mathscr{D}_{y_1}+i\mathscr{D}_{y_2})$,
		$\partial_\xi=2^{-1}(\mathscr{D}_{y_1}-i\mathscr{D}_{y_2})$,
		$\mathscr{D}_{y_j}=\partial/\partial y_j$, $j=1,2$,
		$\mu(x,y)$ --- измеримая комплекснозначная 1-периодическая по $x$ и $y$ функция, удовлетворяющая условию эллиптичности
		\begin{equation}\label{smm-Sirazhudinov:5}
			\mathop{\mathrm{vrai\, sup}}\limits_{(x,y)\in\overline\square\times\square} (|\mu(x,y)|)     \leqslant k_0<1,
		\end{equation}
		$k_0>0$ --- постоянная {(}константа эллиптичности{)}.
		Тогда для  каждого $x\in \overline\square$ справедливы следующие утверждения:
		\begin{enumerate}
			\item[1)]       Периодическая задача \eqref{smm-Sirazhudinov:4} фредгольмова, причем каждое из ядер $\text{Ker\,}\mathscr{A}=\mathbb{C}$, $\text{Ker\,}\mathscr{A}^\ast$ --- одномерное подпространство соответствующего пространства {(}напомним,
			что в работе используются пространства над полем действительных чисел{)}.
			\item[2)] Один из векторов $p$ базиса ядра $\text{ Ker\,}\mathscr{A}^\ast$ обладает свойством: $\left\langle p(x,\cdot)\right\rangle_y=1$.
		\end{enumerate}
	\end{lemma}
	\begin{lemma}\label{smm-lem1.5}
		Рассмотрим периодическую задачу
		\begin{equation}\label{smm-Sirazhudinov:6}
			\mathscr{A}_\varepsilon w_\varepsilon\equiv
			\partial_{\bar z}w_\varepsilon+\mu^\varepsilon(x)\partial_zw_\varepsilon
			=f\in L_2(\square),\quad w_\varepsilon\in W_2^1(\square),
		\end{equation}
		где $\mu^\varepsilon(x)=\mu(x,\varepsilon^{-1}x)$ 	--- {1}-периодические функции, удовлетворяющие условию эллиптичности \eqref{smm-Sirazhudinov:5}.
		Тогда для каждого $\varepsilon=n^{-1}$, $n\in \mathbb N$, справедливы утверждения:
		\begin{enumerate}
			\item[1)] Для периодической задачи \eqref{smm-Sirazhudinov:6} имеет место <<неравенство острого угла>>
			\begin{equation}\label{smm-Sirazhudinov:7}
				(1-k_0)\left<|\partial_{\bar{z}}w_\varepsilon|^2\right>\leqslant \text{Re\,}\left<\mathscr{A}_\varepsilon w_\varepsilon\cdot \overline{\partial_{\bar{z}}w_\varepsilon}\,\right>,\qquad  w_\varepsilon\in W_2^1(\square;\mathbb{C}),
			\end{equation}
			где $k_0$ --- постоянная эллиптичности.
			\item[2)] Задача \eqref{smm-Sirazhudinov:6} фредгольмова, причем ядро  $\text{Ker\,}\mathscr{A}_\varepsilon^\ast$, $\text{ Ker\,}\mathscr{A}_\varepsilon\,(=\mathbb{C})$
			--- одномерные подпространства соответствующих пространств,
			причем один из базисов $\left\{P_\varepsilon\right\}$ ядра
			$\text{Ker\,}\mathscr{A}_\varepsilon^\ast$ обладает свойством
			\begin{equation*}%\label{smm-f:1.10}
				\left<P_\varepsilon\right>=1.
			\end{equation*}
			\end{enumerate}
	\end{lemma}
	\begin{lemma}
		Для периодической задачи
		\begin{equation}\label{smm-Sirazhudinov:8}
			\mathscr{A}w\equiv \partial_{\bar{z}}w +\mu(x) \partial_{z}w =f\in L_2(\square;\mathbb{C}),\quad
			w\in W_2^1(\square;\mathbb{C}),
		\end{equation}
		где коэффициент $\mu(x)$  --- {T}-периодическая функция, удовлетворяющая условию эллиптичности \eqref{smm-Sirazhudinov:5}, имеют место  утверждения:
		\begin{enumerate}
			\item[1)] Задача  \eqref{smm-Sirazhudinov:8} фредгольмова с одномерным ядром $\text{Ker\,}\mathscr{A}^\ast$, $\text{ Ker\,}\mathscr{A}\,(=\mathbb{C})$,
			причем один из базисов $p$ ядра
			$\text{Ker\,}\mathscr{A}^\ast$ обладает свойством
			\begin{equation*}%\label{smm-f:1.10}
				\left< p\right>=1.
			\end{equation*}
			\item[2)] Для периодической задачи \eqref{smm-Sirazhudinov:8} имеет место неравенство острого угла
			\begin{equation}\label{smm-Sirazhudinov:9}
				(1-k_0)\left<|\partial_{\bar{z}}w|^2\right>\leqslant \text{ Re\,}
				\left<\mathscr{A} w\cdot \overline{\partial_{\bar{z}}w}\,\right>,\qquad
				w \in W_2^1(\square_T;\mathbb{C}).
			\end{equation}
		\end{enumerate}
	\end{lemma}
	Утверждения этих лемм --- очевидные следствия результатов работы \cite{smm-10}.


	\subsection{Усреднение и оценки погрешности усреднения}
	
	Имеет место (см. \cite{smm-10})  следующая
	
\begin{theorem}[Об усреднении]
 Для семейства $\{A_\varepsilon\}$ операторов краевой
		задачи Римана -- Гильберта \eqref{smm-f:1.8} имеет место усреднение, причем коэффициент
		усредненного оператора $A_0$, $A_0w\equiv\partial_{\overline{z}}w+\mu^0(x)\partial_zw$, $w\in W_0(Q)$ --- равномерно непрерывен  по Липшицу в $\overline{Q}$ и он даётся равенством \eqref{smm-f:1.11}.
\end{theorem}

Рассмотрим специальную периодическую задачу \eqref{smm-f:1.11} на ячейке, где вместо $\overline{Q}$ имеем $\square$,  которая понадобится для применения асимптотических методов при усреднении периодической задачи. Для каждого $x\in\square$ эта задача однозначно разрешима, причем решение её обладает свойствами (см. \cite{smm-10})

	1) Существует показатель $q>2$ (повышенной) суммируемости, определяемый только по константе эллиптичности $k_0$, такой, что  для каждого $x\in\overline\square$ решение $N(x,y)$ как функция $y$ принадлежит пространству $W_q^1(\square)$ и имеют место равномерные по $x\in\overline\square$ оценки:
	$$
	\|N(x,\cdot)\|_{C^\alpha(\overline\square)}\leqslant c,\quad \|N(x,\cdot)\|_{W_q^1(\square)}\leqslant c,
	$$
	где $c>o$ --- зависящая только от $k_0$, $\alpha=(q-2)/q$.
	
	2) Функция $N(x,y)$ и её первые производные по $x_1$ и $x_2$ ограничены постоянной,
	зависящей только от $k_0$ и $L$.
	
	\textit{Основное утверждение.}
	Рассмотрим периодическую задачу:
	\begin{equation}\label{smm-00}
	\left\{\begin{aligned}
	&\mathscr{A}_\varepsilon w_\varepsilon\equiv \partial_{\bar{z}}w_\varepsilon +\mu^\varepsilon(x) \partial_{z}w_\varepsilon =f-\langle f\overline{P_\varepsilon}\rangle,\quad f\in L_2(\square;\mathbb{C}),\\
	&w_\varepsilon\in W_2^1(\square;\mathbb{C}),	\quad \langle w_\varepsilon\rangle=0,
	\end{aligned}\right.	
	\end{equation}
	   		где $\langle P_\varepsilon\rangle=1$, --- базисный вектор ядра оператора $\mathscr{A}_\varepsilon^\ast$, сопряженного оператору $\mathscr{A}_\varepsilon$ периодической задачи \eqref{smm-f:1.11}, $\square$ --- ячейка периодов, $\langle g\rangle=\int_\square g(x)dx$ --- среднее значение функции  $g=g(x)$. Для того чтобы задача \eqref{smm-00} для любого $f\in L_2(\square)$ имела периодическое (периода 1) решение необходимо положить малый параметр $\varepsilon$ равным $n^{-1}$, $n\in \mathbb{N}$.
		Очевидно, что правая часть  уравнения  \eqref{smm-00} ортогональна ядру оператора $\mathscr{A}_\varepsilon^\ast:L_2 (\square)\to W_2^(-1) (\square)$. Следовательно, из леммы \ref{smm-lem1.5} вытекает: Периодическая задача \eqref{smm-00} имеет единственное решение для любой функции $f\in L_2(\square)$.
		
	
	В качестве первого приближения к решению $w_\varepsilon$  периодической задачи
	для уравнения Бельтрами \eqref{smm-00} с локально-периодическим коэффициентом возьмем функцию:
	$$ w_1^\varepsilon(x)=w^0(x)+\varepsilon N(x,y)\partial_zw^0(x)), \quad y=\varepsilon^{-1}x,$$
где $N(x,y)$  --- периодическое решение задачи на ячейке (см. теорему \ref{smm-th1.3}), $w^0$ --- решение усредненной задачи $A_0w\equiv\partial_{\overline{z}}w+\mu^0(x)\partial_zw=f\in W_2^1(Q;\mathbb{C})$, $w\in W_0(Q)$. Заметим, что $w^0\in W_2^2(Q;\mathbb{C})\cap W_0(Q)$, ввиду 	теоремы \ref{smm-th1.1}, так как $f\in W_2^1(Q;\mathbb{C})$.
	
\begin{theorem}[Оценки погрешности усреднения]\label{smm-th1.8}
Пусть правая часть $f$ периодической задачи \eqref{smm-00} принадлежит пространству $W_2^1(Q;\mathbb{C})$, тогда для малых $\varepsilon$ имеют место оценки
		\begin{equation*}
			\|w_\varepsilon-w_1^\varepsilon\|_{W_2^1 (\square; \mathbb{C})}\leqslant c\sqrt{\varepsilon}\|f
			\|_{L_2 (\square; \mathbb{C})}, \quad \|w_\varepsilon-w^0\|_{W_2^1 (Q; \mathbb{C})}\leqslant c\sqrt{\varepsilon}\|f
			\|_{W_2^1 (\square; \mathbb{C})},
		\end{equation*}
		где $c>0$ --- постоянная, зависящая только от постоянной эллиптичности $k_0$ и постоянной Липшица $L$.
\end{theorem}

 %%%%++++++
% \newcommand{\langle}{\langle}
% \newcommand{\>}{\rangle}
\section{Оценки локально-периодического усреднения задачи Римана -- Гильберта для обобщенного уравнения Бельтрами}	

\begin{abstract}\noindent
	Метод усреднения дифференциальных операторов, основанный на асимптотическом разложении по малому параметру, широко используется  в математической и физической литературе. Этот метод позволяет помимо теоремы усреднения получить
	оценки разности точного решения и его приближений.
	Рассмотрены оценки погрешности усреднения обобщенного уравнения Бельтрами с локально-периодическими коэффициентами $\mu(x,\varepsilon^{-1} x)$, $\nu(x,\varepsilon^{-1} x)$.
	\medskip\\

\end{abstract}

\subsection{Формулировка результатов}



\subsubsection{Задача Римана -- Гильберта.} В ограниченной односвязной области $Q$ с кусочно-гладкой границей рассмотрим задачу Римана -- Гильберта для обобщенного уравнения Бельтрами:
\begin{equation}\label{smm-0sirazhf:1.1}
	\left\{\begin{array}{l}
		A w\equiv\partial_{\bar{z}}w+\mu\,\partial_z w+\nu\,\partial_{\bar z}\overline{w}=f\in L_2(Q;\mathbb{C}), \\[3mm]
		w\in W_0(Q),
	\end{array}\right.
\end{equation}
где коэффициенты $\mu=\mu(x)$, $\nu=\nu(x)$ --- измеримые ограниченные комплекснозначные функции, удовлетворяющие условию эллиптичности
\begin{equation}\label{smm-00sirazhf:1.2}
	\mathop{vrai\,sup}_{x\in Q}\left(|\mu(x)|+|\nu(x)|\right)\leqslant k_0 <1,
\end{equation}
$k_0>0$ --- постоянная (\textit{константа эллиптичности}).

Как известно, имеет место

%\smallskipвыкин
\begin{theorem}\label{smm2-th1}
 Задача Римана -- Гильберта {(\ref{smm-0sirazhf:1.1})} однозначно разрешима для любой правой части $f$ из $L_2(Q;\mathbb{C})$,
	причем имеют место априорные оценки:
\begin{gather}\label{smm-0sirazhf:1.3}
	(1-k_0)\left\|\partial_{\bar{z}}w\right\|^2_{L_2(Q;\mathbb{C})}\leqslant \text{Re\,} \int_{Q} Aw\cdot \overline{\partial_{\bar{z}}w}\,dx,\quad w\in W_0(Q),\\
	\label{smm-0sirazhf:1.4}
	(1-k_0)\left\|\partial_{\bar{z}}w\right\|_{L_2(Q;\mathbb{C})}\leqslant \left\| Aw\right\|_{L_2(Q;\mathbb{C})} \leqslant
	(1+k_0)\left\|\partial_{\bar{z}}w\right\|_{L_2(Q;\mathbb{C})}, \quad w\in W_0(Q).
\end{gather}
\end{theorem}
Выражение
$
\left\| w\right\|_{W_0(Q)}=\left\| \partial_{\bar{z}}w\right\|_{L_2(Q;\mathbb{C})}$, $w\in W_0(Q),
$
задает в подпространстве $W_0(Q)$ норму, эквивалентную норме исходного пространства $W_2^1(Q;\mathbb{C})$ (см.  \cite{smm-12,smm-13}),
поэтому имеют место оценки:
\begin{equation}\label{smm-0sirazhf:1.5}
	c_1\|w\|_{W^1_2(Q;\mathbb{C})}\leqslant\|Aw\|_{L_2(Q;\mathbb{C})}\leqslant c_2 \|w\|_{W^1_2(Q;\mathbb{C})},
\end{equation}
где $c_1$, $c_2>0$ --- постоянные, зависящие только от постоянной эллиптичности $k_0$ и области $Q$.

\begin{theorem}\label{smm2-th2} Пусть $w\in W_2^1(Q;\mathbb{C})$ --- решение уравнения  {(\ref{smm-0sirazhf:1.1})}, коэффициенты $\mu=\mu(x)$, $\nu=\nu(x)$ которого равномерно непрерывны по Липшицу в $\overline Q$, т.е.
	$$
	|\varphi(x^\prime)-\varphi(x)|\leqslant L\,|x^\prime-x|,\quad x,\,x^\prime\in \overline{Q},\quad \varphi\in\{\mu,\,\nu\},
	$$
	где $L>0$ --- постоянная. И пусть  $f\in W_2^1(Q;\mathbb{C})$, тогда $w\in W_{2,\text{loc}}^2(Q)$ и в любой компактной подобласти $Q_1\Subset Q$ имеет место оценка
	\begin{equation}\label{smm-0sirazhf:1.6_1}
		\|w\|_{W^2_2(Q_1; \mathbb{C})}\leqslant c\left(\|f\|_{W^1_2(Q; \mathbb{C})}+\|w\|_{W^1_2(Q; \mathbb{C})}\right),
	\end{equation}
	где $c>0$ --- постоянная, зависящая только от $k_0$, $L$ и $\text{ dist\,}(Q_1,\partial Q)$.


Пусть дополнительно граница $\partial Q$ принадлежит классу $C^2$ и $w$ принадлежит $W_0(Q)$, тогда $w$ принадлежит $W_0(Q)\cap W_2^2(Q; \mathbb{C})$ и имеет место оценка
\begin{equation}\label{smm-0sirazhf:1.6_2}
	\|w\|_{W^2_2(Q; \mathbb{C})}\leqslant c\|f\|_{W^1_2(Q; \mathbb{C})},
\end{equation}
где $c>0$ --- постоянная, зависящая только от $k_0$, $L$ и $Q$.
\end{theorem}

\subsubsection{$G$-сходимость}
Обозначим через $A(k_0)$ --- множество обобщенных  операторов Бельтрами
\eqref{smm-0sirazhf:1.1}, $\tilde{A}(k_0)$ --- подмножество $A(k_0)$  операторов Бельтрами
(\eqref{smm-0sirazhf:1.1} с $\nu=0$).

\begin{definition}
Скажем, что последовательность операторов $\left\{ \textit{A}_k\right\}$ из класса $A(k_0)$
$G$-сходится в области $Q$ к оператору $A\in A(k_0)$
(и будем писать $A_k\overset{G}{\longrightarrow} A$),
если для любого $f\in L_2(Q;\mathbb{C})$ последовательность $\{w_k\}$ решений задачи Р -- Г: $A_kw_k=f$, $w_k\in W_0(Q)$,
сходится в $L_2(Q;\mathbb{C})$. Сходимость в
$L_2(Q;\mathbb{C})$ можно заменить на слабую сходимость в
$W_2^1(Q;\mathbb{C})$ к решению задачи Р -- Г: $Aw=f$, $w\in W_0(Q)$. Эквивалентность полученных при этом определений следует из оценок \eqref{smm-0sirazhf:1.5} и компактности вложения
$W_2^1(Q;\mathbb{C})\subset L_2(Q;\mathbb{C})$.
\end{definition}

$G$-предел определен единственным образом и
классы $A(k_0)$, $\tilde{A}(k_0)$
%обобщенных операторов Бельтрами, операторов Бельтрами
$G$-компактны (см. \cite{smm-11}).

$G$-сходимость обладает следующим свойством \textit{сходимости <<произвольных>>  решений\,}:

\noindent\textit{пусть $A_k\overset{G}{\longrightarrow} A$, $w_k\rightharpoonup w$ в $W_2^1(Q;\mathbb{C})$, $f_k\to f$ в $L_2(Q;\mathbb{C})$,
	$A_kw_k=f_k$, тогда $Aw=f$ {(см. \cite{smm-11})}.}


\subsubsection{Понятие усреднения}
Рассмотрим задачу Р -- Г с малым параметром $\varepsilon$, $\varepsilon>0$:
\begin{equation}\label{smm-0sirazhf:1.8}
	\left\{\begin{array}{l}
		A_\varepsilon w_\varepsilon\equiv\partial_{\bar{z}}w_\varepsilon+\mu^{\varepsilon}\,\partial_z w_\varepsilon
		+\nu^\varepsilon\,\partial_{\bar z}\overline w_\varepsilon=f\in L_2(Q;\mathbb{C}), \\[3mm]
		w_\varepsilon\in W_0(Q).
	\end{array}\right.
\end{equation}
Всюду в пункте   { коэффициенты
	$\mu^\varepsilon=\mu^\varepsilon(x)$, $\nu^\varepsilon=\nu^\varepsilon(x)$ --- измеримые ограниченные функции, имеющие локально-периодическую структуру:
	$\mu^\varepsilon(x)=\mu(x,\varepsilon^{-1}x)$, $\nu^\varepsilon(x)=\nu(x,\varepsilon^{-1}x)$,
	т.\,е. функции $\mu(x,y)$, $\nu(x,y)$ периодические
	по второй переменной $y$ (периодичность по первой переменной $x$ не требуется). Кроме того, функции $\mu(x,y)$, $\nu(x,y)$ равномерно непрерывны по Липшицу  по первой переменной $x$
	
	\begin{equation}\label{smm-0sirazhf:1.2'}
		|\varphi(x^\prime,y)-\varphi(x,y)|\leqslant L\,|x^\prime-x|,\quad x,\,x^\prime\in \overline{Q},\quad \text{п.\,в.\quad} y\in\square,\quad  \varphi\in\{\mu,\,\nu\},
	\end{equation}
	где $L>0$ --- постоянная, и $\mu(x,y)$, $\nu(x,y)$ удовлетворяют условию эллиптичности
	\begin{equation}\label{smm-0sirazhf:1.2''}
		\mathop{vrai\,sup}\limits_{(x,y)\in \overline{Q}\times\square}\left(|\mu(x,y)|+|\nu(x,y)|\right)\leqslant
		k_0 <1.
\end{equation}}


Очевидно, что оператор $A_\varepsilon$ принадлежит классу $A(k_0)$.


\begin{definition}
Скажем, что семейство $\left\{ A_\varepsilon \right\}$ операторов
краевой задачи \eqref{smm-0sirazhf:1.8}  допускает усреднение, если $A_\varepsilon \overset{G}{\longrightarrow}A\in A(k_0) $ при
$\varepsilon\to 0$.
\end{definition}
В вопросах усреднения важную роль играет ядро оператора $\mathcal{A}^*,$ сопряженного оператору периодической по $y=(y_1,y_2)$ задачи:
\begin{equation}\label{smm-0sirazhf:1.7}
	\mathcal{A}w\equiv \partial_{\overline{\xi}}w +\mu(x,y) \partial_{\xi}w +\nu(x,y)\,
	\partial_{\overline \xi}\overline w=f\in L_2(\square;\mathbb{C}),\quad
	w(x,\cdot)\in W_2^1(\square;\mathbb{C}),
\end{equation}
где  $x\in \overline Q$, играет роль параметра,
$$
\partial_{\bar \xi}=2^{-1}\Big(\frac{\partial}{\partial y_1}+i\frac{\partial}{\partial y_2}\Big),\qquad  \partial_{\xi}=2^{-1}\Big(\frac{\partial}{\partial y_1}-i\frac{\partial}{\partial y_2}\Big).
$$

Сформулируем в виде теоремы результаты по периодической задаче из работы \cite{smm-15}, необходимые в дальнейшем.



\begin{theorem}\label{smm2-th3}  Пусть $x\in\overline Q$, тогда справедливы следующие утверждения:
\begin{itemize}
	\item  {Для периодической задачи имеют место оценки}
	\begin{gather}\label{smm-0sirazhf:1.9}
		c_1\big<|\partial_{\bar{\xi}}w(x,\cdot)|^2\big>_y\leqslant \text{ Re\,}\big<\mathcal{A}w(x,\cdot)\cdot \overline{\partial_{\bar{\xi}}w(x,\cdot)}\,\big>_y,\quad  w(x,\cdot)\in W_2^1(\square;\mathbb{C}),\\
		c_1\big<|\partial_{\bar{\xi}}w(x,\cdot)|^2\big>_y^{1/2}\leqslant
		\big<|\mathcal{A}w(x,\cdot)|^2\big>_y^{1/2}\leqslant c_2\big<|\partial_{\bar{\xi}}w(x,\cdot)|^2\big>_y^{1/2},\quad  w(x,\cdot)\in W_2^1(\square;\mathbb{C}),\label{smm-0sirazhf:1.9'}
	\end{gather}
	{ где $c_1=1-k_0$, $c_2=1+k_0$. Первое из этих неравенств будем называть <<неравенством острого угла>>}.
	\item { Периодическая задача \eqref{smm-0sirazhf:1.7} фредгольмова.}
	\item {Ядра  $\text{Ker\,}\mathcal{A}^\ast$ и $\text{ Ker\,}\mathcal{A}\,(=\mathbb{C})$
		--- двумерные подпространства пространств $L_2(\square;\mathbb{C})$ и $W_2^1(\square;\mathbb{C})$ соответственно {(}напомним, что
		наши пространства --- пространства над полем $\mathbb{R}${)},
		причем один из базисов $\left\{p_1,p_2\right\}$ ядра
		$\text{Ker\,}\mathcal{A}^\ast$ обладает свойствами
		\begin{equation}\label{smm-0sirazhdf:1.10}
			\left< p_1(x,\cdot)\right>_y=1,\quad \left< p_2(x,\cdot)\right>_y=i.
		\end{equation}
		В случае оператора Бельтрами {(}$\nu=0${)} имеем: $p_2=ip_1$.}
\end{itemize}
\end{theorem}
Пусть $w=w(y)$, $w=w_1+iw_2\in W_2^1(\square;\mathbb{C})$, тогда имеют место равенства
\begin{equation}\label{smm-0sirazhdf.1.11}
	\left<|\partial_{\bar{\xi}}w|^2\right>_y=\left<|\partial_{\xi}w|^2\right>_y=4^{-1}\sum_{j=1}^2\left<
	\,|\nabla w_j|^2\right>_y.
\end{equation}
Справедливость равенств \eqref{smm-0sirazhdf.1.11} достаточно проверить  для  гладких периодических $w=w_1+iw_2\in C^2(\square;\mathbb{C})$. Легко видеть, что имеет место равенство
\begin{equation}\label{smm-0sirazhdf.1.111}
	\begin{gathered}
		\left<\,|\partial_{\bar \xi} w|^2\right>_y=4^{-1}\sum_{j=1}^2\left<
		\,|\nabla w_j|^2\right>_y+\\
		+2^{-1}|\square|^{-1}\int_{\square}\left(\mathscr{D}_{y_2}w_1\mathscr{D}_{y_1}w_2-\mathscr{D}_{y_1}w_1\mathscr{D}_{y_2}w_2\right)\,dy,\quad w\in C^2(\square;\mathbb{C}).
	\end{gathered}
\end{equation}
(Аналогичное равенство, со знаком минус перед вторым слагаемым справа, имеет место и для $\left<\,|\partial_{\xi} w|^2\right>_y$). Второе слагаемое справа в \eqref{smm-0sirazhdf.1.111} равен нулю, в чем можно убедиться перебросив производные с
$w_1$ на $w_2$ с учетом периодичности. (При этом граничный интеграл равен нулю, так как нормали к противоположным
сторонам ячейки периодов противоположно направлены.) Отсюда вытекают равенства \eqref{smm-0sirazhdf.1.11}.

\subsubsection{Задача на ячейке периодов}
Для применения асимптотических методов при получении оценок погрешности усреднения
уравнения \eqref{smm-0sirazhf:1.8} нам потребуются периодические
решения следующей задачи на ячейке периодов
\begin{equation}\label{smm-0sirazhf:1.11}
	\left\{\begin{aligned}
		&\mathcal{A}N_j\equiv \partial_{\bar\xi}N_j(x,y)+\mu(x,y)\,\partial_\xi N_j(x,y)+\nu(x,y)\,
		\partial_{\bar\xi}\overline{N_j(x,y)}=\chi_j(x,y),\\
		& N_j(x,\cdot)\in W_2^{1}(\square;\mathbb{C}),\quad \left<N_j(x,\cdot)\right>_y=0,\quad j=1;2,
	\end{aligned}\right.
\end{equation}
где $x\in \overline Q$ выступает в роли параметра, $\square$ --- квадрат периодов со стороной $T$,
\begin{equation}\label{smm-0sirazhdf:1}
	\begin{aligned}
		&\chi_1(x,y)=2^{-1}\left(\mu^0(x)+\nu^0(x)-\mu(x,y)-\nu(x,y)\right),\\
		&\chi_2(x,y)=2^{-1}i\left(\mu^0(x)-\nu^0(x)-\mu(x,y)+\nu(x,y)\right).
	\end{aligned}
\end{equation}

\textit{Здесь и всюду далее  $\mu^0(x)$, $\nu^0(x)$ --- функции, определённые
	формулами}
\begin{equation}\label{smm-0sirazhdf:1.11}
	\mu^0(x)=\big<\,\mu(x,\cdot)\, \mathscr Q(x,\cdot)+\overline{\nu(x,\cdot)}\,\mathscr{P}(x,\cdot)\,\big>_y, \quad
	\nu^0(x)=\big<\,\overline{\mu(x,\cdot)}\,\mathscr{P}(x,\cdot)+\nu(x,\cdot)\, \mathscr Q(x,\cdot)\,\big>_y,
\end{equation}
\textit{где}
$
\mathscr{P}(x,y)=2^{-1}(p_1(x,y)+ip_2(x,y)),
\quad \mathscr Q(x,y)=2^{-1}(\overline{p_1(x,y)}+i\,\overline{p_2(x,y)}),
$
\textit{(}$p_1$, $p_2$ ---  базисные векторы из теоремы { \ref{smm2-th3}},
$\overline{p_1}$, $\overline{p_2}$ --- комплексно сопряженные $p_1$, $p_2$
функции{)}.

\textit{В случае уравнения Бельтрами {($\nu=0$)} имеем{\,:} $\mu^0(x)=\big<\mu(x,\cdot)\,\overline{p_1(x,\cdot)}\,\big>_y$,
	$\nu^0=0$.}



\begin{theorem}\label{smm2-th4} Для каждого $x\in\overline{Q}$ задача \eqref{smm-0sirazhf:1.11}
	однозначно разрешима.
\end{theorem}
Действительно, согласно теореме \ref{smm2-th3}, для разрешимости задачи необходимо и достаточно,
чтобы для каждого $x\in\overline{Q}$ функции $\chi_1(x,y)$, $\chi_2(x,y)$, как функции $y$, были ортогональны
базисным векторам $p_1$ и $p_2$ из ядра оператора $\text{ Ker\,}\mathcal{A}^\ast$.
Это легко проверить, используя равенс  тва \eqref{smm-0sirazhdf:1.10}. Единственность
решения следует из неравенства острого угла \eqref{smm-0sirazhf:1.9}, ввиду
неравенства Пуанкаре.

В случае уравнения Бельтрами имеем $\nu=0$, $\nu^0=0$, следовательно, $\chi_2=i\chi_1$.
Значит, $N_2=iN_1$.


\subsection{Свойства решений задачи на ячейке}

\begin{property}\label{smm-prop1}
\textit{Пусть $N_j$, $j=1,2$, --- решение задачи \eqref{smm-0sirazhf:1.11}, тогда отображение $\overline{Q}\ni x \mapsto  N_j(x,\cdot)\in W_2^1(\square;\mathbb{C})$  ограничено, причем имеет место оценка
	\begin{equation}\label{smm-0sirazhdf:1.14}
		\|N_j(x,\cdot)\|_{W_2^1(\square;\mathbb{C})}\leqslant c,\quad x\in \overline{Q},\quad j=1,2,
	\end{equation}
	%      ?N(x,?)?_(W_2^1 (?; C))?c ,     x??Q,                 (1.14)
	где $c>0$ --- постоянная, определяемая только по постоянной эллиптичности $k_0$.}
\end{property}

\begin{corollary}
Функции $\mu^0(x)$, $\nu^0(x)$, $x\in\overline{Q}$, определенные формулами \eqref{smm-0sirazhdf:1.11},  ограничены постоянной, зависящей только от постоянной эллиптичности $k_0$.
\end{corollary}

\begin{property}\label{smm-prop2}
{Найдется число $q>2$ (показатель повышенной суммируемости), зависящее только от постоянной
	эллиптичности $k_0$, такое, что
	решения $N_j$, $j=1,2$  принадлежат $W_q^1(\square;\mathbb{C})$ и имеют место неравенства
	\begin{equation}\label{smm-0sirazhdf:1.15}
		\|N_j(x,\cdot)\|_{C^\alpha(\overline\square;\mathbb{C)}}      \leqslant c,\quad
		\|N_j(x,\cdot)\|_{W_r^1(\square;\mathbb{C})}\leqslant c,\quad j=1,2, \quad\text{для всех\,}\quad x
		\in\overline{Q},
	\end{equation}
	где $c>0$ --- постоянная, зависящая только от постоянной эллиптичности  $k_0$, $2<r\leqslant q$, $\alpha=(r-2)/r$.}
\end{property}

\begin{property}\label{smm-prop3}
{Пусть $N_j$, $j=1,2$ --- решение задачи \eqref{smm-0sirazhf:1.11}, тогда отображение
	$\overline{Q}\ni x \mapsto  N_j(x,\cdot)\in W_2^1(\square;\mathbb{C})$ липшицево, т.\,е.
	\begin{equation}\label{smm-0sirazhdf:1.16}
		\|N_j(x,\cdot)-N_j(x^\prime,\cdot)\|_{W_2^1(\square;\mathbb{C})}\leqslant c\,|x-x^\prime|,
		\qquad x,x^\prime\in\overline{Q},
	\end{equation}
	где $c>0$ --- постоянная, зависящая только от постоянной эллиптичности $k_0$
	и постоянной Липшица $L$.}
\end{property}

\begin{corollary}
{Функции $\mu^0 (x)$ и $\nu^0(x)$, определенные формулами \eqref{smm-0sirazhdf:1.11},
	равномерно непрерывны по Липшицу в замыкании $\overline{Q}$, т.\,е.
	\begin{equation*}
		|\varphi(x)-\varphi(x^\prime)|\leqslant L_1|x-x^\prime|,\quad x,x^\prime\in\overline{Q},\qquad
		\varphi\in\{\mu^0,\nu^0\},
	\end{equation*}
	где  $L_1>0$ --- постоянная, зависящая только от $k_0$ и $L$.}
\end{corollary}

\begin{property}\label{smm-prop4}
{Пусть $q>2$ --- показатель повышенной суммируемости из
	свойства \ref{smm-prop2}  и пусть $2<r\leqslant q$. Тогда отображение $\overline{Q}\ni x \mapsto  N_j(x,\cdot)\in W_r^1(\square;\mathbb{C})$, $j=1,2$,
	липшицево:
	\begin{equation}\label{smm-0sirazhdf:21}
		\|N_j(x,\cdot)-N_j(x^\prime,\cdot)\|_{W_r^1(\square;\mathbb{C})}\leqslant c\,|x-x^\prime|,\quad x,x^\prime\in\overline{Q},\quad
		j=1,2,
	\end{equation}
	где $c>0$ --- постоянная, зависящая только от постоянной эллиптичности $k_0$ и постоянной Липшица $L$}.
\end{property}

\begin{corollary}
{Решение $N_j=N_j(x,y)$, $j=1,2$, задачи \eqref{smm-0sirazhf:1.11} непрерывно в $\overline{Q}\times\overline\square$,
	липшицево по $x$ в $\overline{Q}$ и гельдерово по $y$ в $\overline\square$ с показателем $\alpha=(r-2)/r$,
	($2<r\leqslant q$,  $q>2$ --- показатель повышенной суммируемости), т.\,е. для любых
	$x, x^\prime \in \overline{Q}$,  $y, y^\prime \in \overline\square$ имеем:}
\begin{equation}\label{smm-0sirazhdf:1.18}
	|N_j(x,y)-N_j(x^\prime,y^\prime)|\leqslant c_0|x-x^\prime|+c_1|y-y^\prime|^\alpha, \quad j=1,2,
\end{equation}
{где $c_0$, $c_1>0$ --- постоянные, зависящие только от $k_0$ и $L$.}

{ Кроме того, производные $\mathscr{D}_{x_l} N_j(x,y)={\partial N_j(x,y)}/{\partial x_l}$ , $j,l=1,2$,  принадлежат
	пространству $L_\infty(Q\times\square;\mathbb{C})$  и имеют место оценки}
\begin{equation}\label{smm-0sirazhdf:1.19}
	\|N_j\|_{L_\infty(Q\times\square;\mathbb{C})}\leqslant c_0,\quad \|
	\mathscr{D}_{x_l} N_j(x,y)\|_{L_\infty(Q\times\square;\mathbb{C})} \leqslant c_1,
\end{equation}
{ где $c_0$, $c_1>0$ --- постоянные, $c_0$ --- зависит только от $k_0$, $c_1$ --- от $k_0$ и $L$.}
\end{corollary}

\subsection{Вспомогательные утверждения.}
Пусть $Q$ --- ограниченная гладкая (класса $C^2$) область плоскости,
тогда справедлива следующая

\begin{lemma}
{ Пусть $\mathfrak{a}=\mathfrak{a}(x,y)\in \text{Lip}(\overline{Q}$,$L_r (\square;\mathbb{C}))$, $1<r<2$, (т.\,е. $\mathfrak{a}$  периодична по
	$y$ и равномерно непрерывна  по Липшицу как функция $x\in\overline{Q}$
	со значениями в $L_r (\square;\mathbb{C})$),  $\mathfrak{a}(x,y)\geqslant
	0$, {($x\in\overline{Q}$, $y\in\square$)}  и пусть  $\mathfrak{a}^\varepsilon(x)
	=\mathfrak{a}(x,\varepsilon^{-1} x))$, $x\in\overline{Q}$. Тогда для
	любого $w\in W_2^1 (Q;\mathbb{C})$ имеют место неравенства
	\begin{gather}\label{smm-0sirazhVf:1}
		\int_Q\mathfrak{a}^\varepsilon(x) |w|^2 dx\leqslant c\left(\|w\|_{L_2 (Q;\mathbb{C})}^2+
		\varepsilon^2 \|w\|_{W_2^1 (Q;\mathbb C)}^2 \right),  \\\label{smm-0sirazhVf:2}
		\int_{Q\cap Q_\varepsilon}\mathfrak{a}^\varepsilon(x) |w|^2 dx\leqslant c\varepsilon\|w\|_{W_2^1 (Q;\mathbb{C})}^2,
	\end{gather}
	для всех достаточно малых $\varepsilon$, $\varepsilon\leqslant\varepsilon_0
	(Q)$,
	где $Q_\varepsilon$ --- $\varepsilon$-окрестность границы $\partial Q$,
	постоянная $c>0$ зависит только от области $Q$ и $\mathop{ max}\limits_{x\in\overline{Q}}\|
	\mathfrak{a}(x,\cdot)\|_{L_r (\square;\mathbb C)}$}.
\end{lemma}


Пусть $Q\subset\mathbb{R}^2$ --- ограниченная
односвязная область с гладкой (класса $C^2$) границей,  коэффициенты
уравнения \eqref{smm-0sirazhf:1.8} ---  локально-периодические функции, удовлетворяющие соотношениям  \eqref{smm-0sirazhf:1.2'}, \eqref{smm-0sirazhf:1.2''} и пусть
$w^0\in W^2_2(Q;\mathbb{C})\cap W_0(Q)$, $f=A_0w^0$, где $A_0w^0\equiv\partial_{\overline{z}}w^0+\mu^0(x)\partial_zw^0+\nu^0(x)\partial_{\overline{z}}
\overline{w^0}$ (коэффициенты $\mu^0(x)$, $\nu^0(x)$   определены  формулами \eqref{smm-0sirazhdf:1.11}),

\begin{equation}\label{smm-0sirazhL:1}
	w_1^\varepsilon(x)=w^0(x)+\varepsilon\left(N(x,y)\partial_zw^0(x)+M(x,y)\partial_{\bar{z}}\overline{w^0(x)}\right),
	\quad  y=\varepsilon^{-1}x,
\end{equation}
где $N(x,y)=N_1(x,y)-iN_2(x,y)$, $M(x,y)=N_1(x,y)+iN_2(x,y)$, $N_1$ и $N_2$ --- периодические решения задачи на ячейке (см. теорему \ref{smm2-th4}).
Легко видеть, что
имеет место равенство
\begin{equation}\label{smm-0sirazhL:2}
	A_{\varepsilon}w_1^{\varepsilon}=f+\varepsilon r_\varepsilon,
\end{equation}

\vspace{-0.8cm}
\begin{multline}\label{smm-0sirazhL:3}
	{\hspace{-0.33cm}}      \text{где \quad} r_\varepsilon=N(x,y)
	\left(\partial^2_{\bar{z}z}w^0(x)+\mu(x,y)\,\partial^2_{zz}w^0(x)\right)+\nu(x,y)\overline{N(x,y)}\,\partial^2_{\bar{z}\bar{z}}\overline{w^0(x)
	}+\\
	+M(x,y)
	\left(\partial^2_{\bar{z}\bar{z}}\overline{w^0(x)}+\mu(x,y)\,
	\partial^2_{z\bar{z}}\overline{w^0(x)}\right)+
	\nu(x,y)\,\overline{M(x,y)}\,\partial^2_{\bar{z}z}w^0(x)+\\
	+\partial_{z}\overline{w^0(x)}\left(\partial_{\overline{z}}\overline{N(x,y)}+\mu(x,y)
	\partial_{z}N(x,y)\right)+\nu(x,y)\partial_{\overline z}\overline{N(x,y)}\,
	\partial_{\overline{z}}\overline{w^0(x)}+\\
	+\partial_{\overline{z}}\overline{w^0(x)}\left(\partial_{\overline{z}}M(x,y)+\mu(x,y)
	\partial_{z}M(x,y)\right)+\nu(x,y)\partial_{\overline z}\overline{M(x,y)}\,
	\partial_{z}w^0(x),
	\quad y=\varepsilon^{-1}x.
\end{multline}

Справедлива следующая


\begin{lemma} { Невязка \eqref{smm-0sirazhL:3} $r_\varepsilon$ ограничена в $L_2(Q;\mathbb{C})$, причем}
\begin{equation}\label{smm-0sirazhL:0}
	\|r_\varepsilon\|_{L_2(Q;\mathbb{C})}\leqslant c\|w^0\|_{W_2^2(Q;\mathbb{C})},
\end{equation}
{ где $c>0$ --- постоянная, зависящая только от постоянной эллиптичности $k_0$ и постоянной Липшица $L$. И имеют место следующие оценки}
\begin{align}\label{smm-0sirazh4.12}
	&\|A_\varepsilon w_1^\varepsilon-A_0w^0\|_{L_2(Q; \mathbb{C})}\leqslant c\,\varepsilon\|w^0\|_
	{W_2^2 (Q; \mathbb{C})},\quad\|A_\varepsilon w_1^\varepsilon-A_\varepsilon w_\varepsilon\|_{L_2(Q; \mathbb{C})}\leqslant c\,\varepsilon\|w^0\|_
	{W_2^2 (Q; \mathbb{C})},\\\label{smm-0sirazh4.13}
	& \|w_1^\varepsilon-w^0\|_{L_2 (Q; \mathbb{C})}\leqslant c\,\varepsilon\|w^0\|_
	{W_2^2 (Q; \mathbb{C})},\quad w_1^\varepsilon\rightharpoonup w^0 \text{\ \ в \ \  } W_2^1(Q;\mathbb{C})\text{\  при \ \ } \varepsilon\to0,
\end{align}
{ где $c>0$ --- постоянная, зависящая только от $k_0$ и $L$,
	$w_\varepsilon$ ---  решение задачи Римана -- Гильберта \eqref{smm-0sirazhf:1.8}, функция
	$w_1^\varepsilon$ определена в формуле \eqref{smm-0sirazhL:1}}.
\end{lemma}



\subsection{Усреднение и оценки погрешности усреднения}

Имеет место  следующая

\begin{theorem} { Для семейства $\{A_\varepsilon\}$ операторов краевой
	задачи Римана -- Гильберта \eqref{smm-0sirazhf:1.8} имеет место усреднение, причем коэффициенты
	усредненного оператора $A_0$, $A_0w\equiv\partial_{\overline{z}}w+\mu^0(x)\partial_zw+\nu^0(x)\partial_{\overline{z}}
	\overline w$, $w\in W_0(Q)$ --- равномерно непрерывные  по Липшицу в $\overline{Q}$ функции и они даются равенствами \eqref{smm-0sirazhdf:1.11}.}
\end{theorem}

В качестве первого приближения к решению $w_\varepsilon$  задачи
Римана -- Гильберта  для обобщенного уравнения Бельтрами \eqref{smm-0sirazhf:1.8} с локально-периодическими коэффициентами, по аналогии с первым приближением к решению  задачи Римана -- Гильберта для обобщенного уравнения Бельтрами (см. \cite{smm-8}) с периодическими коэффициентами, возьмем функцию \eqref{smm-0sirazhL:1}:
$$ w_1^\varepsilon(x)=w^0(x)+\varepsilon\left(N(x,y)\partial_zw^0(x)+M(x,y)\partial_{\bar{z}}\overline{w^0(x)}\right),$$ где теперь $w^0$ --- решение усредненной задачи $A_0w\equiv\partial_{\overline{z}}w+\mu^0(x)\partial_zw+\nu^0(x)\partial_{\overline{z}}
\overline{w}=f\in W_2^1(Q;\mathbb{C})$, $w\in W_0(Q)$. Заметим, что $w^0\in W_2^2(Q;\mathbb{C})\cap W_0(Q)$, ввиду
теоремы \ref{smm2-th2}, так как $f\in W_2^1(Q;\mathbb{C})$.

\begin{theorem}
{ Пусть $Q$ --- ограниченная односвязная область с гладкой (класса $C^2$) границей и пусть правая часть $f$ задачи Р -- Г \eqref{smm-0sirazhf:1.8} принадлежит пространству $W_2^1(Q;\mathbb{C})$, тогда для малых $\varepsilon$, $\varepsilon\leqslant \varepsilon(Q)$ имеют место оценки
	\begin{equation}\label{smm-0sirazhT6:1}
		\|w_\varepsilon-w_1^\varepsilon\|_{W_2^1 (Q; \mathbb{C})}\leqslant c\sqrt{\varepsilon}\|f
		\|_{W^1_2 (Q; \mathbb{C})}, \quad \|w_\varepsilon-w^0\|_{L_2 (Q; \mathbb{C})}\leqslant c\sqrt{\varepsilon}\|f
		\|_{W_2^1 (Q; \mathbb{C})},
	\end{equation}
	где $c>0$ --- постоянная, зависящая только от постоянной эллиптичности $k_0$,  постоянной Липшица $L$ и области $Q$.}
\end{theorem}




\begin{theorem}
{ Справедливы следующие операторные оценки усреднения задачи Римана -- Гильберта \eqref{smm-0sirazhf:1.8}
	\begin{gather}\label{smm-0sirazhOpop1}
		\left\|\left(A_\varepsilon^{-1}-A_0^{-1}\right)\partial_{\overline z}^{-1}\right\|_{L_2(Q;\,\mathbb{C})\to L_2(Q;\,\mathbb{C})}
		\leq c\,\sqrt\varepsilon,\quad
		\left\|A_\varepsilon^{-1}-A_0^{-1}\right\|_{W_0(Q)\to L_2(Q;\,\mathbb{C})}
		\leq c\,\sqrt\varepsilon,\\\label{smm-0sirazhOpop2}
		\left\|\left(A_\varepsilon^{-1}-A_0^{-1}\right)\partial_{\overline z}^{-1}-\varepsilon\Big(N^\varepsilon
		\partial_zA_0^{-1}
		\partial_{\overline z}^{-1}\right.
		+\left.M^\varepsilon\partial_{\overline z}\overline{A_0^{-1}\partial_{\overline z}^{-1}}\Big)
		\right\|_{L_2(Q;\,\mathbb{C})\to W^1_2(Q;\,\mathbb{C})}
		\leq c\,\sqrt\varepsilon,\\\label{smm-0sirazhOpop3}
		\left\|A_\varepsilon^{-1}-A_0^{-1}-\varepsilon\left(N^\varepsilon\partial_zA_0^{-1}
		+M^\varepsilon\partial_{\overline z}\overline{A_0^{-1}}\right)
		\right\|_{W^1_2(Q;\,\mathbb{C})\to W^1_2(Q;\,\mathbb{C})}
		\leq c\,\sqrt\varepsilon,
	\end{gather}
	где $c>0$ --- постоянная из \eqref{smm-0sirazhT6:1}; $\partial_{\overline z}^{-1}$ --- оператор обратный к
	оператору краевой задачи
	\eqref{smm-0sirazhf:1.8} для уравнения Коши -- Римана;
	$\overline{A_0^{-1}\partial_{\overline z}^{-1}}$ --- оператор,
	определенный равенством $\overline{A_0^{-1}\partial_{\overline z}^{-1}}{\,v}=\overline{A_0^{-1}\partial_{\overline z}^{-1}{v\,}}$,
	аналогичный смысл имеет и оператор $\overline{A_0^{-1}}$. Здесь
	$N^\varepsilon=N_1(x,\varepsilon^{-1}x)-iN_2(x,\varepsilon^{-1}x)$, $M^\varepsilon=N_1(x,\varepsilon^{-1}x)+iN_2(x,\varepsilon^{-1}x)$,  где $N_1(x,y)$, $N_2(x,y)$ --- решения задачи
	на ячейке (см. теорему \ref{smm2-th4}); $A_\varepsilon^{-1}$, $A_0^{-1}$
	--- обратные операторы к операторам соответствующих  задач Римана -- Гильберта}.
\end{theorem}

В случае оператора Бельтрами ($\nu=0$), ввиду теоремы \ref{smm2-th4}, имеем
$N^\varepsilon=2N_1^\varepsilon$, $M^\varepsilon=0$. Следовательно, корректоры в \eqref{smm-0sirazhOpop2}, \eqref{smm-0sirazhOpop3} упрощаются
и мы имеем более простые оценки.
%\begin{equation*}
%\left\|\left(A_\varepsilon^{-1}-A_0^{-1}-\varepsilon N
%\partial_zA_0^{-1}\right)
%\partial_{\overline z}^{-1}
%\right\|_{L_2(Q;\,\mathbb{C})\to W^1_2(Q;\,\mathbb{C})}
%\leq c\,\sqrt\varepsilon,
%\end{equation*}
%\begin{equation*}
%\left\|A_\varepsilon^{-1}-A_0^{-1}-\varepsilon N\partial_zA_0^{-1}
%\right\|_{W^1_2(Q;\,\mathbb{C})\to W^1_2(Q;\,\mathbb{C})}
%\leq c\,\sqrt\varepsilon.
%\end{equation*}



 \section{Заключение.}

 Задачи рассмотренные нами возникают при изучении плоско-парал\-лельных физических процессов в сильно неоднородных
 средах периодической или локально-периодической структуры.

 Получены результаты по оценке погрешности усреднения задачи Римана -- Гильберта для обобщенного уравнения Бельтрами
 с локально-периодическими коэффициентами.



