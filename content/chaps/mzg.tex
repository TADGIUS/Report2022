\chapter{Обращение $V$-преобразования Радона со степенным весом на плоскости}\label{MZG}

В 1997 году Р. Баско в работе \cite{mzg-Basko} представил преобразование Радона для пары лучей, образующих букву $V$, в попытке смоделировать формирование изображения в так называемой одномерной камере Комптона. Ось этой $V$-образной линии поворачивается вокруг точки плоскости таким образом, что ее вершина лежит на прямой (представляющей детектор <<рассеяния>>), а угол раскрытия двух лучей (представляющий угол рассеяния Комптона) является переменной данных изображения.
Трунг Т. и Нгуен М. К. (\cite{mzg-Truong}) выдвинули идею $V$-образных преобразований Радона с фиксированным направлением оси симметрии. Такие преобразования могут представлять теоретический интерес в интегральной геометрии. Они возникают в результате связанного томографического процесса пропускания-отражения. Интегральные преобразования по преломленным лучам изучаются и во многих других работах. Из недавних работ можно отметить статьи \cite{mzg-Ambartsoumian2, mzg-Kuchment} и др.

В различных статьях упоминается, что в зависимости от конструкции детектора в криволинейном интеграле могут появляться различные веса. Новизна результатов нашей работы состоит в том, что в интегралах участвуют веса, равные некоторым степеням расстояния до вершины уголка ломаной.

Пусть в круге $S_R$ радиуса $R$ с центром в начале координат дано семейство ломаных вида
$$\Gamma(\beta,d)=L_d\cup L, \, 0\leq\beta\leq 2\pi, \, 0\leq d\leq R,$$
где
$$L_d=\{(x_1, x_2)=(R-s)\omega,  0\leq s\leq d\}, \omega=(\cos\beta, \sin \beta),$$
$$L=\{(x_1, x_2)=d\omega-s(\cos (\beta+\theta), \sin (\beta+\theta)), s\geq 0.$$
Здесь $\theta$ -- острый угол.

Задача состоит в том, чтобы по заданным интегралам

\begin{equation}
\label{mzg-eq-one3}
Vf(\beta,d)=\int_{\Gamma(\beta,d)}f(x)ds
\end{equation}
функции $f(x)$ вдоль ломаных $\Gamma(\beta,d)$ определить эту функцию.

Легко заметить, в общем случае семейство ломаных $\Gamma(\beta,d)$ имеет размерность четыре. Можно, однако, ограничить данные 2D-набором, предположив, что лучи входят в круг нормально к его границе и ломаются под фиксированным углом $\theta$.

Г. Амбарцумян и С. Мун \cite{mzg-Ambartsoumian1} решили поставленную задачу методом разложений в ряды Фурье (методом Кормака).

В статье \cite{mzg-Medzhidov2} обнозначность восстановления неизвестной функции доказана, когда угол $\beta$ меняется в ограниченном угловом диапазоне: $0<\beta<\alpha_0$ и $|\pi-\beta|<\alpha_0$, где $\alpha_0$ - произвольный острый угол. При этих условиях доказана формула обращения.

В упомянутой выше работе \cite{mzg-Truong} рассматривается следующее двухпараметрическое семейство ломаных на плоскости

\begin{equation}
\label{mzg-eq-one4}
\Gamma(\xi,\varphi)=\{(x,y)=(\xi\pm r \sin \varphi, r \cos \varphi),\quad 0\leq\varphi\leq\frac{\pi}{2}, \quad \xi\in\mathbb R\},
\end{equation}

Интегральное преобразование на этом семействе ($V$-преобразование Радона) имеет вид:

$$g(\xi,\varphi)=\int_0^\infty f(\xi\pm r \sin \varphi, r \cos \varphi)dr.$$

В этой работе получена формула обращения

$$f(x,y)=\frac{1}{2\pi^2}\int_0^\infty \left( (p.v.)\int_\mathbb R d\xi\left(\frac{g'(\xi,\lambda)}{\xi-x-y\lambda}+\frac{g'(\xi,\lambda)}{\xi-x+y\lambda}\right)\right), \lambda=tg \varphi$$
(внутренний интеграл понимается в смысле главного значения).

Мы рассмотриваем аналогичное семейство ломаных в круге $S_R$:

$$L(\beta,\psi)=L_1(\beta,\psi)\cup L_2(\beta,\psi), 0\leq \beta\leq2\pi,\quad 0<\psi<\pi/2,$$
где
$$L_i(\beta,\psi)=\{x\in S_R:(x,\tau^{(i)})=R \sin \psi\},$$
$$\tau^{(1)}=(\sin(\psi-\beta), \cos(\psi-\beta)),\quad \tau^{(2)}=(\sin(\psi+\beta), -\cos(\psi+\beta)).$$

Пусть
$$g(\beta,\psi)=\int_{L(\beta,\psi)}f(x)ds.$$

Положим
$$h(\beta,r)=g\left(\beta, \arcsin \frac{r}{R}\right).$$

Пусть $f_n(\rho), \quad h_n(r)$ - коэффициенты разложений в ряды Фурье функций $f(\varphi,\rho)$ и $h(\beta,r)$, записанных в полярных координатах.
Эти коэффициенты связаны интегральным уравнением (\cite{mzg-Medzhidov1})
$$h_n(r)=4T_n\left(\frac{r}{R}\right)\int_r^{+\infty}f_n(\rho)\frac{T_n\left(\frac{r}{\rho}\right)}{\sqrt{\rho^2-r^2}}\rho d\rho,$$
где $T_n(t)=\cos (n \arccos t)$ -- полиномы Чебышева первого рода.

Решение этого интегрального уравнения имеет вид

$$f_n(\rho)=-\frac{1}{2\pi}\frac{d}{d\rho}\int_\rho^{+\infty}\frac{\rho}{r}\cdot\frac{T_n\left(\frac{r}{\rho}\right)}{T_n\left(\frac{r}{R}\right)}\frac{h_n(r)}{\sqrt{r^2-\rho^2}}dr.$$

Рассмотрим на семействе $\Gamma(\xi,\varphi)$ ломаных \eqref{mzg-eq-one4} и интегральное преобразование вида

\begin{equation}
\label{mzg-eq-one5}
Vf(\xi,\varphi)=\int_0^\infty\left( f(\xi+ r \sin \varphi, r \cos \varphi)+f(\xi-r \sin \varphi, r \cos \varphi)\right)r^k dr,
\end{equation}
заданное на ломаных этого семейства. Здесь $k$ -- неотрицательное вещественное число.

Пусть неизвестная функция $f$ имеет компактный носитель, принадлежащий верхней полуплоскости.

Мы решаем задачу восстановления функции $f$ по заданной функции $g(\xi,\varphi)=Vf(\xi,\varphi)$ для значений переменных $\xi$ и $\varphi$, $\xi\in\mathbb R,\, 0<\varphi<\frac{\pi}{2}$.

Похожая задача с другой весовой функцией решена в \cite{mzg-Kuchment}.

Формула обращения, полученная в работе \cite{mzg-Medzhidov2}, имеет вид

\begin{equation}
\label{mzg-eq-one8}
f(x,y)=\frac{1}{2\pi}\frac{1}{|x|^k y^k}\int_0^\infty \frac{\cos (\tau xy) }{\left(\sqrt{1+\tau^2}\right)^{k+1}}(-L)^{\frac{k+1}{2}}g(x,\tau)d\tau;
\end{equation}
здесь $Lh(\xi,\tau)=\frac{\partial^2}{\partial\xi^2 }h(\xi,\tau)$, а дробная степень оператора Лапласа $L$ определяется формулой

$$(-L)^lh(\xi,\tau)=I^{-2l}h(\xi,\tau),$$
где $I^\alpha$ -- потенциал Рисса:

$$\widetilde{\left(I^\alpha h\right)}(p,\tau)=|p|^{2l}\tilde h(p,\tau).$$


%\section{Заключение}
%Решена задача обращения $V$-преобразования, или обобщенного преобразования Радона на плоскости. $V$-образные ломаные, по которым берутся интегралы, имеют вершину внутри области восстановления. Полученная формула обобщает известные формулы на случай весовой степенной функции.  Решена также задача обращения интегрального преобразования на семействе ломаных, входящие в круг нормально к окружности и преломляющихся внутри круга.
%Методы решения приведенных задач могут быть применены при обращении $V$-преобразований векторных и тензорных полей.





