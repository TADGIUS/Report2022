\chapter{Устойчивость систем дифференциальных и разностных уравнений Ито}\label{KRI}

%Краткая аннотация важнейших и основных научных результатов,
%полученных в 2022 г.
%\smallskip
%
%1. Изучены вопросы устойчивости для  нового класса стохастических
%систем с запаздыванием, содержащий одновременно компоненты с
%непрерывным и дискретным временем. Предложен и обоснован
%модифицированный метод регуляризации для анализа различных видов
%устойчивости таких систем, основанный на выборе вспомогательного
%уравнения и применении теории положительно обратимых матриц.
%Разработка этого метода для детерминированных
%функционально-дифференциальных уравнений была осуществлена Н.В.
%Азбелевым и его учениками. Получены достаточные условия моментной
%устойчивости решений как в терминах положительной обратимости
%матриц, построенных по параметрам этих систем, так и в терминах
%коэффициентов. Проверены выполнимость этих условий для конкретных
%систем уравнены. Результаты исследований опубликованы в работах
%\cite{kri-1,kri-2,kri-6}.
%
%
%2. Исследована глобальная  моментная устойчивость решений систем
%нелинейных дифференциальных уравнений Ито с запаздываниями.
%Предложен и обоснован модифицированный метод регуляризации для
%анализа различных видов устойчивости таких систем, основанный на
%выборе вспомогательного уравнения и применении теории положительно
%обратимых матриц. Разработка этого метода для детерминированных
%функционально-дифференциальных уравнений была осуществлена Н.В.
%Азбелевым и его учениками. Получены достаточные условия моментной
%устойчивости решений в терминах коэффициентов как для достаточно
%общих, так и конкретных классов уравнений Ито. Результаты
%исследований  опубликованы в работе \cite{kri-3} и приняты к опубликованию в
%журнале  Functional Differential Equations \cite{kri-4}.
%\smallskip
%
%3. Исследованы вопросы  экспоненциональной $p$-устойчивости
%систем линейных дифференциальных уравнений Ито с ограниченными
%запаздываниями, используя теорию положительно обратимых матриц. Для
%упомянутых систем уравнений получены достаточные условия
%экспоненциональной $p$-устойчивости в терминах положительной
%обратимости матрица, построенной по параметрам этих систем.
%Проверена выполнимость этих условий для конкретных систем уравнений.
%Результаты исследований опубликованы в работе \cite{kri-5}.
%\smallskip



%\smallskip

\section{Постановка задачи.}\label{kri-s1}
В дальнейшем систематически
используются следующие обозначения:
%\begin{itemize}
-- ($\Omega$, ${\mathcal F}$, $({\mathcal F})_{t\ge0}$, $P$)
-- стохастический базис, где $ \Omega $ -- множество элементарных
событий, ${\mathcal F}$ -- $\sigma$--алгебра событий на $\Omega$,
$({\mathcal F})_{t\ge0}$-- непрерывный справа поток $\sigma$--алгебр
на $\Omega$, $P$ -- полная вероятностная мера на $\mathcal{F}$;
\\ -- $E$ --
символ математического ожидания на этом пространстве;\\
-- $k^n$ -- линейное пространство  $n$--мерных
${\mathcal F}_0$--измеримых случайных величин;\\
  --  $\mathcal B_i$ ($i=2,\dots,m$)
-- независимые стандартные скалярные винеровские процессы;\\
-- $D^n$ -- линейное пространство $n$--мерных прогрессивно измеримых
случайных процессов на $[0, \infty )$, траектории которых п.н.
непрерывны справа и имеют пределы слева;\\
-- $L^n$ -- линейное пространство $n$--мерных случайных
процессов на $(-\infty , 0)$, которые не зависят от винеровских процессов
$\mathcal B_i$ ($i=2,\dots,m$) и имеют п.н. ограниченные в
существенном траектории;\\
-- $|.|$ -- некоторая норма в $R^n$;\\
 -- $||.||$ -- норма $m\times n$--матриц, согласованная с нормой в
$R^n$;\\
 -- $\bar E$ -- единичная $m \times m$--матрица;\\
 -- $e$ --
$n$--мерный вектор-столбец, все элементы которого равны единице;\\
  -- $N$ --
множество натуральных чисел; $N_+ = \{0\}\cup N$;\\
-- $||.||_X$ -- норма в нормированном
пространстве $X$;\\
-- $\mu $ -- мера Лебега на $[0, \infty)$;\\
%\item $p\ge 1$ -- некоторое конечное число,
-- $[t]$
-- целая часть числа $t$; \\
%$p$ -- действительное число;
%-- $1 \leq p < \infty $;
--  $\gamma :[0, \infty) \rightarrow R^1 $ -- некоторая
положительная непрерывная функция;\\
--
$ M_q^{\gamma } = \left\{x: x \in D^n, ||x||_{M_q^\gamma }
 \mathrel
 {\mathop {=} \limits ^{def}} \mathrel {\mathop {\sup}
 \limits _{t
 \ge 0}} (E|\gamma (t)x(t)|^q)^{1/q} < \infty \right\},  \ \ \
 M_q^1 =  M_q$ ($1\le q <\infty$);\\
 -- $k_q^n = \left\{\alpha: \alpha \in k^n, ||\alpha ||_{k_q^n}
 \mathrel
 {\mathop {=} \limits ^{def}} (E|\alpha |^q)^{1/q} < \infty
 \right \}$ ($1\le q <\infty$);\\
 -- $ L_q^n = \left\{\varphi: \varphi \in L^n,
 ||\varphi||_{L_q^n}
 \mathrel {\mathop {=} \limits ^{def}} \mathrel {\mathop
 {v r a i \sup}
 \limits _{\varsigma < 0}}(E|\varphi (\varsigma ) |^q)^{1/q} < \infty
 \right\}$ ($1\le q <\infty$).
  %\end{itemize}

%$c_p$ -- положительное число, зависящее от $p$ ([9], с. 65) и используемое в оценке \eqref{kri-6};

Для описания класса непрерывно-дискретных систем будет зафиксировано
натуральное число $l$ ($1 \le l < n$), для которого $x_1(t), \dots,
x_l(t)$ $ (t \ge 0)$ будут определять компоненты вектора состояний
системы с непрерывным временем, тогда как $x_{l+1}(s), \dots, x_n(s)$
$(s \in N_+)$ будут задавать его компоненты с дискретным временем. В
векторных обозначениях это будет выглядеть так: $\hat x(t) = col
(x_1(t), \dots, x_l(t))$ $(t \ge 0)$, $\tilde x(s) = col(x_{l+1}(s),
\dots, x_n(s))$ $ (s \in N_+)$ и $x(t) = col(\hat x(t), \tilde x([t]))
= col (x_1(t), \dots, x_l(t), x_{l+1}([t]), \dots, x_n([t]))$ $ (t \ge
0)$.


Исследованы вопросы моментной устойчивости решений системы линейных
дифференциальных и разностных уравнений Ито с последействием вида
\begin{equation}\label{kri-1}
\begin{array}{crl}
 d\hat x(t) = - \sum
 \limits_{j=1}^{m_1}A_{1j}(t)x(h_{1j}(t))dt +
 \sum \limits_{i=2}^m\sum \limits_{j=1}^{m_i}
 A_{ij}(t)x(h_{ij}(t))d\mathcal B_i(t) \, \, (t \ge 0),\\
\tilde x(s+1) = \tilde x(s) - \sum \limits _{j=-\infty }^{s}
A_{1}(s,j)x(j)h + \\
\sum \limits _{i=2}^{m}\sum \limits _{j=-\infty }^{s}
A_{i}(s,j)x(j)(\mathcal B_i((s+1)h) - \mathcal B_i(sh))\, \,  (s
\in N_+)\\
\end{array}
\end{equation}
по начальным данным
\begin{equation}\label{kri-1a}
x(\varsigma)=\varphi (\varsigma) {\,} {\,} (\varsigma < 0),
\end{equation}
\begin{equation}\label{kri-1b}
x(0) = b.
\end{equation}

Здесь\\
%\begin{itemize}
\noindent
  -- $x(t) = col (x_1(t), \dots, x_l(t), x_{l+1}([t]), \dots,
x_n([t]))$ $ (t \ge 0)$  -- $n$--мерный неизвестный случайный
процесс;\\
  -- $A_{ij}(t)$ -- $l \times n$--матрицы ($i = 1,\dots,m$, $j =
1,\dots,m_i$), причём элементами матриц $A_{1j}(t)$, $j = 1,\dots,m_1$
являются прогрессивно измеримые скалярные случайные процессы на
интервале $[0, \infty)$ с почти наверно (п.н.) локально суммируемыми
траекториями, а элементами матриц $A_{ij}(t)$, $i = 2,\dots,m$, $j =
1,\dots,m_i$ являются прогрессивно измеримые скалярные случайные
процессы на $[0, \infty)$, траектории которых п.н. локально
суммируемы с квадратом;\\
  -- $ h_{ij}(t)$, $i = 1,\dots,m$, $j = 1,\dots,m_i$ -- измеримые по
Борелю функции, заданные на $[0, \infty)$ и такие, что $h_{ij}(t)
\leq \ t {\,} {\,} (t \geq 0)$ $\mu $--почти всюду, $i = 1,\dots,m$,
$j = 1,\dots,m_i$;\\
  -- $h$ -- положительное действительное число;\\
  -- $A_i(s,j)$ -- $(n-l)\times n$--матрицы, элементами которых являются
${\mathcal F}_s$--измеримые скалярные случайные величины при
$i=1,\dots,m$, $s\in N_+$, $j=-\infty ,\dots,s$;\\
  -- $\varphi (\varsigma ) = col (\varphi _1 (\varsigma),\dots, \varphi
_l (\varsigma), \varphi _{l+1} ([\varsigma]), \dots, \varphi _n
([\varsigma])) \, \, (\varsigma < 0)$ -- ${\mathcal F}_0$--измеримый
$n$--мерный случайный процесс с п.н. ограниченными в существенном
траекториями;\\
  -- $b = col (b_1,.., b_n)$ -- ${\mathcal F}_0$--измеримая
$n$--мерная случайная величина, т.е. $b \in k^n$.
%\end{itemize}


\begin{definition}\label{kri-def1} Под решением задачи \eqref{kri-1}, \eqref{kri-1a}, \eqref{kri-1b}
понимается случайный процесс $x(t) = col (x_1(t), \dots, x_l(t),
x_{l+1}([t]), \dots, x_n([t]))\, \,(t \in (-\infty , \infty))$,
являющийся прогрессивно измеримым при  $t \ge 0$ и удовлетворяющий
соотношениям $x(\varsigma)=\varphi (\varsigma)\,\, (\varsigma < 0)$,
$x(0) = col (\hat x(0), \tilde x(0)) = b$, а также $P$-почти всюду
системе
$$
\begin{array}{crl}
 \hat x(t) =  \hat x(0)- \sum
 \limits_{j=1}^{m_1}\int \limits _0^tA_{1j}(\varsigma)x(h_{1j}(\varsigma))d\varsigma +
 \sum \limits_{i=2}^m\sum \limits_{j=1}^{m_i}
 \int \limits _0^t A_{ij}(\varsigma)x(h_{ij}(\varsigma))d\mathcal B_i(\varsigma) \,\, (t \ge
 0),\\
 \tilde x(s+1) = \tilde x(s) - \sum \limits _{j=-\infty }^{s}
A_{1}(s,j)x(j)h +
\end{array}
$$
$$
\begin{array}{crl}
\sum \limits _{i=2}^{m}\sum \limits _{j=-\infty }^{s}
A_{i}(s,j)x(j)(\mathcal B_i((s+1)h) - \mathcal B_i(sh)) \,\, (s
\in N_+),\\
\end{array}
$$
где первый интеграл -- это интеграл Лебега, а второй -- интеграл
Ито.
\end{definition}

Используя технику сжимающих отображений, можно убедиться, что при
сделанных предположениях задача \eqref{kri-1}, \eqref{kri-1a}, \eqref{kri-1b} имеет единственное
решение. В частности, при нулевых начальных условиях эта задача
имеет только тривиальное (т.е. нулевое) решение.

Обозначим решение задачи \eqref{kri-1}, \eqref{kri-1a}, \eqref{kri-1b} через $x(t, b,
\varphi)\,\,(t \in (-\infty , \infty ) )$. Очевидно, $x(., b,
\varphi) $ $ \in D^n$.

Пусть $1\le q<\infty$.

\begin{definition}\label{kri-def2} Систему \eqref{kri-1} назовем:\\
%\begin{itemize}
\noindent
  -- \textit{$q$-устойчивой} по начальным данным,
 если
 для любого $\epsilon > 0$ найдется такое $\delta (\epsilon)
 > 0$,
 что при всех $b \in k^n_q$, $\varphi \in L^n_q$ и
 $\|b\|_{k^n_q} + \|\varphi \|_{L^n_q} < \delta (\epsilon)$
 будет
 выполнено неравенство $(E|x(t, b, \varphi)|^q)^{1/q} \le
 \epsilon $
 для любого $t \ge 0$;\\
  -- \textit{ асимптотически $q$-устойчивой }относительно
 начальных данных, если
 оно $q$-устойчиво, и, кроме того, для всех $b \in k^n_q$,
 $\varphi \in L^n_q$ и $\|b\|_{k^n_q} + \|\varphi \|_{L^n_q}
 <
 \delta (\epsilon)$ будет выполнено соотношение $\lim \limits_{t  \rightarrow
 +\infty
 }(E|x(t, b, \varphi)|^q)^{1/q} = 0$;\\
-- экспоненциально \textit{ $q$-устойчивой}  относительно
 начальных
 данных, если существуют положительные числа $c, \lambda$
 такие, что
 для решения $x(t, b, \varphi)$ $(t \in (-\infty , \infty ))$ задачи \eqref{kri-1}, \eqref{kri-1a}, \eqref{kri-1b}
 при всех $b \in k^n_q$, $\varphi \in L^n_q$ выполнено неравенство
 $$
 (E|x(t, b, \varphi)|^q)^{1/q} \leq c\exp \{-\lambda
 t\}\left(\|b\|_{k_q^n} + \|\varphi \|_{L^n_q}\right){\,}
 {\,} (t \geq 0).
 $$
\end{definition}

Изучена \textit{экспоненциальная $q$-устойчивость}, однако
используемый метод регуляризации применим в гораздо более общем
контексте. Поскольку все виды устойчивости из определения \ref{kri-def2}
представляют практический интерес, метод исследования будет ниже
изложен в максимально общем виде.

\smallskip

\section{Метод регуляризации в задачах устойчивости}\label{kri-s2}
В этом пункте излагается основной метод исследования, применяемый в
настоящем исследовании и адаптированный к непрерывно-дискретным стохастическим
системам. Для этой цели свойства стохастической устойчивости из
определения \ref{kri-def2} будут переформулированы в более удобном виде.

Следующее свойство объединяет все виды стохастической устойчивости
из определения \ref{kri-def2}. Пусть $1\le q <\infty$.

\begin{definition}\label{kri-def3}  Систему \eqref{kri-1} назовем  $M_q^\gamma
 $-устойчивой, если при любых $b \in k^n_q$, $\varphi \in L^n_q$ для
 решения $x(., b, \varphi)$ $(t \in (-\infty , \infty ))$ задачи \eqref{kri-1}, \eqref{kri-1a},
 \eqref{kri-1b}  на интервале
 $[0,\infty)$ выполняются соотношение $$x(., b, \varphi) \in M_q^\gamma$$ и
 неравенство
 $$
 \|x(., b, \varphi)\|_{M_q^\gamma} \le c\left(\|b\|_{k^n_q} +
 \|\varphi \|_{L^n_q}\right) \
 $$
 для некоторого положительного числа $c$.
\end{definition}

 Непосредственное сравнение Определений \ref{kri-def2} и \ref{kri-def3} приводит к следующим очевидным выводам,
 на которых основан метод регуляризации (т.е. $W$-метод Н.В. Азбелева) для непрерывно-дискретных стохастических систем:\\
%\begin{itemize}
\noindent
  -- из  $M_q$-устойчивости системы \eqref{kri-1}
 следует $q$-устойчивость этой же системы относительно
 начальных
 данных;\\
  -- из $M_q^\gamma $-устойчивости системы \eqref{kri-1}
 (где $\gamma (t) \ge \delta > 0$ $(t \ge 0)$ и $\lim \limits
 _{t
 \rightarrow +\infty } \gamma (t) = +\infty )$ следует
 асимптотическая $q$-устойчивость этой же системы
 относительно
 начальных данных;\\
  -- из $M_q^\gamma $-устойчивости системы \eqref{kri-1}
 (где $\gamma (t) = \exp \{\lambda t\}$ $(t \geq 0)$, $\lambda$ --
 некоторое
 положительное число) следует экспоненциальная
 $q$-устойчивость
 этой же системы по начальным данным.


Для дальнейшего описания метода регуляризации рассмотрим
вспомогательную (<<модельную>>) непрерывно-дискретную систему линейных
стохастических уравнений вида
\begin{equation}\label{kri-2}
\begin{array}{crl}
 d\hat x(t) =  (- B(t)\hat x(t)+f_1(t))dt + \sum\limits_{i=2}^{m}f_i(t)d\mathcal
B_i(t)\ \ (t \ge 0),\\
\tilde x(s+1) = \tilde x(s) + (- \bar B(s)\tilde x(s)+ g_1(s))h +
\\
+ \sum\limits_{i=2}^{m}g_i(s)(\mathcal B_i((s+1)h)- \mathcal
B_i((s+1)h))\ \
(s \in N_+),
\end{array}
\end{equation}
где  $B(t)$-- $l\times l$--матрица, элементами которой являются
прогрессивно измеримые случайные процессы на интервале $[0, \infty
)$ с п.н. локально суммируемыми траекториями, $f_1(t)$ --
$l$--мерный прогрессивно измеримый случайный процесс на $[0, \infty
)$ с п.н. локально суммируемыми траекториями, $f_i(t), i=2,\dots,m$ --
$l$--мерные прогрессивно измеримые случайные процессы на $[0, \infty
)$ с п.н. локально суммируемыми с квадратом траекториями, $\bar B(s)$ --
$(n-l)\times (n-l)$--матрица, элементами которой являются ${\mathcal
F}_s$--измеримые скалярные случайные величины ($s\in N_+$), $g_i(s), i=1,\dots,m$
-- $(n-l)$--мерные ${\mathcal F}_s$ --измеримые случайные величины
($s\in N_+$), $h>0$ -- константа из уравнения \eqref{kri-1}.


Справедливость следующей леммы непосредственно следует из известных
формул представлений решений  для линейных обыкновенных неоднородных
дифференциальных и разностных уравнений.

\begin{lemma}\label{kri-lem1} Для решений $x(t)$ системы \eqref{kri-2}  имеет место представление
$$
\begin{array}{crl}
\hat  x(t) = \hat X(t,0)\hat x(0) + \int \limits _0^t\hat X(t,
\varsigma)f(\varsigma)d\varsigma +\sum\limits_{i=2}^{m} \int \limits _0^t\hat X(t,
\varsigma)f_i(\varsigma)dB_i(\varsigma)\, \, (t \ge
0),\\
\tilde x(s) = \tilde X(s,0)\tilde x(0) + \sum \limits _{\tau =0
}^{s-1} \tilde X(s,\tau+1)g_1(\tau)h + \sum\limits_{i=2}^{m} \sum
\limits _{\tau =0 }^{s-1} \tilde X(s,\tau+1)g_i(\tau) \int \limits
_{\tau h}^{(\tau +1)h}  dB_i(\varsigma)\, \, (s \in N_+),
\end{array}
$$
где $\hat X(t, \varsigma) \, (t \ge 0, 0 \leq \varsigma \leq t)$ --
$l\times l$--матрица, столбцы которой являются решениями системы $
d\hat x(t) = - B(t)\hat x(t)dt \, \, (t \ge 0)$, причём $ \hat
X(t,t)\,\, (t \geq 0)$ -- единичная матрица размерности  $l\times
l$, а $\tilde  X(s,\tau)$ $(s, \tau \in  N_+, 0 \le \tau \le s)$ --
$(n-l) \times (n-l)$--матрица, столбцы которой являются решениями
системы $\tilde x(s+1) = \tilde x(s) + \bar B(s)\tilde x(s)h \, \,
(s \in N_+)$, причём $\tilde X(s,s) (s \in N_+)$ -- единичная
матрица размерности $(n-l) \times (n-l)$.
\end{lemma}


Используя вспомогательную систему \eqref{kri-2} и лемму \ref{kri-lem1}, перепишем задачу
\eqref{kri-1}, \eqref{kri-1a}, \eqref{kri-1b}
 в следующем эквивалентном виде:
\begin{equation}\label{kri-3}
\begin{array}{crl}
\bar x(t) = X(t)b + (\Theta \bar x)(t) + (K \varphi)(t) \ \ (t \ge
0),
\end{array}
\end{equation}
где  $X(t)$ -- блочно-диагональная матрица, у которой на главной
диагонали находятся матрицы $\hat X(t, 0)$ и $\tilde X([t],0)$, а
вне неё -- нулевые матрицы $\hat 0$ и $\tilde 0$, имеющие
размерности $l\times (n-l)$ и $(n-l)\times l$, соответственно.
Операторы $\Theta$ и $K$ имеют следующее представление:
$$
\begin{array}{crl}
(\Theta \bar x)(t) = col \left(\int \limits _0^t\hat X(t,
\varsigma)\left(B(\varsigma)\hat {\bar{x}}(\varsigma) - \sum
\limits_{j=1}^{m_1}A_{1j}(\varsigma) \bar x(h_{1j}(\varsigma))\right) d\varsigma +
\right. \\
 +
 \sum \limits_{i=2}^m\sum \limits_{j=1}^{m_i}\int \limits
_0^t\hat X(t, \varsigma)A_{ij}(\varsigma)\bar
x(h_{ij}(\varsigma))d\mathcal
B_i(\varsigma) - \\
\sum \limits _{\tau=0 }^{[t]-1}\tilde X([t],\tau
 + 1)\left(\bar B(\tau )\tilde {\bar x}(\tau) - \sum \limits _{j=0
}^{\tau}A_{1}(\tau,j)\bar x(j)\right) h +
\\ \left.
+ \sum \limits
_{i=2}^m\sum \limits _{\tau=0 }^{[t]-1}\tilde X([t],\tau + 1)\sum
\limits _{j=0 }^{\tau} A_{i}(\tau,j)\bar x(j) \int \limits
_{\tau h}^{(\tau +1)h} d\mathcal B_i(\zeta) \right), \\

(K\varphi )(t)
= col\left(- \sum \limits_{j=1}^{m_1}\int \limits _0^t\hat X(t,
\varsigma)A_{1j}(\varsigma)\bar \varphi
(h_{1j}(\varsigma))d\varsigma + \sum \limits_{i=2}^m\sum
\limits_{j=1}^{m_i}\int \limits _0^t\hat X(t,
\varsigma)A_{ij}(\varsigma)\bar \varphi (h_{ij}(\varsigma))d\mathcal
B_i(\varsigma)- \right.\\
\left. \sum \limits _{\tau=0 }^{[t]-1}\tilde X([t],\tau +1)\sum
\limits _{j=-\infty }^{-1}A_{1}(\tau,j)\varphi (j)h + \sum \limits
_{i=2}^{m}\sum \limits _{\tau=0 }^{[t]-1}\tilde X([t],\tau +1)\sum
\limits _{j=-\infty }^{-1} A_{i}(\tau,j)\varphi(j)\int \limits
_{\tau h}^{(\tau +1)h}d\mathcal B_i(\zeta)\right),
\end{array}
$$
$\bar x(t) = col (\bar x_1(t), \dots, \bar x_l(t), \bar x_{l+1}([t]),
\dots, \bar x_n([t]))$ -- неизвестный $n$--мерный случайный процесс на
$(-\infty, \infty)$ такой, что $\bar x(t) = 0$ при $t < 0$ и $\bar
x(t) = x(t)$ при $t \geq 0$, $$\hat {\bar x}(t) = col (\bar x_1(t),
\dots, \bar x_l(t)), \tilde {\bar x}(t) = col (\bar x_{l+1}([t]), \dots,
\bar x_n([t])),$$ а $\bar  \varphi (t)$ -- известные $n$--мерные
случайные процессы на $(-\infty,, \infty)$, причём $\bar \varphi(t)
= \varphi (t)$ при
 $t < 0$ и $\bar \varphi(t) = 0$ при $t \geq 0$. Этот способ представления решений систем с запаздыванием,
заданных на $(-\infty, \infty)$, в виде уравнений на полуоси $[0,
\infty)$, широко используется в теории
функционально-дифференциальных уравнений.

\begin{theorem}\label{kri-th1}Пусть $1\le q < \infty $ и существует
система \eqref{kri-2} такая, что при любых $b \in k^n_q$, $\varphi \in L^n_q$,
$\bar x \in M_q^\gamma $ для системы \eqref{kri-3} на интервале $[0, \infty)$
справедливы оценки
$$
\| Xb\|_{M_q^\gamma} \le c_1\|b\|_{k_q^n}, \ \ \ \|\Theta \bar
x\|_{M_q^\gamma} \le c_2\|\bar x\|_{M_q^\gamma}, \ \ \  \|K\varphi
\|_{M_q^\gamma} \le c_3  \|\varphi \|_{L^n_q},
$$
где $c_1, c_2, c_3$ -- некоторые положительные числа, причём $c_2 <
1$. Тогда система \eqref{kri-1} $M_q^\gamma$-устойчива.
\end{theorem}

Теорему \ref{kri-th1} можно использовать для получения достаточных условий
устойчивости системы \eqref{kri-1} в терминах параметров этой системы, как это
делается в классической версии $W$-метода. Однако такие условия
получаются более точными, если использовать покомпонентные оценки
решений. Ниже предлагается, поэтому, улучшенный метод регуляризации
для случая непрерывно-дискретных стохастических систем с
последействием.

\begin{definition}\label{kri-def4} Обратимая матрица $B = (b_{ij})^m_{i,j=1}$
называется положительно обратимой, если все элементы матрицы
$B^{-1}$ положительны.
\end{definition}

Как известно, матрица $B$ положительно обратима, если $b_{ij} \leq
0$ при $i, j = 1,\dots,m$, $i\neq j$ и выполнено одно из следующих
условий:
\begin{enumerate}
  \item  Все диагональные миноры матрицы $B$ положительны.
  \item Существуют
$\xi _i>0$, $i = 1,\dots,m$ такие, что $\xi_i b_{ii} > \sum \limits
_{j=1\\, i \neq j}^m\xi_j |b_{ij}|$, $i = 1,\dots,m$.
  \item Существуют
$\xi _i>0$, $i = 1,\dots,m$ такие, что $\xi_j b_{jj} > \sum \limits
_{i=1\\, i \neq j}^m\xi_i |b_{ij}|$, $j = 1,\dots,m$.
\end{enumerate}

В частности, если положить $\xi _i = 1$, $i = 1,\dots,m$, то мы
получим класс матриц со строгим диагональным преобладанием и
неположительными внедиагональными элементами.
%%%%%%%%%%%%%%%%%

Для случайного процесса $\bar x(t) = col(\bar x_1(t)$, $\dots$, $\bar
x_l(t)$, $\bar x_{l+1}([t]),\dots,\bar x_n([t]))$ и константы $1 \leq q
< \infty $ введем обозначение $ \bar x^\gamma (q) = col (\bar
x_1^\gamma (q),\dots,\bar x_n^\gamma (q))$, где \\$\bar x_i^\gamma (q) =
\sup \limits _{t \geq 0}\left(E|\gamma (t) \bar x_i(t)|^{q}\right)^{1/q}$ при   $i = 1,\dots,l$ и $\bar x_i^\gamma (q) = \sup \limits
_{t \geq 0}\left(E|\gamma (t)\bar x_i([t])|^{q}\right)^{1/q}$ при
$i = l+1,\dots,n$,

Пусть для некоторых $1\le q < \infty $ и положительной непрерывной
функции $\gamma :[0, \infty) \rightarrow R^1 $ нам удалось с помощью
покомпонентных оценок решений системы \eqref{kri-3}, получить матричное
неравенство следующего вида:
\begin{equation}\label{kri-4}
\bar E\bar x^\gamma (q) \leq C\bar x^\gamma (q) + \bar
c\|b\|_{k^n_{q}}e + \hat c \|\varphi \|_{L^n_q} e,
\end{equation}
где $C$ -- некоторая неотрицательная матрица размерности $n\times
n$, а $\bar c, \ \hat c$ -- некоторые неотрицательные числа. Тогда
справедлива следующая теорема.

\begin{theorem}\label{kri-th2} Пусть существует вспомогательная система
\eqref{kri-2} такая, что в неравенстве \eqref{kri-4} матрица $\bar E - C$ является
положительно обратимой. Тогда система \eqref{kri-1} $M_q^\gamma$-устой\-чи\-ва.
\end{theorem}

На основе теоремы \ref{kri-th2} в следующем параграфе будут получены достаточные
условия моментной устойчивости системы \eqref{kri-1} по начальным данным в
терминах параметров этой системы.

\section{Достаточные условия \texorpdfstring{$M_q^\gamma$}{Mq}-устойчивости}

В этом
пункте изучается задача устойчивости системы \eqref{kri-1} в смысле
Определения \ref{kri-def3}. При этом рассматриваются пространства  $M_q^\gamma$ с
весом и без него (т.е. $\gamma=1$). В последнем случае система \eqref{kri-1}
будет устойчива в смысле определения \ref{kri-def2}, т.е. по начальным данным.
$M_q^\gamma$-устойчивость с экспоненциальным весом $\gamma$ будет
использована в четвертом пункте для доказательства признаков
экспоненциальной моментной устойчивости системы \eqref{kri-1} по начальным
данным. В дальнейшем будем считать, что $1 \leq p < \infty $.

Сформулируем две леммы, которые понадобятся нам в дальнейшем.

\begin{lemma}\label{kri-lem2} Пусть $f(\varsigma )$ -- скалярный прогрессивно
измеримый  случайный процесс, интегрируемый по винеровскому процессу
$\mathcal B(\varsigma)$ на отрезке $[0, t]$. Тогда справедливо
неравенство
\begin{equation}\label{kri-6}
\left(E\left|\int \limits _0^tf(\varsigma )d\mathcal B(\varsigma
)\right|^{2p}\right)^{1/(2p)} \leq c_p \left(E\left(\int \limits
_0^t|f(\varsigma )|^2d\varsigma\right)^p\right)^{1/(2p)},
\end{equation}
где $c_p$ -- некоторое число, зависящее от $p$.
\end{lemma}

Справедливость неравенства \eqref{kri-6} следует из известного неравенства,
где указаны и конкретные оценки для $c_p$.

\begin{lemma}\label{kri-lem3} Пусть $g(\varsigma)$ -- скалярная функция на $[0,
\infty)$, квадрат которой локально суммируем, а $f(\varsigma)$ --
скалярный случайный процесс такой, что $\sup \limits _{\varsigma
\geq 0}(E|f(\varsigma )|^{2p})^{1/2p} < \infty$. Тогда справедливы
следующие неравенства
\begin{equation}\label{kri-7}
\sup \limits _{t \geq 0}\left(E\left|\int \limits
_0^tg(\varsigma)f(\varsigma)d\varsigma\right|^{2p}\right)^{1/(2p)}
\leq \sup \limits _{t \geq 0}\left(\int \limits
_0^t|g(\varsigma)|d\varsigma\right)\sup \limits _{\varsigma \geq
0}\left(E\left|f(\varsigma)\right|^{2p}\right)^{1/(2p)},
\end{equation}
\begin{equation}\label{kri-8}
\sup \limits _{t \geq 0}\left(E|\int \limits
_0^tg(\varsigma)^2f(\varsigma)^2d\varsigma|^{p}\right)^{1/(2p)} \leq
\sup \limits _{t \geq 0}\left(\int \limits
_0^tg(\varsigma)^2d\varsigma\right)^{1/2}\sup \limits _{\varsigma
\geq 0}\left(E\left|f(\varsigma)\right|^{2p}\right)^{1/(2p)}.
\end{equation}
\end{lemma}

В дальнейшем используются  обозначения, введенные в предыдущих
параграфах. Кроме этого, элементы матрицы $A_{ij}(t)$ из системы \eqref{kri-1}
обозначаются ниже $a^{ij}_{kr}(t), k =1,\dots,l, r = 1,\dots,n$  при $i
= 1,\dots,m$, $j = 1,\dots,m_i$, $t \geq 0$, а элементы матрицы
$A_i(s,j)$ из этой системы обозначаются $a^i_{kr}(s,j), k =
l+1,\dots,n, r = 1,\dots,n$ при $i=1,\dots,m$, $s\in N_+$,
$j=-\infty,\dots,s$.

Предположим выполнение следующих условий для системы \eqref{kri-1}:\\
%\begin{itemize}
\noindent
  -- существуют  суммируемые функции $\bar
a^{1j}_{kr}(t)\,\, (t\geq 0)$, $j = 1,\dots,m_1, k =1,\dots,l, r =
1,\dots,n$ и суммируемые с квадратом функции $\bar a^{ij}_{kr}(t)\,\,
(t\geq 0)$, $i = 2,\dots,m, j = 1,\dots,m_i, k =1,\dots,l, r = 1,\dots,n$
такие, что $|a^{ij}_{kr}(t)|\leq \bar a^{ij}_{kr}(t)\,\, (t\geq 0)$
$P\times\mu$--почти всюду при $i = 1, \dots, m, j = 1,\dots,m_i, k
=1,\dots,l, r = 1,\dots,n$;\\
  -- существуют
неотрицательные числа $\bar a^i_{kr}(s,j),i=1,\dots,m, k = l+1, \dots,
n, r = 1,\dots,n, s\in N_+, j=-\infty,\dots,s$ такие, что
$|a^i_{kr}(s,j)| \leq \bar a^i_{kr}(s,j)$ $P$--почти всюду при
$i=1,\dots,m, k = l+1, \dots, n, r = 1,\dots,n, s\in N_+, j=-\infty,\dots,s
$;\\
  -- $ \sum \limits _{\tau=0 }^{\infty }\sum \limits _{j = -\infty
}^{\tau}\bar a^{i}_{kr}(\tau,j) < \infty$ при $i=1,\dots,m, k = l+1,
\dots, n, r = 1,\dots,n$.

%\end{itemize}
Определим элементы $n\times n$--матрицы $C$ следующим образом:
$$
\begin{array}{crl}
c_{kr} = \sum \limits_{j=1}^{m_1}\int \limits _0^{\infty } \bar
a^{1j}_{kr}(\varsigma)d\varsigma + c_p \sum \limits_{i=2}^m \sum
\limits_{j=1}^{m_i}\left(\int \limits _0^{\infty } \bar
a^{ij}_{kr}(\varsigma )^2d\varsigma \right)^{1/2},  k
= 1, \dots ,l, r = 1, \dots, n,\\
c_{kr} = h\sum \limits _{\tau=0 }^{\infty }\sum \limits _{j=0
}^{\tau}\bar a^{1}_{kr}(\tau,j) + c_p \sqrt{h}\sum \limits
_{i=2}^{m}\sum \limits _{\tau=0 }^{\infty }\sum \limits _{j=0
}^{\tau} \bar a^{i}_{kr}(\tau,j), k = l+1, \dots,n, r = 1, \dots, n.
\end{array}
$$
В этих обозначениях справедлива

\begin{theorem}\label{kri-th3} Если матрица $\bar E - C$ положительно
обратима, то система \eqref{kri-1} $M_{2p}$--устой\-чи\-ва.
\end{theorem}

Предположим, что для системы \eqref{kri-1}
выполнены условия:\\
%\begin{itemize}
\noindent
  -- $h_{11}(t) = t \,\,
(t \geq 0)$;\\
  -- диагональные элементы матриц $A_{11}(t)\,\, (t \geq
0)$, $A_1(s,s) \,\, (s \in N_+)$ имеют вид $a_{kk}^{11}(t) + \lambda
_k \,\, (t \geq 0), k=1, \dots, l$ и $a_{kk}^{1}(s,s) + \lambda _k
\,\, (s \in N_+), k=l + 1, \dots, n$ соответственно, где $\lambda _k,
k = 1, \dots, n$ некоторые положительные числа, причём $0 < \lambda
_kh < 1, k = l + 1, \dots, n$;\\
  -- существуют  неотрицательные числа  $\bar
a^{ij}_{kr}, i = 2,\dots,m, j = 1,\dots,m_i, k =1,\dots,l, r = 1,\dots,n$
 такие, что $|a^{ij}_{kr}(t)|\leq \bar
a^{ij}_{kr} \,\, (t\geq 0) $ $P\times\mu$--почти всюду при всех этих
индексах;\\
  -- существуют  неотрицательные числа
$\bar a^i_{kr}(s,j),i=1,\dots,m, k = l+1, \dots, n, r = 1,\dots,n, s\in
N_+, j=-\infty,\dots,s$ такие, что  $|a^i_{kr}(s,j)| \leq \bar
a^i_{kr}(s,j)$ $P$--почти всюду при всех этих индексах, причём $
\mathrel {\mathop {\sup} \limits _{\tau \in N_+ }} \sum \limits
_{j=-\infty }^{\tau}\bar a^{i}_{kr}(\tau,j) < \infty$ при
$i=1,\dots,m, k = l+1, \dots, n, r = 1,\dots,n$.
  %\end{itemize}

Определим  элементы $n\times n$--матрицы $C$ следующим образом:
$$
\begin{array}{crl}
c_{kr} = \frac{1}{\lambda _k}\sum \limits_{j=1}^{m_1} \bar
a^{1j}_{kr} + \frac{c_p}{\sqrt{2\lambda _k}} \sum \limits_{i=2}^m
\sum \limits_{j=1}^{m_i} \bar a^{ij}_{kr}, \ \  k
= 1, \dots ,l, r = 1, \dots, n,\\
c_{kr} = \frac{1}{\lambda _k}\mathrel {\mathop {\sup}  \limits
_{\tau \in N_+ }} \sum \limits _{j=0 }^{\tau}\bar a^{1}_{kr}(\tau,j)
+ \frac{c_p}{\lambda _k\sqrt{h}} \sum \limits _{i=2}^{m}\mathrel
{\mathop {\sup}  \limits _{\varsigma \geq 0 }} \sum \limits _{j=0
}^{\tau} \bar a^{i}_{kr}(\tau,j), \ \ k = l+1, \dots,n, r = 1, \dots, n.
\end{array}
$$
Тогда справедлива

\begin{theorem}\label{kri-th4} Если матрица $\bar E - C$ положительно
обратима, то система \eqref{kri-1} $M_{2p}$--устой\-чи\-ва.
\end{theorem}

В следующей теореме рассматривается $M_{2p}^\gamma $-устойчивость
системы \eqref{kri-1} с экспоненциальным весом $\gamma (t) = \exp \{\lambda
t\} \,\, (t \geq 0)$, где $\lambda$ --  некоторое  положительное
число. Эта теорема является главным источником признаков
экспоненциальной $2p$-устойчивости системы \eqref{kri-1} по начальным данным,
которые будут доказаны в следующем параграфе. Отметим, что примеры
показывают, что экспоненциальная устойчивость решений систем
детерминированных линейных функционально-дифференциальных уравнений,
как правило,  наблюдается только в случае ограниченных запаздываний.
Это объясняет, в частности, первое из условий, накладываемых на
систему \eqref{kri-1} в приводимой ниже теореме \ref{kri-th5}. Предположим, что

\noindent
%\begin{itemize}
--
существуют неотрицательные числа $\tau_{ij}$, $i = 1,\dots,m$, $j =
1,\dots,m_i$  такие, что  $0 \leq t- h_{ij}(t) \leq \tau _{ij} {\,}
{\,} (t \geq 0)$ $\mu $--почти всюду при всех этих индексах;\\
--  существуют неотрицательные числа $\bar a^{ij}_{kr}, i =
1,\dots,m, j = 1,\dots,m_i, k =1,\dots,l, r = 1,\dots,n$ такие, что
$|a^{ij}_{kr}(t)|\leq \bar a^{ij}_{kr} \,\, (t\geq 0) $
$P\times\mu$--почти всюду при всех этих индексах.
%\end{itemize}
Кроме этого, пусть существуют положительные числа $\lambda _k, k = 1, \dots,
n$, для которых\\
%\begin{itemize}
\noindent
  -- диагональные элементы матрицы $ A_1(s,s)$ $(s \in
N_+)$ имеют вид $a_{kk}^{1}(s,s) + \lambda _k \,\, (s \in N_+), k=l
+ 1, \dots, n$;\\
  --  $\sum \limits_{j\in I_k}a^{1j}_{kk}(t)  \geq
\lambda _k \,\, (t\geq 0)$ $P\times\mu$--почти всюду при $k =
1,\dots,l$ для некоторых подмножеств $I_k \subset \{1,\dots, m_1\}, k =
1,\dots, l$;\\
  -- $0 < \lambda _kh < 1$ при $ k = l + 1, \dots, n$.
%\end{itemize}
Наконец, пусть существуют $d_i \in N_+$, $i = 1,\dots,m$, для которых\\
%\begin{itemize}
\noindent
 -- элементы матриц
$A_i(s,j)$ равны нулю $P$--почти всюду при $s \in N_+$,
$j=-\infty,\dots,s-d_i-1, i=1,\dots,m$;\\
--  $|a^i_{kr}(s,j)| \leq \bar a^i_{kr}(s,j)$ $P$--почти
всюду при $i=1,\dots,m, k = l+1, \dots, n, r = 1,\dots,n, s\in N_+,
j=s-d_i,\dots,s$, причём для всех $i= 1,\dots,m, \ k = l+1, \dots, n, \ r
= 1,\dots,n$
$$\mathrel {\mathop {\sup}
\limits _{\tau \in N_+}}\sum \limits _{j=\nu _i (\tau)}^{\tau}\bar
a^{1}_{kr}(\tau,j) < \infty, $$ где $\nu _i (\tau) = 0$ при $0 \le
\tau \le d_i$ и  $\nu _i (\tau) = \tau - d_i$ при $\tau
> d_i$.

 %\end{itemize}
Элементы $n\times n$--матрицы $C$ определим следующим образом:
$$
\begin{array}{crl}
c_{kk}  = \frac{1}{\lambda _k }\left(\sum \limits_{j \in I_k}\bar
a^{1j}_{kk}\left(\sum \limits_{\nu=1}^{m_1}\bar a^{1\nu}_{kk}\tau
_{1j}  + c_p\sum \limits_{i=2}^m \sum \limits_{\nu=1}^{m_i}\bar
a^{i\nu}_{kr}\sqrt{\tau _{1j}}\right) + \sum \limits_{j=1, j \in
\{1,\dots,m_1\}/ I_k}^{m_1} \bar a^{1j}_{kk}\right)+ \\
\frac{c_p}{\sqrt{2\lambda_k }}\sum
\limits_{i=2}^m \sum \limits_{j=1}^{m_i}\bar a^{ij}_{kk}, \ \ \ k = 1,\dots,l,\\
c_{kr} = \frac{1}{\lambda _k }\left(\sum \limits_{j \in I_k}\bar
a^{1j}_{kr}\left(\sum \limits_{\nu=1}^{m_1}\bar a^{1\nu}_{kr} \tau
_{1j} +  c_p\sum \limits_{i=2}^m \sum \limits_{\nu=1}^{m_i}\bar
a^{i\nu}_{kr}\sqrt{\tau _{1j}}\right) + \sum \limits_{j=1}^{m_1}
\bar a^{1j}_{kr}\right)+ \\
\frac{c_p}{\sqrt{2\lambda_k }}\sum \limits_{i=2}^m \sum
\limits_{j=1}^{m_i}\bar a^{ij}_{kr}, \ \ \ k =
1,\dots,l,r = 1, \dots, n, k \neq r,\\
 c_{kr} = \frac{1}{\lambda_kh}
\left(h\mathrel {\mathop {\sup} \limits _{\tau \in N_+}}\sum \limits
_{j=\nu _1 (\tau)}^{\tau}\bar a^{1}_{kr}(\tau,j) +
\right. \\ \left. +
c_p\sqrt{h}\sum
\limits _{i=2}^{m}\mathrel {\mathop {\sup} \limits _{\tau \in
N_+}}\sum \limits _{j=\nu _i (\tau)}^{\tau}\bar
a^{i}_{kr}(\tau,j)\right), \ k = 1,\dots,l,\, r  = 1,\dots,n.
\end{array}
$$
Тогда справедлива

\begin{theorem}\label{kri-th5}
Если матрица $\bar E - C$ положительно
обратима, то система \eqref{kri-1} $M_{2p}^\gamma $--устой\-чи\-вa с
экспоненциальным весом $\gamma (t) = \exp \{\lambda t\} \,\, (t \geq
0)$, где \begin{equation}\label{kri-9}0<\lambda < \min \{\lambda _i, \ i = 1, \dots,l; \ \ -\ln
(1-\lambda _ih), \ i = l+1, \dots, n \}.
\end{equation}
\end{theorem}

\section{Признаки экспоненциальной моментной устойчивости}\label{kri-s4}
В этом пункте предполагается, что $\gamma (t) = \exp \{\lambda t\}
\,\, (t \geq 0)$, где $\lambda$ --  некоторое  положительное число.

Предположим, что для системы \eqref{kri-1} выполняются следующие условия:\\
\noindent
%\begin{itemize}
  -- элементы матриц
$A_{ij}(t) \,\, (t \geq 0)$, $i = 2,\dots,m, j = 1,\dots,m_i$ равны нулю
$P\times \mu$--почти всюду;\\
  -- элементы матриц $A_i(s,j)\,\, (s \in
N_+, j=-\infty,\dots,s)$, $i=2,\dots,m$  равны нулю $P$--почти всюду;\\
    --  существуют неотрицательные числа $\tau_{1j}$, $j = 1,\dots,m_1$ такие, что $0 \leq t- h_{1j}(t) \leq \tau _{1j} {\,} {\,} (t
\geq 0)$ $\mu $--почти всюду при $j = 1,\dots,m_1$;\\
  -- существуют неотрицательные числа $\bar a^{1j}_{kr}, j = 1,\dots,m_1, k =1,\dots,l, r = 1,\dots,n$
 такие, что $|a^{1j}_{kr}(t)|\leq \bar a^{1j}_{kr} \,\, (t\geq 0) $
$P\times\mu$--почти всюду при $j = 1,\dots,m_1, k =1,\dots,l, r =
1,\dots,n$.

%\end{itemize}
Пусть также существуют положительные числа $\lambda _k, k = 1, \dots,
n$, для которых\\
\noindent
%\begin{itemize}
-- диагональные элементы
матрицы $ A_1(s,s)$ $(s \in N_+)$ имеют вид $a_{kk}^{1}(s,s) +
\lambda _k \,\, (s \in N_+), k=l + 1, \dots, n$;\\
--  $\sum \limits_{j\in I_k}a^{1j}_{kk}(t)
\geq \lambda _k \,\, (t\geq 0)$ $P\times\mu$--почти всюду при $k =
1,\dots,l$ для некоторых подмножеств $I_k \subset \{1,\dots, m_1\}, k =
1,\dots, l$;\\
-- $0 < \lambda _kh < 1$ при $ k
= l + 1, \dots, n$.

%\end{itemize}
Наконец, предположим, что существует $d_1 \in N_+$, для которого\\
%\begin{itemize}
\noindent
  -- элементы матрицы $A_1(s,j)$ равны нулю $P$--почти всюду
при $ s \in N_+$, $j=-\infty,\dots,s-d_1-1$;\\
--
$|a^1_{kr}(s,j)| \leq \bar a^1_{kr}$ $P$--почти всюду при $ k = l+1,
\dots, n, r = 1,\dots,n, s\in N_+, j=s-d_1,\dots,s $ для некоторых
неотрицательных чисел $\bar a^1_{kr},k = l+1, \dots, n, r = 1,\dots,n$.

%\end{itemize}
Элементы $n\times n$--матрицы $C$ определим следующим образом:
$$
\begin{array}{crl}
c_{kk} = \frac{1}{\lambda _k }\left(\sum \limits_{j \in I_k}\bar
a^{1j}_{kk}\sum \limits_{\nu=1}^{m_1}\bar a^{1\nu}_{kk} \tau _{1j} +
\sum \limits_{j=1, j \in \{1,\dots,m_1\}/ I_k}^{m_1} \bar
a^{1j}_{kk}\right), \ \ \ k = 1,\dots,l,\\
c_{kr} = \frac{1}{\lambda _k }\left(\sum \limits_{j \in I_k}\bar
a^{1j}_{kk}\sum \limits_{\nu=1}^{m_1}\bar a^{1\nu}_{kr}\tau _{1j} +
\sum \limits_{j=1}^{m_1} \bar a^{1j}_{kr}\right),\ \ \ k =
1,\dots,l,r = 1, \dots, n, k \neq r,\\
 c_{kr} = \frac{(d_1+1)\bar a^{1}_{kr}}{\lambda_k}, \ \ \ k = 1 + 1,\dots,l, \
 r = 1,\dots,n.
\end{array}
$$
Тогда справедливо следующее следствие из теоремы \ref{kri-th5}.

\begin{proposition}\label{kri-prop1}
Если матрица $\bar E - C$ является
положительно обратимой, то система \eqref{kri-1} является экспоненциально
$2p$-устойчивой по начальным данным (т.е. в смысле Определения \ref{kri-def2}),
причём для показателя  $\lambda$ имеет место оценка \eqref{kri-9}.
\end{proposition}
\begin{remark}\label{kri-rem1}  Если при выполнении условий Предложения \ref{kri-prop1}
элементы матриц $A_{1j} (t) \, (t \geq 0$, $j = 1,..,m_1$) являются
измеримыми локально суммируемыми функциями, а элементами матрицы
$A_1(s,j)\,\, (s \in N_+$, $j=s-d_1, \dots, s)$ являются
действительные числа,  то система \eqref{kri-1} является детерминированной
гибридной системой линейных дифференциальных  и разностных уравнений
с ограниченными запаздываниями в смысле работы \cite{kri-5}, и эта
система будет экспоненциально устойчива по начальным данным.
\end{remark}

Для формулировки Предложения \ref{kri-prop2} предположим, что для системы \eqref{kri-1}
выполнены следующие условия:\\
\noindent
%\begin{itemize}
  -- $A_{1j}(t) \,\, (t \geq 0)$, $j = 2,\dots,m_1$ и $A_{ij}(t) \,\, (t
\geq 0)$, $i = 2,\dots,m, j = 1,\dots,m_i$ равны нулю $P\times
\mu$--почти всюду, а элементы матриц $A_i(s,j)\,\, (s \in N_+,
j=-\infty,\dots,s)$, $i=2,\dots,m$  равны нулю $P$--почти всюду;\\
  -- существует неотрицательное число $\tau_{11}$ такое, что $0 \leq
t- h_{11}(t) \leq \tau _{11} {\,} {\,} (t \geq 0)$ $\mu $--почти
всюду;\\
  -- существуют неотрицательные числа $\bar
a^{11}_{kr}, k =1,\dots,l, r = 1,\dots,n$,  такие, что
  $|a^{11}_{kr}(t)|\leq \bar a^{11}_{kr} \,\, (t\geq 0) $
$P\times\mu$--почти всюду при $k =1,\dots,l, r = 1,\dots,n$.

 %\end{itemize}
 Кроме этого, существуют положительные числа $\lambda _k, k = 1, \dots, n$, для которых\\
 %\begin{itemize}
 \noindent
-- диагональные элементы матрицы $ A_1(s,s)$
$(s \in N_+)$ имеют вид $a_{kk}^{1}(s,s) + \lambda _k \,\, (s \in
N_+), k=l + 1, \dots, n$;\\
   -- $a^{11}_{kk}(t) \geq \lambda _k \,\, (t\geq 0)$
$P\times\mu$--почти всюду при $k = 1,\dots,l$;\\
   -- $0 < \lambda _kh < 1$ при $ k = l + 1, \dots, n$.

 %\end{itemize}
 Наконец, предположим, что существует $d_1 \in N_+$, для которого\\
 \noindent
  %\begin{itemize}
  -- элементы
матрицы $A_1(s,j)$ равны нулю $P$--почти всюду при $ s \in N_+$,
$j=-\infty,\dots,s-d_1-1$;\\
 -- для некоторых неотрицательных чисел $\bar a^1_{kr}, k = l+1, \dots,
n, r = 1,\dots,n$ выполняются оценки $|a^1_{kr}(s,j)| \leq \bar
a^1_{kr}$ $P$--почти всюду при $ k = l+1, \dots, n, r = 1,\dots,n, s\in
N_+, j=s-d_1,\dots,s $.

%\end{itemize}
Элементы $n\times n$--матрицы $C$ определим следующим образом:
$$
\begin{array}{crl}
c_{kk} = \frac{(\bar a^{11}_{kk})^2\tau_{11}}{\lambda _k }, \ \ k =
1,\dots,l, \ \ \ c_{kr} = \frac{\bar a^{11}_{kk}\bar a^{11}_{kr}\tau
_{11} + \bar a^{11}_{kR}}{\lambda _k }, \ \ k =
1,\dots,l,\ r = 1, \dots, n, \ k \neq r,\\
 c_{kr} = \frac{(d_1+1)\bar a^{1}_{kr}}{\lambda_k}, \ \ \ k = 1 + 1,\dots,l, \ r = 1, \dots,
 n.
\end{array}
$$
Тогда в силу предложения \ref{kri-prop1} справедливо

\begin{proposition}\label{kri-prop2}  Если матрица $\bar E - C$ является
положительно обратимой, то система \eqref{kri-1} является экспоненциально
${2p}$-устойчивой по начальным данным, причём для показателя
$\lambda$ имеет место оценка \eqref{kri-9}.
\end{proposition}

Для формулировки предложений \ref{kri-prop3} и \ref{kri-prop4} предположим, что для системы \eqref{kri-1} выполнены следующие условия:\\
%\begin{itemize}
\noindent
  -- $m_1 =1$;\\
   -- существуют неотрицательные числа
$\tau_{11}, \tau_{ij}$, $i = 2,\dots,m$, $j = 1,\dots,m_i$  такие, что
$0 \leq t- h_{11}(t) \leq \tau _{11} {\,} {\,} (t \geq 0)$ $\mu
$--почти всюду, $0 \leq t- h_{ij}(t) \leq \tau _{ij} {\,} {\,} (t
\geq 0)$ $\mu $--почти всюду при $i = 2,\dots,m$, $j = 1,\dots,m_i$;\\
  -- существуют неотрицательные числа
$\bar a^{11}_{kr},  k =1,\dots,l, r = 1,\dots,n$, $\bar a^{ij}_{kr}, i =
2,\dots,m, j = 1,\dots,m_i,$ $k =1,\dots,l, r = 1,\dots,n$,  для которых
$|a^{11}_{kr}(t)|\leq \bar a^{11}_{kr} \,\, (t\geq 0) $
$P\times\mu$--почти всюду при $k =1,\dots,l,$ $r = 1,\dots,n$,
$|a^{ij}_{kr}(t)| \leq \bar a^{ij}_{kr} \,\, (t\geq 0) $
$P\times\mu$--почти всюду при $i = 2, \dots, m, j = 1,\dots,m_i, k
=1,\dots,l, r = 1,\dots,n$.

%\end{itemize}
Кроме этого, предположим, что существуют положительные числа
$\lambda _k, k = 1, \dots, n$, для которых\\
%\begin{itemize}
\noindent
-- диагональные
элементы матрицы $ A_1(s,s)$ $(s \in N_+)$ имеют вид
$a_{kk}^{1}(s,s) + \lambda _k \,\, (s \in N_+), k=l + 1, \dots, n$;\\
-- $\sum \limits_{j\in I_k}a^{1j}_{kk}(t)  \geq \lambda _k
\,\, (t\geq 0)$ $P\times\mu$--почти всюду при $k = 1,\dots,l$;\\
-- $0 < \lambda _kh < 1$
при $ k = l + 1, \dots, n$.

%\end{itemize}
Наконец, предположим существование чисел $d_i \in N_+$, $i =
1,\dots,m$, для которых\\
%\begin{itemize}
\noindent
 --
 элементы матрицы $A_i(s,j)$ равны нулю
$P$--почти всюду при $s \in N_+$, $j=-\infty,\dots,s-d_i-1,
i=1,\dots,m$;\\
-- для некоторых неотрицательных чисел $\bar
a^i_{kr},i=1,\dots,m, k = l+1, \dots, n, r = 1,\dots,n$ выполняются оценки
$|a^i_{kr}(s,j)| \leq \bar a^i_{kr}$ $P$--почти всюду при
$i=1,\dots,m, k = l+1, \dots, n, r = 1,\dots,n$, $s \in N_+, j = s - d_i,
\dots, s$;\\
--  $\mathrel {\mathop {\sup} \limits _{\tau \in N_+}}\sum
\limits _{j=\nu _i (\tau)}^{\tau}\bar a^{1}_{kr}(\tau,j) < \infty $,
где $\nu _i (\tau) = 0$ при $0 \le \tau \le d_i$, $\nu _i (\tau) =
\tau - d_i$ при $\tau > d_i$ для  $i= 1,\dots,m$.

%\end{itemize}
Определим элементы $n\times n$--матрицы $C$ следующим образом:
$$
\begin{array}{crl}
c_{kk}  = \frac{\bar a^{11}_{kk}}{\lambda _k }\left(\bar
a^{11}_{kk}\tau _{11}  + c_p\sum \limits_{i=2}^m \sum
\limits_{\nu=1}^{m_i}\bar a^{i\nu}_{kr}\sqrt{\tau _{11}}\right) +
\frac{c_p}{\sqrt{2\lambda_k }}\sum
\limits_{i=2}^m \sum \limits_{j=1}^{m_i}\bar a^{ij}_{kk}, \ \ k = 1,\dots,l,\\
c_{kr} = \frac{1}{\lambda _k }\left(\bar a^{11}_{kr}\left(\bar
a^{11}_{kr} \tau _{11} + c_p\sum \limits_{i=2}^m \sum
\limits_{\nu=1}^{m_i}\bar a^{i\nu}_{kr}\sqrt{\tau _{11}}\right)
+ \bar a^{11}_{kr}\right)+
\\ +
\frac{c_p}{\sqrt{2\lambda_k }}\sum
\limits_{i=2}^m \sum \limits_{j=1}^{m_i}\bar a^{ij}_{kr}, \ k =
1,\dots,l,\ r = 1,\dots, n, \ k \neq r,\\
 c_{kr} = \frac{(d_1+1)\bar a^{1}_{kr}}{\lambda_k} +
\frac{c_p}{\lambda_k\sqrt{h}} \sum \limits _{i=2}^{m}\mathrel (d_i +
1)\bar a^{i}_{kr}, \ \ k = 1 + 1,\dots,l,\ r = 1,\dots,n.
\end{array}
$$
Тогда справедливы

\begin{proposition}\label{kri-prop3}
Если матрица $\bar E - C$ является
положительно обратимой, то система \eqref{kri-1} является экспоненциально
${2p}$-устойчивой по начальным данным, причём для показателя
$\lambda$ имеет место оценка \eqref{kri-9}.
\end{proposition}

\begin{proposition}\label{kri-prop4}
Пусть в системе \eqref{kri-1} $n = 2, l = 1$ и для этих констант выполнены все условия предложения \ref{kri-prop3}. Пусть, далее, для $c_{ij}, i,j = 1, 2$, выполнены неравенства: $ 1 - c_{11} > 0; (1 - c_{11})(1 - c_{22}) > c_{12}c_{21}$. Тогда система \eqref{kri-1}
является экспоненциально ${2p}$-устойчивой по начальным данным,
причём для показателя  $\lambda$ имеет место оценка \eqref{kri-9}, где $n = 2,
l = 1$.
\end{proposition}
Справедливость предложения \ref{kri-prop4} следует из предложения \ref{kri-prop3} и из того, что
при данных предположениях $2\times 2$--матрица $\bar E - C$ будет
положительно обратимой, поскольку её диагональные миноры
положительны.

Для формулировки двух последних предложений предположим, что для
системы \eqref{kri-1} выполнены следующие условия:\\
%\begin{itemize}
\noindent
  -- $m_1 =1$;\\
  -- $h_{11}(t) = t \,\, (t \geq 0)$ $\mu $--почти всюду;\\
  -- существуют неотрицательные числа $\tau_{ij}$, $i =
2,\dots,m$, $j = 1,\dots,m_i$  такие, что
 $0 \leq t- h_{ij}(t)
\leq \tau _{ij} {\,} {\,} (t \geq 0)$ $\mu $--почти всюду при $i =
2,\dots,m$, $j = 1,\dots,m_i$;\\
-- существуют неотрицательные числа
  $\bar a^{11}_{kr},  k =1,\dots,l, r =
1,\dots,n$, $\bar a^{ij}_{kr}, i = 2,\dots,m, j = 1,\dots,m_i,$ $k =1,\dots,l,
r = 1,\dots,n$  такие, что $|a^{11}_{kr}(t)|\leq \bar a^{11}_{kr} \,\,
(t\geq 0) $ $P\times\mu$--почти всюду при $k =1,\dots,l, r = 1,\dots,n$,
$|a^{ij}_{kr}(t)|\leq \bar a^{ij}_{kr} \,\, (t\geq 0) $
$P\times\mu$--почти всюду при $i = 2, \dots, m, j = 1,\dots,m_i, k
=1,\dots,l, r = 1,\dots,n$.
%\end{itemize}

Кроме этого, предположим, что существуют положительные числа
$\lambda _k, k = 1, \dots, n$, для которых\\
%\begin{itemize}
\noindent
  -- диагональные элементы
матрицы $ A_1(s,s)$ $(s \in N_+)$ имеют вид $a_{kk}^{1}(s,s) +
\lambda _k \,\, (s \in N_+), k=l + 1, \dots, n$;\\
-- $a^{11}_{kk}(t)  \geq
\lambda _k \,\, (t\geq 0)$ $P\times\mu$--почти всюду при $k =
1,\dots,l$;\\
 -- $0 < \lambda _kh < 1$ при $ k = l + 1, \dots, n$.

%\end{itemize}
Наконец, предположим существование чисел $d_i \in N_+$, $i =
1,\dots,m$, для которых\\
%\begin{itemize}
\noindent
 -- элементы матрицы $A_i(s,j)$ равны нулю
$P$--почти всюду при $s \in N_+$, $j=-\infty,\dots,s-d_i-1,
i=1,\dots,m$;\\
-- для некоторых чисел $\bar a^i_{kr},i=1,\dots,m, k = l+1, \dots, n, r =
1,\dots,n$ выполняются оценки $|a^i_{kr}(s,j)| \leq \bar a^i_{kr}$
$P$--почти всюду при $i=1,\dots,m, k = l+1, \dots, n, r = 1,\dots,n$, $s
\in N_+, j = s - d_i, \dots, s$;\\
-- $\mathrel {\mathop {\sup} \limits _{\tau \in N_+}}\sum
\limits _{j=\nu _i (\tau)}^{\tau}\bar a^{1}_{kr}(\tau,j) < \infty $,
где $\nu _i (\tau) = 0$ при $0 \le \tau \le d_i$, $\nu _i (\tau) =
\tau - d_i$ при $\tau > d_i$ для $i= 1,\dots,m$.

  %\end{itemize}
        Определим элементы
$n\times n$--матрицы $C$ следующим образом:
$$
\begin{array}{crl}
c_{kk}  = \frac{c_p}{\sqrt{2\lambda_k }}\sum
\limits_{i=2}^m \sum \limits_{j=1}^{m_i}\bar a^{ij}_{kk}, \ \ \ k = 1,\dots,l,\\
c_{kr} = \frac{\bar a^{11}_{kr}}{\lambda _k }+
\frac{c_p}{\sqrt{2\lambda_k }}\sum \limits_{i=2}^m \sum
\limits_{j=1}^{m_i}\bar a^{ij}_{kr}, \ \ \ k =
1,\dots,l, \ r = 1, \dots, n, \ k \neq r,\\
c_{kr} = \frac{(d_1+1)\bar a^{1}_{kr}}{\lambda_k} +
\frac{c_p}{\lambda_k\sqrt{h}} \sum \limits _{i=2}^{m}\mathrel (d_i +
1)\bar a^{i}_{kr}, \ \ \ k = 1 + 1,\dots,l,\ r = 1,\dots,n.
\end{array}
$$
Тогда в силу предложения \ref{kri-prop3} справедливы следующие

\begin{proposition}\label{kri-prop5}  Если матрица $\bar E - C$ является
положительно обратимой, то система \eqref{kri-1} является экспоненциально
${2p}$-устойчивой по начальным данным, причём для показателя
$\lambda$ имеет место оценка \eqref{kri-9}.
\end{proposition}

\begin{proposition}\label{kri-prop6} Пусть в системе \eqref{kri-1} $n = 2, l =1$, и
для этих констант выполнены все условия предложения \ref{kri-prop5}. Пусть, далее,
для $c_{ij}, i,j = 1, 2$, выполнены неравенства: $ 1 - c_{11}
> 0; (1 - c_{11})(1 - c_{22}) > c_{12}c_{21}$. Тогда система \eqref{kri-1}
является экспоненциально ${2p}$-устойчивой по начальным данным,
причём для показателя  $\lambda$ имеет место оценка \eqref{kri-9}, где $n = 2,
l =1$.
\end{proposition}

Справедливость предложения \ref{kri-prop6} следует из предложения \ref{kri-prop5} и из того, что в этих предположениях $2\times 2$-матрица $\bar E - C$ будет
положительно обратимой, поскольку её диагональные миноры положительны.


\section{Примеры}
 Рассмотрим непрерывно-дискретную систему
стохастических уравнений с постоянными коэффициентами и
ограниченными запаздываниями вида:
\begin{equation}\label{kri-10}
\begin{array}{crl}
 d\hat x(t) = - \sum
 \limits_{j=1}^{m_1}A_{1j}x(t - h_{1j})dt +
 \sum \limits_{i=2}^m\sum \limits_{j=1}^{m_i}
 A_{ij}x(t - h_{ij})d\mathcal B_i(t) \, \, (t \ge 0),\\
\tilde x(s+1) = \tilde x(s) - A_{1}\sum \limits _{j=s -d_1 }^{s}
x(j)h + \\
\sum \limits _{i=2}^{m}A_{i}\sum \limits _{j=s-d_i}^{s}
x(j)(\mathcal B_i((s+1)h) - \mathcal B_i(sh))\, \,  (s
\in N_+),\\
\end{array}
\end{equation}
где  $A_{ij}=(a^{ij}_{kr})_{k,r=1}^{l,n}, \,\,i = 1,\dots,m, j =
1,\dots,m_i$ -- $l \times n$--матрицы и $A_i =
(a^i_{kr})_{k=l+1,r=1}^n, \,\,i = 1,\dots,m$ -- $(n-l)\times
n$--матрицы, элементами которых являются произвольные действительные
числа,  $h_{ij}, i = 1,\dots,m, j = 1,\dots,m_i$ --- неотрицательные
действительные числа, $h$ --- достаточно малое положительное
действительное число. Положим также $\sum \limits_{j=1}^{m_1}
a^{1j}_{kk} = a_k, k = 1,\dots,l$ и определим элементы $n\times
n$--матрицы $C$ следующим образом:
$$
\begin{array}{crl}
c_{kk}  = \frac{1}{a _k }\sum \limits_{j =1}^{m_1}
|a^{1j}_{kk}|\left(\sum \limits_{\nu=1}^{m_1} |a^{1\nu}_{kk}|h _{1j}
+ c_p\sum \limits_{i=2}^m \sum \limits_{\nu=1}^{m_i}
|a^{i\nu}_{kr}|\sqrt{h_{1j}}\right) + \frac{c_p}{\sqrt{2a_k }}\sum
\limits_{i=2}^m \sum \limits_{j=1}^{m_i}|a^{ij}_{kk}|, \ \ \ k = 1,\dots,l,\\
c_{kr} = \frac{1}{a _k }\left(\sum \limits_{j
=1}^{m_1}|a^{1j}_{kr}|\left(\sum
\limits_{\nu=1}^{m_1}|a^{1\nu}_{kr}| h_{1j} +  c_p\sum
\limits_{i=2}^m \sum \limits_{\nu=1}^{m_i}
|a^{i\nu}_{kr}|\sqrt{h_{1j}}\right) + \sum \limits_{j=1}^{m_1}
|a^{1j}_{kr}|\right)+ \\
\frac{c_p}{\sqrt{2a_k }}\sum \limits_{i=2}^m \sum
\limits_{j=1}^{m_i} |a^{ij}_{kr}|,  \ \ \ k =
1,\dots,l, \ r = 1, \dots, n, \ k \neq r,\\
 c_{kk} = \frac{c_p(d_i+1)}{a^1_{kk}\sqrt{h}}
\sum\limits _{i=2}^{m}|a^{i}_{kk}|, \ \ \ k = 1 + 1,\dots,l, \\
c_{kr} = \frac{(d_1 + 1)|a^{1}_{kr}|}{a^1_{kk}}
+\frac{c_p(d_i+1)}{a^1_{kk}\sqrt{h}} \sum\limits
_{i=2}^{m}|a^{i}_{kr}|, \ \ \ k = 1 + 1,\dots,l, \ r = 1, \dots, n, \ k
\neq r.
\end{array}
$$

Тогда имеет место следующее

\begin{proposition}\label{kri-prop7} Если $a_k
> 0, \ k = 1,\dots,l$, $a^1_{kk} > 0, \ k = l +1,\dots,n$, а
матрица $\bar E - C$ положительно обратимой, то система \eqref{kri-10}
экспоненциально $2p$-устойчива по начальным данным.
\end{proposition}

Справедливость утверждения следует из теоремы \ref{kri-th5}, где $I_k = \{1,
\dots, m_1\}$ для всех $k = 1, \dots, l$.

\begin{corollary}\label{kri-cor1}  Пусть в системе \eqref{kri-10} $n = 2, l = 1$,
причём для этих констант выполнены все условия предложения \ref{kri-prop7}. Пусть,
далее, для $c_{ij}, i,j = 1, 2$, выполнены неравенства: $ 1 - c_{11}
> 0; (1 - c_{11})(1 - c_{22}) > c_{12}c_{21}$. Тогда система \eqref{kri-10}
экспоненциально $2p$-устойчива по начальным данным.
\end{corollary}

Справедливость следствия вытекает из предложения \ref{kri-prop7} и из того, что при
предположениях следствия $2\times 2$--матрица $\bar E - C$ будет
положительно обратимой, поскольку её диагональные миноры
положительны.

Определим теперь элементы  $n\times n$--матрицы $C$ определены
следующим образом:
$$
\begin{array}{crl}
c_{kk}  = \frac{1}{a^{11}_{kk}}\sum \limits_{j =2}^{m_1}
|a^{1j}_{kk}| + \frac{c_p}{\sqrt{2a_k }}\sum
\limits_{i=2}^m \sum \limits_{j=1}^{m_i}|a^{ij}_{kk}|, \ \ \ k = 1,\dots,l,\\
c_{kr} = \frac{1}{a _{kk}^{11}} + \sum \limits_{j=1}^{m_1}
|a^{1j}_{kr}|+ \frac{c_p}{\sqrt{2a_k }}\sum \limits_{i=2}^m \sum
\limits_{j=1}^{m_i} |a^{ij}_{kr}|, \ \ \ k =
1,\dots,l, \ r = 1, \dots, n, \ k \neq r,\\
 c_{kk} = \frac{c_p(d_i+1)}{a^1_{kk}\sqrt{h}}
\sum\limits _{i=2}^{m}|a^{i}_{kk}|, \ \ \ k = 1 + 1,\dots,l, \\
c_{kr} = \frac{(d_1 + 1)|a^{1}_{kr}|}{a^1_{kk}}
+\frac{c_p(d_i+1)}{a^1_{kk}\sqrt{h}} \sum\limits
_{i=2}^{m}|a^{i}_{kr}|, \ \ \ k = 1 + 1,\dots,l, \ r = 1, \dots,n, \ k
\neq r.
\end{array}
$$
Тогда справедливо следующее

\begin{proposition}\label{kri-prop8}
 Если $h_{11} =0$, $ a^{11}_{kk}
> 0, k = 1,\dots,l$, $a^1_{kk} > 0, k = l + 1,\dots,n$, а матрица $\bar E - C$ является
положительно обратимой, то система \eqref{kri-10} экспоненциально
$2p$-устойчива по начальным данным.
\end{proposition}

Справедливость утверждения следует из теоремы \ref{kri-th5}, где $I_k = \{1\}$
для всех $k = 1, \dots, l$.

\begin{corollary}\label{kri-cor2}
 Пусть в системе \eqref{kri-10} $n = 2, l =1$, причём для этих констант выполнены все условия Предложения 8. Пусть, далее, для $c_{ij}, i,j = 1, 2$, выполнены неравенства: $ 1 - c_{11} > 0; (1 - c_{11})(1 - c_{22}) > c_{12}c_{22}$. Тогда система \eqref{kri-10}
экспоненциально $2p$-устойчива по начальным данным.
\end{corollary}

Справедливость следствия вытекает из предложения \ref{kri-prop8} и из того, что при
предположениях следствия $2\times 2$--матрица $\bar E - C$ будет
положительно обратимой, поскольку её диагональные миноры
положительны.

%\section{Заключение} В отчете изложен метод регуляризации для
%анализа вопросов моментной устойчивости систем стохастических
%уравнений, содержащих компоненты с непрерывным и дискретным
%временем. В качестве вспомогательной, т.е. регуляризирующей, системы
%использовалась сравнительно простая система линейных
%дифференциальных и разностных уравнений, поскольку для неё легко
%можно было проверить необходимые для применения метода свойства
%устойчивости. Отметим, однако, что для получения более тонких
%признаков устойчивости можно брать и более сложные регуляризирующие
%системы линейных дифференциальных и разностных уравнений с
%последействием, так как в этом случае также имеют место аналоги
%леммы \ref{kri-lem1} и теорем \ref{kri-th1} и \ref{kri-th2}. Эти исследования
%планируется провести в будущем.





%\end{document}
