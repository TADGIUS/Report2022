\chapter{Ряды Фурье по полиномам Якоби --- Соболева}\label{MMG}
\section{Введение}
В данном разделе мы остановимся более подробно на некоторых результатах, связанных со скалярным произведением \eqref{mmg-sob-prod} в случае $\rho(x)=\rho(\alpha,\beta; x)=(1-x)^\alpha(1+x)^\beta$.

Пусть $\mathcal{P}^{\alpha,\beta}=\{ \hat{P}_n^{\alpha,\beta} \}_{n=0}^\infty$ --- система полиномов, ортонормированных на отрезке $[-1,1]$ с весом Якоби $\rho(\alpha,\beta; x)$ (система полиномов Якоби). Введем в рассмотрение новую систему функций $\mathcal{P}^{\alpha,\beta}_r$, $r \ge 1$, с помощью равенств:
\begin{gather}
	\label{mmg-sob-def1}
	P_{r,k}^{\alpha,\beta}(x) =\frac{(x+1)^k}{k!}, \quad k=0,1,\ldots, r-1,\\
	\label{mmg-sob-def2}
	P_{r,k}^{\alpha,\beta}(x) =\frac{1}{(r-1)!}\int\limits_{-1}^x(x-t)^{r-1}\hat{P}_{k-r}^{\alpha,\beta}(t)dt, \quad k=r,r+1,\ldots.
\end{gather}
Можно показать, что таким образом определенная система будет ортонормирована относительно \eqref{mmg-sob-prod} при $\rho(x)=\rho(\alpha,\beta; x)$ \cite[с.~231]{mmg-Shii-izvran2018}. Систему $\mathcal{P}_r^{\alpha,\beta}$ будем называть системой полиномов, ортогональной в смысле Соболева и ассоциированной с полиномами Якоби $P_n^{\alpha,\beta}$.
Ряд Фурье функции $f \in W^r_{L^2_{\rho(\alpha,\beta)}}[-1,1]$ по системе $\mathcal{P}_r^{\alpha,\beta}$ и частичная сумма этого ряда имеют следующий вид \cite[с.~227]{mmg-Shii-izvran2018}:
\begin{gather}
	\label{mmg-sob-fourier-series}
	f(x) \sim \sum_{k=0}^{r-1} f^{(k)}(-1)\frac{(x+1)^k}{k!}+ \sum_{k=r}^\infty c^{\alpha,\beta}_{r,k}(f) P_{r,k}^{\alpha,\beta}(x),\\
	\label{mmg-sob-part-sum}
	S^{\alpha,\beta}_{r,n}(f,x) = \sum_{k=0}^{r-1} f^{(k)}(-1)\frac{(x+1)^k}{k!}+ \sum_{k=r}^n c^{\alpha,\beta}_{r,k}(f) P_{r,k}^{\alpha,\beta}(x), \quad n \ge r,
\end{gather}
где $c^{\alpha,\beta}_{r,k}(f)=\int_{-1}^1 f^{(r)}(t)\hat{P}_{k-r}^{\alpha,\beta}(t)\rho(\alpha,\beta; t)dt$.

Одно из замечательных свойств соболевского ряда Фурье \eqref{mmg-sob-fourier-series} состоит в том, что его частичная сумма \eqref{mmg-sob-part-sum} при $n \ge r$ совпадает $r$-кратно с исходной функцией $f(x)$ в точке $x=-1$ \cite[с. 228]{mmg-Shii-izvran2018}:
\begin{equation}\label{mmg-sob-part-sum-coins}
	(S^{\alpha,\beta}_{r,n})^{(\nu)}(f,-1)=f^{(\nu)}(-1), \quad 0\le\nu\le r-1.
\end{equation}
Это очень важное свойство, которое в сочетании с хорошими аппроксимативными свойствами сумм Фурье \eqref{mmg-sob-part-sum}
делает их весьма эффективным инструментом приближённого решения краевых задач для обыкновенных дифференциальных уравнений спектральными методами. Заметим при этом, что суммы Фурье по классическим ортогональным полиномам таким свойством не обладают.

\section{Вспомогательные сведения}
%Рассмотрим некоторые свойства частичных сумм \eqref{mmg-sob-part-sum}.
Из \eqref{mmg-sob-def1}, \eqref{mmg-sob-def2} вытекает соотношение
\begin{equation}\label{mmg-Prk-deriv-prop}
	(P^{\alpha,\beta}_{r,k}(x))^{(\nu)} =
	\begin{cases}
		P^{\alpha,\beta}_{r-\nu,k-\nu}(x), &\nu \le \min\{k,r\},\\
		(P^{\alpha,\beta}_{k-r})^{(\nu-r)}(x), &r < \nu \le k,\\
		0, &\nu > k,
	\end{cases}
\end{equation}
где полагаем $P^{\alpha,\beta}_{0,k}(x)=P^{\alpha,\beta}_{k}(x)$, $r \ge 0$.

Как отмечалось во введении, коэффициенты ряда Фурье по системе $\mathcal{P}^{\alpha,\beta}_r$ имеют вид \cite[с. 10]{mmg-SharapudinovIzvRan2019}:
\begin{equation}\label{mmg-crk}
	c^{\alpha,\beta}_{r,k}(f)=
	\begin{cases}
		f^{(k)}(-1), &k < r,\\
		\int_{-1}^1 f^{(r)}(t)\hat{P}_{k-r}^{\alpha,\beta}(t)\rho(t;\alpha,\beta)dt, &k \ge r.
	\end{cases}
\end{equation}

Отсюда нетрудно показать, что коэффициенты $c^{\alpha,\beta}_{r,k}(f)$ при различных $r$ связаны равенством
\begin{equation}\label{mmg-crk-deriv}
	c^{\alpha,\beta}_{r,k}(f)=c^{\alpha,\beta}_{r-\nu,k-\nu}(f^{(\nu)}), \quad 0 \le \nu \le \min\{r,k\}.
\end{equation}

Из соотношений \eqref{mmg-Prk-deriv-prop} и \eqref{mmg-crk-deriv} получаем:
\begin{equation*}
	(S^{\alpha,\beta}_{r,n})^{(\nu)}(f,x)=
	\sum_{k=0}^{n}c^{\alpha,\beta}_{r,k}(f)(P^{\alpha,\beta}_{r,k})^{(\nu)}(x)=
	\sum_{k=\nu}^{n}c^{\alpha,\beta}_{r-\nu,k-\nu}(f^{(\nu)})P^{\alpha,\beta}_{r-\nu,k-\nu}(x).
\end{equation*}
Последнее выражение представляет собой частичную сумму порядка $n-\nu$ ряда Фурье функции $f^{(\nu)}$ по системе $\mathcal{P}^{\alpha,\beta}_{r-\nu}$. Таким образом, имеет место равенство
\begin{equation}\label{mmg-Srn-deriv}
	(S^{\alpha,\beta}_{r,n})^{(\nu)}(f,x)=
	S^{\alpha,\beta}_{r-\nu,n-\nu}(f^{(\nu)},x), \quad 0 \le \nu \le \min\{r,n\},
\end{equation}
при этом считаем, что $S^{\alpha,\beta}_{0,n}(f)=S^{\alpha,\beta}_n(f)$ --- частичная сумма ряда Фурье по полиномам Якоби $\hat{P}^{\alpha,\beta}_n$.

Поскольку $P^{\alpha,\beta}_{r,k}(x)$ --- полином степени $k$, то $S^{\alpha,\beta}_{r,n}(f,x)$ будет полиномом степени $n$. Кроме того, $S^{\alpha,\beta}_{r,m}(f,x)$, $m \ge n$, оставляет на месте полиномы степени $n$:
\begin{equation}\label{mmg-Srk-proj}
	S^{\alpha,\beta}_{r,m}(p_n,x)=p_n(x), \quad m \ge n,
\end{equation}
где $p_n(x)$ --- полином степени $n$, что нетрудно показать индукцией по $r$. В самом деле, для $r=0$ соотношение \eqref{mmg-Srk-proj}, очевидно, выполнено. Шаг индукции доказывается с помощью следующего соотношения, которое вытекает из \eqref{mmg-Srn-deriv} и \eqref{mmg-sob-part-sum-coins}:
\begin{equation}\label{mmg-Srn-int-1}
	S^{\alpha,\beta}_{r,m}(f,x)=f(-1)+\int_{-1}^x S^{\alpha,\beta}_{r-1,m-1}(f',t)dt, \quad r \ge 1.
\end{equation}

Далее, из \eqref{mmg-sob-part-sum} можно получить следующее интегральное представление для частичных сумм $S^{\alpha,\beta}_{r,n}(f,x)$ при $r=1$:
\begin{equation}\label{mmg-S1n-int-repr}
	S^{\alpha,\beta}_{1,1+n}(f,x) = f(-1)+\int_{-1}^{1}f'(t)\rho(\alpha, \beta; t)K^{\alpha,\beta}_{1,1+n}(x,t)dt,
\end{equation}
где
\begin{equation*}
	K^{\alpha,\beta}_{1,1+n}(x,t)=\sum_{k=1}^{n+1}\hat{P}^{\alpha,\beta}_{k-1}(t)P^{\alpha,\beta}_{1,k}(x).
\end{equation*}
С помощью формулы \eqref{mmg-sob-def2} для ядра $K^{\alpha,\beta}_{1,1+n}(x,t)$ можно получить соотношение:
\begin{equation}\label{mmg-K1n-Kn-repr}
	K^{\alpha,\beta}_{1,1+n}(x,t)=\sum_{k=1}^{n+1}\hat{P}^{\alpha,\beta}_{k-1}(t)\int_{-1}^{x}\hat{P}^{\alpha,\beta}_{k-1}(u)du=\int_{-1}^{x}K^{\alpha,\beta}_n(t,u)du,
\end{equation}
где $K^{\alpha,\beta}_n(t,u)=\sum_{k=0}^{n}\hat{P}^{\alpha,\beta}_{k}(t)\hat{P}^{\alpha,\beta}_{k}(u)$.

\section{Некоторые сведения о полиномах Якоби}
Для произвольных действительных чисел $\alpha$ и $\beta$ полиномы Якоби $P_n^{\alpha,\beta}(x)$ можно определить~\cite{Ram-Sege} с помощью формулы Родрига
$$
P_n^{\alpha,\beta}(x)=\frac{(-1)^n}{2^nn!}\frac{1}{\kappa(x)}\frac{d^n}{dx^n}\{\kappa(x)\sigma^n(x)\},
$$
где $\kappa(x)=\kappa(x;\alpha,\beta)=(1-x)^\alpha(1+x)^\beta$, $\sigma(x)=1-x^2$. Отметим также следующие свойства полиномов $P_n^{\alpha,\beta}(x)$, которые можно найти в~\cite{Ram-Sege}:
\begin{itemize}
	\item
	соотношение ортогональности
	$$
	\int_{-1}^{1}P_n^{\alpha,\beta}(x)P_m^{\alpha,\beta}(x)\kappa(x)dx=h_n^{\alpha,\beta}\delta_{n,m},\quad \alpha, \beta>-1,
	$$
	где
	$$
	h_n^{\alpha,\beta}=\frac{\Gamma(n+\alpha+1)\Gamma(n+\beta+1)2^{\alpha+\beta+1}}{n!(2n+\alpha+\beta+1)\Gamma(n+\alpha+\beta+1)};
	$$
	
	\item
	равенства
	\begin{multline}\label{alpha_1}
	(1-x)P_n^{\alpha+1,\beta}(x)=\\
	\frac{2}{2n+\alpha+\beta+2}\left[(n+\alpha+1)P_n^{\alpha,\beta}(x)-(n+1)P_{n+1}^{\alpha,\beta}(x)\right],
	\end{multline}
	\begin{multline}\label{beta_1}
	(1+x)P_n^{\alpha,\beta+1}(x)=\\
	\frac{2}{2n+\alpha+\beta+2}\left[(n+\beta+1)P_n^{\alpha,\beta}(x)+(n+1)P_{n+1}^{\alpha,\beta}(x)\right];
	\end{multline}
	
	\item
	формула Кристоффеля--Дарбу
	\begin{multline}\label{Kris_Dar}
	K_n^{\alpha,\beta}(x,t)=\sum_{k=0}^{n}\frac{P_k^{\alpha,\beta}(x)P_k^{\alpha,\beta}(t)}{h_k^{\alpha,\beta}}=\frac{2^{-\alpha-\beta}}{2n+\alpha+\beta+2}\times\\
	\frac{\Gamma(n+2)\Gamma(n+\alpha+\beta+2)}{\Gamma(n+\alpha+1)\Gamma(n+\beta+1)}
	\frac{P_{n+1}^{\alpha,\beta}(x)P_n^{\alpha,\beta}(t)-P_n^{\alpha,\beta}(x)P_{n+1}^{\alpha,\beta}(t)}{x-t};
	\end{multline}
	
	\item
	весовая оценка ($-1\le x\le1$)
	\begin{equation*}
	\sqrt{n}|P_n^{\alpha,\beta}(x)|\le c(\alpha,\beta)\left(\sqrt{1-x}+\frac{1}{n}\right)^{-\alpha-\frac12}
	\left(\sqrt{1+x}+\frac{1}{n}\right)^{-\beta-\frac12},
	\end{equation*}
\end{itemize}
где здесь и всюду в дальнейшем $c$, $c(\alpha,\beta)$, $c(\alpha,\beta,r)$ -- положительные числа, зависящие только от указанных параметров и различные в разных местах.

\section{Пространство Соболева}
В этом подразделе будут рассмотрены некоторые свойства пространств Соболева $W^r_{L^1_{\rho(\alpha, \beta)}}$, $\rho(\alpha, \beta; x)=(1-x)^\alpha(1+x)^\beta$.

\begin{lemma}\label{mmg-complete-W1L1rho}
	Пространство Соболева $W^r_{L^1_{\rho(\alpha, \beta)}}$ является полным при $-1 < \alpha, \beta \le 0$.
\end{lemma}

Пространство $W^r_{L^1_\rho}=W^r_{L^1_{\rho(\alpha, \beta)}}$ перестает быть полным при $\max\{\alpha,\beta\} > 0$. Покажем это для случая $r=1$. Пусть для определенности $\alpha>0$. Рассмотрим последовательность функций
\begin{equation*}
	f_n(x)=
	\begin{cases}
		0, &-1 \le x \le 0,\\
		-\ln (1-x), &0<x<1-\frac{1}{n},\\
		\ln n + n(x-(1-\frac{1}{n})), &1-\frac{1}{n} \le x \le 1.
	\end{cases}
\end{equation*}
Легко видеть, что $f_n(x)$ --- абсолютные непрерывные на $[-1,1]$ функции, производные которых
\begin{equation*}
	f_n'(x)=
	\begin{cases}
		0, &-1 \le x < 0,\\
		\frac{1}{1-x}, &0<x<1-\frac{1}{n},\\
		n, &1-\frac{1}{n} \le x \le 1,
	\end{cases}
\end{equation*}
принадлежат $L^1_{\rho}$. Следовательно, $f_n(x) \in W^1_{L^1_{\rho}}$.
Нетрудно также убедиться в том, что $\|f_n'-g\|_{L^1_\rho} \to 0$, где
\begin{equation*}
	g(x)=
	\begin{cases}
		0, &-1 \le x < 0,\\
		\frac{1}{1-x}, &0<x<1.
	\end{cases}	
\end{equation*}
Отсюда в силу равенства $\|f_n-f_m\|_{W^1_{L^1_{\rho}}} = \|f_n'-f_m'\|_{L^1_\rho}$ вытекает фундаментальность последовательности $\{f_n\}$ в $W^1_{L^1_{\rho}}$. Покажем теперь, что указанная последовательность не имеет предела в $W^1_{L^1_{\rho}}$. Предположим противное, что существует функция $f \in W^1_{L^1_{\rho}}$, такая что $\|f_n-f\|_{W^1_{L^1_{\rho}}} \to 0$. Тогда, пользуясь определением нормы, получаем $\|f'_n-f'\|_{L^1_{\rho}} \to 0$, откуда в силу единственности предела следует равенство $f'(x)=g(x)$ почти всюду. Но такого быть не может, поскольку $f' \in L^1$, а $g \in L^1_\rho \setminus L^1$.

\begin{lemma}\label{mmg-pol-dense}
	Множество полиномов всюду плотно в пространстве $W^1_{L^1_{\rho(\alpha,\beta)}}$, $\alpha, \beta > -1$.
\end{lemma}




\section{Сходимость в равномерной метрике и метрике пространств Соболева}
Из результатов, полученных в общем случае для систем вида \eqref{mmg-sob-def1}, \eqref{mmg-sob-def2} \cite[с. 38, теорема 2]{mmg-SharapudinovUMN}, следует, что при $-1 < \alpha, \beta < 1$ ряд Фурье \eqref{mmg-sob-fourier-series} сходится равномерно на $[-1,1]$ к функциям $f \in W^r_{L^2_{\rho(\alpha,\beta)}}[-1,1]$ \cite[с. 68, следствие 5]{mmg-SharapudinovUMN}.
Возникает естественный вопрос о том, сохранится ли свойство равномерной сходимости рядов Фурье \eqref{mmg-sob-fourier-series} для функций $f(x)\in W^r_{L^p_{\rho(\alpha,\beta)}}[-1,1]$, когда $1 \le p < 2$.
В статье \cite{mmg-Shii-izvran2018}, опираясь на результаты Макенхоупта, получены условия равномерной сходимости рядов Фурье по соболевской системе $\mathcal{P}^{-\frac12,-\frac12}_r=\{\hat{P}^{-\frac12,-\frac12}_{r,k}\}_{k=0}^\infty$, порождённой системой полиномов Чебышева первого рода $\{\hat{P}^{-\frac12,-\frac12}_{k}\}$.
\begin{theoremA}\label{mmg-st-sob-cheb-uniconv}
	Пусть $A, B \in \mathbb{R}$, $p>1$ таковы, что
	\begin{equation}\label{mmg-}
		\left|\frac{A+1}{p}-\frac{1}{4}\right|<\frac{1}{4},\quad
		\left|\frac{B+1}{p}-\frac{1}{4}\right|<\frac{1}{4}.
	\end{equation}
	Тогда если $f\in W^r_{L_{\rho(A,B)}^p}[-1,1]$, $r \ge 1$, то ряд Фурье функции $f$ по системе $\mathcal{P}^{-\frac12,-\frac12}_r$ равномерно на $[-1,1]$ сходится к $f(x)$.
\end{theoremA}
Из этой теоремы непосредственно выводится следствие.
\begin{corollaryA}
	Если $f \in W^r_{L_{\rho(-\frac{1}{2},-\frac{1}{2})}^p}[-1,1]$, $p>1$, то ряд Фурье функции $f$ по системе $\mathcal{P}^{-\frac12,-\frac12}_r$ равномерно на $[-1,1]$ сходится к $f(x)$.
\end{corollaryA}
Заметим, что ряд Фурье по системе $\mathcal{P}^{-\frac12,-\frac12}_r$ может быть построен для любой функции $f \in W^r_{L_{\rho(-\frac{1}{2},-\frac{1}{2})}^1}[-1,1]$. Однако приведённое выше следствие справедливо только для функций $W^r_{L_{\rho(-\frac{1}{2},-\frac{1}{2})}^p}[-1,1]$ при $p>1$. Интерес представляет вопрос о том, будет ли справедливо утверждение следствия для более общего случая, когда $f \in W^r_{L_{\rho(-\frac{1}{2},-\frac{1}{2})}^1}[-1,1]$. Положительный ответ на этот вопрос получен в \cite[теорема 6]{mmg-Shii-izvran2018}.
\begin{theoremA}
	Если $f \in W^r_{L_{\rho(-\frac{1}{2},-\frac{1}{2})}^1}[-1,1]$, то ряд Фурье функции $f$ по системе $\mathcal{P}^{-\frac12,-\frac12}_r$ равномерно на $[-1,1]$ сходится к $f(x)$.
\end{theoremA}

Аналог теоремы \ref{mmg-st-sob-cheb-uniconv} доказан также для системы соболевских функций \\$\mathcal{P}_r^{\alpha,0}=\{P_{r,k}^{\alpha,0}\}_{k=0}^\infty$, порождённых полиномами Якоби $P_k^{\alpha,0}(x)$ при $-1<\alpha\le\frac{1}{2}$, $\alpha$ --- дробное \cite[с. 10, следствие 1]{mmg-Shii-matzam2017}.
\begin{theoremA}\label{mmg-st-sob-jac-0-uniconv}
	Пусть
	$-1<\alpha\le\frac{1}{2}$, $A,B\in\mathbb{R}$, $p>1$ таковы, что
	\begin{equation*}
		\left|\frac{A+1}{p}-\frac{\alpha+1}{2}\right|<
		\min\left\{\frac{1}{4},\frac{\alpha+1}{2}\right\}, \quad
		\left|\frac{B+1}{p}-\frac{1}{2}\right|<\frac{1}{4}.
	\end{equation*}
	Тогда если $f\in W^r_{L_{\rho(A,B)}^p}[-1,1]$, $r \ge 1$, то ряд Фурье функции $f$ по системе $\mathcal{P}_r^{\alpha,\beta}$ равномерно на $[-1,1]$ сходится к $f(x)$.
\end{theoremA}

Одним из результатов данного раздела является обобщение теорем \ref{mmg-st-sob-cheb-uniconv}, \ref{mmg-st-sob-jac-0-uniconv}.
\begin{theorem}\label{mmg-sob-uni-conv-muck}
	Пусть $-1 < \alpha,\beta$, $A,B \in \mathbb{R}$, $p>1$, таковы, что
	\begin{gather}
		\label{mmg-muck-A}
		\left|\frac{A+1}{p}-\frac{\alpha+1}{2}\right|<
		\min\left\{\frac{1}{4},\frac{\alpha+1}{2}\right\},\\
		\label{mmg-muck-B}
		\left|\frac{B+1}{p}-\frac{\beta+1}{2}\right|<\min\left\{\frac{1}{4},\frac{\beta+1}{2}\right\},\\
		\label{mmg-Lpw-in-L1-cond}
		\frac{A+1}{p} < 1, \quad \frac{B+1}{p} < 1.
	\end{gather}
	Тогда если $f\in W^r_{L_{\rho(A,B)}^p}[-1,1]$, $r \ge 1$, то ряд Фурье функции $f$ по системе $\mathcal{P}_r^{\alpha,\beta}$ равномерно на $[-1,1]$ сходится к $f(x)$.
\end{theorem}
На самом деле указанную теорему можно несколько усилить.
\begin{theorem}\label{mmg-sob-conv-muck}
	Пусть $\alpha,\beta > -1$, $A,B \in \mathbb{R}$, $p>1$. Для каждой функции $f\in W^r_{L_{\rho(A,B)}^p}[-1,1]$, $r \ge 1$, ряд Фурье по системе $\mathcal{P}_r^{\alpha,\beta}$ сходится к $f(x)$ по норме пространства $W^r_{L_{\rho(A,B)}^p}[-1,1]$ тогда и только тогда, когда
	\begin{gather}
		\label{mmg-muck-A1}
		\left|\frac{A+1}{p}-\frac{\alpha+1}{2}\right|<
		\min\left\{\frac{1}{4},\frac{\alpha+1}{2}\right\},\\
		\label{mmg-muck-B1}
		\left|\frac{B+1}{p}-\frac{\beta+1}{2}\right|<\min\left\{\frac{1}{4},\frac{\beta+1}{2}\right\}.
	\end{gather}
\end{theorem}

При условиях \eqref{mmg-Lpw-in-L1-cond} норма $W^r_{L_{\rho(A,B)}^p}[-1,1]$ сильнее равномерной нормы, поэтому теорема \ref{mmg-sob-uni-conv-muck} вытекает из теоремы \ref{mmg-sob-conv-muck}.

Для $A=\alpha$, $B=\beta$ аналог теоремы \ref{mmg-sob-conv-muck} получен в работе \cite{mmg-Diaz-Gonzalez2020} (см. теорему 5). В ней получены достаточные условия сходимости в пространстве $W^{\alpha,\beta}_{\overline{w},p}$, $w=(w_0,\ldots,w_{r-1}) \in \mathbb{R}^r$, рядов Фурье по системе полиномов, ортогональных относительно более общего скалярного произведения
\begin{equation*}
	\langle f, g \rangle = \sum_{k=0}^{r-1} f^{(k)}(\omega_k)g^{(k)}(\omega_k)+\int_{-1}^{1}f^{(r)}(t)g^{(r)}(t)\rho(\alpha,\beta;t)dt.
\end{equation*}



Как и в случае системы $\mathcal{P}^{-\frac12,-\frac12}_r$, для построения ряда Фурье по системе $\mathcal{P}_r^{\alpha,\beta}$ необходимо и достаточно, чтобы $f\in W^r_{L_{\rho(\alpha,\beta)}^1}[-1,1]$. Теорема \ref{mmg-sob-uni-conv-muck} справедлива только при $p>1$. Основным результатом настоящего раздела является следующая теорема.

\begin{theorem}\label{mmg-sob-uni-conv-L1}
	Если $f \in W^r_{L^1_{\rho(\alpha,\beta)}}$, $r \ge 1$, $-1<\alpha,\beta \le 0$, то ряд Фурье функции $f$ по системе полиномов $\mathcal{P}_r^{\alpha,\beta}$ равномерно на $[-1,1]$ сходится к $f$.
\end{theorem}

Данная теорема носит окончательный характер в том смысле, что при заданных $\alpha, \beta$ нельзя расширить множество рассматриваемых функций (ряд Фурье по $\mathcal{P}_r^{\alpha,\beta}$ не определяется для функций $f \notin  W^r_{L^1_{\rho(\alpha,\beta)}}$).

Доказательство теоремы \ref{mmg-sob-uni-conv-L1} основано на теореме \ref{mmg-st-S1n-norm-est}. Теорема \ref{mmg-st-S1n-norm-est} доказывается с помощью леммы \ref{mmg-st-K1n-bounded}. В доказательстве леммы \ref{mmg-st-K1n-bounded} используется следующее утверждение, которое вытекает из лемм, доказанных в работе А.\,В. Зорщикова \cite[леммы 2 и 3]{mmg-Zorschikov1967}.
\begin{lemma}\label{mmg-st-Zor}
	Пусть $-1 < \alpha,\beta \le 0$, $x \in [-1,1]$. Тогда существует такая постоянная $c(\alpha,\beta)$, зависящая только от $\alpha,\beta$, что для любых $n$  и любых $-1 \le a \le b \le 1$ выполняется неравенство
	\begin{equation*}
		\Bigl| \int_a^b \rho(\alpha,\beta;u)K_n^{\alpha,\beta}(t,u)du \Bigr| \le c(\alpha,\beta).
	\end{equation*}
\end{lemma}

\begin{lemma}\label{mmg-st-K1n-bounded}
	Справедлива оценка
	\begin{equation}\
		|K^{\alpha,\beta}_{1,1+n}(x,t)| \le c(\alpha, \beta), \quad -1 < \alpha, \beta \le 0, x,t \in [-1,1], n \ge 0.
	\end{equation}
\end{lemma}
\begin{theorem}\label{mmg-st-S1n-norm-est}
	Для частичных сумм рядов Фурье по системе полиномов $\mathcal{P}_1^{\alpha,\beta}$ при $-1< \alpha, \beta \le 0$ имеет место неравенство:
	\begin{equation}\label{mmg-S1n-bound}
		\|S^{\alpha,\beta}_{1,n}\|_{W^1_{L^1_{\rho(\alpha,\beta)}} \to C} \le c(\alpha,\beta),
	\end{equation}
	где $c(\alpha,\beta)$ --- константа, не зависящая от $n$.
\end{theorem}









