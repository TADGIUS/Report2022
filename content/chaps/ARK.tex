%\chapter{Рациональные сплайн-функции и их приложения}

\chapter{Динамическое решение интегрального уравнения Вольтерры в виде
рациональных сплайн-функций}
%\begin{abstract}
%
%Приближенное решение интегрального уравнения Вольтерры второго рода
%представлено в виде коллокационных рациональных сплайн-функций на последовательных отрезках,
%исчерпывающих всю область решения. Получены также оценки скорости сходимости
%приближенных решений к точному в равномерной метрике через модуль непрерывности решения и его
%производных первого и второго порядков.
%\end{abstract}

\section{Введение}

Возросший в последние годы интерес к дальнейшему исследованию интегральных уравнений
Вольтерры, в частности, связан с поиском более эффективных методов решения задач
математической физики, с востребованностью решения задач, описывающих модели
биологии, экологии, экономики, а также с потребностью приложений интегральных уравнений
к решению задач по моделированию развивающихся динамических систем (см., напр., \cite{ark-3}
и цитированные там источники). Эти задачи показывают также специфику
уравнений Вольтерры, которая не позволяет получить полное их решение методами
исследований более общих интегральных уравнений Фредгольма, и востребованность новых методов
приближенного решения уравнений Вольтерры.

В данном разделе рассматривается вопрос приближенного решения интегрального уравнения
 Вольтерры
\begin{equation}\label{ark-eq-1}
y(x)-\lambda \int_a^x K(x,t)y(t) dt=\varphi(x),\quad x\in[a,b],
\end{equation}
с непрерывным на треугольнике $a\leqslant t\leqslant x\leqslant b$ ядром $K(x,t)$
и непрерывной на отрезке $[a,b]$ правой частью $\varphi(x)$. Именно, для натурального
$N\geqslant 2$ берется сетка произвольных узлов $\Delta:a=x_0<x_1<\dots <x_N=b$,  и
в качестве приближенного решения уравнения \eqref{ark-eq-1} на отрезках вида $[a,x_n]$,
$n=1,2,\dots,N$, строятся коллокационные рациональные сплайн-функции с параметром.

Отметим, что вопросы приближенного решения интегральных уравнений с помощью полиномиальных
сплайнов рассматривались в \cite{ark-4,ark-5,ark-6,ark-7} и других работах. Но известно
\cite{ark-7, ark-8}, что классические полиномиальные сплайны для непрерывных функций
по произвольным сеткам узлов с бесконечно малыми диаметрами могут не сходиться.
В отличие от них сплайн-функции по рациональным интерполянтам \cite{ark-9} для любой непрерывной
на данном отрезке функции по любой последовательности сеток узлов с бесконечно малыми диаметрами
равномерно сходятся. В подобной <<безусловной>> сходимости сплайн-функций существенную роль
играет выбор полюсов рациональных интерполянтов, на основе которых строятся сплайн-функции.
При этом интерполянты содержат также параметр, соответствующий выбор которого
влияет на скорость сходимости. Представлен также достаточно эффективный способ нахождения
соответствующего дискретного решения.


\section{Основные результаты}
Для данной сетки произвольных узлов $\Delta: a=x_0<x_1<\dots <x_N=b$ $(N\geqslant 2)$
рассмотрим последовательно отрезки вида $[a,x_n]$, $n=1,2,\dots,N$, и промежуточные
сетки узлов $\Delta_n: a=x_0<x_1<\dots<x_n$.

Обозначим $h_i=x_i-x_{i-1}$, $i=1,2,\dots,N$, и относительно произвольного параметра
$\mu>0$ определим набор чисел $g=\{g_1,g_2,\dots,g_{N-1}\}$ таких, что при $i=1,2,\dots,N-1$
имеем
\begin{equation}\label{ark-eq-1.1}
g_i=\begin{cases}
x_{i+1}+\mu h_{i+1} \quad\text{ при }\quad h_{i+1}\leqslant h_i,\\
x_{i-1}-\mu h_i \quad \text{ при }\quad h_{i+1}> h_i.
\end{cases}
\end{equation}

Положим также $G_n=\{g_1,g_2,\dots,g_{n-1}\}$, $n=2,3,\dots,N$.
Всюду ниже $y(x)$ считается точным непрерывным решением интегрального уравнения \eqref{ark-eq-1}.

При $n=2,3,\dots,N$ для троек узлов $x_{i-1}<x_i<x_{i+1}$, $i=1,2,\dots,n-1$,
построим рациональные интерполянты вида
\begin{equation}\label{ark-eq-1.2}
R_i(x,y)=R_i(x,y,[a,x_n])=\alpha_i+\beta_i(x-x_i)+\gamma_i \frac 1{x-g_i},
\end{equation}
которые однозначно определяются условиями $R_i(x_j,y)=y(x_j)$ при $j=i-1,i,i+1$.

Функцию $R_i(x,y, [a,x_n])$ при $x\in[x_{i-1},x_{i+1}]$,
$i=1,2,\dots,n-1$, используя интерполяционные условия,  можно представить также в виде
\begin{equation}\label{ark-eq-1.3}
R_i(x,y,[a,x_n])=a_i(x) y(x_{i-1})+b_i(x) y(x_i)+c_i(x) y(x_{i+1}),
\end{equation}
$$
a_i(x)=\frac{(x-x_i)(x-x_{i+1})(x_{i-1}-g_i)}{(x_{i-1}-x_i)(x_{i-1}-x_{i+1})(x-g_i)},\quad
b_i(x)=\frac{(x-x_{i-1})(x-x_{i+1})(x_i-g_i)}{(x_i-x_{i-1})(x_i-x_{i+1})(x-g_i)},
$$
$$
c_i(x)=\frac{(x-x_{i-1})(x-x_i)(x_{i+1}-g_i)}{(x_{i+1}-x_{i-1})(x_{i+1}-x_i)(x-g_i)}.
$$

При этом $a_i(x)+b_i(x)+c_i(x)=1$ для $x\in[x_{i-1}, x_{i+1}]$, а с учетом расположения узлов
и полюса $g_i$ при $x\in [x_{i-1}, x_i]$ получим неравенства $a_i(x)\geqslant 0$,
$b_i(x)\geqslant 0$, $c_i(x)\leqslant 0$, при $x\in[x_i, x_{i+1}]$ -- неравенства
$a_i(x)\leqslant 0$, $b_i(x)\geqslant 0$, $c_i(x)\geqslant 0$.

Учитывая эти неравенства, в случае $h_{i+1}\leqslant h_i$ последовательно имеем:
$$
|a_i(x)|=\frac{|x-x_i|}{x_i-x_{i-1}}\cdot \frac{x_{i+1}-x}{g_i-x}\cdot
 \frac{g_i-x_{i-1}}{x_{i+1}-x_{i-1}}\leqslant 1,\quad \text{если}\quad x\in [x_{i-1},x_{i+1}];
$$
$$
|c_i(x)|=\mu \frac{x-x_{i-1}}{x_{i+1}-x_{i-1}}\cdot \frac{x_i-x}{g_i-x}
\leqslant \mu,\quad \text{если}\quad x\in [x_{i-1},x_i];
$$
$$
|c_i(x)|=\mu \frac{x-x_{i-1}}{g_i-x}\cdot \frac{x-x_i}{x_{i+1}-x_{i-1}}
\leqslant 1,\quad \text{если}\quad x\in [x_i,x_{i+1}];
$$
$$
|b_i(x)|=b_i(x)=1-a_i(x)-c_i(x)\leqslant 1-c_i(x)\leqslant 1+\mu,
\quad \text{если}\quad x\in [x_{i-1},x_i];
$$
$$
|b_i(x)|=b_i(x)=1-a_i(x)-c_i(x)\leqslant 1-a_i(x)\leqslant 2,
\quad \text{если}\quad x\in [x_i,x_{i+1}].
$$

Значит, если $x\in[x_{i-1}, x_i]$, то $0\leqslant a_i(x)\leqslant 1$,
$0\leqslant b_i(x)\leqslant 1+\mu$,  $-\mu\leqslant c_i(x)\leqslant 0$.

Если же $x\in[x_i, x_{i+1}]$, то $-1\leqslant a_i(x)\leqslant 0$,
$ 0\leqslant b_i(x)\leqslant 2$, $0\leqslant c_i(x)\leqslant 1$.

В случае $h_{i+1}>h_i$ аналогично получим неравенства $0\leqslant a_i(x)\leqslant 1$,
$0\leqslant b_i(x)\leqslant 2$,   $-1\leqslant c_i(x)\leqslant 0$, если $x\in[x_{i-1}, x_i]$,
и неравенства $-\mu\leqslant a_i(x)\leqslant 0$, $0\leqslant b_i(x)\leqslant 1+\mu$,
$0\leqslant c_i(x)\leqslant 1$, если $x\in[x_i,x_{i+1}]$.

Из полученных неравенств, в частности, при $x\in[x_{i-1}, x_{i+1}]$ для $i=1,2,\dots,n-1$ и
$\mu>0$ вытекают единые оценки
\begin{equation}\label{ark-eq-1.5}
|a_i(x)|\leqslant 1+\mu,\quad 0\leqslant b_i(x)\leqslant \max\{2,1+\mu\},
\quad |c_i(x)|\leqslant 1+\mu.
\end{equation}

Положим также
\begin{equation}\label{ark-eq-1.6}
\begin{array}{l}
 R_0(x,y,[a,x_n])=R_1(x,y,[a,x_n]),\quad n=1,2,\dots, N-1,\\
 R_n(x,y,[a,x_n])=R_{n-1}(x,y,[a,x_n]),\quad n=2,3,\dots, N.
\end{array}
\end{equation}

Равенства \eqref{ark-eq-1.6} играют роль краевых условий, с учетом которых
 для $r=1,2$ построим рациональную сплайн-функцию
$R_{n,r}(x,y,[a,x_n])=R_{n,r}(x,y,\Delta_n, G_n, \mu)$
на отрезке $[a,x_n]$, $n=1,2,\dots,N$, такую, что
при $x\in[x_{i-1},x_i]$, $i=1,2,\dots,n$, выполняется равенство
\begin{equation}\label{ark-eq-1.7}
R_{n,r}(x,y,[a,x_n])=R_i(x,y,[a,x_n])A_{i,r}(x)+R_{i-1}(x,y,[a,x_n])B_{i,r}(x),
\end{equation}
где
$$
A_{i,r}(x)=\frac{(x-x_{i-1})^r}{(x-x_{i-1})^r+(x_i-x)^r},\quad B_{i,r}(x)=1-A_{i,r}(x).
$$

Как следует из \cite{ark-9}, $R_{n,r}(x,y,[a,x_n])$ является гладкой функцией класса $C^r_{[a,b]}$.
В \cite{ark-11} показано, что если $f \in C^2_{[a,b]}$, то для $k=0,1,2$ соответственно
$R_{n,2}^{(k)}(x,f,[a,b])$ сходятся равномерно к $f^{(k)}(x)$ на $[a,b]$ при
$\|\Delta\|=\max\{h_i:i=1,2,\dots,N\}\to 0$.

Отметим также, что из \eqref{ark-eq-1.7}, \eqref{ark-eq-1.6} и \eqref{ark-eq-1.2} имеем
\begin{equation}\label{ark-eq-1.8}
R_{n,r}(x_n,y,[a,x_n])=R_{n-1}(x_n,y,[a,x_n])=y(x_n).
\end{equation}

Важным является вопрос об оценке скорости сходимости приближенных решений интегрального
уравнения \eqref{ark-eq-1} к точному его решению из данного класса гладкости.

Для решения этой задачи применяются приводимые далее аппроксимативные свойства
трехточечных рациональных интерполянтов $R_i(x,f)=R_i(x, f,[a,b])$ вида \eqref{ark-eq-1.2}
для функций $f(x)$, определенных на отрезке $[a,b]$, и данной сетки узлов
$\Delta: a=x_0<x_1<\dots<x_N=b$ $(N\geqslant 2)$. Будем пользоваться также
обозначениями
 $\rho_\Delta=\max\{h_i /h_j:\,|i-j|=1, 1\leqslant i, j\leqslant N\}$,
\newline $\omega(\delta, f)=\omega(\delta, f,[a,b])=\sup\{|f(x+h)-f(x)|:
0\leqslant h\leqslant \delta;\, x, x+h\in[a,b]\}$,
\newline $\|f\|_{[a,b]}=\sup\{|f(x)|: x\in[a,b]\}$.

Как показано в \cite{ark-10}, если $f\in C_{[a,b]}$, то при $x\in[x_{i-1}, x_{i+1}]$,
$i=1,2,\dots,N-1$, выполняется неравенство
\begin{equation}\label{ark-eq-1.9}
|f(x)-R_i(x,f)|\leqslant (2+\max\{1,\mu\})\omega(\|\Delta\|, f).
\end{equation}

Следует отметить, что
справедливость неравенства \eqref{ark-eq-1.9} для любой функции $f\in C_{[a,b]}$ и каждой сетки
произвольных узлов на отрезке $[a,b]$ обеспечивает рациональным сплайн-функциям вида
\eqref{ark-eq-1.7}  безусловную сходимость на всем классе функций $C_{[a,b]}$, а именно
без дополнительных ограничений на сетки узлов, кроме
стремления к нулю максимального расстояния между соседними узлами (в отличие от
классических полиномиальных сплайнов).

Если же $f\in C^1_{[a,b]}$, то при $x\in[x_{i-1}, x_{i+1}]$, $i=1,2,\dots,N-1$,
имеем

\begin{equation}\label{ark-eq-1.10}
|f(x)-R_i(x,f)|\leqslant
\left(4+\frac 2\mu\right)\|\Delta\|\omega(\|\Delta\|, f^\prime).
\end{equation}

В \cite{ark-11} установлено, что если $f\in C^2_{[a,b]}$, то при $x\in[x_{i-1},x_{i+1}]$,
$i=1,2,\dots,N-1$, выполняется неравенство
$$
|f(x)-R_i(x,f)|\leqslant  \left(\omega(x_{i+1}-x_{i-1}, f^{\prime\prime})+
\frac 1{4\mu} \rho_\Delta \|f^{\prime\prime}\|_{[x_{i-1},x_{i+1}]}\right)
\max\{h_i^2, h_{i+1}^2\}\leqslant
$$
\begin{equation}\label{ark-eq-1.11}
\leqslant \left(2\omega(\|\Delta\|, f^{\prime\prime})+\frac 1{4\mu}\rho_\Delta \|f^{\prime\prime}\|_{[a,b]}\right)
\|\Delta\|^2.
\end{equation}


Для данного интегрального уравнения \eqref{ark-eq-1}  с непрерывной на отрезке $[a,b]$ правой частью
и непрерывного на треугольнике $a\leqslant t \leqslant x \leqslant b$ ядра $K(x,t)$
будем строить параллельно дискретное решение и гладкие решения в виде рациональных сплайн-функций.

Введем дискретную функцию $Y(x)$ с искомыми значениями $y_0, y_1, \dots, y_N$
в соответствующих узлах данной сетки $\Delta_N: a=x_0<x_1<\dots<x_N=b$ $(N\geqslant 2)$
и всюду далее будем считать, что функция
 $R_{n,r}(x, Y, [a, x_n])= R_{n,r}(x, Y, \Delta_n, G_n, \mu)$ для $n=1,2,\dots,N$ и $r=1,2$
получается из выражения рациональной сплайн-функции $R_{n,r}(x,y,[a,x_n])$,
определенной равенством \eqref{ark-eq-1.7}, если там вместо решения $y(x)$ подставить дискретную функцию
$Y(x)$.

Значит, при $x\in[x_{i-1},x_i]$, $i=1,2,\dots,n$, будет выполняться равенство
\begin{equation}\label{ark-eq-2.1}
R_{n,r}(x,Y,[a,x_n])=R_i(x,Y,[a,x_n])A_{i,r}(x)+R_{i-1}(x,Y,[a,x_n]) B_{i,r}(x),
\end{equation}
а также в соответствии с \eqref{ark-eq-1.6} и \eqref{ark-eq-1.8} получим равенства
\begin{equation}\label{ark-eq-2.2}
\begin{array}{l}
 R_{1,r}(x,Y,[a,x_1])=R_1(x,Y,[a,x_1]),\quad R_1(x_1,Y,[a,x_1])=y_1,\\
 R_{n,r}(x,Y,[a,x_n])=R_{n-1}(x,Y,[a,x_n]),\\
R_{n,r}(x_n,Y,[a,x_n])=R_{n-1}(x_n,Y,[a,x_n])=y_n,\quad n=2,3,\dots,N.
\end{array}
\end{equation}

Положим $y_0=\varphi(x_0)=\varphi(a)$ и составим систему алгебраических уравнений относительно
$y_1, y_2, \dots, y_N$ вида
\begin{equation}\label{ark-eq-2.3}
y_n-\lambda \int_a^{x_n} K(x_n,t) R_{n,r}(t,Y,[a,x_n])dt=\varphi(x_n),\quad n=1,2,\dots,N.
\end{equation}

Следующая теорема дает условия однозначной разрешимости системы \eqref{ark-eq-2.3}.
\begin{theorem} \label{ark-theo1}
Если для данных значений параметров $\mu>0$ и $\lambda$ и ядра $K(x,t)$ выполняется
неравенство
\begin{equation}\label{ark-eq-2.4}
|\lambda| M(K)< \frac 1{3(1+\mu)},
\end{equation}
где
$$
M(K)=\sup\left\{\int_a^x |K(x,t)|dt: x\in [a,b]\right\},
$$
то система уравнений \eqref{ark-eq-2.3} имеет единственное решение $(y_1,y_2,\dots,y_N)$.
\end{theorem}


Как следует из теоремы \ref{ark-theo1}, если $y_0=\varphi(x_0)$ и для данных значений параметров
$\mu>0$, $\lambda$ и ядра $K(x,t)$ выполняется условие \eqref{ark-eq-2.4}, то дискретная функция
$Y(x)$ со значениями $y_0, y_1, \dots, y_N$ в соответствующих узлах сетки
$\Delta: a=x_0<x_1<\dots <x_N=b$ $(N\geqslant 2)$ однозначно определяется.

Более того, в качестве динамического решения интегрального уравнения \eqref{ark-eq-1} на
расширяющихся отрезках вида $[a,x_n]$, $n=1,2,\dots,N$, можно взять рациональные
сплайн-функции
$$
R_{n,r}(x,Y,[a,x_n])=R_{n,r}(x,Y,\Delta_n, G_n,\mu)
$$
из класса $C^r_{[a,x_n]}$ $(r=1,2)$, для которых при $x\in[x_{i-1},x_i]$, $i=1,2,\dots,n$,
выполняется равенство
\begin{equation}\label{ark-eq-2.8}
R_{n,r}(x,Y,[a,x_n])=R_i(x,Y,[a,x_n])A_{i,r}(x)+R_{i-1}(x,Y,[a,x_n])B_{i,r}(x).
\end{equation}


Как и выше, будем предполагать, что правая часть $\varphi(x)$ уравнения \eqref{ark-eq-1} и ядро
$K(x,t)$ являются непрерывными функциями соответственно на отрезке $[a,b]$ и на треугольнике
$a\leqslant t\leqslant x\leqslant b$.

Для данной сетки из произвольных узлов $\Delta_N: a=x_0<x_1<\dots <x_N=b$ $(N\geqslant 2)$
будем также придерживаться принятых выше обозначений, в соответствии с которыми рациональные
сплайн-функции $R_{n,r}(x,Y,[a,x_n])=R_{n,r}(x,Y,\Delta_n, G_n,\mu)$, $n=1,2,\dots,N$,
из класса $C^r_{[a,x_n]}$ $(r=1,2)$ определяются равенствами \eqref{ark-eq-2.8}.

Следующее утверждение дает оценку скорости сходимости приближенных решений
$R_{n,r}(x,Y,[a,x_n])$ интегрального уравнения \eqref{ark-eq-1} к его точному решению $y(x)$ на
отрезках вида $[a,x_n]$, $n=1,2,\dots,N$, которые расширяясь исчерпывают
всю область определения $[a,b]$ этого решения $y(x)$.

Оценка получена в терминах величины
\begin{equation}\label{ark-eq-3.1}
E_N(y)=\max\{\|y-R_j(\cdot, y, [a,b])\|_{[x_{j-1},x_{j+1}]}:j=1,2,\dots,N-1\},
\end{equation}
что позволяет оценить скорость сходимости приближенных решений к точному решению
$y(x)$ с учетом его гладкостных свойств. Для этого, как показано далее, можно воспользоваться
аппроксимативными свойствами трехточечных рациональных интерполянтов $R_j(x,y,[a,b])$
вида \eqref{ark-eq-1.2} с соответствующим выбором параметра $\mu>0$.

\begin{theorem} \label{ark-theo2}
Если для данных значений параметров $\mu>0$ и $\lambda$ и ядра $K(x,t)$ выполняется
условие
\begin{equation}\label{ark-eq-3.2}
|\lambda| M(K)< \frac 1{4(1+\mu)},
\end{equation}
то для непрерывного решения $y(x)$ уравнения \eqref{ark-eq-1} и при $r=1,2$ для
рациональных сплайн-функций $R_{n,r}(x,Y,[a,x_n])=R_{n,r}(x,Y,\Delta_n, G_n,\mu)$,
$n=1,2,\dots,N$, из \eqref{ark-eq-2.8} при $x\in[a,x_n]$ имеем
\begin{equation}\label{ark-eq-3.3}
|y(x)-R_{n,r}(x, Y,[a,x_n])|\leqslant  \frac 1{1-4(1+\mu)|\lambda|M(K)}E_N(y).
\end{equation}
\end{theorem}


\section{Заключение}
В случае равномерных сеток узлов $\Delta: a=x_0<x_1<\dots<x_N=b$ $(N\geqslant 2)$ для любого
значения $\mu>0$ вполне аналогично неравенствам \eqref{ark-eq-1.5} при $x\in[x_{i-1}, x_{i+1}]$
для $i=1,2,\dots,n-1$ получаются оценки
$$
|a_i(x)|\leqslant 1,\quad |b_i(x)|\leqslant 2,\quad |c_i(x)|\leqslant 1.
$$

Поэтому для равномерных сеток узлов $\Delta$ заключение
теоремы \ref{ark-theo1} остается справедливым, если в ней условие
на $|\lambda|M(K)$  заменить на неравенство $|\lambda|M(K)<1/3$.

Что касается теоремы \ref{ark-theo2}, в ней в случае равномерных сеток
узлов $\Delta$ условие на $|\lambda|M(K)$  можно заменить
на неравенство $|\lambda|M(K)<1/4$, а правую часть неравенства из
ее заключения -- на выражение $1/(1-4|\lambda|M(K))E_N(y)$.

Отсюда для равномерных сеток узлов и из теоремы \ref{ark-theo2} в общем случае с учетом
 неравенств \eqref{ark-eq-1.9}--\eqref{ark-eq-1.11}
непосредственно вытекают оценки скорости сходимости приближенных гладких решений
$R_{n,r}(x,Y,[a,x_n])$ к точному решению $y(x)$ интегрального уравнения \eqref{ark-eq-1}.
Эти оценки выражаются через модуль непрерывности решения $\omega(\|\Delta\|,y)$ в случае
непрерывности $y(x)$, а в случае существования гладких решений $y(x)$ --
через модули непрерывности производных
  $\omega(\|\Delta\|,y^\prime)$ и $\omega(\|\Delta\|,y^{\prime\prime})$
с соответствующим выбором параметра $\mu>0$.

К примеру, если уравнение \eqref{ark-eq-1} допускает решение $y(x)$ с непрерывной
второй производной  $y^{\prime\prime}(x)$
на отрезке $[a,b]$, то, как следует из \eqref{ark-eq-1.11}, в случае равномерных сеток узлов
$\Delta: a=x_0<x_1<\dots<x_N=b$ $(N\geqslant 2)$ с $x_i-x_{i-1}=h$ $(i=1,2,\dots,N)$ при
выполнении условия $|\lambda|M(K)<1/4$ и выборе параметра $\mu=1/(4h)$
справедливо неравенство
$$
\|y-R_{n,r}(\cdot,Y,[a,x_n])\|_{[a,x_n]}\leqslant \frac 1{1-4|\lambda|M(K)}
\left(2\omega(h,y^{\prime\prime},[a,x_n])+h\|y^{\prime\prime}\|_{[a,x_n]}\right) h^2
$$
для $r=1,2$ и каждого $n=2,3,\dots,N$.



\chapter{Решение интегральных уравнений Фредгольма методом
коллокационных рациональных сплайн-функций}

%\begin{abstract}
%Для произвольных сеток узлов получено приближенное решение интегрального
% уравнения Фредгольма второго рода в виде коллокационной рациональной
%сплайн-функции.
%
%Представлены оценки скорости равномерной сходимости приближенных решений
%к точному решению из класса гладкости $C^r$ для $r=0,1,2$.
%\end{abstract}

\section{Введение}
Многие непрерывно текущие процессы физического характера, химических реакций,
экологии и др., как хорошо известно, моделируются с привлечением
интегральных уравнений.
Учитывая, что к таким уравнениям точные методы решения не всегда применимы,
актуальным
остается вопрос об эффективных приближенных методах их решения. При этом
в случае наличия
элементов
неопределенности в изучаемых процессах востребованы уравнения с параметрами,
а если
предлагается приближенное решение, то желательно, чтобы это решение также
содержало некоторые управляемые параметры.

В данном разделе изучается вопрос приближенного решения интегрального уравнения
Фредгольма второго рода
\begin{equation}\label{ark2-eq-1}
y(x)-\lambda \int_a^b K(x,t)y(t)dt=\varphi(x),\quad x\in[a,b],
\end{equation}
с помощью сплайн-функций относительно рациональных интерполянтов с параметрами.

Правую часть $\varphi(x)$ и ядро $K(x,t)$ полагаем непрерывными функциями соответственно
на отрезке $[a,b]$ и прямоугольнике $[a,b]\times[a,b]$, а величину $\lambda$ -- действительным
параметром, и при этом считаем, что уравнение \eqref{ark2-eq-1} имеет единственное решение $y(x)$, непрерывное
на отрезке $[a,b]$.

Схема приближенного решения интегральных уравнений вида \eqref{ark2-eq-1} с помощью кубических сплайнов
в общих чертах описана в \cite{ark-4} (гл. II, п.2.8).

В \cite{ark-7} (гл.VI, п.3) с помощью периодических кубических и параболических сплайнов
в случае равномерных сеток узлов дано приближенное решение уравнений вида \eqref{ark2-eq-1}
и изучена скорость сходимости приближенных решений к точному.

Оценка погрешности приближенного решения уравнений вида \eqref{ark2-eq-1} периодическими полиномиальными
сплайнами по равномерным сеткам узлов в интегральных метриках исследована в
\cite{ark-12} (гл. 5, п.5.1).

Как известно \cite{ark-4, ark-7}, классические полиномиальные сплайны непрерывных функций
в случае последовательностей произвольных сеток узлов с диаметрами, стремящимися к нулю,
могут не сходиться.

Известно также \cite{ark-9}, что последовательность сплайн-функций относительно рациональных
 трехточечных интерполянтов для любой непрерывной на данном отрезке функции в случае
любой последовательности сеток узлов с диаметрами, стремящимися к нулю, сходится равномерно
 на этом отрезке.

В данном разделе рассматриваются произвольные узлы, точнее, предлагается приближенное
решение интегральных уравнений вида \eqref{ark2-eq-1}, имеющих единственное решение, с помощью
 рациональных сплайн-функций в случае произвольных сеток узлов и получена оценка
 скорости сходимости приближенного решения к точному в зависимости от гладкостных
 свойств точного решения.

Отметим, что интегральные уравнения имеют многочисленные приложения самого разного
характера. В связи, в частности, с этим разработаны различные методы их решения, некоторые
из которых приведены также в \cite{ark-1,ark-2,ark-3}.


\section{Обозначения и вспомогательные утверждения}

Для сетки произвольных узлов $\Delta: a=x_0<x_1<\dots<x_N=b$ $(N\geqslant 2)$ положим
$h_i=x_i-x_{i-1},$ $i=1,2,\dots,N,$ и с параметром  $\mu>0$ построим набор чисел
$g=\{g_1,g_2, \dots,g_{N-1}\}$ таких, что
\begin{equation}\label{ark2-eq-2}
g_i=\begin{cases}
x_{i+1}+\mu h_{i+1}, \,\text{ если }\, h_{i+1}\leqslant h_i,\\
x_{i-1}-\mu h_i, \,\text{ при }\, h_{i+1}> h_i, \quad i=1,2,\dots,N-1.
\end{cases}
\end{equation}

Для функции $f(x)$, определенной на сетке узлов $\Delta$,
при $i=1,2,\dots,N-1$
рассмотрим рациональные интерполянты
\begin{equation}\label{ark2-eq-3}
R_i(x)=R_i(x,f)=\alpha_i+\beta_i(x-x_i)+\gamma_i\frac 1{x-g_i}
\end{equation}
такие, что $R_i(x_j)=f(x_j)$ для $j=i-1,i,i+1$. Из этих условий с использованием
 разделенных разностей имеем
$$
\begin{array}{lcl}
\alpha_i=f(x_i)-f(x_{i-1}, x_i, x_{i+1})(x_{i-1}-g_i)(x_{i+1}-g_i),\\
\beta_i=f(x_{i-1}, x_{i+1})+f(x_{i-1}, x_i, x_{i+1})(x_i-g_i),\\
\gamma_i=f(x_{i-1}, x_i, x_{i+1})(x_{i-1}-g_i)(x_i-g_i)(x_{i+1}-g_i).
\end{array}
$$

Будем считать также, что $R_0(x,f)\equiv R_1(x,f)$, $R_N(x,f)\equiv R_{N-1}(x,f)$.

Всюду ниже для натурального $r$ при $i=1,2,\dots,N$ обозначим
$$
A_{i,r}(x)=\frac{(x-x_{i-1})^r}{(x-x_{i-1})^r+(x_i-x)^r},\quad B_{i,r}(x)=1-A_{i,r}(x)
$$
 и рассмотрим рациональные  сплайн-функции
$R_{N,r}(x,f)=R_{N,r}(x,f,\Delta,g,\mu)$ такие, что при $x\in [x_{i-1}, x_i]$,
 $i=1,2,\dots,N$, выполняется равенство
\begin{equation}\label{ark2-eq-4}
R_{N,r}(x,f)=R_i(x,f)A_{i,r}(x)+R_{i-1}(x,f)B_{i,r}(x).
\end{equation}

Как следует из \cite{ark-9}, $R_{N,r}(x,f)$ представляет собой гладкую функцию
 класса $C^r_{[a,b]}$. При этом из \eqref{ark2-eq-4} имеем
\begin{equation}\label{ark2-eq-5}
R_{N,r}(x_i,f)=R_i(x_i,f)=f(x_i),\quad i=0,1,\dots,N.
\end{equation}

В \cite{ark-10} показано, что для любой непрерывной на данном отрезке $[a,b]$ функции $f(x)$,
произвольной сетки узлов $\Delta: a=x_0<x_1<\dots<x_N=b$ $(N\geqslant 2)$, любого
$\mu>0$ и соответствующей интерполяционной сплайн-функции
$R_{N,r}(x,f)=R_{N,r}(x,f,\Delta,g,\mu)$ для всех $x\in[a,b]$ выполняется неравенство
\begin{equation}\label{ark2-eq-6}
|f(x)-R_{N,r}(x,f)|\leqslant (3+\mu)\omega(\|\Delta\|,f),
\end{equation}
где $\|\Delta\|=\max\{h_i|i=1,2,\dots,N\}$ и, как обычно,
$$
\omega(\delta, f)=\sup\{|f(x+h)-f(x)|: |h|\leqslant \delta; x,x+h\in [a,b]\}
$$
означает равномерный модуль непрерывности функции $f(x)$ на данном отрезке $[a,b]$.

Заметим, что рациональные интерполянты $R_i(x,f)$ из \eqref{ark2-eq-3} для всех $x\in[x_{i-1},x_{i+1}]$
$(i=1,2,\dots,N-1)$ допускают в силу интерполяционности следующее представление:
\begin{equation}\label{ark2-eq-7}
R_i(x,f)=a_i(x)f(x_{i-1})+b_i(x)f(x_i)+c_i(x)f(x_{i+1}),
\end{equation}
в котором
$$
a_i(x)=\frac{(x-x_i)(x-x_{i+1})(x_{i-1}-g_i)}
{(x_{i-1}-x_i)(x_{i-1}-x_{i+1})(x-g_i)},
\quad
b_i(x)=\frac{(x-x_{i-1})(x-x_{i+1})(x_i-g_i)}
{(x_i-x_{i-1})(x_i-x_{i+1})(x-g_i)},
$$
$$
c_i(x)=\frac{(x-x_{i-1})(x-x_i)(x_{i+1}-g_i)}
{(x_{i+1}-x_{i-1})(x_{i+1}-x_i)(x-g_i)},
$$
причем $a_i(x)+b_i(x)+c_i(x)=1$.
Далее будут использоваться также неравенства
\begin{equation}\label{ark2-eq-8}
|a_i(x)|<1+\mu, \quad |b_i(x)|<2(1+\mu),\quad |c_i(x)|<1+\mu,
\end{equation}
которые справедливы при $\mu>0$ для $x\in[x_{i-1},x_{i+1}]$, $i=1,2,\dots,N-1$.

Эти неравенства проще получаются, если воспользоваться другими представлениями
для коэффициентов $a_i(x)$ и $b_i(x)$  , которые получаются \cite{ark-10} из \eqref{ark2-eq-3}.
Так, в случае $h_{i+1}\leqslant h_i$ последовательно имеем:
$$
|a_i(x)|=\left\vert \frac{x-x_i}{x_{i-1}-x_i}+\frac{(x-x_{i-1})(x-x_i)(x_{i+1}-g_i)}
{(x_{i-1}-x_i)(x_{i-1}-x_{i+1})(x-g_i)}\right\vert=
$$
$$
=\frac{|x-x_i|}{x_i-x_{i-1}}
\left[1-\frac{(x-x_{i-1})(g_i-x_{i+1})}{(x_{i+1}-x_{i-1})(g_i-x)}\right]\leqslant 1<1+\mu;
$$
$$
|b_i(x)|=\left\vert \frac{x-x_{i-1}}{x_i-x_{i-1}}+\frac{(x-x_{i-1})(x-x_i)(x_{i+1}-g_i)}
{(x_i-x_{i-1})(x_i-x_{i+1})(x-g_i)}\right\vert=
\frac{x-x_{i-1}}{x_i-x_{i-1}}\cdot\frac{|g_i-x-\mu(x-x_i)|}{g_i-x}=
$$
$$
=(1+\mu)\frac{(x-x_{i-1})(x_{i+1}-x)}{(x_i-x_{i-1})(g_i-x)}<2(1+\mu);
$$
$$
|c_i(x)|=\mu
\frac{(x-x_{i-1})|x-x_i|}{(x_{i+1}-x_{i-1})(g_i-x)},
$$
отсюда
$$
|c_i(x)|=\mu\frac{x-x_{i-1}}{x_{i+1}-x_{i-1}}\cdot \frac{x-x_i}{g_i-x}\leqslant 1<1+\mu,
$$
если $x\in[x_i, x_{i+1}]$, и
$$
|c_i(x)|=\mu\frac{x-x_{i-1}}{x_{i+1}-x_{i-1}}\cdot \frac{x_i-x}{g_i-x}\leqslant \mu<1+\mu,
$$
если $x\in[x_{i-1}, x_i]$.

Случай, когда $h_{i+1}>h_i$, рассматривается вполне аналогично. Неравенства
\eqref{ark2-eq-8} доказаны.


\section{Основные результаты}

Пусть интегральное уравнение \eqref{ark2-eq-1} для данных $\lambda$, непрерывной
на $[a,b]$ правой части $\varphi(x)$ и непрерывном на прямоугольнике $[a,b]\times [a,b]$
ядре $K(x,t)$ имеет единственное решение $y(x)$, непрерывное на $[a,b]$.

Рассмотрим дискретную функцию $Y(x)$, определенную на данной сетке с произвольными узлами
$\Delta: a=x_0<x_1<\dots<x_N=b$ $(N\geqslant 2)$, со значениями
$Y(x_i)=y_i$ для $i=0,1,\dots,N$.

Для этой функции $Y(x)$, параметра $\mu>0$ и набора полюсов $g=\{g_1,g_2,\dots,g_{N-1}\}$
в соответствии со значениями \eqref{ark2-eq-2} построим рациональные интерполянты $R_i(x,Y)$ вида \eqref{ark2-eq-3}
и соответствующую им сплайн-функцию
$R_{N,r}(x,Y)=R_{N,r}(x,Y,\Delta,g,\mu)$ типа \eqref{ark2-eq-4} для значений $r=1,2$.

Как следует из \eqref{ark2-eq-5} и конструкции рациональной сплайн-функции $R_{N,r}(x,Y)$,
будут выполняться равенства
\begin{equation}\label{ark2-eq-9}
R_{N,r}(x_i,Y)=R_i(x_i,Y)=y_i,\quad i=0,1,\dots,N.
\end{equation}

Составим систему линейных алгебраических уравнений относительно неизвестных
$y_0,y_1,\dots,y_N$ с помощью следующих условий коллокации узлов сетки:
\begin{equation}\label{ark2-eq-10}
R_{N,r}(x_i,Y)-\lambda \int_a^b K(x_i,t)R_{N,r}(t,Y)dt=\varphi(x_i),
\end{equation}
$i=0,1,\dots,N$.

Тогда имеет место

\begin{theorem}\label{teor1}
Если для данных значений $\lambda$ и $\mu>0$ и ядра $K(x,t)$ выполняется неравенство
\begin{equation}\label{ark2-eq-11}
|\lambda| \sup_{a\leqslant x\leqslant b} \,\int_a^b |K(x,t)|dt<\frac 1{8(1+\mu)},
\end{equation}
то системой \eqref{ark2-eq-10} однозначно определяется коллокационная рациональная сплайн- функция
$R_{N,r}(x,Y)=R_{N,r}(x,Y,\Delta,g,\mu)$ $(N\geqslant 2; r=1,2)$ вида \eqref{ark2-eq-4}.
 \end{theorem}

В условиях теоремы \ref{teor1} имеет место также

\begin{theorem}\label{teor2}
Если для данных значений $\lambda$ и $\mu>0$ и ядра $K(x,t)$ выполняется неравенство \eqref{ark2-eq-11},
то для непрерывного решения $y(x)$ интегрального уравнения \eqref{ark2-eq-1} и коллокационной
рациональной сплайн-функции $R_{N,r}(x,Y)$ $(N\geqslant 2; r=1,2)$, определяемой системой \eqref{ark2-eq-10},
при любом $x\in[a,b]$ выполняется неравенство
$$
|y(x)-R_{N,r}(x,Y)|\leqslant (10+8\mu)(3+\mu)\omega(\|\Delta\|,y).
$$
\end{theorem}

\section{Заключение}

 Отметим, что из неравенства (18) можно получить также оценки скорости сходимости
приближенных решений $R_{N,r}(x,Y)$ интегрального уравнения \eqref{ark2-eq-1} к его точному решению $y(x)$
класса $C^r_{[a,b]}$ в случаях $r=1$ и $r=2$.

Действительно, по теореме~1.1 из \cite{ark-10} для решения $y(x)$ из класса $C^1_{[a,b]}$ и
его интерполяционной рациональной сплайн-функции $R_{N,1}(x,y)$ вида \eqref{ark2-eq-4} получим
$$
|y(x)-R_{N,1}(x,y)|\leqslant
\left(4+\frac 2\mu\right)\|\Delta\|\omega(\|\Delta\|,y^\prime),\quad x\in[a,b].
$$

Если же решение $y(x)$ принадлежит классу $C^2_{[a,b]}$, то для
его интерполяционной рациональной сплайн-функции $R_{N,2}(x,y)$ вида \eqref{ark2-eq-4}
по теореме~1 из \cite{ark-11} при
$\rho_\Delta=\max\{h_ih_j^{-1}|\,|i-j|=1, 1\leqslant i,j\leqslant N\}$ имеем
$$
|y(x)-R_{N,2}(x,y)|\leqslant \|\Delta\|^2\left(2\omega(\|\Delta\|,y^{\prime\prime})+
\frac1{4\mu} \rho_\Delta \|y^{\prime\prime}\|\right),\quad x\in[a,b].
$$

Подставляя правые части последних двух неравенств в правую часть неравенства (18), получим
соответствующие оценки равномерной сходимости на отрезке $[a,b]$ коллокационных сплайн-функций
$R_{N,r} (x,Y)$ к решению $y(x)$ в случаях $r=1$ и $r=2$.




\chapter{Гладкая интерполяция локальными полиномиальными сплайнами}
%\begin{abstract}
%
%По дискретной функции, заданной в узлах произвольной сетки из данного отрезка
%$[a,b]$ числовой оси, построены локальные полиномиальные  интерполяционные
%сплайны пятой степени.
%
%Для произвольных сеток узлов доказано, что построенные интерполяционные
%сплайны имеют на отрезке $[a,b]$ непрерывные производные до второго порядка
%включительно.
%В случае непрерывных на отрезке $[a,b]$ функций и равномерных сеток узлов дана
%оценка равномерной сходимости построенных локальных сплайнов к функции
%на этом отрезке через равномерный модуль непрерывности функции.
%
%Изучены также аппроксимативные свойства самих сплайнов и их производных
%в случае функций, имеющих непрерывные производные второго порядка на данном
%отрезке $[a,b]$.
%
%Представлены оценки скорости одновременной равномерной сходимости самих
%сплайнов к функции, производных первого и второго порядков от сплайнов
%к соответствующим производным от исходной функции.
%
%При этом оценка скорости сходимости для самих локальных интерполяционных
%сплайнов к функции получена  в терминах модуля непрерывности этой функции,
%а оценки скорости сходимости производных первого  и второго порядков
%от сплайнов соответственно к первой и второй производным функции получены
%в терминах модуля непрерывности соответствующей производной функции.
%
%\end{abstract}

\section{Введение}

Вопросы о сплайн-аппроксимациях и сплайн-интерполяциях особенно актуальны
в численных методах современного анализа и математической физики.
Как математический аппарат для описания кривых и поверхностей сплайн-функции
применяются в задачах построения и оптимизации сложных поверхностей с помощью
компьютеров, а также для исследования различных физических, биологических и
других явлений,  для программного обеспечения медицинского диагностического
оборудования, для решения многих других прикладных задач
(см., например,\cite{ark-4,ark-5,ark-6,ark-7} и цитированные в них источники).

Следует отметить, что интерполяционные сплайны первой степени обладают
хорошими аппроксимативными свойствами, но не являются гладкими. Наибольший
интерес представляют интерполяционные сплайн-функции достаточно высокой
степени гладкости. Поэтому широкую известность получили глобальные
кубические сплайны Шенберга \cite{ark-4}, которые являются интерполяционными и
имеют максимальную для кубических сплайнов гладкость второго порядка.
Для их построения одновременно используются интерполяционные условия
во всех узлах исходной сетки, что приводит к решению систем уравнений
с большим числом неизвестных, а сама задача их построения имеет решение
при определенных краевых условиях.

Поэтому исследуются также локальные сплайны, для построения каждого фрагмента
которых используется лишь несколько интерполяционных условий.

В данном разделе построены локальные полиномиальные сплайны пятой степени,
имеющие гладкость второго порядка, и изучены аппроксимативные свойства самих
сплайнов и их производных до второго порядка включительно.


\section{Основные результаты}

Пусть на некотором отрезке $[a,b]$ задана произвольная сетка узлов
$a=x_0<x_1<\dots <x_N=b$,  $N\geqslant 3$. Присоединим к ним точки
$ x_{-2}<x_{-1}<a, b<x_{N+1}<x_{N+2}$ и возьмем любую конечную функцию $f(x)$,
определенную на множестве $\Delta=\{x_k: -2\leqslant k \leqslant N+2\}$.
Тогда для $k=1,2,\dots,N$ однозначно определяются интерполяционные третьей степени
 полиномы Ньютона $p_k (x)=p_k (x,f)$
такие, что  $p_k (x_j )=f(x_j)$ при $j=k-2,k-1,k,k+1$.

Исходя из этих полиномов, для узлов сетки $\Delta$ построим новые полиномы
$$
P_j (x)=P_j (x,f)=p_j (x) \frac{x-x_{j-2}}{x_j-x_{j-2}}+
p_{j-1}(x)\frac{x_j-x}{x_j-x_{j-2}},
$$
$$
Q_k (x)=Q_k (x,f)=P_{k+1}(x)\frac{x-x_{k-1}}{x_k-x_{k-1}}+
P_k (x)\frac{x_k-x}{x_k-x_{k-1}}.
$$
На отрезке $[a,b]$  рассмотрим кусочно-полиномиальную функцию
 $S_N (x)=S_N (x,f,\Delta)$ такую, что при каждом $k=1,2,\dots,N$
для всех $x\in [x_{k-1},x_k]$ выполняется равенство $S_N (x)=Q_k (x,f)$.

Для сокращения записи далее рассмотрим случай равностоящих узлов.
Всюду ниже будем считать функцию $f(x)$ непрерывной $(b-a)-$периодической,
$h=(b-a)/N$, и рассмотрим узлы $\Delta:x_k=a+kh$, $k=0,\pm 1,\pm 2,\dots$

Доказаны следующие утверждения:

1) Для любой конечной функции $f(x)$, определенной на системе узлов
$\Delta=\{x_k:-2\leqslant k \leqslant N+2\}$, функция
$S_N (x)=S_N (x,f,\Delta)$ на отрезке $[a,b]$
является дважды непрерывно дифференцируемым полиномиальным сплайном.

2) Для любой $(b-a)-$периодической непрерывной функции $f(x)$
при всех $x\in [a,b]$ выполняется неравенство
$$
|S_N (x,f,\Delta)-f(x)|\leqslant 10 \omega(h,f).
$$

3) Для любой $(b-a)-$периодической непрерывной функции $f(x)$,имеющей
непрерывные производные второго порядка, при всех $x\in [a,b]$ выполняются
неравенства
$$
|S_N (x,f,\Delta)-f(x)|\leqslant 9h^2 \omega(h,f^{\prime\prime}),
$$
$$
|S_N^\prime(x,f,\Delta)-f^\prime(x)|\leqslant 18h\omega(h,f^{\prime\prime}),
$$
$$
|S_N^{\prime\prime}(x,f,\Delta)-f^{\prime\prime}(x)|\leqslant 99\omega(h,f^{\prime\prime}).
$$


\section{Заключение}
Отметим, что применения находят также локальные эрмитовы сплайны,
для построения которых
дополнительно требуются интерполяционные условия на производные,
и базисные сплайны, которые не являются интерполяционными.
Поэтому определенный интерес представляет задача построения локальных
интерполяционных сплайн-функций наперед заданной гладкости.




































