\newtheorem{theorem}{Теорема}[chapter]
\newtheorem*{theorem*}{Теорема}
\newtheorem*{lemma*}{Лемма}
\newtheorem*{abstract}{Аннотация}
\newtheorem{property}{Свойство}
\newtheorem{lemma}{Лемма}[chapter]
\newtheorem{statement}{Утверждение}[chapter]
\newtheorem{definition}{Определение}[chapter]
\newtheorem{example}{Пример}[chapter]
\newtheorem{corollary}{Следствие}[chapter]
\newtheorem*{corollary*}{Следствие}
\newtheorem{remark}{Замечание}[chapter]
\newtheorem*{remark*}{Замечание}
\newtheorem{hypothesis}{Гипотеза}[chapter]

\newtheorem{cond}{Условие}
\newtheorem{theoremA}{Теорема}[chapter]
\newtheorem{lemmaA}{Лемма}[chapter]
\newtheorem{corollaryA}{Следствие}[chapter]

\newtheorem{state}{Предложение}
\newtheorem{proposition}{Предложение}
\renewcommand{\thetheoremA}{\thechapter.\Alph{theoremA}}
\renewcommand{\thelemmaA}{\thechapter.\Alph{lemmaA}}
\renewcommand{\thecorollaryA}{\thechapter.\Alph{corollaryA}}
\newcommand{\No}{\textnumero}

\newcommand{\norm}[1]{\|#1\|_{p(\cdot),w}}
\newcommand{\ip}[2]{\langle #1, #2 \rangle}

\DeclareMathOperator*{\esssup}{ess\,sup}
\DeclareMathOperator*{\essinf}{ess\,inf}

\numberwithin{equation}{chapter} %
\renewcommand{\theequation}{\thechapter.\arabic{equation}}

\newenvironment{description}{}{}

\DeclareMathOperator*{\sign}{sign}
\newtheorem{theoremrus}{Теорема}
\newtheorem{lemmarus}{Лемма}
\newtheorem{statementrus}{Утверждение}
\newtheorem{remarkrus}{Замечание}
\newtheorem{corollaryrus}{Следствие}



\def\Ker{\operatorname{Ker}}
