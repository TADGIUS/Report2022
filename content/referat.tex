\Referat %Реферат отчёта, не более 1 страницы

Отчет X с., X рис., X таблиц, X источников.

\MakeUppercase{скалярное произведение типа Соболева, полиномы Якоби, ряд Фурье, равномерная сходимость, пространство Соболева, условия Макенхоупта.}

Объектом исследования являются системы функций, ортогональные относительно дискретно-непрерывных скалярных произведений типа Соболева.

В ходе выполнения НИР изучена сходимость рядов Фурье по соболевской системе, ассоциированной с полиномами Якоби, в равномерной метрике и метрике пространств Соболева.

Полученные результаты могут найти применение в задачах аналитического и численного решения систем дифференциальных уравнений и обработки цифровых сигналов.

Рассмотрена задача об отклонении от функции $f$ из пространства $W^r$ частичных сумм ряда Фурье по системе полиномов Якоби $\{P_n^{\alpha-r,-r}(x)\}$, ортогональной относительно скалярного произведения типа Соболева. Исследовано поведение функции типа Лебега частичных сумм ряда Фурье по системе $\{P_n^{\alpha-r,-r}(x)\}$. Получены оценки в терминах модуля непрерывности $r$ - ой производной функции $f$.

Исследована задача о сходимости ряда Фурье по системе полиномов $\{m_{n,N}^{\alpha,r}(x)\}$, ортонормированной по Соболеву и порожденной системой модифицированных полиномов Мейкснера. В частности, показано, что ряд Фурье по этой системе сходится к $f\in W^r_{l^p_{\rho_N}(\Omega_\delta)}$ поточечно на сетке $\Omega_\delta$ при $p\ge2$. Получены оценки для соответствующей функции Лебега частичных сумм ряда Фурье по системе $\{m_{n,N}^{0,r}(x)\}$. 