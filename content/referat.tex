\Referat %Реферат отчёта, не более 1 страницы

Отчет X с., X рис., X таблиц, X источников.

\MakeUppercase{скалярное произведение типа Соболева, полиномы Якоби, полиномы Мейкснера, ряд Фурье, равномерная сходимость, пространства Соболева и Лебега, условия Макенхоупта, уравнение Бельтрами, задача Римана -- Гильберта.
}
Объектом исследования являются системы полиномов, ортогональные относительно дискретно-непрерывных скалярных произведений типа Соболева; интерполяционные рациональные сплайн-функции;
базисность полиномов Лежандра в пространствах Лебега с переменным показателем;
стохастические системы с запаздыванием;
усреднения периодической задачи для уравнения Бельтрами;
интегральные преобразования скалярных и векторных полей;
проверка интервальной раскрашиваемости всех биграфов заданного порядка;
чистые и разбавленные модели Поттса.

В ходе выполнения НИР изучены сходимость и аппроксимативные свойства рядов Фурье по полиномам Якоби -- Соболева, Мейкснера -- Соболева,
получены динамическое решение интегрального уравнения Вольтерры второго рода и приближенное решение уравнения Фредгольма второго рода и точные по порядку оценки скорости сходимости приближенных решений,
были получены оценки и представления для ядра, играющего важную роль при изучении базисности системы полиномов Лежандра в пространствах Лебега с переменным показателем.
Предложен и обоснован модифицированный метод регуляризации для анализа различных видов устойчивости стохастических систем, содержащий одновременно компоненты с непрерывным и дискретным временем и получены достаточные условия моментной устойчивости решений для таких систем.
Изучены вопросы усреднения обобщенного уравнения Бельтрами с локально периодическими коэффициентами и получены оценки погрешности усреднения периодической задачи для обобщенного уравнения Бельтрами в пространствах Соболева и Лебега.
Получены новые формулы обращения интегральных преобразований скалярных и векторных полей, определенных на некоторых семействах ломаных на плоскости.
Разработано программное обеспечение для решения перечислительных проблем дискретной математики, задач компьютерной графики, а также для создания демонстрационного материала по дисциплинам компьютерных наук.
Методом Ванга-Ландау исследована трех-вершинная
модель Поттса на решетке Кагоме. Определены структуры основного состояния и построена
фазовая диаграмма. 2. Построена фазовая диаграмма зависимости параметра от величины внешнего
магнитного поля для четырех-вершинной модели Поттса на гексагональной решетке. Проведен
анализ фазовых переходов этой модели при различных значениях величины магнитного поля; 3.
Однокластерным алгоритмом Вольфа исследованы фазовые переходы в двумерных однородных
моделях Поттса с числом состояний спина q=4 и q=5 на гексагональной решетке. Показано, что в
двумерной модели Поттса с q=5 на гексагональной решетке наблюдается фазовый переход
первого рода в соответствии с предсказаниями аналитических теорий, а в модели Поттса с q=4 на гексагональной решетке – ФП второго рода.
Отдельно исследована трехмерная слабо разбавленная модель Поттса с q=5 на простой кубической решетке. 


Полученные результаты могут найти применение в задачах аналитического и численного решения систем дифференциальных уравнений и обработки цифровых сигналов.
Область применения результатов -- области науки и техники, связанные с неразрушающей реконструкцией: биомедицинская визуализация, национальная безопасность, гамма-астрономия и др. На основе точных формул обращения могут быть составлены алгоритмы численного восстановления.
Интервальная раскраска двудольного графа является графической интерпретацией беспростойного расписания мультипроцессорной системы. Задача о димерных числах возникает в исследованиях свойств химических соединений, а также при исследовании адсорбции двухатомных молекул на поверхности.
