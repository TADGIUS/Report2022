\Referat %Реферат отчёта, не более 1 страницы

Отчет X с., X рис., X таблиц, X источников.

\MakeUppercase{полиномы Якоби -- Соболева, полиномы Мейкснера -- Соболева, ряд Фурье, равномерная сходимость, пространства Соболева и Лебега, условия Макенхоупта, рациональные сплайны, уравнение Бельтрами, задача Римана -- Гильберта, W-метод, уравнение Ито, метод Ванга-Ландау.
}

Объектом исследования являются системы полиномов, ортогональные относительно дискретно-непрерывного скалярного произведения Соболева; интерполяционные рациональные сплайн-функции;
базисность полиномов Лежандра в пространствах Лебега с переменным показателем;
стохастические системы с запаздыванием;
усреднения периодической задачи для уравнения Бельтрами;
интегральные преобразования скалярных и векторных полей;
проверка интервальной раскрашиваемости всех биграфов заданного порядка;
чистые и разбавленные модели Поттса.

В ходе выполнения НИР изучена сходимость рядов Фурье по полиномам Якоби -- Соболева, Мейкснера -- Соболева и исследованы аппроксимативные свойства их частичных сумм в различных функциональных пространствах. Получены динамическое решение интегрального уравнения Вольтерры второго рода в виде коллокационных рациональных сплайн-функций и точные по порядку оценки скорости сходимости приближенных решений. Получено также приближенное решение интегрального уравнения Фредгольма второго рода в случае произвольных сеток узлов.
Получены утверждения, играющие важную роль при изучении базисности системы полиномов Лежандра в пространствах Лебега с переменным показателем.
Предложен и обоснован модифицированный метод регуляризации для анализа различных видов устойчивости стохастических систем, содержащий одновременно компоненты с непрерывным и дискретным временем, и получены достаточные условия моментной устойчивости решений для таких систем.
Изучены вопросы усреднения обобщенного уравнения Бельтрами с локально периодическими коэффициентами и получены оценки погрешности усреднения периодической задачи для обобщенного уравнения Бельтрами в пространствах Соболева и Лебега.
Получены новые формулы обращения интегральных преобразований скалярных и векторных полей, определенных на некоторых семействах ломаных на плоскости.
Разработано программное обеспечение для решения перечислительных проблем дискретной математики, задач компьютерной графики, а также для создания демонстрационного материала по дисциплинам компьютерных наук.
Методом Ванга-Ландау исследована трехвершинная модель Поттса на решетке Кагоме. 
Определены структуры основного состояния и построена фазовая диаграмма. 

Полученные результаты могут найти применение в задачах математической физики, цифровой обработки сигналов, теории управления и в задачах составления расписаний мультипроцессорной системы.
