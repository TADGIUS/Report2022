\Referat %Реферат отчёта, не более 1 страницы

Отчет X с., X рис., X таблиц, X источников.

\MakeUppercase{скалярное произведение типа Соболева, полиномы Якоби, ряд Фурье, равномерная сходимость, пространство Соболева, условия Макенхоупта,
%Сиражудинов
оценки погрешности усреднения периодической задачи  для уравнения бельтрами
с локально-периодическим коэффициентом, оценки локально-периодического усреднения задачи римана-гильберта для обобщенного уравнения бельтрами.
}

%MMG объект
Объектом исследования являются системы функций, ортогональные относительно дискретно-непрерывных скалярных произведений типа Соболева.


%Рамазанов А.Р.-К.

Объектом исследования являются интерполяционные рациональные сплайн-функции.

%ШТН
Объектом исследования является базисность полиномов Лежандра в пространствах Лебега с переменным показателем.

%КРИ
Объект исследования:
Стохастические системы с запаздыванием, содержащие одновременно компоненты с непрерывным и дискретным временем.

%СММ ошибка 404


%Меджидов
Объект исследования - интегральные преобразования скалярных и векторных полей, заданных на семействах прямых и ломаных на плоскости и в трехмерном евклидовом пространстве.

%АКМ
\underline{Объектом исследования}
явились две перечислительные задачи комбинаторики: 1) проверка интервальной раскрашиваемости всех биграфов заданного порядка; 2) вопросы подсчета димерного числа $T(m, n)$ -- числа совершенных паросочетаний в решёточном графе $m\times n$. (при фиксированной ширине m принято обозначать $T(m, n)$ через $a_n$); \underline{объектом разработки} явились программы, предназначенные для компьютерного сопровождения задач 1-2, а также для решения ряда актуальных прикладных задач.

%Физики
Объектом исследования являются – чистые и разбавленные модели Поттса.

..............................................................................................................................................

%GRM

Рассмотрена задача об отклонении от функции $f$ из пространства $W^r$ частичных сумм ряда Фурье по системе полиномов Якоби $\{P_n^{\alpha-r,-r}(x)\}$, ортогональной относительно скалярного произведения типа Соболева. Исследовано поведение функции типа Лебега частичных сумм ряда Фурье по системе $\{P_n^{\alpha-r,-r}(x)\}$. Получены оценки в терминах модуля непрерывности $r$ - ой производной функции $f$.

Исследована задача о сходимости ряда Фурье по системе полиномов $\{m_{n,N}^{\alpha,r}(x)\}$, ортонормированной по Соболеву и порожденной системой модифицированных полиномов Мейкснера. В частности, показано, что ряд Фурье по этой системе сходится к $f\in W^r_{l^p_{\rho_N}(\Omega_\delta)}$ поточечно на сетке $\Omega_\delta$ при $p\ge2$. Получены оценки для соответствующей функции Лебега частичных сумм ряда Фурье по системе $\{m_{n,N}^{0,r}(x)\}$.

%MMG

В ходе выполнения НИР изучена сходимость рядов Фурье по соболевской системе, ассоциированной с полиномами Якоби, в равномерной метрике и метрике пространств Соболева.

Полученные результаты могут найти применение в задачах аналитического и численного решения систем дифференциальных уравнений и обработки цифровых сигналов.

%Рамазанов А.Р.-К.

%Цель - решение интегральных уравнений Вольтерры и Фредгольма второго рода
%в виде коллокационных рациональных сплайн-функций.

Получено динамическое решение интегрального уравнения Вольтерры второго рода,
которое представлено в виде коллокационных рациональных сплайн-функций
на последовательных отрезках, исчерпывающих всю область решения. Представлены
точные по порядку оценки скорости сходимости приближенных решений
к точному. Получено также приближенное решение интегрального уравнения Фредгольма
второго рода в случае произвольных сеток узлов.

%Возросший в последние годы интерес к дальнейшему исследованию интегральных
%уравнений Вольтерры, в частности, связан с поиском более эффективных методов
%решения задач математической физики, с востребованностью решения задач,
%описывающих модели биологии, экологии, экономики, а также с потребностью
%приложений интегральных уравнений к решению задач по моделированию
%развивающихся динамических систем.

%ШЭТН
Была предпринята попытка ослабить условия базисности системы полиномов Лежандра в пространствах Лебега с переменным показателем. Были получены оценки и представления для ядра, играющего важную роль при изучении вопроса базисности данной системы.

%КРИ
Полученные результаты:
Предложен и обоснован модифицированный метод регуляризации для анализа различных видов устойчивости стохастических систем, содержащий одновременно компоненты с непрерывным и дискретным временем. основанный на вы-боре вспомогательного уравнения и применении теории положительно обра-тимых матриц. Получены достаточные условия моментной устойчивости реше-ний для таких систем.




%Сиражудинов


%Локальные характеристики математических моделей сильно неоднородных сред, как правило, описываются быстро осциллирующими функциями. Следовательно, соответствующие математические модели --- дифференциальные уравнения с быстро осциллирующими коэффициентами.
Изучены вопросы усреднения уравнения Бельтрами и обобщенного уравнения Бельтрами с локально периодическими коэффициентами.

Получены оценки погрешности усреднения периодической задачи для уравнения Бельтрами в пространствах Соболева и Лебега и аналогичные оценки по задаче Римана-Гильберта для обобщенного уравнения Бельтрами.


%Меджидов
Цель работы  -- получить формулы обращения V-преобразования Радона и лучевых преобразований векторных полей.

Метод исследования -- разложение в ряд Фурье (метод Кормака), применение преобразования Фурье по части переменных, применение формул обращения классического преобразования Радона.

Результат работы -- получены новые формулы обращения интегральных преобразований скалярных и векторных полей, определенных на некоторых семействах ломаных на плоскости.

Область применения результатов -- области науки и техники, связанные с неразрушающей реконструкцией: биомедицинская визуализация, национальная безопасность, гамма-астрономия и др. На основе точных формул обращения могут быть составлены алгоритмы численного восстановления.

%АКМ


\underline{Цели работы.}
1) Найти способ перечисления представителей классов изоморфизма множества всех биграфов заданного порядка и разработать алгоритм проверки интервальной раскрашиваемости заданного биграфа. Ввиду NP-полноты задач от искомых алгоритмов требуется лишь реализуемость <<за практически приемлемое время>>. 2) Доказать существование представления $a_n$ в виде линейной комбинации $a_0, a_1, ..., a_{n-1}$ с целыми постоянными коэффициентами и выяснить единственность такого представления.

\underline{Методы/методология проведения работы.}
 Ввиду особенностей вычислительной сложности задач необходимым условием успеха является компьютерное сопровождение исследований; например, при исследовании задачи 2 для реальных данных приходится решать системы л.а.у., где число неизвестных и достигает несколько тысяч.

\underline{Результаты работы и их новизна.}
Авторский алгоритм, разработанный ранее для проверки интервальной раскрашиваемости, усилен фрагментом «Метод увеличивающего пути», что несколько сократило время счета (новый результат). Доказано существование прямого рекуррентного соотношения для $a_n$ (новым является принципиально иной подход к доказательству и его элементарность) и показать, что оно не единственно.

Разработано программное обеспечение для решения перечислительных проблем дискретной математики, задач компьютерной графики, а также для создания демонстрационного материала по дисциплинам компьютерных наук.

\underline{Область применения результатов}.
Интервальная раскраска двудольного графа является графической интерпретацией беспростойного расписания мультипроцессорной системы. Задача о димерных числах возникает в исследованиях свойств химических соединений, а также при исследовании адсорбции двухатомных молекул на поверхности.

%Физики

%Цель работы – разработка микроскопических моделей функционально важных реальных
%чистых и разбавленных магнитных структур, а также исследование их свойств в широком диапазоне
%изменения термодинамических параметров методами современного компьютерного моделирования
%с использованием новейших, специальных и высокоэффективных алгоритмов метода Монте-Карло.
%Методы или методология проведения работы – для проведения исследований использовались
%современные высокоэффективные алгоритмы метода Монте-Карло (Метрополиса, одно-кластерный,
%Ванга-Ландау, репличный, гибридные).
Результаты работы и их новизна – 1. Методом Ванга-Ландау исследована трех-вершинная
модель Поттса на решетке Кагоме. Определены структуры основного состояния и построена
фазовая диаграмма. 2. Построена фазовая диаграмма зависимости параметра от величины внешнего
магнитного поля для четырех-вершинной модели Поттса на гексагональной решетке. Проведен
анализ фазовых переходов этой модели при различных значениях величины магнитного поля; 3.
Одно-кластерным алгоритмом Вольфа исследованы фазовые переходы в двумерных однородных
моделях Поттса с числом состояний спина q=4 и q=5 на гексагональной решетке. Показано, что в
двумерной модели Поттса с q=5 на гексагональной решетке наблюдается фазовый переход
первого рода в соответствии с предсказаниями аналитических теорий, а в модели Поттса с q=4 на
гексагональной решетке – ФП второго рода. 4. Отдельно исследована трехмерная слабо
разбавленная модель Поттса с q=5 на простой кубической решетке. Внесение слабого
вмороженного беспорядка в виде немагнитных примесей концентрацией с (с = 0.10) в эту модель
не приводит к фазовому переходу второго рода.
%Область применения результатов – полученные результаты могут найти применение при
%подготовке к производству различных функциональных элементов электроники.
%Рекомендации по внедрению или итоги внедрения результатов НИР – результаты
%исследования дают возможность оценить применимость той или иной магнитной структуры в
%качестве элемента микроэлектроники.


