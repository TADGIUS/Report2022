\Conclusion

В 2022 году в Отделе математики и информатики Дагестанского федерального иссле\-довательского центра РАН по теме НИР были получены следующие основные результаты.

%GRM

Продолжено исследование задачи об отклонении от функции $f\in W^r$ на $(-1,1)$ ряда Фурье по системе полиномов Якоби--Соболева $\{P_n^{\alpha-r,-r}(x)\}$, начатое в \cite{Ram-SharMN}. В частности были получены оценки для функций типа Лебега частичных сумм ряда Фурье по системе полиномов $\{P_n^{\alpha-r,-r}(x)\}$ (см. теоремы \ref{Ram-theo1}-\ref{Ram-theo3}).

Была рассмотрена система полиномов $\{m_{n,N}^{\alpha,r}(x)\}$, ортонормированная по Соболеву на сетке $\Omega_\delta=\{0, \delta, 2\delta, \ldots\}$ и порожденная системой модифицированных полиномов Мейкснера $\{m_{n,N}^{\alpha}(x)\}$. Показано, что ряд Фурье по этой системе сходится к $f\in W^r_{l^p_{\rho_N}(\Omega_\delta)}$ поточечно на сетке $\Omega_\delta$ при $p\ge2$. А в случае, когда $1\le p<2$ показано, что существуют функция и сетка $\Omega_\delta$, ряд Фурье которой расходится в некоторой точке $x_0\in\Omega_\delta$. Кроме того, исследованы аппроксимативные свойства частичных сумм ряда Фурье по системе $\{m_{n,N}^{0,r}(x)\}$.

%MMG

Для функций из $W^r_{L^1_\rho(\alpha,\beta)}$, $-1 <\alpha, \beta  \le 0$, исследована равномерная сходимость рядов Фурье по системам полиномов, ортогональных в смысле Соболева и порожденных системами полиномов Якоби с показателями $\alpha, \beta$.

Получены необходимые и достаточные условия сходимости в пространстве $W^r_{L^p_\rho(A,B)}$, ($p > 1$, $A, B \in \mathbb{R}$) рядов Фурье по соболевской системе полиномов, порожденной полиномами Якоби с показателями $\alpha, \beta  > -1$. Показано также, что при дополнительном условии на $A, B$ и $p$ указанные ряды сходятся равномерно на отрезке $[-1,1]$.

%ARK

Представлен новый метод динамического решения в виде рациональных
сплайн-функций интегрального уравнения Вольтерры второго рода.

Найдены условия, при выполнении которых интегральное уравнение
Фредгольма второго рода допускает приближенное решение в виде
коллокационных рациональных сплайн-функций.

Как для уравнения Вольтерры, так и для уравнения Фредгольма
представлены точные по порядку оценки скорости сходимости
приближенных решений к точному решению интегрального уравнения.

Применяемый метод поиска приближенного решения в виде
коллокационных рациональных сплайн-функций в силу
сравнительной простоты их структуры
может находить эффективные применения
при решении различных задач математической
физики и других задач численных методов.