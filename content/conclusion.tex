\Conclusion

В 2022 году в Отделе математики и информатики Дагестанского федерального иссле\-довательского центра РАН по теме НИР были получены следующие основные результаты.

%GRM

%Была рассмотрена задача об отклонении от функции $f$ из пространства $W^r$ частичных сумм ряда Фурье по ортогональной по Соболеву системе полиномов $\{\varphi_n(x)\}_{n=0}^\infty$, в которой $\varphi_n(x)=\frac{(x+1)^n}{n!}$ при $0\le n\le r-1$ и $\varphi_n(x)=\frac{2^r}{(n+\alpha-r)^{[r]}\sqrt{h_{n-r}^{\alpha,0}}}P_n^{\alpha-r,-r}(x)$ при $n\ge r$, где $P_n^{\alpha-r,-r}(x)$ -- полином Якоби степени $n$. Получены оценки сверху для функции типа Лебега частичных сумм ряда Фурье по системе $\{\varphi_n(x)\}_{n=0}^\infty$.

%Была исследована задача о сходимости ряда Фурье по полиномам, ортогональным по Соболеву и порожденным полиномами Мейкснера. Кроме того, были исследованы аппроксимативные свойства частичных сумм Фурье по указанным полиномам. В частности, получены оценки для функции Лебега, зависящие от расположения переменной $x$ на полуоси $[r\delta, \infty)$.


Продолжено исследование задачи об отклонении от функции $f\in W^r$ на $(-1,1)$ ряда Фурье по системе полиномов Якоби--Соболева $\{P_n^{\alpha-r,-r}(x)\}$, начатое в \cite{mmg-Shii-matzam2017}. В частности были получены оценки для функций типа Лебега частичных сумм ряда Фурье по системе полиномов $\{P_n^{\alpha-r,-r}(x)\}$ (см. теоремы \ref{Ram-theo1}-\ref{Ram-theo3}).

Была рассмотрена система полиномов $\{m_{n,N}^{\alpha,r}(x)\}$, ортонормированная по Соболеву на сетке $\Omega_\delta=\{0, \delta, 2\delta, \ldots\}$ и порожденная системой модифицированных полиномов Мейкснера $\{m_{n,N}^{\alpha}(x)\}$. Показано, что ряд Фурье по этой системе сходится к $f\in W^r_{l^p_{\rho_N}(\Omega_\delta)}$ поточечно на сетке $\Omega_\delta$ при $p\ge2$. А в случае, когда $1\le p<2$ показано, что существуют функция и сетка $\Omega_\delta$, ряд Фурье которой расходится в некоторой точке $x_0\in\Omega_\delta$. Кроме того, исследованы аппроксимативные свойства частичных сумм ряда Фурье по системе $\{m_{n,N}^{0,r}(x)\}$.

%MMG

Для функций из $W^r_{L^1_\rho(\alpha,\beta)}$, $-1 <\alpha, \beta  \le 0$, исследована равномерная сходимость рядов Фурье по системам полиномов, ортогональных в смысле Соболева и порожденных системами полиномов Якоби с показателями $\alpha, \beta$.

Получены необходимые и достаточные условия сходимости в пространстве $W^r_{L^p_\rho(A,B)}$, ($p > 1$, $A, B \in \mathbb{R}$) рядов Фурье по соболевской системе полиномов, порожденной полиномами Якоби с показателями $\alpha, \beta  > -1$. Показано также, что при дополнительном условии на $A, B$ и $p$ указанные ряды сходятся равномерно на отрезке $[-1,1]$.

%ARK

Представлен новый метод динамического решения в виде рациональных
сплайн-функций интегрального уравнения Вольтерры второго рода.

Найдены условия, при выполнении которых интегральное уравнение
Фредгольма второго рода допускает приближенное решение в виде
коллокационных рациональных сплайн-функций.

Как для уравнения Вольтерры, так и для уравнения Фредгольма
представлены точные по порядку оценки скорости сходимости
приближенных решений к точному решению интегрального уравнения.

Применяемый метод поиска приближенного решения в виде
коллокационных рациональных сплайн-функций в силу
сравнительной простоты их структуры
может находить эффективные применения
при решении различных задач математической
физики и других задач численных методов.

%ШЭТН

При изучении вопроса базисности системы Лежандра в пространствах Лебега с переменным показателем важную роль играет ограниченность операторов $T_i(f)$ в $L^{p(\cdot)}$ ($i=1,2$). В ходе исследования ограниченности  этих операторов  оказалось удобным разбить интеграл на три части:
\begin{equation*}
\int_{-1}^1 T_i(f)(x)^{p(x)}dx=\Bigl(\int_{-1}^{-1+\varepsilon} + \int_{-1+\varepsilon}^{1-\varepsilon} + \int_{1-\varepsilon}^{1}\Bigr)T_i(f)(x)^{p(x)}dx=J_1+J_2+J_3,
\end{equation*}
и оценить их по отдельности. Неравенство
$$
J_2\le c(p),
$$
было показано в \cite{tad-SHII-Leg}, а чтобы показать ограниченность величин $J_1$ и $J_3$ там потребовалось наложить на показатель условие постоянства на концах $[-1,1]$. Полученные нами результаты, связанные с изучением свойств ядра $K(x,y)$, позволяют рассчитывать на устранение условия постоянства на концах отрезка $[-1,1]$.

%КРИ
За отчетный период развит метод регуляризации для анализа различных видов устойчивости  для  стохастических систем с запаздыванием, содержащий одновременно компоненты с непрерывным и дискретным временем. Получены  достаточные условия моментной устойчивости решений как в терминах положительной обратимости матриц, построенных по параметрам этих систем, так и в терминах коэффициентов. Проверяется выполнимость этих условий для конкретных систем уравнений.

%СММ
Задачи рассмотренные нами возникают при изучении плоско-парал\-лельных физических процессов в сильно неоднородных
 средах периодической или локально-периодической структуры.
Получены результаты по оценке погрешности усреднения задачи Римана -- Гильберта для обобщенного уравнения Бельтрами
 с локально-периодическими коэффициентами.


%Меджидов

Решена задача обращения $V$-преобразования, или обобщенного преобразования Радона на плоскости. $V$-образные ломаные, по которым берутся интегралы, имеют вершину внутри области восстановления. Полученная формула обобщает известные формулы на случай весовой степенной функции.  Решена также задача обращения интегрального преобразования на семействе ломаных, входящие в круг нормально к окружности и преломляющихся внутри круга.
Методы решения приведенных задач могут быть применены при обращении $V$-преобразований векторных и тензорных полей.


%АКМ

Было разботано методическое, алгоритмическое и программное обеспечение. Всего было выполнено около 20 проектов для целей компьютерного сопровождения монографии в.н.с. ОМИ Магомедова А.М., изданной в этом году. По результатам обсуждения на заседаниях семинара ОМИ выяснилось, что некоторые из них имеют перспективы, которые сложно было распознать и обозначить в исходной формулировке задачи. Так, например, проект, рассчитанный на визуализацию дискретных звуковых сигналов для помощи лицам с ограниченными возможностями, представляется естественным продолжить включением функций для общения с обездвиженными больными с сохранением движений глаз.

%Физики

Исследование магнитных структур основного состояния, фазовых переходов и
термодинамических свойств двумерной модели Поттса с числом состояний спина $q=3$ на решетке
Кагоме с учетом взаимодействий первых и вторых ближайших соседей выполнено с
использованием алгоритма Ванга-Ландау метода Монте-Карло. Получены магнитные структуры
основного состояния в широком интервале значений величины взаимодействия вторых
ближайших соседей. Построена фазовая диаграмма зависимости критической температуры от
величины взаимодействия вторых ближайших соседей. Для значения $J_1J_2=0.5$ наблюдается
вырождение основного состояния, и система становится фрустрированной.
В данной работе нами показано, что в двумерной ФМ модели Поттса с числом состояний
спина $q=4$ на гексагональной решетке с учетом взаимодействий первых и вторых ближайших
соседей в интервалах значений магнитного поля $0.0 \le h \le 1.0$ и $2.0 \le h \le 3.5$ наблюдается ФП
первого рода, а при значении поля $h = 1.5$ -- ФП второго рода. Обнаружено, что в интервале $4.0 \le h
\le 7.0$ магнитное поле снимается вырождение основного состояния и ФП размывается.
Полученные в результате наших исследований данные свидетельствуют о том, что в
двумерной модели Поттса с $q=5$ на гексагональной решетке наблюдается фазовый переход
первого рода в соответствии с предсказаниями аналитических теорий, а в модели Поттса с $q=4$ --
фазовый переход второго рода. В случае трехмерной модели Поттса с $q=5$ на простой кубической
решетке внесение слабого немагнитного беспорядка при концентрации спинов $p=0.90$ не приводит
к смене фазового перехода первого рода на фазовый переход второго рода.
Результаты, полученные в ходе исследований, могут быть полезными для описания
различных низкоразмерных магнитных материалов, имеющих структуру типа решетки Кагоме,
таких как Гербертсметиты, Делафосситы, Капелласиты, Фольбортиты и т.д. Результаты,
полученные для слабо разбавленных магнитных структур, имеют важное значение при создании
новых магнитных материалов. 