\begin{thebibliography}{111}

% Ниже указаны примеры форматирования литературы

% \bibitem{mmg-MarcellanXu2015}
% Marcellán F., Xu Y. On Sobolev orthogonal polynomials // Expositiones Math. --- 2015. --- Vol 33. P. 308---352.

% \bibitem{mmg-mmg-walsh-Shii-UMN}
% Шарапудинов И.И. Ортогональные по Соболеву системы функций и некоторые их приложения // УМН. --- 2019. --- Т. 74, \No 4(448). --- С. 87---164.

% \bibitem{ark-bib-2}
% Стечкин С.Б., Субботин Ю.Н.
% Сплайны в вычислительной математике.
% --- М.: Наука,
% 1976. --- 248~с.

% \bibitem{ark-bib-3}
% Завьялов Ю.С., Квасов Б.И., Мирошниченко В.Л.
% Методы сплайн-функций.
% --- М.: Наука,
% 1980.
% --- 352 c.


%MMG

\bibitem{mmg-MarcellanXu2015}
Marcellan F., Xu Y.
On Sobolev orthogonal polynomials
//
Expo Math.
--- 2015.
--- Vol 33.
--- P. 308---352.

\bibitem{mmg-MarcellanJacobiSobolev}
Marcellan, F., Quintana, Y., Urieles, A.
On the Pollard decomposition method applied to some Jacobi--Sobolev expansions.
//
Turk. J. Math.
--- 2013.
--- 37(6).
--- P. 934---948

\bibitem{mmg-CiaurriJacobiSobolev}
Ciaurri, O., Minguez, J.
Fourier series of Jacobi--Sobolev polynomials.
//
Integral Transf. Spec. Funct.
--- 2019.
--- V. 30.
--- P. 334---346.

\bibitem{mmg-CiaurriCoherentPairs}
Ciaurri O., Minguez J.
Fourier series for coherent pairs of Jacobi measures. Preprint.

\bibitem{mmg-Fejzullahu2010}
Fejzullahu B.\,Xh.
Asymptotic properties and Fourier expansions of orthogonal polynomials with a non-discrete Gegenbauer--Sobolev inner product.
//
Journal of Approximation Theory.
--- 2010.
--- V. 162, iss. 2.
--- P. 397---406.
ISSN 0021-9045, https://doi.org/10.1016/j.jat.2009.07.002.

\bibitem{mmg-Fejzullahu2013}
Fejzullahu B.\,Xh., Marcellan F., Moreno--Balcazar J.J.
Jacobi-Sobolev orthogonal polynomials: Asymptotics and a Cohen type inequality
//
Journal of Approximation Theory.
--- 2013.
--- V. 170.
--- P. 78---93.
ISSN 0021-9045, https://doi.org/10.1016/j.jat.2012.05.015.

\bibitem{mmg-IserlesKoch1991}
Iserles A., Koch P.E., Norsett S.P., Sanz--Serna J.M.
On polynomials orthogonal with respect to	certain Sobolev inner product.
//
J. Approx. Theory.
--- 1991.
--- V. 65.
--- P. 151---175.

\bibitem{mmg-Marcellan2002}
Marcellan F., Osilenker B.P., Rocha I.A.
On Fourier Series of a Discrete Jacobi--Sobolev Inner Product
//
Journal of Approximation Theory.
--- 2002.
--- V. 117, iss. 1.
--- P. 1---22.
ISSN 0021-9045, https://doi.org/10.1006/jath.2002.3681.

\bibitem{mmg-Rocha2003}
Rocha I.A., Marcellan F., Salto L.
Relative asymptotics and Fourier series of orthogonal polynomials with a discrete Sobolev inner product
//
J. Approx. Theory.
--- 2003.
--- V. 121.
--- P. 336---356.

\bibitem{mmg-OsilenkerFourier2012}
Осиленкер Б.П.
Сходимость и суммируемость рядов Фурье--Соболева
//
Вестник МГСУ.
--- 2012.
--- №5.
--- URL: https://cyberleninka.ru/article/n/shodimost-i-summiruemost-ryadov-furie-soboleva-1 (дата обращения: 23.10.2020).

\bibitem{mmg-OsilenkerLinearMethods2015}
Осиленкер Б.П.
О линейных методах суммирования рядов Фурье по многочленам, ортогональным в дискретных пространствах Соболева
//
Сиб. матем. журн.
--- 2015.
--- Т. 56, №2.
--- С. 420---435.	

\bibitem{mmg-Fejzullahu2009}
Fejzullahu, Marcellan.
On convergence and divergence of Fourier expansions with respect to some Gegenbauer--Sobolev type inner product.
//
Communications in the Analytic Theory of Continued Fractions.
--- 2009.
--- n. 16.
--- P. 1---11.

\bibitem{mmg-CiaurriSigma2018}
Ciaurri O., Minguez J.
Fourier series of Gegenbauer--Sobolev polynomials. SIGMASymm. Integrabi.
//
Geom. Methods Appl.
--- 2018.
--- V. 14.
--- P. 1---11.

\bibitem{mmg-SharapudinovUMN}
Шарапудинов И.И.
Ортогональные по Соболеву системы функций и некоторые их приложения
//
УМН
--- 74:4(448)
--- 2019.
--- С. 87---164.

\bibitem{mmg-SharapudinovIzvRan2019}
Шарапудинов И.И.
Системы функций, ортогональные по Соболеву, ассоциированные с ортогональной системой
//
Изв. РАН. Сер. матем.
--- 82:1
--- 2018.
--- С. 225---258.

\bibitem{mmg-MMG2019}
Магомед-Касумов М.Г.
Система функций, ортогональная в смысле Соболева и порожденная системой Уолша
//
Матем. заметки
--- 105:4
--- 2019.
--- С. 545---552.
Math. Notes, 105:4 (2019), 543---549.

\bibitem{mmg-Gadzhimirzaev2019}
Gadzhimirzaev R.M.
Sobolev--orthonormal system of functions generated by the system of Laguerre functions
//
Пробл. анал. Issues Anal.
--- 8(26):1
--- 2019.
--- P. 32---46.

\bibitem{mmg-Diaz-Gonzalez2020}
Diaz-Gonzalez A., Marcellan F., Pijeira-Cabrera H. et al.
Discrete--Continuous Jacobi--Sobolev Spaces and Fourier Series
//
Bull. Malays. Math. Sci. Soc.
--- 2020. https://doi.org/10.1007/s40840-020-00950-7.

\bibitem{mmg-Shii-izvran2018}
Шарапудинов И.И.
Ортогональные по Соболеву системы функций, ассоциированные с ортогональной системой функций
//
Изв. РАН. Сер. матем.
--- 2018.
--- Т. 82, вып. 1.
--- С. 225---258. (\url{http://mi.mathnet.ru/izv8536}).

\bibitem{mmg-Shii-matzam2017}
Шарапудинов И.И.
Аппроксимативные свойства рядов Фурье по многочленам, ортогональным по Соболеву с весом Якоби и дискретными массами
//
Матем. заметки.
--- 101:4
--- 2017.
--- С. 611---629.
Math. Notes, 101:4 (2017), 718–734.

\bibitem{mmg-Muck1969}
Muckenhoupt B.
Mean convergence of Jacobi series.
//
Proc. Am. Math. Soc.
--- 1969.
--- V. 23.
--- P. 306---310

\bibitem{mmg-Zorschikov1967}
Зорщиков А.В.
О равномерности сходимости рядов Фурье по многочленам Якоби
//
Докл. АН СССР.
--- 1967.
--- Т. 176, \No 1.
--- P. 35---38.

\bibitem{mmg-Fiht2}
Фихтенгольц Г.М.
Курс дифференциального и интегрального исчисления. В 3 т. Т. II / Пред. и прим. А.А. Флоринского.
--- 8-е изд.
--- М.: ФИЗМАТЛИТ, 2003.
--- 864 с.
--- ISBN 5-9221-0157-9.


%GRM

\bibitem{Ram-Ba-Ra-Pe}
{Barry P. Rajkovi\'c P.M., Petkovi\'c M.D.}
An application of Sobolev orthogonal polynomials to the computation of a special Hankel determinant
//
In book: Approximation and Computation (Chapter 4).
--- 2011.
--- Vol. 42.
--- P. 53---60.

\bibitem{Ram-Mar-Xu}
{Marcell\'an F., Xu Y.}
On Sobolev orthogonal polynomials
//
Expo Math.
--- 2015.
--- Vol. 33.
--- P. 308---352.

\bibitem{Ram-Shar-UMN}
{Шарапудинов И.И.}
Ортогональные по Соболеву системы функций и некоторые их приложения
//
УМН.
--- 2019.
--- Т. 74, вып. 4.
--- С. 87---164.

\bibitem{Ram-SharMN}
{Шарапудинов И.И.}
Аппроксимативные свойства рядов Фурье по многочленам, ортогональным по Соболеву с весом Якоби и дискретными массами
//
Матем. заметки.
--- 2017.
--- Т. 101, вып. 4.
--- С. 611---629.	

\bibitem{Ram-Sege}
{Сеге Г.}
Ортогональные многочлены.
--- Москва. Физматгиз. 1962.

\bibitem{Ram-Shar-VMJ}
{Шарапудинов И.И., Гаджиева З.Д., Гаджимирзаев Р.М.}
Разностные уравнения и полиномы, ортогональные по Соболеву, порожденные многочленами Мейкснера
//
Владикавк. матем. журн.
--- 2017.
--- Т. 19, вып. 2.
--- С. 58---72.

\bibitem{Ram-Ar-Go-Mar}
{Area I., Goboy E., Marcell\'an F.}
Inner products involving differences: The Meixner--Sobolev polynomials
//
J. Differ. Equations Appl.
--- 2000.
--- Vol. 6.
--- P. 1---31.

\bibitem{Ram-Kh-Old}
{Khwaja S.F., Olde-Daalhuis A.B.}
Uniform asymptotic approximations for the Meixner--Sobolev polynomials
//
Analysis and Applications.
--- 2012.
--- Vol. 10. № 3.
--- P. 345---361.

\bibitem{Ram-Bav1}
{Bavinck H., Haeringen H.V.}
Difference equations for generalized Meixner polynomials
//
J. Math. Anal. Appl.
--- 1994.
--- Vol. 1994.
--- P. 453---463.

\bibitem{Ram-Bav2}
{Bavinck H., Koekoek R.}
Difference operators with sobolev type Meixner polynomials as eigenfunctions
//
Comput. Math. Appl.
--- 1998.
--- Vol. 36.
--- P. 163---177.

\bibitem{Ram-Shar-Sar}
{Шарапудинов И.И., Гаджиева З.Д.}
Полиномы, ортогональные по Соболеву, порожденные многочленами Мейкснера
//
Изв. Сарат. ун-та. Нов. сер. Сер. Математика. Механика. Информатика.
--- 2016.
--- Т. 16, вып. 3.
--- С. 310---321.

\bibitem{Ram-Mor-Bal}
{Moreno-Balc\'azar J.}
$\delta$--Meixner--Sobolev orthogonal polynomials: Mehler--heine type formula and zeros
//
J. Comput. Appl. Math.
--- 2015.
--- Vol. 284.
--- P. 228---234.

\bibitem{Ram-Co-So-Vil}
{Costas-Santos R.S., Soria-Lorente A., Vilaire J.-M.}
On polynomials orthogonal with respect to an inner product involving higher-order differences: the Meixner case
//
Mathematics.
--- 2022.
--- Vol. 10.
--- P. 1952.

\bibitem{Ram-SharBook}
{Шарапудинов И.И.}
Многочлены, ортогональные на сетках.
--- Махачкала. Изд-во Даг. гос. пед. ун-та, 1997.

\bibitem{Ram-MN2019}
{Гаджимирзаев Р.М.}
Оценка функции Лебега сумм Фурье по модифицированным полиномам Мейкснера
//
Матем. заметки.
--- 2019.
--- Vol. 106. № 4.
--- P. 519---530.


%ARK

\bibitem{ark-1}
Цалюк~З.Б.
Интегральные уравнения Вольтерра
//
Итоги науки и техн. Сер. Мат. анал.
--- 1977.
--- Т.~15.
--- С.~131---198.

\bibitem{ark-2}
Полянин~А.Д., Манжиров~А.В.
Интегральные уравнения. Часть 1:~справочник для вузов. 2-е изд.
--- М.: Юрайт, 2017.
--- 365~c.

\bibitem{ark-3}
Сидоров~Д.Н.
Методы анализа интегральных динамических моделей: теория и приложения.
--- Иркутск: Изд-во ИГУ, 2013.
--- 293~c.

\bibitem{ark-4}
Алберг~Дж., Нильсон~Э., Уолш~Дж.
Теория сплайнов и ее приложения.
--- М.: Мир, 1972.
--- 319~c.

\bibitem{ark-5}
El Tom~M.E.A.
Numerical solution of Volterra integral equations by spline functions
//
BIT.
--- 1972.
--- V.~13.
--- P.~1---7.

\bibitem{ark-6}
Netravali~A.N.
Spline approximation to the solution of the Volterra integral equation of the second kind
//
Math. Comput.
--- 1973.
--- V.~27, iss.~121.
--- P.~99---106.

\bibitem{ark-7}
Стечкин~С.Б., Субботин~Ю.Н.
Сплайны в вычислительной математике.
--- М.: Наука, 1976.
--- 248~с.

\bibitem{ark-8}
Nord~S.
Approximation properties of the spline fit
//
BIT.
--- 1967.
--- V.~7.
--- P.~132---144.

\bibitem{ark-9}
Рамазанов~А.-Р.К., Магомедова~В.Г.
Безусловно сходящиеся
интерполяционные рациональные сплайны
//
Мат. заметки.
--- 2018.
--- Т.~103, вып.~4.
--- С.~592---603.

\bibitem{ark-10}
Рамазанов~А.-Р.К., Магомедова~В.Г.
Сплайны по трехточечным рациональным интерполянтам с автономными полюсами
//
Дагестанские электронные математические известия.
--- 2017.
--- Вып.~7.
--- C.~16---28.

\bibitem{ark-11}
Рамазанов~А.-Р.К., Магомедова~В.Г.
О приближенном решении дифференциальных уравнений с помощью рациональных сплайн--функций
//
Журнал вычислительной математики и математической физики.
--- 2019.
--- Т.~59, №~4.
--- С.~579---586.

\bibitem{ark-12}
Сендов~Б., Попов~В.А.
Усредненные модули гладкости.
--- М.: Мир, 1988.
--- 328~c.

\bibitem{ark-13}
Рамазанов~А.-Р.К., Рамазанов А.К., Магомедова~В.Г.
О динамическом решении интегрального уравнения Вольтерры в виде рациональных сплайн-функций
//
Мат. заметки.
--- 2022.
--- Т.~111, вып.~4.
--- С.~581---591.

\bibitem{ark-14}
Ramazanov~A.-R.K., Ramazanov A.K., Magomedova V.G.
On the Dynamic Solution of the Volterra Integral Equation in the Form Rational Spline Functions
//
Mathematical Notes.
--- 2022.
--- Vol. 111, \No. 4.
--- P.~596---603.

\bibitem{ark-15}
Рамазанов~А.-Р.К., Магомедова~В.Г.
О приближенном решении интегральных уравнений Фредгольма методом коллокационных рациональных сплайн-функций
//
Дагестанские электронные математические известия.
--- 2022.
--- Вып.~17.
--- C.~20---31.

\bibitem{ark-16}
Рамазанов~А.-Р.К., Алиева Р.Ш.
О гладкой интерполяции локальными полиномиальными сплайнами
//
Вестник Дагестанского государственного университета. Серия 1. Естественные науки.
--- 2022.
--- Том~37, вып.~1.
--- С.~32---39.


% MMA

\bibitem{mma-bib-1}
Prewitt C.T., Shannon R.D., Rogers D.B.
Chemistry of noble metal oxides. II. Crystal structures of PtCoO2, PdCoO2, CuFeO2 and AgFeO2.
//
Inorg. Chem.
--- 1971.
--- V. 10, №4.
--- P. 719---723.

\bibitem{mma-bib-2}
Hirakawa K., Kadowaki H., Ubukoch K.
Experimental studies of triangular lattice antiferromagnets with S = ½: NaTiO2 and LiNiO2.
//
J. Phys. Soc. Japan.
--- 1985.
--- V. 54, №9.
--- P. 3526---3536.

\bibitem{mma-bib-3}
Townsend M.G., Longworth G. and Roudaut E.
Triangular-spin, kagome plane in jarosites
//
Physical Review В.
--- 1986.
--- V. 33.
--- P. 4919---4926.

\bibitem{mma-bib-4}
Li J., Sleight A.W.
Structure of $\beta$-AgAlO2and structural systematics of tetrahedral MM'X2compounds.
//
J. Solid State Chem.
--- 2004.
--- V. 177, №3.
--- P. 889---894.

\bibitem{mma-bib-5}
Sachdev, S.
Kagome- and triangular-lattice Heisenberg antiferromagnets: ordering from quantum fluctuations and quantum-disordered ground states with un-confined bosonic spinons. Phys. Rev. B 45, 12377---12396 (1992).

\bibitem{mma-bib-6}
Xu G., Lian B., Zhang S.-C.
Intrinsic quantum anomalous Hall effect in the Kagome lattice Cs2LiMn3F12
//
Phys. Rev. Lett. 115, 186802 (2015).

\bibitem{mma-bib-7}
Chen H., Niu Q., Macdonald A.H.
Anomalous hall effect arising from noncollinear antiferromagnetism. Phys. Rev. Lett. 112, 17205 (2014)

\bibitem{mma-bib-8}
Balents L.
Spin liquids in frustrated magnets. Nature 464, 199---208 (2010)

\bibitem{mma-bib-9}
Kang M., Ye L., Fang S., et al.
Dirac fermions and flat bands in the ideal kagome metal FeSn
//
Nature Materials.
--- 2020.
--- V. 19.
--- P. 163---170.

\bibitem{mma-bib-10}
Ramazanov M.K., Murtazaev A.K., Magomedov M.A., Badiev M.K.
Phase transitions and thermodynamic properties of antiferromagnetic Ising model with next-nearest-neighbor interactions on the Kagomé lattice
//
Phase Transitions. ---2018. ---V. 91. ---P. 610-618.

\bibitem{mma-bib-11}
Магомедов М.А., Муртазаев А.К.
Плотность состояний и структура основного состояния модели Изинга на решетке Кагоме с учетом взаимодействия ближайших и следующих соседей
//
ФТТ.
--- 2018.
--- Т. 60.
--- С. 1173---1177.

\bibitem{mma-bib-12}
Ramazanov M.K., Murtazaev A.K., Magomedov M.A. Rizvanova T.R., Murtazaeva A.A.
Phase diagram of the Potts model with the number of spin states q=4 on a Kagome lattice
//
Low Temperature Physics.
--- 2021.
--- V. 47, \No 5.
--- P. 396---400.

\bibitem{mma-bib-13}
Landau D.P., Wang F., Tsai S.-H.
Critical endpoint behavior: A Wang-Landau study
//
Comp. Phys. Comm.
--- 2008.
--- V. 179.
--- P. 8.

\bibitem{mma-bib-14}
Körner M., Troyer M., in Computer Simulation Studies in Condensed-Matter Physics XVI, edited by D. Landau, S. Lewis and H.-B. Schütler (Springer Berlin Heidelberg, 2006), Vol. 103, p. 142.

\bibitem{mma-bib-15}
Chiaki Y., Yutaka O.
Three-dimensional antiferromagnetic q -state Potts models: application of the Wang-Landau algorithm
//
Journal of Physics A: Mathematical and General.
--- 2001.
--- V. 34.
--- 8781.

\bibitem{mma-bib-16}
Zhou C.Bhatt R.N.
Understanding and improving the Wang-Landau algorithm
//
Physical Review E.
--- 2005.
--- V. 72(2).
--- P. 025701.


% RMK

\bibitem{rmk-bib-1}
Diep H.T.
Frustrated Spin Systems.
--- World Scientific Publishing Co. Pte. Ltd., Singapore, 2004.
--- P. 624.

\bibitem{rmk-bib-2}
Baxter R.J.
Exactly Solved Models in Statistical Mechanics.
--- Academic, New York, 1982;
--- Mir, Moscow, 1985.

\bibitem{rmk-bib-3}
Wu F.Y.
Exactly Solved Models: A Journey in Statistical Mechanics
--- World Scientific, New Jersey, 2008.

\bibitem{rmk-bib-4}
Ramazanov M.K., Murtazaev A.K., Magomedov M.A.
Phase transitions in the frustrated Potts model in the magnetic field
//
Physica E: Low-dimensional Systems and Nanostructures.
--- 2022.
--- V. 140.
--- P. 115226-1-115226-6.

\bibitem{rmk-bib-5}
Рамазанов М.К., Муртазаев А.К., Магомедов М.А.
Фрустрированная модель Поттса с числом состояний спина $q = 4$ в магнитном поле
//
ЖЭТФ.
--- 2022.
--- Т. 161, вып. 6.
--- С. 816---824.


%KRI

\bibitem{kri-1}
Кадиев Р.И., Поносов А.В.
Исследование устойчивости решений непрерывно-дискретных стохастических систем с последействием методом регуляризации
//
Дифференциальные уравнения.
--- 2022.
--- Т. 58, № 4.
--- С. 435---455.

\bibitem{kri-2}
Кадиев Р.И., Шахбанова З.И.
Экспоненциальная устойчивость решений одной непрерывно--дискретной линейной системы Ито с ограниченными запаздываниями
//
Вестник ДГУ, Серия 1. Естественные науки.
--- 2022.
--- Том 37, вып. 3.
--- С. 7---17.

\bibitem{kri-3}
Кадиев Р.И., Поносов А.В.
Глобальная устойчивость систем нелинейных дифференциальных уравнений ИТО с последействием и $W$--метод Н.В. Азбелева
//
Известия высших учебных заведений. Математика.
--- 2022.
--- № 1.
--- С. 38---56.

\bibitem{kri-4}
Kadiev R.I., Ponosov A.V.
The W-method in staility analysis of stochastic functional differential equations
//
Functional Differential Equations
(журнал издается в Израиле)

\bibitem{kri-5}
Kadiev R., Ponosov A. positive invertibility of matrices and
exponential stability of linear stochastic systems with delay
//
International Journal of Differential Equations.
--- 2022.
--- Т. 2022.
--- С. 5549693.

\bibitem{kri-6}
Kadiev R.I., Ponosov A.V.
Global stability of nonlinear delay ITO equations and N.V. Azbelev's W-method
//
International Workshop QUALITDE.
--- 2021, December 1 - 3, 2022, Tbilisi, Georgia.

%SMM

 \bibitem{smm-1} Жиков В. В, Козлов С. М., Олейник О. А. Усреднение дифференциальных операторов. --- М: Наука, 1993.

 \bibitem{smm-2} Жиков В. В., Пастухова С. Е. Об операторных оценках в теории усреднения // УМН. --- 2016. --- Т. 71., № 3. --- С. 27---122.

 \bibitem{smm-3} Бирман М. Ш., Суслина Т. А. Периодические дифференциальные операторы второго порядка. Пороговые свойства  и усреднения // Алгебра  и  анализ. --- 2003. --- Т. 15, № 5. --- С. 1---108.

 \bibitem{smm-4} Борисов Д. И. Асимптотики решений эллиптических систем с быстро осциллирующими коэффициентами // Алгебра и анализ. --- 2008. --- Т.20, № 2. --- С. 19---42.

 \bibitem{smm-5} Сеник Н. Н. Об усреднении несамосопряженных локально-периодических эллиптических операторов // Функц. Анализ и его прил. --- 2017. --- Т. 51, № 2. --- С. 92---96.

 \bibitem{smm-6} Пастухова C. Е., Тихомиров Р. Н. Операторные оценки повторного  и локально-периодического  усреднения // Доклады РАН. --- 2007. --- Т. 415, № 3. ---  С. 304---309.

 \bibitem{smm-7} Жиков В. В. Об операторных оценках в теории усреднения // Доклады РАН. --- 2005. --- Т. 403, № 3. --- С. 305---308.

 \bibitem{smm-8} Сиражудинов М. М. Асимптотический метод усреднения обобщенных  операторов Бельтрами // Матем. сборник. --- 2017. --- Т. 208, № 4. --- С. 87---110.

 \bibitem{smm-9} Сиражудинов М. М., Тихомирова С. В. Оценки погрешности усреднения периодической задачи для обобщенного уравнения Бельтрами // Дифференциальные уравнения. --- 2020. --- Т. 56, № 12.  С. 1651---1659.

 \bibitem{smm-10} Сиражудинов М. М., Джамалудинова С. П. Оценки погрешности усреднения задачи Римана–Гильберта для уравнения Бельтрами с локально-периодическим коэффициентом //  Вестник Даг. Гос. университета. Серия1: Естественные науки. --- 2021. --- Том 36, № 4. --- С. 23---38.

 \bibitem{smm-11} Сиражудинов М. М. О G-сходимости и усреднении обобщенных операторов Бельтрами // Матем. сборник. --- 2008. --- Том 199, № 5 --- С. 127---158

  \bibitem{smm-12}	Сиражудинов М.М., Ибрагимов М.Г., Магомедова М.Г. Оценки погрешности локально-периодического усреднения периодической задачи для уравнения Бельтрами // Вестник Дагестанского государственного университета. Серия1: Естественные науки. --- 2022. --- № 1. --- С. 24---31.

  \bibitem{smm-13}	Сиражудинов М.М. Оценки локально-периодического усреднения Задачи Римана–Гильберта для обобщённого уравнения Бельтрами // Дифференциальные уравнения. --- 2022. --- Т. 58, № 6. --- С. 777---794.

  \bibitem{smm-14}	Sirazhudinov M.M. Estimates for the Locally Periodic Homogenization of the Riemann–Hilbert Problem for a Generalized Beltrami Equation// Differential Equations. --- 2022. --- Vol. 58. No. 6. --- Pp. 771---790.

  \bibitem{smm-15} Сиражудинов М.М. О периодических решениях одной
  	эллиптической системы первого порядка //  Матем. заметки.
  	--- 1990.  --- Т. 48. № 5. --- С. 153---155.


%Меджидов

\bibitem{mzg-Basko}
Basko R., Zeng G.L., Gullberg G.T.
Analytical reconstruction formula for the one-dimensional Compton camera
//
IEEE Trans. Nucl. Sci.
--- 1997.
--- V. 44, iss. 3.
--- P. 1342---1346.

\bibitem{mzg-Truong}
Truong T.T., Nguen M.K.
On $V$-line Radon transform in $R^2$ and thear inversion
//
J. Phis. A: Math. Theor.
--- 2011.
--- V. 44, \No 075206.
--- P. 13.

\bibitem{mzg-Ambartsoumian1}
Gaik Ambartsoumian and Sunghwan Moon.
A series formula for inversion of the $V$-line Radon transform in a disc
//
Comp. Math. Appl.
--- 2013.
--- V. 19, iss. 66.
--- P. 1567---1572.

\bibitem{mzg-Ambartsoumian2}
Gaik Ambartsoumian, Latifi-Jebelli M.J., Mishra R.K.
Generalized $V$-line transforms in 2D vector tomography
//
Inverse Problems.
--- 2020.
--- V. 36., \No 10,
--- 104002.

\bibitem{mzg-Medzhidov1}
Меджидов З.Г., Гаммадов Ш.М.
Обращение $V$-преобразования Радона в круге по неполным данным
//
ДЭМИ.
--- 2019.
--- Вып. 10.
--- С. 61---65.

\bibitem{mzg-Medzhidov2}
Меджидов З.Г.
Обращение $V$-преобразования Радона со степенным весом на плоскости
//
ДЭМИ.
--- 2022.
--- Вып. 17.
--- C. 32---43.

\bibitem{mzg-Kuchment}
Peter Kuchment, Fatma Terzioglu.
Inversion of weighted divergent beam and cone transforms
//
Inverse Problems and Imaging.
--- 2017.
--- V. 11, \No 6.
--- P. 1071---1090.


% AKM

\bibitem{akm-bib-m1}
Магомедов A.М.
Задания, алгоритмы, программы, результаты.
--- Махачкала, издательство ДГУ. 2022.
--- 160 с.
ISBN 978-5-9913-0256-2.

\bibitem{akm-bib-m2}
Магомедов А.М., Раджабова Н.Ш.
Замощение клетчатой полосы шириной 4
//
Информатика в школе.
--- 2022.
--- № 1 (174).
--- С. 81---84.

\bibitem{akm-bib-m3}
Lawrencenko S., Magomedov A.M.
Enumeration of Abstract Complexes of a Given Kind (Labeled vs. Unlabeled)
//
A Mathematical Approach. Novel Research Aspects in Mathematical and Computer Science.
--- Vol. 8.
--- P. 106---126.
https://doi.org/10.9734/bpi/nramcs/v8/3103C.

\bibitem{akm-bib-m4}
Магомедов А.М., Лавренченко С.А.
Некоторые свойства прямых рекуррентных соотношений для последовательностей димерных чисел
//
Вестник Дагестанского государственного университета. Серия 1. Естественные науки.
--- 2022.
--- Т. 37, вып. №1.
--- С. 51---62.
DOI: 10.21779/2542-0321-2022-37-1-51-62.

\bibitem{akm-bib-m5}
Магомедов А.М.
<<Компьютерное>> решение и обобщение классической арифметической задачи
//
Вестник ДГУ.
--- 2022.
--- Т. 37, вып. 3.
--- С. 25---29.
DOI: 10.21779/2542-0321-2022-37-3-25-29

\bibitem{akm-bib-m6}
Магомедов А.М., Якубов Р.А.
Некоторые подходы к определению четности числа разбиений прямоугольной полосы
//
Вестник ДГУ.
--- 2022.
--- Т. 37, вып. 3.
--- С. 30---33.
DOI: 10.21779/2542-0321-2022-37-3-30-33
http://vestnik.dgu.ru/Stat/v2022-est-3-4.pdf

\bibitem{akm-bib-m7}
Магомедов А.М., Якубов Р.А.
Два подхода к определению четности числа разбиений прямоугольной полосы
//
Труды III Всероссийской научной конференции 7-9 февраля 2022 г. <<Актуальные проблемы математики и информационных технологий>>.

\bibitem{akm-bib-m8}
Магомедов А.М., Лавренченко С.А.
Пример применения Wolfram Mathematica для решения задачи перечислительной комбинаторики
//
Труды III Всероссийской научной конференции 7-9 февраля 2022 г.  <<Актуальные проблемы математики и информационных технологий>>.

\bibitem{akm-bib-m9}
Магомедов А.М., Лавренченко С.А.
Свидетельство № 2022615932 государственной регистрации программы для ЭВМ <<Программа вычисления коэффициентов прямой рекурсии для последовательности димерных чисел>>. Заявка №2022615303, дата поступления 22 марта 2022, дата государственной регистрации в Реестре программ для ЭВМ 04 апреля 2022 г.

\bibitem{akm-bib-m10}
Мирзоев Г.М., Магомедов А.М.
Свидетельство № 2022619428 о государственной регистрации программы для ЭВМ <<Программа построения двудольного графа с функцией раскрашивания ребер>>. Заявка № 2022617984, дата поступления 28 апреля 2022 г., дата государственной регистрации в Реестре программ для ЭВМ 20 Мая 2022 г

\bibitem{akm-bib-m11}
Магомедов А.М., Мирзоев Г.М.
Свидетельство № 2022662175 о государственной регистрации программы для ЭВМ <<Интеллектуальная система перемещения пакетов между заданными ячейками склада>>. Заявка № 2022661007, дата поступления 14 июня 2022 г., дата государственной регистрации в Реестре программ для ЭВМ 30 июня 2022 г.

\bibitem{akm-bib-m12}
Магомедов А.М.
Свидетельство № 2022668726 о государственной регистрации программы для ЭВМ <<Программа реализации <<компьютерного>> уточнения формулировки классической задачи теории чисел>>. Заявка № 2022667909, дата поступления 30 сентября 2022 г., дата государственной регистрации в Реестре программ для ЭВМ 11 октября 2022 г.

\bibitem{akm-bib-m13}
Магомедов А.М.
Свидетельство № 2022668725 о государственной регистрации программы для ЭВМ <<Программа обработки цветов фона и текста при реставрации старой рукописи>>. Заявка № 2022667909, дата поступления 30 сентября 2022 г., дата государственной регистрации в Реестре программ для ЭВМ 11 октября 2022 г.

\bibitem{akm-bib-m14}
Магомедов А.М.
Свидетельство № 2022668847 о государственной регистрации программы для ЭВМ <<Программа визуализации взвешенного двудольного графа без налегания реберных меток>>. Заявка № 2022667822, дата поступления 30 сентября 2022 г., дата государственной регистрации в Реестре программ для ЭВМ 12 октября 2022 г.

\bibitem{akm-bib-m15}
Магомедов А.М.
Свидетельство № 2022668825 о государственной регистрации программы для ЭВМ <<Программа интервальной раскраски графа методом жадного алгоритма с возвратами>>.
Заявка № 2022667821, дата поступления 30 сентября 2022 г., дата государственной регистрации в Реестре программ для ЭВМ 12 октября 2022 г.

\bibitem{akm-bib-m16}
Магомедов А.М., Якубов Р.А.
Свидетельство № 2022682652 о государственной регистрации программы для ЭВМ <<Программа оцифровки последовательности звуковых сигналов>>.
Заявка № 2022681087, дата поступления 03 ноября 2022 г., дата государственной регистрации в Реестре программ для ЭВМ 24 ноября 2022 г.


% ShTN

\bibitem{tad-SHII-Haar}
Шарапудинов И.И.
О базисности системы Хаара в пространстве $L^{p(t)}([0,1])$ и принципе локализации в среднем
//
Мат. сборник.
--- 1986.
--- Т. 130(172), № 2(6).
--- С. 275---283.

\bibitem{tad-SHII-AnalisysMath}
Some aspects of approximation theory in the spaces $L^{p(x)}(E)$
//
Analysis Mathematica.
--- 2007.
--- Vol 33.
--- Pp. 135---153.

\bibitem{tad-SHII-Leg}
О базисности системы полиномов Лежандра в пространстве Лебега $L^{p(x)}(-1,1)$ с переменным показателем $p(x)$
//
Мат. сборник.
--- 2009.
--- Т. 200, № 1.
--- С. 137---160.

\bibitem{tad-SHII-Jacob}
Шарапудинов И.И., Шах-Эмиров Т.Н.
Сходимость рядов Фурье по полиномам Якоби в весовом пространстве Лебега с переменным показателем
//
Дагестанские электронные математические известия.
--- 2017.
--- Вып. 8.
--- С. 27---47.

\bibitem{tad-MMG-Haar}
Магомед-Касумов М.Г.
Базисность системы Хаара в весовых пространствах Лебега с переменным показателем
//
Владикавк. матем. журн.
--- 2014.
--- Т. 16, № 3.
--- С. 38---46.

\bibitem{tad-SHII-Ult}
Шарапудинов И.И.
О базисности ультрасферических полиномов Якоби в весовом пространстве Лебега с переменным показателем
//
Мат. заметки.
--- 2019.
--- Т. 106, № 4.
--- С. 616---638.

\bibitem{tad-RAM-Jacob}
Shakh-Emirov T.N., Gadzhimirzaev R.M.
The Convergence of the Fourier-Jacobi Series in Weighted Variable Exponent Lebesgue Spaces
//
Operator Theory and Differential Equations. Trends in Mathematics. Birkhäuser, Cham.  In: Kusraev, A.G., Totieva, Z.D. (eds).
--- 2021.
--- Pp. 205---227.

\bibitem{tad-lpxtopology}
Шарапудинов И.И.
О топологии пространства $L^{p(t)}([0,1])$
//
Мат. заметки.
--- 1979.
--- Т. 26, № 4.
--- С. 613---632.

\end{thebibliography} 