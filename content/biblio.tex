\begin{thebibliography}{111}

  % Ниже указаны примеры форматирования литературы

  \bibitem{mmg-MarcellanXu2015}
  Marcellán F., Xu Y. On Sobolev orthogonal polynomials // Expositiones Math. --- 2015. --- Vol 33. P. 308---352.

  \bibitem{mmg-mmg-walsh-Shii-UMN}
  Шарапудинов И.И. Ортогональные по Соболеву системы функций и некоторые их приложения // УМН. --- 2019. --- Т. 74, \No 4(448). --- С. 87---164.

  \bibitem{ark-bib-2}
  Стечкин С.Б., Субботин Ю.Н.
  Сплайны в вычислительной математике.
  --- М.: Наука,
  1976. --- 248~с.

  \bibitem{ark-bib-3}
  Завьялов Ю.С., Квасов Б.И., Мирошниченко В.Л.
  Методы сплайн-функций.
  --- М.: Наука,
  1980.
  --- 352 c.
  
  %MMG
  	\bibitem{mmg-MarcellanXu2015}
	Marcellan F., Xu Y. On Sobolev orthogonal polynomials // Expo Math. --- 2015. --- Vol 33. P. 308---352.
	
	
	
	
	\bibitem{mmg-MarcellanJacobiSobolev}
	Marcellan, F., Quintana, Y., Urieles, A.: On the Pollard decomposition method applied to some Jacobi–
	Sobolev expansions. Turk. J. Math. 37(6), 934–948 (2013).
	
	
	
	
	\bibitem{mmg-CiaurriJacobiSobolev}
	Ciaurri, O., Minguez, J.: Fourier series of Jacobi–Sobolev polynomials. Integral Transf. Spec. Funct.
	30, 334–346 (2019).
	
	
	
	
	\bibitem{mmg-CiaurriCoherentPairs}
	Ciaurri, O., Minguez, J.: Fourier series for coherent pairs of Jacobi measures. Preprint.
	
	
	
	
	\bibitem{mmg-Fejzullahu2010}
	B.\,Xh. Fejzullahu. Asymptotic properties and Fourier expansions of orthogonal polynomials with a non-discrete Gegenbauer–Sobolev inner product. Journal of Approximation Theory, Volume 162, Issue 2, 2010, Pages 397-406, ISSN 0021-9045, https://doi.org/10.1016/j.jat.2009.07.002.
	
	
	
	
	\bibitem{mmg-Fejzullahu2013}
	B.\,Xh. Fejzullahu, F. Marcellan, J.J. Moreno-Balcazar, Jacobi–Sobolev orthogonal polynomials: Asymptotics and a Cohen type inequality, Journal of Approximation Theory,
	Volume 170, 2013, Pages 78-93, ISSN 0021-9045, https://doi.org/10.1016/j.jat.2012.05.015.
	
	
	
	
	
	\bibitem{mmg-IserlesKoch1991}
	Iserles, A., Koch, P.E., Norsett, S.P., Sanz-Serna, J.M.: On polynomials orthogonal with respect to
	certain Sobolev inner product. J. Approx. Theory 65, 151–175 (1991).
	
	
	
	
	\bibitem{mmg-Marcellan2002}
	F. Marcellan, B.P. Osilenker, I.A. Rocha, On Fourier Series of a Discrete Jacobi–Sobolev Inner Product, Journal of Approximation Theory, Volume 117, Issue 1, 2002,Pages 1-22, ISSN 0021-9045, https://doi.org/10.1006/jath.2002.3681.
	
	
	
	
	\bibitem{mmg-Rocha2003}
	Rocha I. A., Marcellan F., Salto L. Relative asymptotics and Fourier series of orthogonal
	polynomials with a discrete Sobolev inner product // J. Approx. Theory. 2003. V. 121.
	P. 336–356.
	
	
	
	
	\bibitem{mmg-OsilenkerFourier2012}
	Осиленкер Борис Петрович Сходимость и суммируемость рядов Фурье - Соболева // Вестник МГСУ. 2012. №5. URL: https://cyberleninka.ru/article/n/shodimost-i-summiruemost-ryadov-furie-soboleva-1 (дата обращения: 23.10.2020).
	
	
	
	
	\bibitem{mmg-OsilenkerLinearMethods2015}
	Б. П. Осиленкер, О линейных методах суммирования рядов Фурье по многочленам, ортогональным в дискретных пространствах Соболева, Сиб. матем. журн., 2015, том 56, номер 2, 420–435.
	
	
	
	
	\bibitem{mmg-Fejzullahu2009}
	Fejzullahu, Marcellan. On convergence and divergence of Fourier expansions with respect to some Gegenbauer-Sobolev type inner product.
	Communications in the Analytic Theory of Continued Fractions, 2009, n. 16, p. 1-11.
	
	
	
	
	\bibitem{mmg-CiaurriSigma2018}
	Ciaurri, O., Minguez, J.: Fourier series of Gegenbauer–Sobolev polynomials. SIGMASymm. Integrabi.
	Geom. Methods Appl. 14, 1–11 (2018).
	
	
	
	
	
	\bibitem{mmg-SharapudinovUMN}
	И. И. Шарапудинов, “Ортогональные по Соболеву системы функций и некоторые их приложения”, УМН, 74:4(448) (2019), 87–164.
	
	
	
	
	\bibitem{mmg-SharapudinovIzvRan2019}
	И. И. Шарапудинов, “Системы функций, ортогональные по Соболеву, ассоциированные с ортогональной системой”, Изв. РАН. Сер. матем., 82:1 (2018), 225–258.
	
	
	
	
	\bibitem{mmg-MMG2019}
	М. Г. Магомед-Касумов, “Система функций, ортогональная в смысле Соболева и порожденная системой Уолша”, Матем. заметки, 105:4 (2019), 545–552; Math. Notes, 105:4 (2019), 543–549.
	
	
	
	
	\bibitem{mmg-Gadzhimirzaev2019}
	R. M. Gadzhimirzaev, “Sobolev-orthonormal system of functions generated by the system of Laguerre functions”, Пробл. анал. Issues Anal., 8(26):1 (2019), 32–46.
	
	
	
	
	\bibitem{mmg-Diaz-Gonzalez2020}
	Diaz-Gonzalez, A., Marcellan, F., Pijeira-Cabrera, H. et al. Discrete–Continuous Jacobi–Sobolev Spaces and Fourier Series. Bull. Malays. Math. Sci. Soc. (2020). https://doi.org/10.1007/s40840-020-00950-7.
	
	
	
	
	
	\bibitem{mmg-Shii-izvran2018}
	И. И. Шарапудинов. Ортогональные по Соболеву системы функций, ассоциированные с ортогональной системой функций // Изв. РАН. Сер. матем., 2018, том 82, выпуск 1, с. 225--258. (\url{http://mi.mathnet.ru/izv8536}).  	
	
	
	
	
	\bibitem{mmg-Shii-matzam2017}
	И. И. Шарапудинов, Аппроксимативные свойства рядов Фурье по многочленам, ортогональным по Соболеву с весом Якоби и дискретными массами, Матем. заметки, 101:4 (2017), 611–629; Math. Notes, 101:4 (2017), 718–734.
	
	\bibitem{mmg-Muck1969}
	Muckenhoupt, B.: Mean convergence of Jacobi series. Proc. Am. Math. Soc. 23, 306–310 (1969).
	
	
	\bibitem{mmg-Zorschikov1967}
	А. В. Зорщиков, О равномерности сходимости рядов Фурье по многочленам Якоби, Докл. АН СССР, 1967, том 176, номер 1, 35–38.
	
	
	\bibitem{mmg-Fiht2}
	Г.М. Фихтенгольц. Курс дифференциального и интегрального исчисления. В 3 т. Т. II / Пред. и прим. А.А. Флоринского. --- 8-е изд. --- М.: ФИЗМАТЛИТ, 2003. --- 864 с. --- ISBN 5-9221-0157-9.

%GRM

\bibitem{Ram-Ba-Ra-Pe}
{Barry P. Rajkovi\'c P.M., Petkovi\'c M.D.} An application of Sobolev orthogonal polynomials to the computation of a special Hankel determinant // In book: Approximation and Computation (Chapter 4). 2011. Vol. 42. Pp. 53--60.

\bibitem{Ram-Mar-Xu}
{Marcell\'an F., Xu Y.} On Sobolev orthogonal polynomials // Expo Math. 2015. Vol. 33. Pp. 308--352.

\bibitem{Ram-Shar-UMN}
{Шарапудинов И.И.} Ортогональные по Соболеву системы функций и некоторые их приложения // УМН. 2019. Т. 74. Вып. 4. С. 87--164.

\bibitem{Ram-SharMN}
{Шарапудинов И.И.} Аппроксимативные свойства рядов Фурье по многочленам, ортогональным по Соболеву с весом Якоби и дискретными массами // Матем. заметки. 2017. Т. 101. Вып. 4. С. 611--629.	
	
\bibitem{Ram-Sege}
{Сеге Г.} Ортогональные многочлены. Москва. Физматгиз. 1962.	

\bibitem{Ram-Shar-VMJ}
{Шарапудинов И.И., Гаджиева З.Д., Гаджимирзаев Р.М.} Разностные уравнения и полиномы, ортогональные по Соболеву, порожденные многочленами Мейкснера //  Владикавк. матем. журн. 2017. Т. 19. Вып. 2. С. 58--72.

\bibitem{Ram-Ar-Go-Mar}
{Area I., Goboy E., Marcell\'an F.} Inner products involving differences: The Meixner–Sobolev polynomials // J. Differ. Equations Appl. 2000. Vol. 6. Pp. 1--31.

\bibitem{Ram-Kh-Old}
{Khwaja S.F., Olde-Daalhuis A.B.} Uniform asymptotic approximations for the Meixner–Sobolev polynomials // Analysis and Applications. 2012. Vol. 10. № 3. Pp. 345--361.

\bibitem{Ram-Bav1}
{Bavinck H., Haeringen H.V.} Difference equations for generalized Meixner polynomials // J. Math. Anal. Appl. 1994. Vol. 1994. Pp. 453--463.

\bibitem{Ram-Bav2}
{Bavinck H., Koekoek R.} Difference operators with sobolev type Meixner polynomials as eigenfunctions // Comput. Math. Appl. 1998. Vol. 36. Pp. 163--177.

\bibitem{Ram-Shar-Sar}
{Шарапудинов И.И., Гаджиева З.Д.} Полиномы, ортогональные по Соболеву, порожденные многочленами Мейкснера // Изв. Сарат. ун-та. Нов. сер. Сер. Математика. Механика. Информатика. 2016. Т. 16. Вып. 3. С. 310--321.

\bibitem{Ram-Mor-Bal}
{Moreno-Balc\'azar J.} $\delta$-Meixner-Sobolev orthogonal polynomials: Mehler–heine type formula and zeros // J. Comput. Appl. Math. 2015. Vol. 284. Pp. 228--234.

\bibitem{Ram-Co-So-Vil}
{Costas-Santos R.S., Soria-Lorente A., Vilaire J.-M.} On polynomials orthogonal with respect to an inner product involving higher-order differences: the Meixner case // Mathematics. 2022. Vol. 10. Pp. 1952.

\bibitem{Ram-SharBook}
{Шарапудинов И.И.} Многочлены, ортогональные на сетках. Махачкала. Изд-во Даг. гос. пед. ун-та. 1997.

\bibitem{Ram-MN2019}
{Гаджимирзаев Р.М.} Оценка функции Лебега сумм Фурье по модифицированным полиномам Мейкснера // Матем. заметки. 2019. Vol. 106. № 4. Pp. 519--530.

%ARK

\bibitem{ark-1} Цалюк~З.Б. Интегральные уравнения Вольтерра //
 Итоги науки и техн. Сер. Мат. анал. 1977. Т.~15.
  С.~131--198.

\bibitem{ark-2} Полянин~А.Д., Манжиров~А.В.
Интегральные уравнения. Часть 1:~справочник для вузов.  2-е изд.
 М.: Юрайт, 2017. 365~c.

\bibitem{ark-3} Сидоров~Д.Н. Методы анализа интегральных динамических
моделей: теория и приложения.  Иркутск: Изд-во ИГУ, 2013. 293~c.

\bibitem{ark-4} Алберг~Дж., Нильсон~Э., Уолш~Дж. Теория сплайнов и ее приложения. М.: Мир, 1972. 319~c.

\bibitem{ark-5} El Tom~M.E.A. Numerical solution of Volterra integral equations by spline
 functions //BIT.  1972. V.~13. P.~1--7.

\bibitem{ark-6} Netravali~A.N. Spline approximation to the solution of the Volterra integral equation of the
 second kind // Math. Comput. 1973. V.~27.
 Iss.~121. P.~99--106.

\bibitem{ark-7} Стечкин~С.Б., Субботин~Ю.Н. Сплайны в вычислительной математике.
 М.: Наука, 1976. 248~с.

\bibitem{ark-8} Nord~S. Approximation properties of the spline fit//~ BIT.
1967. V.~7. P.~132—144.

\bibitem{ark-9} Рамазанов~А.-Р.К., Магомедова~В.Г. Безусловно сходящиеся
интерполяционные рациональные сплайны // Мат. заметки. 2018. Т.~103.
 Вып.~4. С.~592--603.

\bibitem{ark-10} Рамазанов~А.-Р.К., Магомедова~В.Г. Сплайны по трехточечным рациональным интерполянтам
с автономными полюсами //
Дагестанские электронные математические известия. 2017. Вып.~7. C.~16--28.

\bibitem{ark-11} Рамазанов~А.-Р.К., Магомедова~В.Г. О приближенном решении дифференциальных
уравнений с помощью рациональных сплайн-функций // Журнал вычислительной математики и
математической физики. 2019. Т.~59. №~4. С.~579–586.

\bibitem{ark-12} Сендов~Б., Попов~В.А. Усредненные модули гладкости. М.: Мир,
1988. 328~c.

\bibitem{ark-13} Рамазанов~А.-Р.К., Рамазанов А.К., Магомедова~В.Г.
 О динамическом решении интегрального уравнения Вольтерры в виде
 рациональных сплайн-функций // Мат. заметки. 2022. Т.~111.
 Вып.~4. С.~581--591.

\bibitem{ark-14} Ramazanov~A.-R.K., Ramazanov A.K., Magomedova V.G.
On the Dynamic Solution of the Volterra Integral Equation in the Form
Rational Spline Functions// Mathematical Notes. 2022. Vol. 111, No. 4.
P.~596 – 603.

\bibitem{ark-15} Рамазанов~А.-Р.К., Магомедова~В.Г.
 О приближенном решении интегральных уравнений Фредгольма методом
коллокационных рациональных сплайн-функций //
Дагестанские электронные математические известия. 2022. Вып.~17. C.~20--31.

\bibitem{ark-16} Рамазанов~А.-Р.К., Алиева Р.Ш. О гладкой интерполяции
 локальными полиномиальными сплайнами // Вестник Дагестанского
государственного университета. Серия 1. Естественные науки. 2022. Том~37.
 Вып.~1. С.~32--39.


  % MMA
  
  \bibitem{mma-bib-1}
  Prewitt C.T., Shannon R.D., Rogers D.B.
  Chemistry of noble metal oxides. II. Crystal structures of PtCoO2, PdCoO2, CuFeO2 and AgFeO2.
  //
  Inorg. Chem.
  --- 1971.
  --- V. 10, №4.
  --- P. 719---723.
  
  \bibitem{mma-bib-2}
  Hirakawa K., Kadowaki H., Ubukoch K.
  Experimental studies of triangular lattice antiferromagnets with S = ½: NaTiO2 and LiNiO2.
  //
  J. Phys. Soc. Japan.
  --- 1985.
  --- V. 54, №9.
  --- P. 3526---3536.
  
  \bibitem{mma-bib-3}
  Townsend M.G., Longworth G. and Roudaut E.
  Triangular-spin, kagome plane in jarosites
  //
  Physical Review В.
  --- 1986.
  --- V. 33.
  --- P. 4919---4926.
  
  \bibitem{mma-bib-4}
  Li J., Sleight A.W.
  Structure of $\beta$-AgAlO2and structural systematics of tetrahedral MM'X2compounds.
  //
  J. Solid State Chem.
  --- 2004.
  --- V. 177, №3.
  --- P. 889---894.
  
  \bibitem{mma-bib-5}
  Sachdev, S.
  Kagome- and triangular-lattice Heisenberg antiferromagnets: ordering from quantum fluctuations and quantum-disordered ground states with un-confined bosonic spinons. Phys. Rev. B 45, 12377---12396 (1992).
  
  \bibitem{mma-bib-6}
  Xu G., Lian B., Zhang S.-C.
  Intrinsic quantum anomalous Hall effect in the Kagome lattice Cs2LiMn3F12
  //
  Phys. Rev. Lett. 115, 186802 (2015).
  
  \bibitem{mma-bib-7}
  Chen H., Niu Q., Macdonald A.H.
  Anomalous hall effect arising from noncollinear antiferromagnetism. Phys. Rev. Lett. 112, 17205 (2014)
  
  \bibitem{mma-bib-8}
  Balents L.
  Spin liquids in frustrated magnets. Nature 464, 199---208 (2010)
  
  \bibitem{mma-bib-9}
  Kang M., Ye L., Fang S., et al.
  Dirac fermions and flat bands in the ideal kagome metal FeSn
  //
  Nature Materials.
  --- 2020.
  --- V. 19.
  --- P. 163---170.
  
  \bibitem{mma-bib-10}
  Ramazanov M.K., Murtazaev A.K., Magomedov M.A., Badiev M.K.
  Phase transitions and thermodynamic properties of antiferromagnetic Ising model with next-nearest-neighbor interactions on the Kagomé lattice
  //
  Phase Transitions. ---2018. ---V. 91. ---P. 610-618.
  
  \bibitem{mma-bib-11}
  Магомедов М.А., Муртазаев А.К.
  Плотность состояний и структура основного состояния модели Изинга на решетке Кагоме с учетом взаимодействия ближайших и следующих соседей
  //
  ФТТ.
  --- 2018.
  --- Т. 60.
  --- С. 1173---1177.
  
  \bibitem{mma-bib-12}
  Ramazanov M.K., Murtazaev A.K., Magomedov M.A. Rizvanova T.R., Murtazaeva A.A.
  Phase diagram of the Potts model with the number of spin states q=4 on a Kagome lattice
  //
  Low Temperature Physics.
  --- 2021.
  --- V. 47, \No 5.
  --- P. 396---400.
  
  \bibitem{mma-bib-13}
  Landau D.P., Wang F., Tsai S.-H.
  Critical endpoint behavior: A Wang-Landau study
  //
  Comp. Phys. Comm.
  --- 2008.
  --- V. 179.
  --- P. 8.
  
  \bibitem{mma-bib-14}
  Körner M., Troyer M., in Computer Simulation Studies in Condensed-Matter Physics XVI, edited by D. Landau, S. Lewis and H.-B. Schütler (Springer Berlin Heidelberg, 2006), Vol. 103, p. 142.
  
  \bibitem{mma-bib-15}
  Chiaki Y., Yutaka O.
  Three-dimensional antiferromagnetic q -state Potts models: application of the Wang-Landau algorithm
  //
  Journal of Physics A: Mathematical and General.
  --- 2001.
  --- V. 34.
  --- 8781.
  
  \bibitem{mma-bib-16}
  Zhou C.Bhatt R.N.
  Understanding and improving the Wang-Landau algorithm
  //
  Physical Review E.
  --- 2005.
  --- V. 72(2).
  --- P. 025701.

\end{thebibliography} 